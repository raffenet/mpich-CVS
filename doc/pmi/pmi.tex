% 
%   This is a latex file that generates a reference manual for 
%   PMI
%
\documentclass{article}
\usepackage{tpage}
\usepackage{../adi3/refman}
\usepackage{url} 
\usepackage{epsf}
%\usepackage{psfig}
%hyperref - do not remove this comment

\textheight=9in
\textwidth=6.1in
\oddsidemargin=.2in
\topmargin=-.50in
\newread\testfile

%
% For now, let findex be the same as index.  This will allow us to
% more easily separate function and nonfunction index entries later.
\let\findex=\index
%
% Modify the way titles are handled for no breaks between pages
\def\mantitle#1#2#3{\pagerule\nobreak
\ifmancontents\addcontentsline{toc}{subsection}{#1}\fi
\index{#1}}

\makeindex

\begin{document}

\markright{PMI Reference Manual}

\def\nopound{\catcode`\#=13}
{\nopound\gdef#{{\tt \char`\#}}}
\catcode`\_=13
\def_{{\tt \char`\_}}
\catcode`\_=11
\def\code#1{{\tt #1}}
%\let\url=\code
\def\makeussubscript{\catcode`\_=8}
\def\makeustext{\catcode`\_=11}
%\tpageoneskip
\ANLTMTitle{Process Manager Interface\\
Version 1.0\\
Reference Manual\\\ \\Draft of \today}{\em 
William Gropp\\
Ewing Lusk\\
Brian R. Toonen\\
David Ashton\\
Mathematics and Computer Science Division\\
Argonne National Laboratory}{00}{\today}

\clearpage

\pagenumbering{roman}
\tableofcontents
\clearpage

\pagenumbering{arabic}
\pagestyle{headings}

\section{Introduction}

MPI needs the services of a process manager.
Many MPI implementations include their own process managers.
For greatest flexibility and usability, an MPI program should work with any
process manager, including third-party process managers (e.g., PBS).
MPICH2 uses a ``Process Manager Interface'' (PMI) to separate the process
management functions from the MPI implementation.
PMI should be scalable in design and functional in specification.
MPICH2 executables can be handled by different process managers without
recompiling or relinking.
MPICH2 includes multiple implementations.
We are in discussions with vendors to support PMI directly.

\section{Process group functions}
\input{pmiman/PMI_Init.tex}
\input{pmiman/PMI_Initialized.tex}
\input{pmiman/PMI_Finalize.tex}
\input{pmiman/PMI_Get_rank.tex}
\input{pmiman/PMI_Get_size.tex}
\input{pmiman/PMI_Get_id.tex}
\input{pmiman/PMI_Get_kvs_domain_id.tex}
\input{pmiman/PMI_Get_id_length_max.tex}
\input{pmiman/PMI_Get_clique_size.tex}
\input{pmiman/PMI_Get_clique_ranks.tex}
\input{pmiman/PMI_Barrier.tex}
%\input{ch3man/.tex}

\section{Keyval space functions}
\input{pmiman/PMI_KVS_Get_my_name.tex}
\input{pmiman/PMI_KVS_Get.tex}
\input{pmiman/PMI_KVS_Put.tex}
\input{pmiman/PMI_KVS_Commit.tex}
\input{pmiman/PMI_KVS_Create.tex}
\input{pmiman/PMI_KVS_Destroy.tex}
\input{pmiman/PMI_KVS_Get_name_length_max.tex}
\input{pmiman/PMI_KVS_Get_key_length_max.tex}
\input{pmiman/PMI_KVS_Get_value_length_max.tex}
\input{pmiman/PMI_KVS_Iter_first.tex}
\input{pmiman/PMI_KVS_Iter_next.tex}

\section{Spawn function}
\input{pmiman/PMI_keyval_t.tex}
\input{pmiman/PMI_Spawn_multiple.tex}
\input{pmiman/PMI_Args_to_keyval.tex}

\section{PMI Constants}
\input{pmiman/PMI_CONSTANTS.tex}

% Index
%\openin\testfile{pmi.ind}
%\ifeof\testfile\else
\let\SaveIndex=\theindex
\long\def\theindex#1{\SaveIndex{#1}\addcontentsline{toc}{section}{Index}}
\input pmi.ind
%\fi
%\closein\testfile

\end{document}
