\documentclass{article}
\title{A SPMI Wire Protocol}
\begin{document}
%
% A description of the wire protocol for the simple PMI interface
%
% This initial proposal is (mostly) backward compatible
%
This document describes a human-readable wire protocol for PMI
commands.  All data is transmitted as valid printable ASCII
characters.

This protocol is backward compatible with the current simple PMI code
for all commands except \texttt{cmd=spawn}.   This proposal addresses
deficiencies in the current simple PMI implementation of the
\texttt{spawn} command.  It also adds a version number in the
handshake at initialization time to allow evolution of the format.

% BNF for the wire protocol.
The following BNF describes the simple PMI wire protocol.  Literals are 
described using Perl-style regular expressions, using expressions within single
quotes.
\begin{verbatim}
    SESSION     -> PREAMBLE [ STATEMENTS ]* POSTAMBLE
    STATEMENTS  -> COMMAND
                |  MULTILINE_COMMAND
    PREAMBLE    -> 'cmd=open\s+version=' VERSION_NUM EOL
                | empty
    POSTAMBLE   -> 'cmd=close' EOL
                | empty
    COMMAND     -> 'cmd=' NAME [ '\s+'  RESTRICTED_KEYVAL ]* EOL
    RESTRICTED_KEYVAL -> NAME '=' NAME
    NAME        -> '[^=\s\r\n]+'
    
    MULTILINE_COMMAND => 'mcmd=' NAME '\s+' 
                            [ KEYVAL EOL ]* 'endcmd' EOL
    KEYVAL      -> '\s*' NAME '=' VALUE EOL
    VALUE       -> '.*'

    VERSION_NUM -> '[0-9]+\.[0-9]+'
    EOL         -> '\r?\n'
\end{verbatim}
The following is a short summary of the Perl-style regular expression
terms used above:
\begin{description}
\item[\texttt{[...]}]a character class.  E.g, the digits zero through
 nine are \texttt{[0-9]}.
\item[\texttt{[\char`\^...]}]the complement of a character class.
\item[\texttt{*}]zero or more instances of the previous expression.
\item[\texttt{+}]one or more instances of the previous expression.
\item[\texttt{.}]any character other than newline or null.
\item[\texttt{?}]zero or one instance of the previous expression.
\item[\texttt{\char`\\s}]a space character (blank, tab, etc.).
\item[\texttt{\char`\\n}]a newline.
\item[\texttt{\char`\\r}]a carriage return.
\end{description}
Following common BNF practice, square brackets outside of a string
denote an optional item.  To shorten the BNF, an asterisk (\texttt{*})
is used to indicate zero or more instances of an item, and when
combined with square brackets, indicates an optional sequence of
items.  For example, 
\begin{verbatim}
    [ KEYVAL '\n' ]*
\end{verbatim}
indicates any number of optional KEYVAL items, each followed by a
newline character.

The term \texttt{empty} is a special case; this is the empty item.  In
the above, it lets us consider backward compatibility to PMI
implementations that do not support the preamble command.

\end{document}

