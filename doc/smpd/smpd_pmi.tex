%
%  This is a latex file that generates a reference manual for
%  the wire protocol between an mpich2 process and a smpd manager.
%
\documentclass[dvipdfm,11pt]{article}
\usepackage[dvipdfm]{hyperref} % Upgraded url package
\usepackage{epsf}
\parskip=.1in

\begin{document}
\markright{SMPD PMI Wire Protocol Reference Manual}
\title{SMPD PMI Wire Protocol Reference Manual\\
Version 0.1\\
DRAFT of \today\\
Mathematics and Computer Science Division\\
Argonne National Laboratory}

\author{David Ashton}


\maketitle

\cleardoublepage

\pagenumbering{roman}
\tableofcontents
\clearpage

\pagenumbering{arabic}
\pagestyle{headings}


\section{Introduction}

When a user builds MPICH2 they have the option to choose the SMPD process
manager to launch and manage processes in MPICH2 jobs.  MPICH2 provides
an implementation of smpd and mpiexec to launch MPICH2 jobs.  MPICH2 
applications communicate with the process manager using the PMI interface.
The PMI library for smpd provides an implementation of PMI for communicating
with SMPD process managers. This document describes the environment and wire
protocol between the MPICH2 application and the SMPD manager.

If a process manager implementor replicates the environment and protocol
described in this document, they would be able to launch and manage MPICH2 
jobs compiled for SMPD.

An SMPD manager communicates with its child process through environment
variables and a socket.  This document describes the environment and the 
wire protocol on that socket.

\section{SMPD manager topology}
This section describes how SMPD is organized in MPICH2.  An implementation
of a process manager that uses the protocol described in this document is 
not required to use this topology.  It is provided for reference.

In the idle state, SMPDs reside on each node unconnected.  
When a new job is to be launched, mpiexec first selects a list of hosts 
to launch a job on.  Then it connects to the SMPDs and they fork or spawn 
new managers resulting in a connected tree with mpiexec at the root.
See figure~\ref{fig:tree}.  This tree remains for the duration of the job.
It can grow as a result of spawn commands.  Each SMPD manager has
a id in the tree used to route commands.  Each manager can manage multiple
child processes.  The control socket connections between the SMPD manager
and the child processes are referenced by context ids.  The SMPD manager
provides the context id to the child when it launches a process.

\begin{figure}
\centerline{\epsfbox{smpd_tree.eps}}
\caption{SMPD Manager tree}
\label{fig:tree}
\end{figure}

\section{Child process environment}
SMPD managers launch and manage child processes in an MPICH2 job.
MPICH2 processes compiled with the SMPD PMI library expect the following
environment variables to be set:

\begin{description}
\item PMI\_RANK = my rank in the process group ( 0 to N-1 )
\item PMI\_SIZE = process group size ( N )
\item PMI\_KVS = my keyval space name unique to my process group
\item PMI\_DOMAIN = my keyval space domain name
\item *PMI\_CLIQUE = my node neighbors in the form of a clique.  A clique is
a comma separated list of ranges and numbers.  Example: 0,2..4,7
\item PMI\_SMPD\_ID = my smpd manager node id
\item PMI\_SMPD\_KEY = ctx\_key value to be included with PMI commands 
from this process.
\item PMI\_SPAWN = 0 or 1 if this process was started by a 
PMI\_Spawn\_multiple command.
\item PMI\_APPNUM = index of the command that started the local process
\item **PMI\_SMPD\_FD = file descriptor/handle to convert into the PMI 
socket context.
\item **PMI\_HOST = host description as specified by the MPIDU\_Sock interface
\item **PMI\_PORT = port PMI\_Host is listening on
\item **PMI\_ROOT\_HOST = root host to connect to to establish the PMI 
socket context.
\item **PMI\_ROOT\_PORT = root listening port number
\item **PMI\_ROOT\_LOCAL = 0 or 1 if the root process is to act as the root smpd manager.
If PMI\_ROOT\_LOCAL is specfied and it is 1, the root MPICH process starts a separate
thread or process to act as the smpd manager.  This manager listens on the specified port
for pmi socket contexts to connect from all the processes in the job and handles smpd pmi
commands for them.  It is an error if PMI\_ROOT\_HOST is not the same as the host where
rank 0 is launched.
\end{description}

* If not specified, default clique contains only the local process.

** Only one set of options may be specified.

The three options, PMI\_SMPD\_FD, PMI\_ROOT\_HOST/PMI\_ROOT\_PORT and PMI\_HOST/PMI\_PORT
are mutually exclusive.  If PMI\_SMPD\_FD exists then the process uses that handle
as its connection to the SMPD manager otherwise it makes a socket connection to
the host/port described by PMI\_ROOT\_HOST/PMI\_ROOT\_PORT or PMI\_HOST/PMI\_PORT.

\section{SMPD wire commands}
This section describes the wire protocol for PMI commands from the child
process to the smpd manager.

\subsection{Comand Format}
Commands are variable length.
Each command begins with a 13 byte header.  The header 
is a NULL terminated ascii string representation of the length of the 
command to follow the header.  After the header is a string of the length 
described by the header.  Both the header and the command are NULL terminated.
The header is always 13 bytes no matter where the NULL character falls.  The
command string begins at the 14th byte and the length of the command must 
include the NULL character.

Commands contain \texttt{key=value} strings to describe the components of 
the command.  All commands will have the following keys:

\begin{itemize}
\item \texttt{cmd=command}
\item \texttt{src=my\_smpd\_id}
\item \texttt{dest=dest\_smpd\_id}
\item \texttt{tag=command\_tag}
\item \texttt{ctx\_key=pmi\_smpd\_key}
\end{itemize}

Additional command specific keys are described in the following section.

\subsection{Commands}
\begin{description}
%\item[\texttt{exit\_on\_done}]\mbox{}\\
%Tell the root to exit when all done commands are received - used in rPMI only.
\item[\texttt{done}]\mbox{}\\
No more PMI commands, close the context.  This command is sent from the 
child directly to its SMPD manager and does not receive a reply.

Example: \texttt{cmd=done src=3 dest=3 tag=14 ctx\_key=0}
\item[\texttt{exit\_on\_done}]\mbox{}\\
The root smpd manager can and should exit when all done commands are received.
This command is sent by the root process.

Example: \texttt{cmd=exit\_on\_done src=1 dest=1 tag=13 ctx\_key=0}

Is this command necessary?  Shouldn't the root smpd know that it is a root smpd
and exit automatically when all its pmi contexts close?
\item[\texttt{barrier}]\mbox{}\\
Barrier across a set of processes. 
Add \texttt{name=barrier\_name value=number\_of\_participants}.
The result command returns SUCCESS or FAIL.

Example: \texttt{cmd=barrier src=2 dest=1 tag=3 ctx\_key=1 name=kvsname value=2}
\item[\texttt{dbcreate}]\mbox{}\\
Create a new keyval space.  If \texttt{name=kvsname} is added to the command
then the keyval space is created with the provided name, otherwise the 
implementation chooses a name.
The result command returns SUCCESS or FAIL and \texttt{name=kvsname}.

Example: \texttt{cmd=dbcreate src=1 dest=1 tag=100 ctx\_key=0}
\item[\texttt{dbdestroy}]\mbox{}\\
Destroy a keyval space.  Add \texttt{name=kvsname}.
The result command returns SUCCESS or FAIL.

Example: \texttt{cmd=dbdestroy src=4 dest=1 tag=13 ctx\_key=1 name=kvsname}
\item[\texttt{dbput}]\mbox{}\\
Put a keyval into a kvs space.  Add \texttt{name=kvsname key=user\_key 
value=user\_value}.
The result command returns SUCCESS or FAIL.

Example \texttt{cmd=dbput src=3 dest=1 tag=100 ctx\_key=0 name=kvsname key=foo value=bar}
\item[\texttt{dbget}]\mbox{}\\
Get a kevyal from a kvs space.  Add \texttt{name=kvsname key=user\_key}.
The result command returns SUCCESS or FAIL and \texttt{value=val}.

Example: \texttt{cmd=dbget src=4 dest=1 tag=0 ctx\_key=0 name=kvsname key=foo}
\item[\texttt{dbfirst}]\mbox{}\\
Start the keyval space iterator.  Add \texttt{name=kvsname}.
The result command returns SUCCESS or FAIL and \texttt{key=key value=val}.

Example: \texttt{cmd=dbfirst src=1 dest=1 tag=22 ctx\_key=0 name=kvsname}
\item[\texttt{dbnext}]\mbox{}\\
Get the next keyval from the iterator.  Add \texttt{name=kvsname}.
The result command returns SUCCESS or FAIL and \texttt{key=key value=val}.

Example: \texttt{cmd=dbnext src=2 dest=1 tag=12 ctx\_key=0 name=kvsname}
\item[\texttt{spawn}]\mbox{}\\
Spawn a new process group.  See the next section for a complete description.
\item[\texttt{result}]\mbox{}\\
The result of a previous command.  Result commands will always have two 
fields, \texttt{cmd\_tag=command\_tag} and \texttt{result=result\_string}.
The \texttt{command\_tag} matches the tag of the command the result command 
refers to.  The \texttt{result\_string} is SUCCESS or a failure message. 
Other return fields will be present as specified by the issued command.
\end{description}

\subsection{spawn command}
The \texttt{spawn} command is issued by a single node to launch a set of 
processes in a new process group.

The \texttt{spawn} command is used to implement PMI\_Spawn\_multiple.

The keys to the \texttt{spawn} command are the following:
\begin{description}
\item \texttt{ncmds = x number of commands}
\item \texttt{cmd0 = command}
\item \texttt{cmd1 = command}
\item \texttt{...}
\item \texttt{argv0 = string1 string2 string3 ...}
\item \texttt{argv1 = string1 string2 string3 ...}
\item \texttt{...}
\item \texttt{maxprocs = n0 n1 n2 ... nx-1}
\item \texttt{nkeyvals = n0 n1 n2 ... nx-1}
\item \texttt{keyvals0 = ``0=$\backslash$``key=val$\backslash$'' 
1=$\backslash$``key=val$\backslash$'' ... 
n0-1=$\backslash$``key=val$\backslash$''''}
\item \texttt{keyvals1 = ``0=$\backslash$``key=val$\backslash$'' 
1=$\backslash$``key=val$\backslash$'' ... 
n1-1=$\backslash$``key=val$\backslash$''''}
\item \texttt{...}
\item \texttt{npreput = number of preput keyvals}
\item \texttt{preput = ``0=$\backslash$``key=val$\backslash$'' 
1=$\backslash$``key=val$\backslash$'' ... 
n-1=$\backslash$``key=val$\backslash$''''}
\end{description}

The \texttt{ncmds} key represents the size of the rest of the vector arguments.
There will be \texttt{ncmds} \texttt{cmd} and \texttt{argv} keys.  \texttt{maxprocs}
and \texttt{nkeyvals} will contain \texttt{ncmds} entries.  The values in
\texttt{maxprocs} represent the requested number of processes to launch for the
corresponding \texttt{cmd} command.  There will be \texttt{ncmds} \texttt{keyvals}
keys and each \texttt{keyvals} key will contain nx keys where nx is the 
corresponding value in \texttt{nkeyvals}.  \texttt{npreput} represents the 
number of keys in the \texttt{preput} key.  The keys in the \texttt{preput} key
are to be put in the keyval space of the spawned process group before any
of the processes are launched.

Example: \texttt{cmd=spawn src=3 dest=0 tag=4 ctx\_key=0 ncmds=1 cmd0=myapp
argv0=``one $\backslash$``two args$\backslash$'' three'' maxprocs=4 
nkeyvals=2 keyvals0=``0=$\backslash$``host=toad$\backslash$'' 
1=$\backslash$``path=/home/me$\backslash$'''' 
npreput=1 preput=``0=$\backslash$``port=1244$\backslash$''''}

\section{string format}
This section describes the format of key=value elements in a stream.

stream := frame \vline\ frame stream

frame := element frame\_char \vline\ element separ\_char frame

element := key delim\_char value

key := string

value := string

string := literal \vline\ quoted

literal := array of chars without separators, deliminators, or quotes

quoted := quote\_char array-of-escaped-characters quote\_char

chars := ascii characters

escapted chars := ascii characters with escaped quote\_chars and escape\_chars

quote\_char := "

escape\_char := $\backslash$

delim\_char := =

separ\_char := ' '

frame\_char := '$\backslash$0'

Example:

	a=b "my name"="David Ashton" foo="He said, $\backslash$"Hi there.$\backslash$""
	
\bibliographystyle{plain}
\bibliography{smpd_pmi}

\end{document}
