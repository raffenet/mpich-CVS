\documentclass[dvipdfm,11pt]{article}
\usepackage[dvipdfm]{hyperref} % Upgraded url package

\begin{document}
\markright{MPICH2 User's Manual}
\title{MPICH2 User's Manual\\
Version 0.1\\
Draft of \today\\
Mathematics and Computer Science Division\\
Argonne National Laboratory}

\author{William Gropp\\
Ewing Lusk\\
David Ashton\\
Rob Ross\\
Rajeev Thakur\\
Brian Toonen}

\maketitle

\cleardoublepage

\pagenumbering{roman}
\tableofcontents
\clearpage

\pagenumbering{arabic}
\pagestyle{headings}


Here is a basic outline for the document

1. Setting paths

2. Compiling and linking

  2a. Chosing compilers (e.g., you need not use the same complier that
      MPICH was built with)

  2b. Shared libraries

  2c. Special issues for Fortran 77 and Fortran 90
 
      (mostly the choice module, but also the various name mangling issues)

3. Running with mpiexec

  3a. mpiexec standard options (from MPI-2)

  3b. device and pm specific options

  3c. environment variables

      for example, \texttt{MPIEXEC\_TIMEOUT}

  3d. managing stdin/out/err

  3e. managing files (staging?), including executables

4. Examples 

   4a. Simple programs
  
   4b. Benchmarking (similar or identical to text in the installation guide)

   4c. Pointers to other resources (books, tutorials, sample programs)

5. Debugging

   5a. Working with single-process debuggers

   5b. Working with parallel debuggers such as Totalview

6. Troubleshooting
\end{document}
