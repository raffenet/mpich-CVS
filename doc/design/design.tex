\documentclass{article}
\usepackage{graphics}
\usepackage[dvipdfm]{hyperref}
%
% This is the new (September 2005) MPICH2 design document.  The plan is
% to make this available both as a PDF document and split on the web in
% an easy to read fashion.  We can use latex2html to tohtml to provide a 
% simple version of this, and that may be enough.  However, to retain the
% option of generating web pages directly from this source, only simple 
% Latex should be used, and only the new forms (e.g., \texttt{...} instead of 
% {\tt ...}
%


\makeindex

\begin{document}

\markright{MPICH Design Document}

\title{MPICH2 Design Document}
\author{William D. Gropp and Rajeev Thakur}
\maketitle

\pagenumbering{roman}
\tableofcontents
\clearpage


%\raggedright
%% raggedright resets parindent
%\parindent 1em
%% no parskip when parindent used
%\parskip 0pt

\pagenumbering{arabic}
\pagestyle{headings}

\part{MPICH2 organization for users}

\section{Overview}
     <<up here, explain that MPICH2 is a framework for MPI implementation with many possible choices, including specialized communication devices and interfaces to process managers, not ch3+mpd.  Duplicate this discussion in the top-level entry for developers>>

\section{Communication Devices}

\section{Process Managers}

    other options 
    <<this section partially overlaps the installation manual, but that manual doesn't cover much of the motivation for the design>>

\section{Help Resources}

\subsection{Frequently Asked Questions (FAQ)}

\subsection{Diagnostic Programs}

\subsection{Buglist archives}


\part{MPICH2 organization for developers and hackers}


    <<each section (as appropriate) contains a rationale for the choices and a brief discussion of alternatives and why they were not chosen>>

\section{Goals of MPICH2}


\subsection{Principles}
         No duplicated code (no cut and paste programming) 
         Uniform appearance of code
         Uniform user interface for controls (e.g., standardized parameter handling, output, naming)
         Coverage and Testing

\subsection{Major components}
         devices
              the ch3 device and channels
         process managers and PM interface
         logging
         collectives
         topology

\subsection{Build System}
         rationale
              why not automake
              why not libtool
         using the build system
         using your own configure/make/build scripts

\subsection{Coding Style}
         Code template for uniform error checking and reporting (e.g., common \texttt{fn\_fail} target)
         Macros for common operations
              error reporting
              memory allocation
              debugging
         Safer or more efficient replacements for common routines
              safe string routines
              memcpy hooks
         Error reporting
              Rationale
               Adding new error classes and codes
         Tags for coverage analysis

\subsection{Language Bindings}
           rationale for buildiface
           Outstanding issues

\subsection{Scripts and Standards for user interfaces}
           command line and environment variables

\section{Adding your own implementation of a component}
         General mechanism (--with-<<component-name>>=directory, e.g., --with-pm=/home/me/mypm)
              configure and setup scripts
              standardized variables (\texttt{MPI\_CFLAGS}, CFLAGS, etc.)
         Specific components (describe, explain the API/ABI, and how to work with it):
             collecitves
             pm and pmi
             name server
             mpid
             ch3-channel
             logging
             topology
          Components still evolving (a placeholder for things that we want to do)

\section{Working with MPICH2 itself}
          <<for people that edit MPICH2 rather than use the component interface>>
          Coding standards
          Keeping configure clean
          Keeping MPICH modular
          

\end{document}
