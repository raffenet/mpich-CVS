\documentclass[dvipdfm,11pt]{article}
\usepackage[dvipdfm]{hyperref} % Upgraded url package
\parskip=.1in

% Formatting conventions for contributors
% 
% A quoting mechanism is needed to set off things like file names, command
% names, code fragments, and other strings that would confuse the flow of
% text if left undistinguished from preceding and following text.  In this
% document we use the LaTeX macro '\texttt' to indicate such text in the
% source, which normally produces, when used as in '\texttt{special text}',
% the typewriter font.

% It is particularly easy to use this convention if one is using emacs as
% the editor and LaTeX mode within emacs for editing LaTeX documents.  In
% such a case the key sequence ^C^F^T (hold down the control key and type
% 'cft') produces '\texttt{}' with the cursor positioned between the
% braces, ready for the special text to be typed.  The closing brace can
% be skipped over by typing ^e (go to the end of the line) if entering
% text or ^C-} to just move the cursor past the brace.

% LaTeX mode is usually loaded automatically.  At Argonne, one way to 
% get several useful emacs tools working for you automatically is to put
% the following in your .emacs file.

% (require 'tex-site)
% (setq LaTeX-mode-hook '(lambda ()
%          		 (auto-fill-mode 1)
%          		 (flyspell-mode 1)
%          		 (reftex-mode 1)
% 			 (setq TeX-command "latex")))


\begin{document}
\markright{MPICH2 Windows Development Guide}
\title{{\bf MPICH2 Windows Development Guide}\thanks{This work was supported by the
    Mathematical, Information, and Computational Sciences Division
    subprogram of the Office of Advanced Scientific Computing Research,
    SciDAC Program, Office of Science, U.S. Department of Energy, under
    Contract
    W-31-109-ENG-38.}\\
  Version 1.0.2\\
  Mathematics and Computer Science Division\\
  Argonne National Laboratory}

\author{David Ashton}

\maketitle
\cleardoublepage

\pagenumbering{roman}
\tableofcontents
\clearpage

\pagenumbering{arabic}
\pagestyle{headings}

\section{Introduction}
\label{sec:intro}
This manual describes how to set up a Windows machine to build MPICH2 on.

\section{Build machine}
\label{sec:machine}

Build a Windows XP or Windows Server 2003 machine.

\section{Test machine}
\label{sec:test_machine}

Build a Windows XP or Windows Server 2003 machine on a 32bit CPU.
Also build a Windows Server 2003 X64 machine to test the Win64 distribution.

\section{Software}

This section describes the software necessary to build MPICH2.

\subsection{Prerequisites}
\label{sec:prerequisites}

To build MPICH2 you will need:
\begin{enumerate}
\item Microsoft Developer Studio .NET 2003
\item Microsoft Platform SDK
\item cygwin - full installation
\item Intel Fortran compiler
\item Intel Fortran compiler EMT64
\item Mellanox 32bit and 64bit Infiniband libraries
\end{enumerate}

\end{document}
