\documentclass[dvipdfm,11pt]{article}
\usepackage[dvipdfm]{hyperref} % Upgraded url package
\parskip=.1in

\begin{document}
\markright{MPICH2 Logging}
\title{MPICH2 Logging\\
Version 0.1\\
DRAFT of \today\\
Mathematics and Computer Science Division\\
Argonne National Laboratory}

\author{David Ashton}


\maketitle

\cleardoublepage

\pagenumbering{roman}
\tableofcontents
\clearpage

\pagenumbering{arabic}
\pagestyle{headings}


\section{Introduction}
\label{sec:introduction}

This manual assumes that MPICH2 has already been installed.  For
instructions on how to install MPICH2, see the MPICH2 Installation Guide,
or the README in the top-level MPICH2 directory.  This manual will
explain how the internal logging macros are generated and how the user
can generate log files viewable in jumpshot.


\section{Logging state code generation}
\label{sec:genstates}

This section can be skipped by users.  It describes the internal scripts used
to develop the logging macros.

This is how the maint/genstates logging script works:

\begin{enumerate}
\item maint/updatefiles creates genstates from genstates.in replacing @PERL@

\item genstates finds all .i, .h and .c files in the mpich2 directory tree,
 searches for \_STATE\_DECL in each file and builds a list of all the MPID\_STATEs.

\item genstates finds all the "describe\_states.txt" files anywhere in the mpich2
 tree.

\item genstates matches the states in the describe\_states.txt files with the 
 states it found in the .i, .h and .c files.  It warns if states are described
 without being declared and discards these descriptions.  It produces errors
 if states are declared without being described in a describe\_states.txt file.

\item The describe\_states.txt file format is this:
\begin{verbatim}
MPID_STATE_XXX <user string for the state> <optional rgb color>
\end{verbatim}
 Here is an example line:
\begin{verbatim}
 MPID_STATE_MPI_SEND MPI_Send 0 0 255
\end{verbatim}
 
\item genstates ouputs mpich2/src/include/mpiallstates.h with this enum in it:
\begin{verbatim}
enum MPID_TIMER_STATE
{
    MPID_STATE_XXX,
    ...
};
\end{verbatim}

\item genstates outputs mpich2/src/util/logging/describe\_states.c with the
 MPIR\_Describe\_timer\_states() function in it.  Currently, only the rlog
 version of MPIR\_Describe\_timer\_states() is generated.

\end{enumerate}

\section{Configuring mpich2 to create log files}
\label{sec:configuring}

When users run configure they can specify logging options.  There are three configure
options to control logging.

\begin{description}
\item[\texttt{--enable-timing=<timing\_type>}]\mbox{}\\
Add this option to enable timing.  The two options for timing\_type are \texttt{log}
and \texttt{log\_detailed}.  The \texttt{log} option will log only the MPI functions just
like the MPE logging interface does.  The \texttt{log\_detailed} will log every
function in mpich2.  This option gives fine grained logging information and
also creates large log files.  It must be used in conjunction with a timer-type that
can log very short intervals on the order of 100's of nanoseconds.

\item[\texttt{--with-logging=<logger>}]\mbox{}\\
Specify the logging library to use.  Currently the only logger option is \texttt{rlog}.

\item[\texttt{--enable-timer-type=<timer\_type>}]\mbox{}\\
Specify the timer type.  The options are
\begin{itemize}
\item \texttt{gethrtime}
Solaris timer (Solaris systems only)
\item \texttt{clock\_gettime}
Posix timer (where available)
\item \texttt{gettimeofday}
Most Unix systems
\item \texttt{linux86\_cycle}
Linux x86; uses cycle counts to approximate time in seconds
\item \texttt{linuxalpha\_cycle}
Like linux86\_cycle, but for Linux Alpha
\item \texttt{gcc\_ia64\_cycle}
IPF ar.itc timer
\end{itemize}
\end{description}

Here is an example:
\begin{verbatim}
mpich2/configure
    --enable-timing=log
    --with-logging=rlog
    --enable-timer-type=linux86_cycle
    ...
\end{verbatim}

\section{Generating log files}
\label{sec:genlogs}
Run your mpi application to create intermediate .irlog files.

\begin{verbatim}
mpicc myapp.c -o myapp
mpiexec -n 3 myapp
\end{verbatim}
There will be .irlog files created for each process:
\begin{verbatim}
log0.irlog
log1.irlog
log2.irlog
\end{verbatim}

\section{RLOG tools}
\label{sec:tools}
For performance reasons each process produces a local intermediate log file that
needs to be merged into a single rlog file.  Use the rlog tools to merge the .irlog files
 into an .rlog file.  The rlog tools are found in mpich2\_build/src/util/logging/rlog.
Currently they are not copied to the install directory.

\begin{description}
\item[\texttt{irlog2rlog}]\mbox{}\\
This tool combines the intermediate .irlog files to a single .rlog file.
The usage is: irlog2rlog outname.rlog input0.irlog input1.irlog ...
A shortcut is provided: irlog2rlog outname.rlog num\_files.  Execute irlog2rlog
without any parameters to see the usage options.

\item[\texttt{printrlog}]\mbox{}\\
This tool prints the contents of an .rlog file.

\item[\texttt{printirlog}]\mbox{}\\
This tool prints the contents of an .irlog file.
\end{description}

``\texttt{irlog2rlog myapp.rlog 3}'' will convert log0.irlog, log1.irlog and log2.irlog
to myapp.rlog.

\section{Viewing log files}
This section describes how to view a log file

.rlog files can be printed from a command shell using the printrlog tool but
the more interesting way to view the log files is from jumpshot.  jumpshot
displays slog2 files and has a built in convertor to convert .rlog files to
.slog2 files.  Simply open jumpshot and point it at your .rlog file.

\bibliographystyle{plain}
\bibliography{logging}

\end{document}
