\documentclass[dvipdfm,11pt]{article}
\usepackage[dvipdfm]{hyperref} % Upgraded url package
\usepackage{../mpiman}   % Definitions for this particular version of the
                         % manual

\usepackage{tpage}

\begin{document}
\markright{MPICH2 Installer's Manual}
\ANLTMTitle{MPICH2 Installer's Manual\\
Version 0.1\\
Draft of \today}{\em 
William Gropp\\
Ewing Lusk\\
David Ashton\\
Rob Ross\\
Rajeev Thakur\\
Brian Toonen\\
Mathematics and Computer Science Division\\
Argonne National Laboratory}{00}{\today}

\cleardoublepage

\pagenumbering{roman}
\tableofcontents
\clearpage

\pagenumbering{arabic}
\pagestyle{headings}
\begin{document}

Here is a basic outline for the document

0. Quick start with ``best practices''.  Each step has a reference to
more detailed information later in the document.

1. Acquiring and unpacking.  Using a ``fast'' directory location and
   VPATH

1a. Reporting problems

2. Choosing a device (defer a detailed discussion of each until later)

3. configure, make, and install.  Always use --prefix 

show only basic options for configure

   3a. Optional include of device-specific information

   3b. Optional include of pm-specific information
 
   3c. Optional ``fast'' version

   3d. Shared libraries

4. Testing and benchmarking

4a. make testing

4b. Getting, building, and using mpptest and netpipe

5. Special options

6. Troubleshooting

Appendix:

A. Summary of configure options (particularly the enable and with options)

\end{document}

%
% Comments on subclassing the document
% We can use \ifdevname ... \fi and \ifpmname ... \fi, as in
% \ifdevchiii .. \fi and \ifpmmpd ... \fi
% (these will still need to be defined)
% There should also be a way to select ``all'' in such a way that the
% document can still flow well, such as
% \ifdevall ... \else \ifdevchiii \else \ifdevmm \fi \fi \fi



