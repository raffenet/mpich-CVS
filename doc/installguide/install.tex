\documentclass[dvipdfm,11pt]{article}
\usepackage[dvipdfm]{hyperref} % Upgraded url package
\parskip=.1in

\begin{document}
\markright{MPICH2 Installer's Guide}
\title{MPICH2 Installer's Guide\\
Version 0.2\\
Draft of \today \\
Mathematics and Computer Science Division\\
Argonne National Laboratory}

\author{William Gropp\\
Ewing Lusk\\
David Ashton\\
Anthony Chan\\
Rob Ross\\
Rajeev Thakur\\
Brian Toonen}

\maketitle
\cleardoublepage

\pagenumbering{roman}
\tableofcontents
\clearpage

\pagenumbering{arabic}
\pagestyle{headings}

Here is a basic outline for the document.   (Bill's original outline)

0. Quick start with ``best practices''.  Each step has a reference to
more detailed information later in the document.

1. Acquiring and unpacking.  Using a ``fast'' directory location and
   VPATH

1a. Reporting problems

2. Choosing a device (defer a detailed discussion of each until later)

3. configure, make, and install.  Always use --prefix 

show only basic options for configure

   3a. Optional include of device-specific information

   3b. Optional include of pm-specific information
 
   3c. Optional ``fast'' version

   3d. Shared libraries

4. Testing and benchmarking

4a. make testing

4b. Getting, building, and using mpptest and netpipe

5. Special options

6. Troubleshooting

Appendix:

A. Summary of configure options (particularly the enable and with options)


\section{Introduction}
\label{sec:intro}
This manual describes how to obtain and install MPICH2, the MPI-2
implementation from Argonne National Laboratory.  (Of course, if you are
reading this, chances are good that you have already obtained it and
found this document, among others, in its \texttt{doc} subdirectory.)
This \emph{Guide} will explain how to install MPICH so that you and others can use it to
run MPI applications.  Some particular features are different
if you have system administration privileges (can become ``root'' on a
Unix system), and these are explained here.  It is not necessary to have
such privileges to build and install MPICH2.  In the event of problems,
send mail to \texttt{mpich2-maint@mcs.anl.gov}.  Once MPICH2 is
installed, details on how to run MPI jobs are covered in the \emph{MPICH2
User's Guide}, found in this same \texttt{doc} subdirectory.

MPICH2 has many options.  We will first go through a preferred,
``standard'' installation in a step-by-step fashion, and later describe
alternative possibilities.  This \emph{Installer's Guide} is for MPICH2
Release 0.971.  We are reserving the 1.0 designation for when every last
feature of the MPI-2 Standard is implemented, but most features are
included.  See the \texttt{RELEASE\_NOTES} file in the top-level
directory for details.


\section{Quick Start}
\label{sec:quick}

In this section we describe a ``default'' set of installation steps.  It
uses the default set of configuration options, which builds the \texttt{sock}
communication device and the \texttt{MPD} process manager, for as many
of languages C, C++, Fortran-77, and Fortran-90 compilers as it can
find, with compilers chosen automatically from the user's environment,
without tracing and debugging options.  It uses the \texttt{VPATH}
feature of \texttt{make}, so that the build process can take place on
a local disk for speed.

\subsection{Prerequisites}
\label{sec:prerequisites}

For the default installation, you will need:
\begin{enumerate}
\item A copy of the distribution, \texttt{mpich2.tar.gz}.
\item A C compiler.
\item A fortran-77, Fortran-90, and/or C++ compiler if you wish to write
  MPI programs in any of these languages.
\item Python 2.2 or later version, for building the default process
  management system, MPD.  PyXML and an XML parser like expat (in order
  to use mpiexec with the MPD process manager).  Most systems have
  Python, PyXML, and expat pre-installed, but you can get them free from
  \texttt{www.python.org}.  Assume they are there unless the
  \texttt{configure} step below complains.
\item Any one of a number of Unix operating systems, such as Linux.
  MPICH2 is most extensively tested on Linux;  there remain some
  difficulties on systems we do not currently have access to.  Our
  \texttt{configure} script attempts to adapt MPICH2 to new systems. 
\end{enumerate}
Configure will check for these prerequisites and try to work around
deficiencies if possible.  (If you don't have Fortran, you will
still be able to use MPICH2, just not with Fortran applications.)

This default installation procedure builds and installs MPICH2 ready for
C and Fortran (-77) programs, using the MPD process manager (and it
builds and installs MPD itself), without debugging options.  Regardless
of where the source resides, the build takes place on a local file
system, where compilation is likely to be much faster than on a
network-attached file system, but the installation directory that is
accessed by users can be on a shared file system.  For other options,
see the appropriate sections later in the document.

\subsection{From A Standing Start to Running an MPI Program}
\label{sec:steps}
Here are the steps from obtaining MPICH2 through running your own
parallel program on multiple machines.

\begin{enumerate}
\item 
Unpack the tar file.
\begin{verbatim}
    tar xfz mpich2.tar.gz
\end{verbatim}
If your tar doesn't accept the z option, use
\begin{verbatim}
    gunzip -c mpich2.tar.gz | tar xf mpich2.tar
\end{verbatim}
Let us assume that the directory where you do this is
\texttt{/home/you/libraries}.  It will now contain a subdirectory named
\texttt{mpich2-0.971}.

\item
Choose an installation directory (the default is \texttt{/usr/local/bin)}:
\begin{verbatim}
    mkdir /home/you/mpich2-install
\end{verbatim}
It will be most convenient if this directory is shared by all of the
machines where you intend to run processes.  If not, you will have
to duplicate it on the other machines after installation.  Actually, if
you leave out this step, the next step will create the directory for you.

\item
Choose a build directory.  Building will proceed \emph{much} faster if
your build directory is on a file system local to the machine on which
the configuration and compilation steps are executed.  It is preferable
that this also be separate from the source directory, so that it remains
clean and can be reused to build other copies on other machines.
\begin{verbatim}
    mkdir /tmp/you/mpich2-0.971
\end{verbatim}

\item
Configure MPICH2, specifying the installation directory, and running
the \texttt{configure} script in the source directory:
\begin{verbatim}
    cd  /tmp/you/mpich2-0.971
    /home/you/libraries/mpich2-0.971/configure \
            -prefix=/home/you/mpich2-install |& tee configure.log
\end{verbatim}
where the \texttt{\\} means that this is really one line.  (On
\texttt{sh} and its derivatives, use \verb+2>&1 | tee configure.log+
instead of \verb+|& tee configure.log+).  Other configure options are
described below.  Check the \texttt{configure.log} file to make sure
everything went well.  Problems should be self-explanatory, but if not,
sent \texttt{configure.log} to \texttt{mpich2-maint@mcs.anl.gov}.

\item
Build MPICH2:
\begin{verbatim}
    make |& tee make.log
\end{verbatim}
This step should succeed if there were no problems with the preceding
step.  Check \texttt{make.log}.  If there were problems, send
\texttt{configure.log} and \texttt{make.log} to
\texttt{mpich2-maint@mcs.anl.gov}.

\item
Install the MPICH2 commands:
\begin{verbatim}
    make install |& tee install.log
\end{verbatim}
This step collects all required executables and scripts in the \texttt{bin}
subdirectory of the directory specified by the prefix argument to
configure. 

\item
Add the \texttt{bin} subdirectory of the installation directory to your path:
\begin{verbatim}
    setenv PATH /home/you/mpich2-install/bin:$PATH
\end{verbatim}
for \texttt{csh} and \texttt{tcsh}, or 
\begin{verbatim}
    export PATH=/home/you/mpich2-install/bin:$PATH
\end{verbatim}
for \texttt{bash} and \texttt{sh}.  Check that everything is in order at
this point by doing
\begin{verbatim}
    which mpd
    which mpicc
    which mpiexec
    which mpirun
\end{verbatim}
All should refer to the commands in the \texttt{bin} subdirectory of your
install directory.  It is at this point that you will need to
duplicate this directory on your other machines if it is not
in a shared file system such as NFS.

\item
MPICH2, unlike MPICH, uses an external process manager for scalable
startup of large MPI jobs.  The default process manager is called
MPD, which is a ring of daemons on the machines where you will run
your MPI programs.  In the next few steps, you will get this ring up
and tested.  More details on interacting with MPD can be found in
the README file in \texttt{mpich2/src/pm/mpd}, such as how to list
running jobs, kill, suspend, or otherwise signal them, and how to
use the \texttt{mpigdb} debugger.  The instructions given here should be
enough to get you started.

For security reasons, mpd looks in your home directory for a file named
\texttt{.mpd.conf} containing the line
\begin{verbatim}
    secretword=<secretword>
\end{verbatim}
where \verb+<secretword>+ is a string known only to yourself.  It should
not be your normal Unix password.  Make this file readable and writable
only by you:
\begin{verbatim}
    cd $HOME
    touch .mpd.conf
    chmod 600 .mpd.conf
    echo "secretword=mr45-j9z" >> .mpd.conf
\end{verbatim}
(Of course use a different secret word than verb+mr45-j9z+.)

\item
The first sanity check consists of bringing up a ring of one mpd on
the local machine, testing one mpd command, and bringing the "ring"
down. 
\begin{verbatim}
    mpd & 
    mpdtrace
    mpdallexit
\end{verbatim}
The output of mpdtrace should be the hostname of the machine you are
running on.  The \texttt{mpdallexit} causes the mpd daemon to exit.

\item
Now we will bring up a ring of mpd's on a set of machines.  Create
a file consisting of a list of machine names, one per line.  Name this
file \texttt{mpd.hosts}.  These hostnames will be used as targets for
\texttt{ssh} or \texttt{rsh}, so include full domain names if necessary.  Check that you
can reach these machines with \texttt{ssh} or \texttt{rsh} without
entering a password.  You can test by doing
\begin{verbatim}
    ssh othermachine date
\end{verbatim}
or
\begin{verbatim}
    rsh othermachine date
\end{verbatim}
If you cannot get this to work without entering a password, you will
need to configure \texttt{ssh} or \texttt{rsh} so that this can be done,
or else use the workaround for \texttt{mpdboot} in the next step.

\item
Start the daemons on (some of) the hosts in the file mpd.hosts
\begin{verbatim}
  mpdboot -n <number to start>  
\end{verbatim}
The number to start can be less than 1 + number of hosts in the
file, but cannot be greater than 1 + the number of hosts in the
file.  One mpd is always started on the machine where \texttt{mpdboot} is
run, and is counted in the number to start, whether or not it occurs
in the file.

There is a workaround if you cannot get \texttt{mpdboot} to work because of
difficulties with \texttt{ssh} or \texttt{rsh} setup.  You can start the daemons "by
hand" as follows:
\begin{verbatim}
   mpd &            # starts the local daemon
   mpdtrace -l      # makes the local daemon print its host
                    # and port in the form <host>_<port>
\end{verbatim}
Then log into each of the other machines, put the \texttt{install/bin}
directory in your path, and do:
\begin{verbatim}
   mpd -h <hostname> -p <port> &
\end{verbatim}
where the hostname and port belong to the original mpd that you
started.  From each machine, after starting the mpd, you can do 
\begin{verbatim}
   mpdtrace
\end{verbatim}
to see which machines are in the ring so far.  More details on
\texttt{mpdboot} and other options for starting the mpd's are in
\texttt{mpich2-0.971/src/pm/mpd/README}.

\item
Test the ring you have just created:
\begin{verbatim}
    mpdtrace
\end{verbatim}
The output should consist of the hosts where MPD daemons are now
running.  You can see how long it takes a message to circle this
ring with 
\begin{verbatim}
    mpdringtest
\end{verbatim}
That was quick.  You can see how long it takes a message to go
around many times by giving mpdringtest an argument:
\begin{verbatim}
    mpdringtest 100
    mpdringtest 1000
\end{verbatim}

\item
Test that the ring can run a multiprocess job:
\begin{verbatim}
    mpdrun -n <number> hostname
\end{verbatim}
The number of processes need not match the number of hosts in the
ring;  if there are more, they will wrap around.  You can see the
effect of this by getting rank labels on the stdout:
\begin{verbatim}
    mpdrun -l -n 30 hostname
\end{verbatim}
You probably didn't have to give the full pathname of the hostname
command because it is in your path.  If not, use the full pathname:
\begin{verbatim}
    mpdrun -l -n 30 /bin/hostname
\end{verbatim}

\item
Now we will run an MPI job, using the \texttt{mpiexec} command as specified
in the MPI-2 standard.  There are some examples in the install
directory, which you have already put in your path, as well as in
the directory \texttt{mpich2-0.97/examples}.  One of them is the classic \texttt{cpi}
example, which computes the value of $\pi$ by numerical integration in
parallel.   
\begin{verbatim}
    mpiexec -n 5 cpi
\end{verbatim}
As with \texttt{mpdrun} (which is used internally by \texttt{mpiexec}), the number of
processes need not match the number of hosts.  The \texttt{cpi} example will
tell you which hosts it is running on.  By default, the processes
are launched one after the other on the hosts in the mpd ring, so it
is not necessary to specify hosts when running a job with \texttt{mpiexec}.

There are many options for \texttt{mpiexec}, by which multiple executables
can be run, hosts can be specified (as long as they are in the mpd
ring), separate command-line arguments and environment variables can
be passed to different processes, and working directories and search
paths for executables can be specified.  Do
\begin{verbatim}
    mpiexec --help
\end{verbatim}
for details. A typical example is:
\begin{verbatim}
    mpiexec -n 1 master : -n 19 slave
\end{verbatim}
or
\begin{verbatim}
    mpiexec -n 1 -host mymachine : -n 19 slave
\end{verbatim}
to ensure that the process with rank 0 runs on your workstation.

The arguments between ':'s in this syntax are called "argument sets",
since they apply to a set of processes.  \textbf{Change this to match
new global and local arguments described in User's Guide.}  There is
an extra argument set for arguments that apply to all the processes,
introduced by the -default argument.  For example, to get rank labels on
standard output, use
\begin{verbatim}
    mpiexec -default -l : -n 3 cpi
\end{verbatim}
The \texttt{mpirun} command from the original MPICH is still available,
although it does not support as many options as mpiexec.  You might
want to use it in the case where you do not have the XML parser
required for the use of \texttt{mpiexec}.
\end{enumerate}

If you have completed all of the above steps, you have successfully
installed MPICH2 and run an MPI example.  


\subsection{Common Non-Default Configuration Options}
\label{sec:non-default}

\begin{verbatim}
enable-g, enable-fast, devices, pms, etc.
\end{verbatim}
Reference Section~\ref{configure-options}.


\subsection{Shared Libraries}
\label{sec:shared-libraries}

Options.  Portability problems.  By hand.


\subsection{What to Tell the Users}
\label{sec:telling}

Now that MPICH2 has been installed, the users have to be informed of how
to use it.  Part of this is covered in the \emph{User's Guide}.  Other
things users need to know is covered here.  (E.g. whether they need to
run their own mpd rings or use a system-wide one run by root.)

\section{Installing and Managing Process Managers}
\label{sec:process-managers}


\subsection{MPD}
\label{sec:mpd}

In Section~\ref{sec:steps} you installed the mpd ring.  Several commands
can be used to use, test, and manage this ring.  You can find out about
them by running \texttt{mpdhelp}, whose output looks like this:

\begin{verbatim}
The following mpd commands are available.  For usage of any specific one,
invoke it with the single argument --help .

mpd           start an mpd daemon
mpdtrace      show all mpd's in ring
mpdboot       start a ring of daemons all at once
mpdringtest   test how long it takes for a message to circle the ring 
mpdallexit    take down all daemons in ring
mpdcleanup    repair local Unix socket if ring crashed badly
mpdrun        start a parallel job
mpdlistjobs   list processes of jobs (-a or --all: all jobs for all users)
mpdkilljob    kill all processes of a single job
mpdsigjob     deliver a specific signal to the application processes of a job

Each command can be invoked with the --help argument, which prints usage
information for the command without running it.
\end{verbatim}
So for example, to see a complete list of the possible arguments for
\texttt{mpdboot}, you would run
\begin{verbatim}
    mpdboot --help
\end{verbatim}

\subsubsection{Running MPD as Root}
\label{sec:mpd-root}

How to run mpd as root for other people to use.  Test whether all that
is necessary is for root to be the one who runs the install step.

\subsection{SMPD}
\label{sec:smpd}

\subsubsection{Configuration}
\label{sec:smpd_configure}

You may add the following configure options, 
\texttt{--with-pm=smpd --with-pmi=smpd}, 
to build and install the smpd process manager. The sprocess manager, smpd, 
will be installed to the bin sub-directory of the installation directory 
of your choice specified by the \texttt{--prefix} option.

smpd process managers run on each node as stand-alone daemons and need to
be running on all nodes that will participate in MPI jobs.  smpd process 
managers are not connected to each other and rely on a known port to 
communicate with each other.  Note: If you want multiple users to use the 
same nodes they must each configure their smpds to use a unique port per 
user. 

smpd uses a configuration file to store settings.  The default location is 
\verb+~/.smpd+.  This file must not be readable by anyone other than 
the owner and contains at least one required option - the access passphrase.
This is stored in the configuration file as \texttt{phrase=<phrase>}. Access 
to running smpds is authenticated using this passphrase and it must 
not be your user password.

\subsubsection{Usage and administration}
\label{sec:smpd_usage}

Users will start the smpd daemons before launching mpi jobs.  To get an 
smpd running on a node, execute 
\begin{verbatim}
    smpd -s
\end{verbatim}
Executing this for the first time will prompt the user to create a 
\verb+~/.smpd+ configuration file and passphrase if one does not 
already exist.

Then users can use \texttt{mpiexec} to launch MPI jobs.

All options to smpd:

\begin{itemize}
\item[\texttt{smpd -s}]\mbox{}\\
Start the smpd service/daemon for the current user.  You can add 
\texttt{-p <port>} to specify the port to listen on.  All smpds must use
the same port and if you don't use the default then you will have to
add \texttt{-p <port>} to mpiexec or add the \texttt{port=<port>} to the 
\texttt{.smpd} configuration file.

\item[\texttt{smpd -r}]\mbox{}\\
Start the smpd service/daemon in root/multi-user mode.  This is not yet
implemented.

\item[\texttt{smpd -shutdown [host]}]\mbox{}\\
Shutdown the smpd on the local host or specified host.  Warning: this will
cause the smpd to exit and no mpiexec or smpd commands can be issued to the
host until smpd is started again.

%\item[\texttt{smpd -start}]\mbox{}\\
%Start the Windows smpd service on the local host.
%
%\item[\texttt{smpd -stop}]\mbox{}\\
%Stop the Windows smpd on the local host.
%
%\item[\texttt{smpd -console [host]}]\mbox{}\\
%Connect to a specific smpd to issue console commands.  The currently 
%supported commands are: %get, set, delete, stat, status, shutdown and validate.
%\begin{enumerate}
%\item[\texttt{get <var>}]\mbox{}
%\item[\texttt{set <var>=<value>}]\mbox{}
%\item[\texttt{delete <var>}]\mbox{}
%\item[\texttt{status}]\mbox{}
%\item[\texttt{stat <var>}]\mbox{}
%\item[\texttt{shutdown}]\mbox{}
%\item[\texttt{validate}]\mbox{}
%\end{enumerate}

\end{itemize}

\subsection{Forker}
\label{sec:forker}


\subsection{Other Process Managers}
\label{sec:other-pms}

\section{Testing}
\label{sec:testing}
Running basic tests in the examples directory, the MPICH2 tests,
obtaining and running the assorted test suites.  


\subsection{Using the Intel Test Suite}
\label{sec:intel}

These instructions may be partly local to our test environment at Argonne.

How to run a select set of tests from the Intel test suite:

\begin{verbatim}
1) checkout the Intel test suite (cvs co IntelMPITEST)

2) create a testing directory separate from the IntelMPITEST source
directory

3) cd into that testing directory

4) make sure the process manager (e.g., mpd) is running

5) run "<ITS_SRC_DIR>/configure --with-mpich2=<MPICH2_INSTALL_DIR>", where
<ITS_SRC_DIR> is the path to the directory Intel test suite source (e.g.,
/home/toonen/Projects/MPI-Tests/IntelMPITEST) and <MPICH2_INSTALL_DIR> is
the directory containing your MPICH2 installation

6) mkdir Test; cd Test

7) find tests in <ITS_SRC_DIR>/{c,fortran} that you are interested in
running and place the test names in a file.  For example:

% ( cd /home/toonen/Projects/MPI-Tests/IntelMPITEST/Test ; \
    find {c,fortran} -name 'node.*' -print | grep 'MPI_Test' 
    | sed -e 's-/node\..*$--' ) |& tee testlist
Test/c/nonblocking/functional/MPI_Test
Test/c/nonblocking/functional/MPI_Testall
Test/c/nonblocking/functional/MPI_Testany
Test/c/nonblocking/functional/MPI_Testsome
Test/c/persist_request/functional/MPI_Test_p
Test/c/persist_request/functional/MPI_Testall_p
Test/c/persist_request/functional/MPI_Testany_p
Test/c/persist_request/functional/MPI_Testsome_p
Test/c/probe_cancel/functional/MPI_Test_cancelled_false
Test/fortran/nonblocking/functional/MPI_Test
Test/fortran/nonblocking/functional/MPI_Testall
Test/fortran/nonblocking/functional/MPI_Testany
Test/fortran/nonblocking/functional/MPI_Testsome
Test/fortran/persist_request/functional/MPI_Test_p
Test/fortran/persist_request/functional/MPI_Testall_p
Test/fortran/persist_request/functional/MPI_Testany_p
Test/fortran/persist_request/functional/MPI_Testsome_p
Test/fortran/probe_cancel/functional/MPI_Test_cancelled_false
%

8) run the tests using ../bin/mtest:

% ../bin/mtest -testlist testlist -np 6 |& tee mtest.log
%

NOTE: some programs hang if less they are run with less than 6 processes.

9) examine the summary.xml file.  look for '<STATUS>fail</STATUS>' to see if
any failures occurred.  (search for '>fail<' works as well)

\end{verbatim}


\section{Benchmarking}
\label{sec:benchmarking}
netpipe, mpptest, others (SkaMPI).

\section{MPE}
\label{sec:mpe}

This section describes what MPE is and its potentially separate installation.  It
includes discussion of Java-related problems.

\section{Windows Version}
\label{sec:windows}

\subsection{Binary distribution}
\label{sec:winbin}

The Windows binary distribution uses the Microsoft Installer.  Download and 
execute mpich2.msi to install the binary distribution.  The default 
installation path is \texttt{C:$\backslash$Program Files$\backslash$MPICH2}. 
You must have administrator privileges to install mpich2.msi.  The installer 
installs a Windows service to launch MPICH applications and only administrators
may install services.  This process manager is called smpd.exe.  Access to 
the process manager is passphrase protected.  The installer asks for this 
passphrase.  Do not use your user password.  The same passphrase must be 
installed on all nodes that will participate in a single MPI job.

Under the installation directory are three sub-directories: \texttt{include},
 \texttt{bin}, and \texttt{lib}.  The \texttt{include} and \texttt{lib} 
directories contain the header files and libraries necessary to compile MPI 
applications.  The \texttt{bin} directory contains the process manager, 
\texttt{smpd.exe}, and the the MPI job launcher, \texttt{mpiexec.exe}.  The
dlls that implement MPICH2 are copied to the Windows system32 directory.

You can install MPICH in unattended mode by executing 
\begin{verbatim}
    msiexec /q /I mpich2.msi
\end{verbatim}

The smpd process manager for Windows runs as a service that can launch jobs 
for multiple users.  It does not need to be started like the unix version 
does.  The service is automatically started when it is installed and when 
the machine reboots.  smpd for Windows has additional options:
\begin{enumerate}
\item[\texttt{smpd -start}]\mbox{}\\
Start the Windows smpd service.
\item[\texttt{smpd -stop}]\mbox{}\\
Stop the Windows smpd service.
\item[\texttt{smpd -install}]\mbox{}\\
Install the smpd service.
\item[\texttt{smpd -remove}]\mbox{}\\
Remove the smpd service.
\end{enumerate}

\subsection{Source distribution}
\label{sec:winsrc}

If you want to use a channel other than the default socket channel you need 
to download the mpich2 source distribution and build an alternate channel.  
You must have MS Developer Studio .NET 2003 or later, perl and optionally 
Intel Fortran 8 or later.

\begin{itemize}
\item
Download mpich2.tar.gz and unzip it.
\item
Bring up a Visual Studio Command prompt with the compiler environment 
variables set.
\item
Run winconfigure.wsf. If you don't have a Fortran compiler add the 
``--remove-fortran'' option to winconfigure to remove all the Fortran 
projects and dependencies.  Execute ``winconfigure.wsf /?'' to see all
available options.
\item 
    open mpich2$\backslash$mpich2.sln
\item
    build the ch3sockDebug mpich2 solution
\item
    build the ch3sockDebug mpich2s project
\item
    build the ch3sockRelease mpich2 solution
\item
    build the ch3sockRelease mpich2s project
\item
    build the Debug mpich2 solution
\item
    build the Release mpich2 solution
\item
    build the fortDebug mpich2 solution
\item
    build the fortRelease mpich2 solution
\item
    build the gfortDebug mpich2 solution
\item
    build the gfortRelease mpich2 solution
\item
    build the sfortDebug mpich2 solution
\item
    build the sfortRelease mpich2 solution
\item
    build the channel of your choice.  The options are shm, ssm, sshm, ib.  
The shm channel is for small numbers of processes that will run on a single 
machine using shared memory.  The shm channel should not be used for more 
than about 8 processes.  The sshm (scalable shared memory) is for use with 
more than 8 processes.  The ssm (sock shared memory) channel is for clusters 
of smp nodes.  This channel should not be used if you plan to over-subscribe 
the CPU's.  If you plan on launching more processes than you have processors 
you should use the default sock channel.  The ssm channel uses a polling 
progress engine that can perform poorly when multiple processes compete for 
individual processors.  The ib channel is for clusters with Infiniband
interconnects from Mellanox.

\end{itemize}

\subsection{cygwin}
\label{sec:cygwin}

MPICH2 can also be built under cygwin using the source
distribution and the unix commands described in previous sections.  This
will not build the same libraries as described in this section.  It will 
build a ``unix'' distribution that runs under cygwin.

\section{All Configure Options}
\label{configure-options}
Here is a list of all the configure options currently recognized by the
top-level configure.  It is the MPICH-specific part of the output of 
\begin{verbatim}
    configure --help
\end{verbatim}
Not all of these options may be fully supported yet.  Explain all of them \ldots

\begin{verbatim}

--enable and --with options recognized:
--enable-cache  - Turn on configure caching
--enable-echo  - Turn on strong echoing. The default is enable=no. 
--enable-strict - Turn on strict debugging with gcc
--enable-coverage - Turn on coverage analysis using gcc and gcov
--enable-error-checking=level - Control the amount of error checking.  
level may be 
    no        - no error checking
    runtime   - error checking controllable at runtime through environment 
                variables
    all       - error checking always enabled
--enable-error-messages=level - Control the amount of detail in error 
  messages.  Level may be
    all       - Maximum amount of information
    generic   - Only generic messages (no information about the specific
                instance)
    class     - One message per MPI error class
    none      - No messages
--enable-timing=level - Control the amount of timing information 
collected by the MPICH implementation.  level may be
    none    - Collect no data
    all     - Collect lots of data
    runtime - Runtime control of data collected
The default is none.
--enable-threads=level - Control the level of thread support in the 
MPICH implementation.  The following levels are supported.
    single - No threads (MPI_THREAD_SINGLE)
    funneled - Only the main thread calls MPI (MPI_THREAD_FUNNELED)
    serialized - User serializes calls to MPI (MPI_THREAD_SERIALIZED)
    multiple[:impl] - Fully multi-threaded (MPI_THREAD_MULTIPLE)
The default is funneled.  If enabled and no level is specified, the
level is set to multiple.  If disabled, the level is set to single.
When the level is set to multiple, an implementation may also be
specified.  The following implementations are supported.
    global_mutex - a single global lock guards access to all MPI functions.
    global_monitor - a single monitor guards access to all MPI functions.
The default implementation is global_mutex.
--enable-g=option - Control the level of debugging support in the MPICH
implementation.  option may be a list of common separated names including
    none   - No debugging
    handle - Trace handle operations
    dbg    - Add compiler -g flags
    all    - All of the above choices
--enable-fast - pick the appropriate options for fast execution.  This
                turns off error checking and timing collection
--enable-f77 - Enable Fortran 77 bindings
--enable-f90 - Enable Fortran 90 bindings
--enable-cxx - Enable C++ bindings
--enable-romio - Enable ROMIO MPI I/O implementation
--enable-nmpi-as-mpi - Use MPI rather than PMPI routines for MPI routines,
 such as the collectives, that may be implemented in terms of other MPI 
 routines
--with-device=name - Specify the communication device for MPICH.
--with-pmi=name - Specify the pmi interface for MPICH.
--with-pm=name - Specify the process manager for MPICH.
      Multiple process managers may be specified as long as they all use
      the same pmi interface by separating them with colons.  The 
      mpiexec for the first named process manager will be installed.
      Example: --with-pm=forker:mpd:remshell builds the three process 
      managers forker, mpd, and remshell; only the mpiexec from forker
      is installed into the bin directory.
--with-thread-package=package - Thread package to use.  Supported thread
packages include:
    posix or pthreads - POSIX threads
    solaris - Solaris threads (Solaris OS only)
The default package is posix.
--with-logging=name - Specify the logging library for MPICH.
--with-mpe - Build the MPE (MPI Parallel Environment) routines
--enable-weak-symbols - Use weak symbols to implement PMPI routines (default)
--with-htmldir=dir - Specify the directory for html documentation
--with-docdir=dir - Specify the directory for documentation
--with-cross=file - Specify the values of variables that configure cannot
determine in a cross-compilation environment
--with-flavor=name - Set the name to associate with this flavor of MPICH
--with-namepublisher=name - Choose the system that will support 
                             MPI_PUBLISH_NAME and MPI_LOOKUP_NAME.  Options
                             include
                               no (no service available)
                               pmiext  (service using a pmi extension,
                                        usually only within the same MPD ring)
                               file:directory
                               ldap:ldapservername
                             Only no and file have been implemented so far.
--enable-sharedlibs - This switch is obsolete and is not supported

--enable-dependencies - Generate dependencies for sourcefiles.  This
            requires that the Makefile.in files are also created
            to support dependencies (see maint/updatefiles)

\end{verbatim}

\end{document}

%
% Comments on subclassing the document
% We can use \ifdevname ... \fi and \ifpmname ... \fi, as in
% \ifdevchiii .. \fi and \ifpmmpd ... \fi
% (these will still need to be defined)
% There should also be a way to select ``all'' in such a way that the
% document can still flow well, such as
% \ifdevall ... \else \ifdevchiii \else \ifdevmm \fi \fi \fi




