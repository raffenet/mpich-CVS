\documentclass[dvipdfm,11pt]{article}
\usepackage[dvipdfm]{hyperref} % Upgraded url package
\parskip=.1in

% Formatting conventions for contributors
% 
% A quoting mechanism is needed to set off things like file names, command
% names, code fragments, and other strings that would confuse the flow of
% text if left undistinguished from preceding and following text.  In this
% document we use the LaTeX macro '\texttt' to indicate such text in the
% source, which normally produces, when used as in '\texttt{special text}',
% the typewriter font.

% It is particularly easy to use this convention if one is using emacs as
% the editor and LaTeX mode within emacs for editing LaTeX documents.  In
% such a case the key sequence ^C^F^T (hold down the control key and type
% 'cft') produces '\texttt{}' with the cursor positioned between the
% braces, ready for the special text to be typed.  The closing brace can
% be skipped over by typing ^e (go to the end of the line) if entering
% text or ^C-} to just move the cursor past the brace.

% LaTeX mode is usually loaded automatically.  At Argonne, one way to 
% get several useful emacs tools working for you automatically is to put
% the following in your .emacs file.

% (require 'tex-site)
% (setq LaTeX-mode-hook '(lambda ()
%                        (auto-fill-mode 1)
%                        (flyspell-mode 1)
%                        (reftex-mode 1)
%                        (setq TeX-command "latex")))

%
% When updating the version number of MPICH2, make sure that you change *all*
% instances of the version.  Search for mpich2- and check each match.  Some
% are within a verbatim statement and will need to be changed for each update.
%

\begin{document}
\markright{MPICH2 Installer's Guide}
\title{{\bf MPICH2 Installer's Guide}\thanks{This work was supported by the
    Mathematical, Information, and Computational Sciences Division
    subprogram of the Office of Advanced Scientific Computing Research,
    SciDAC Program, Office of Science, U.S. Department of Energy, under
    Contract
    W-31-109-ENG-38.}\\
  Version 1.0.4\\
  Mathematics and Computer Science Division\\
  Argonne National Laboratory}

\author{William Gropp\\
Ewing Lusk\\
David Ashton\\
Darius Buntinas\\
Ralph Butler\\
Anthony Chan\\
Rob Ross\\
Rajeev Thakur\\
Brian Toonen}

\maketitle
\cleardoublepage

\pagenumbering{roman}
\tableofcontents
\clearpage

\pagenumbering{arabic}
\pagestyle{headings}

%% Here is a basic outline for the document.   (Bill's original outline)

%% 0. Quick start with ``best practices''.  Each step has a reference to
%% more detailed information later in the document.
%% 1. Acquiring and unpacking.  Using a ``fast'' directory location and
%%    VPATH
%% 1a. Reporting problems
%% 2. Choosing a device (defer a detailed discussion of each until later)
%% 3. configure, make, and install.  Always use --prefix 
%% show only basic options for configure
%%    3a. Optional include of device-specific information
%%    3b. Optional include of pm-specific information
%%    3c. Optional ``fast'' version
%%    3d. Shared libraries
%% 4. Testing and benchmarking
%% 4a. make testing
%% 4b. Getting, building, and using mpptest and netpipe
%% 5. Special options
%% 6. Troubleshooting
%% Appendix:
%% A. Summary of configure options (particularly the enable and with options)


\section{Introduction}
\label{sec:intro}
This manual describes how to obtain and install MPICH2, the MPI-2
implementation from Argonne National Laboratory.  (Of course, if you are
reading this, chances are good that you have already obtained it and
found this document, among others, in its \texttt{doc} subdirectory.)
This \emph{Guide} will explain how to install MPICH so that you and others can use it to
run MPI applications.  Some particular features are different
if you have system administration privileges (can become ``root'' on a
Unix system), and these are explained here.  It is not necessary to have
such privileges to build and install MPICH2.  In the event of problems,
send mail to \texttt{mpich2-maint@mcs.anl.gov}.  Once MPICH2 is
installed, details on how to run MPI jobs are covered in the \emph{MPICH2
User's Guide}, found in this same \texttt{doc} subdirectory.

MPICH2 has many options.  We will first go through a recommended,
``standard'' installation in a step-by-step fashion, and later describe
alternative possibilities. 


\section{Quick Start}
\label{sec:quick}

In this section we describe a ``default'' set of installation steps.  It
uses the default set of configuration options, which builds the \texttt{sock}
communication device and the \texttt{MPD} process manager, for 
languages C, C++, Fortran-77, and Fortran-90 (if those compilers exist),
 with compilers chosen automatically from the user's environment,
without tracing and debugging options.  It uses the \texttt{VPATH}
feature of \texttt{make}, so that the build process can take place on
a local disk for speed.

\subsection{Prerequisites}
\label{sec:prerequisites}

For the default installation, you will need:
\begin{enumerate}
\item A copy of the distribution, \texttt{mpich2.tar.gz}.
\item A C compiler.
\item A fortran-77, Fortran-90, and/or C++ compiler if you wish to write
  MPI programs in any of these languages.
\item Python 2.2 or later version, for building the default process
  management system, MPD.
  Most systems have
  Python pre-installed, but you can get it free from
  \texttt{www.python.org}.  You may assume it is there unless the
  \texttt{configure} step below complains.
\item Any one of a number of Unix operating systems, such as IA32-Linux.
  MPICH2 is most extensively tested on Linux;  there remain some
  difficulties on systems we do not currently have access to.  Our
  \texttt{configure} script attempts to adapt MPICH2 to new systems. 
\end{enumerate}
Configure will check for these prerequisites and try to work around
deficiencies if possible.  (If you don't have Fortran, you will
still be able to use MPICH2, just not with Fortran applications.)


\subsection{From A Standing Start to Running an MPI Program}
\label{sec:steps}
Here are the steps from obtaining MPICH2 through running your own
parallel program on multiple machines.

\begin{enumerate}
\item 
Unpack the tar file.
\begin{verbatim}
    tar xfz mpich2.tar.gz
\end{verbatim}
If your tar doesn't accept the z option, use
\begin{verbatim}
    gunzip -c mpich2.tar.gz | tar xf -
\end{verbatim}
Let us assume that the directory where you do this is
\texttt{/home/you/libraries}.  It will now contain a subdirectory named
\texttt{mpich2-1.0.4}.

\item
Choose an installation directory (the default is \texttt{/usr/local/bin)}:
\begin{verbatim}
    mkdir /home/you/mpich2-install
\end{verbatim}
It will be most convenient if this directory is shared by all of the
machines where you intend to run processes.  If not, you will have
to duplicate it on the other machines after installation.  Actually, if
you leave out this step, the next step will create the directory for you.

\item
Choose a build directory.  Building will proceed \emph{much} faster if
your build directory is on a file system local to the machine on which
the configuration and compilation steps are executed.  It is preferable
that this also be separate from the source directory, so that the
source directories remain
clean and can be reused to build other copies on other machines.
\begin{verbatim}
    mkdir /tmp/you/mpich2-1.0.4
\end{verbatim}

\item
Configure MPICH2, specifying the installation directory, and running
the \texttt{configure} script in the source directory:
\begin{verbatim}
    cd  /tmp/you/mpich2-1.0.4
    /home/you/libraries/mpich2-1.0.4/configure \
            -prefix=/home/you/mpich2-install |& tee configure.log
\end{verbatim}
where the \texttt{$\backslash$} means that this is really one line.  (On
\texttt{sh} and its derivatives, use \verb+2>&1 | tee configure.log+
instead of \verb+|& tee configure.log+).  Other configure options are
described below.  Check the \texttt{configure.log} file to make sure
everything went well.  Problems should be self-explanatory, but if not,
send \texttt{configure.log} to \texttt{mpich2-maint@mcs.anl.gov}.
The file \texttt{config.log} is created by \texttt{configure} and
contains a record of the tests that \texttt{configure} performed.  It
is normal for some tests recorded in \texttt{config.log} to fail.  

\item
Build MPICH2:
\begin{verbatim}
    make |& tee make.log
\end{verbatim}
This step should succeed if there were no problems with the preceding
step.  Check \texttt{make.log}.  If there were problems, send
\texttt{configure.log} and \texttt{make.log} to
\texttt{mpich2-maint@mcs.anl.gov}.

\item
Install the MPICH2 commands:
\begin{verbatim}
    make install |& tee install.log
\end{verbatim}
This step collects all required executables and scripts in the \texttt{bin}
subdirectory of the directory specified by the prefix argument to
configure. 

\item
Add the \texttt{bin} subdirectory of the installation directory to your path:
\begin{verbatim}
    setenv PATH /home/you/mpich2-install/bin:$PATH
\end{verbatim}
for \texttt{csh} and \texttt{tcsh}, or 
\begin{verbatim}
    export PATH=/home/you/mpich2-install/bin:$PATH
\end{verbatim}
for \texttt{bash} and \texttt{sh}.  Check that everything is in order at
this point by doing
\begin{verbatim}
    which mpd
    which mpicc
    which mpiexec
    which mpirun
\end{verbatim}
All should refer to the commands in the \texttt{bin} subdirectory of your
install directory.  It is at this point that you will need to
duplicate this directory on your other machines if it is not
in a shared file system such as NFS.

\item
MPICH2, unlike MPICH, uses an external process manager for scalable
startup of large MPI jobs.  The default process manager is called
MPD, which is a ring of daemons on the machines where you will run
your MPI programs.  In the next few steps, you will get this ring up
and tested.
The instructions given here will probably be enough to get you started.
If not, you should refer to Appendix~\ref{app:mpd} for troubleshooting help.
More details on interacting with MPD can be found by running
\texttt{mpdhelp} or any mpd command with the \texttt{--help} option, or by
viewing the README file in \texttt{mpich2/src/pm/mpd}.  The information
provided includes how to list running jobs, kill, suspend, or otherwise
signal them, and how to use the \texttt{gdb} debugger via special
arguments to \texttt{mpiexec}.

For security reasons, MPD looks in your home directory for a file named
\texttt{.mpd.conf} containing the line
\begin{verbatim}
    secretword=<secretword>
\end{verbatim}
where \verb+<secretword>+ is a string known only to yourself.  {\em It should
not be your normal Unix password.}  Make this file readable and writable
only by you:
\begin{verbatim}
    cd $HOME
    touch .mpd.conf
    chmod 600 .mpd.conf
\end{verbatim}
Then use an editor to place a line like:
\begin{verbatim}
  secretword=mr45-j9z
\end{verbatim}
into the file.
(Of course use a different secret word than \verb+mr45-j9z+.)

\item
The first sanity check consists of bringing up a ring of one mpd on
the local machine, testing one mpd command, and bringing the ``ring''
down. 
\begin{verbatim}
    mpd & 
    mpdtrace
    mpdallexit
\end{verbatim}
The output of mpdtrace should be the hostname of the machine you are
running on.  The \texttt{mpdallexit} causes the mpd daemon to exit.
If you encounter problems, you should check the troubleshooting section 
of Appendix~\ref{app:mpd}.

\item
The next sanity check is to run a non-MPI program using the daemon.
\begin{verbatim}
    mpd & 
    mpiexec -n 1 /bin/hostname
    mpdallexit
\end{verbatim}
This should print the name of the machine you are running on.  If not,
you should check the troubleshooting section in Appendix~\ref{app:mpd}.

\item
Now we will bring up a ring of \texttt{mpd}'s on a set of machines.  Create
a file consisting of a list of machine names, one per line.  Name this
file \texttt{mpd.hosts}.  These hostnames will be used as targets for
\texttt{ssh} or \texttt{rsh}, so include full domain names if necessary.  Check that you
can reach these machines with \texttt{ssh} or \texttt{rsh} without
entering a password.  You can test by doing
\begin{verbatim}
    ssh othermachine date
\end{verbatim}
or
\begin{verbatim}
    rsh othermachine date
\end{verbatim}
If you cannot get this to work without entering a password, you will
need to configure \texttt{ssh} or \texttt{rsh} so that this can be done,
or else use the workaround for \texttt{mpdboot} in the next step.

\item
Start the daemons on (some of) the hosts in the file \texttt{mpd.hosts}.
\begin{verbatim}
  mpdboot -n <number to start> -f mpd.hosts
\end{verbatim}
The number to start can be less than 1 + number of hosts in the
file, but cannot be greater than 1 + the number of hosts in the
file.  One mpd is always started on the machine where \texttt{mpdboot} is
run, and is counted in the number to start, whether or not it occurs
in the file.  By default, \texttt{mpdboot} will only start one \texttt{mpd} per
machine even if the machine name appears in the hosts file multiple times.
The \texttt{-1} option can be used to override this behavior, but there is typically
no reason for a user to need multiple mpds on a single host.
The \texttt{-1} option exists mostly to support internal testing.
The \texttt{--help} option explains these as well as other useful
options.  In particular, if your cluster has multiprocessor nodes, you
might want to use the \texttt{--ncpus} argument described in 
Section~\ref{sec:mpd-on-smps}.

Check to see if all the hosts you listed in \texttt{mpd.hosts} are in the output
of 
\begin{verbatim}
    mpdtrace
\end{verbatim}
and if so move on to the next step.

There is a workaround if you cannot get \texttt{mpdboot} to work because of
difficulties with \texttt{ssh} or \texttt{rsh} setup.  You can start the daemons ``by
hand'' as follows:
\begin{verbatim}
   mpd &            # starts the local daemon
   mpdtrace -l      # makes the local daemon print its host
                    # and port in the form <host>_<port>
\end{verbatim}
Then log into each of the other machines, put the \texttt{install/bin}
directory in your path, and do:
\begin{verbatim}
   mpd -h <hostname> -p <port> &
\end{verbatim}
where the hostname and port belong to the original mpd that you
started.  From each machine, after starting the mpd, you can do 
\begin{verbatim}
   mpdtrace
\end{verbatim}
to see which machines are in the ring so far.  More details on
\texttt{mpdboot} and other options for starting the mpd's are in
\texttt{mpich2-1.0.4/src/pm/mpd/README}.

In case of persistent difficulties getting the ring of mpd's up and
running on the machines you want, please see Appendix~\ref{app:mpd}.
There we discuss the mpd's in more detail, together with some programs
for testing the configuration of your systems to make sure that they
allow the mpd's to run.

\item
Test the ring you have just created:
\begin{verbatim}
    mpdtrace
\end{verbatim}
The output should consist of the hosts where MPD daemons are now
running.  You can see how long it takes a message to circle this
ring with 
\begin{verbatim}
    mpdringtest
\end{verbatim}
That was quick.  You can see how long it takes a message to go
around many times by giving mpdringtest an argument:
\begin{verbatim}
    mpdringtest 100
    mpdringtest 1000
\end{verbatim}

\item
Test that the ring can run a multiprocess job:
\begin{verbatim}
    mpiexec -n <number> hostname
\end{verbatim}
The number of processes need not match the number of hosts in the
ring;  if there are more, they will wrap around.  You can see the
effect of this by getting rank labels on the stdout:
\begin{verbatim}
    mpiexec -l -n 30 hostname
\end{verbatim}
You probably didn't have to give the full pathname of the hostname
command because it is in your path.  If not, use the full pathname:
\begin{verbatim}
    mpiexec -l -n 30 /bin/hostname
\end{verbatim}

\item
Now we will run an MPI job, using the \texttt{mpiexec} command as specified
in the MPI-2 standard.  

% There are some examples in the install
% directory, which you have already put in your path, as well as in
% the directory \texttt{mpich2-1.0.4/examples}.  
% One of them is the classic \texttt{cpi}
% example, which computes the value of $\pi$ by numerical integration in
% parallel.   
As part of the build process for MPICH2, a simple program to compute the value
of $\pi$ by numerical integration is created in the
\texttt{mpich2-1.0.4/examples} directory.  If the current directory is the top
level MPICH2 build directory, then you can run this program with
\begin{verbatim}
    mpiexec -n 5 examples/cpi
\end{verbatim}
The number of processes need not match the number of hosts.
The \texttt{cpi} example will tell you which hosts it is running on.
By default, the processes are launched one after the other on the hosts
in the mpd ring, so it is not necessary to specify hosts when running a
job with \texttt{mpiexec}.

There are many options for \texttt{mpiexec}, by which multiple executables
can be run, hosts can be specified (as long as they are in the mpd
ring), separate command-line arguments and environment variables can
be passed to different processes, and working directories and search
paths for executables can be specified.  Do
\begin{verbatim}
    mpiexec --help
\end{verbatim}
for details. A typical example is:
\begin{verbatim}
    mpiexec -n 1 master : -n 19 slave
\end{verbatim}
or
\begin{verbatim}
    mpiexec -n 1 -host mymachine master : -n 19 slave
\end{verbatim}
to ensure that the process with rank 0 runs on your workstation.

The arguments between `:'s in this syntax are called ``argument sets,''
since they apply to a set of processes.  More argments are described in
the \textit{User's Guide}.

The \texttt{mpirun} command from the original MPICH is still available,
although it does not support as many options as \texttt{mpiexec}.  You might
want to use it in the case where you do not have the XML parser
required for the use of \texttt{mpiexec}.
\end{enumerate}

If you have completed all of the above steps, you have successfully
installed MPICH2 and run an MPI example.  


\subsection{Common Non-Default Configuration Options}
\label{sec:non-default}

A list of \texttt{configure} options is found in
Section~\ref{configure-options}.  Here we comment on some of them.  This
section needs much more work.

\subsubsection{The Most Important Configure Options}
\begin{description}
\item[--prefix]Set the installation directories for MPICH2.  
\item[--enable-debuginfo]Provide access to the message queues for
  debuggers such as Totalview.
\item[--enable-g]Build MPICH2 with various debugging options.  This is
  of interest primarily to MPICH2 developers.  The options
  \texttt{--enable-g=dbg,mem,log} are recommended in that case.
\item[--enable-fast]Configure MPICH2 for fastest performance at the
  expense of error reporting and other program development aids.  This
  is recommended only for getting the best performance out of proven
  production applications, and for benchmarking.
\item[--without-mpe]Configure MPICH2 without the MPE package of
  program development tools (including the Jumpshot performance viewer)
\item[--enable-sharedlibs]Build MPICH2 with shared libraries. 
  For example, 
  \texttt{--enable-sharedlibs=gcc} for standard gcc on Linux  
  \texttt{--enable-sharedlibs=osx-gcc} for Mac OS X or
  \texttt{--enable-sharedlibs=solaris-cc} for cc on Solaris
\item[--with-pm]Select the process manager.  The default is
  \texttt{mpd}; also useful is \texttt{gforker}.  You can build with
  both process managers with \texttt{--with-pm=mpd:gforker}.
\end{description}


\subsubsection{Using the Absoft Fortran compilers with MPICH2}
\label{sec:absoft}

For best results, it is important to force the Absoft Fortran compilers to 
make all routine names monocase.  In addition, if lower case is chosen 
(this will match common use by many programs), you must also tell the the 
Absoft compiles to append an underscore to global names in order to access 
routines such as getarg (getarg is not used by MPICH2 but is used in some 
of the tests and is often used in application programs).  We recommend 
configuring MPICH2 with the following options
\begin{verbatim}

setenv F77 f77
setenv MPI_FFLAGS "-f -N15"
setenv MPI_F90FLAGS "-YALL_NAMES=LCS -YEXT_SFX=_"

./configure ....

\end{verbatim}




\subsection{Shared Libraries}
\label{sec:shared-libraries}

Shared libraries are currently only supported for gcc on Linux and
Mac OS X and for cc on Solaris. To have shared libraries created when
MPICH2 is built, specify the following when MPICH2 is configured:
\begin{verbatim}
    configure --enable-sharedlibs=gcc         (on Linux)
    configure --enable-sharedlibs=osx-gcc     (on Mac OS X)
    configure --enable-sharedlibs=solaris-cc  (on Solaris)
\end{verbatim}




\subsection{What to Tell the Users}
\label{sec:telling}

Now that MPICH2 has been installed, the users have to be informed of how
to use it.  Part of this is covered in the \emph{User's Guide}.  Other
things users need to know are covered here.  (For example, whether they need to
run their own mpd rings or use a system-wide one run by root.)


\section{Migrating from MPICH1}
\label{sec:migrating}

MPICH2 is an all-new rewrite of MPICH1.  Although the basic steps for
installation have remained the same (\texttt{configure}, \texttt{make},
\texttt{make install}), a number of things have changed.  In this
section we attempt to point out what you may be used to in MPICH1 that
are now different in MPICH2.

\subsection{Configure Options}
\label{sec:configure-options}

The arguments to \texttt{configure} are different in MPICH1 and MPICH2;
the \texttt{Installer's Guide} discusses \texttt{configure}.  In
particular, the newer \texttt{configure} in MPICH2 does not support the
\verb+-cc=<compiler-name>+ (or \texttt{-fc}, \texttt{-c++}, or
\texttt{-f90}) options.  Instead, many of the items that could be
specified in the command line to configure in MPICH1 must now be set by
defining an environment variable.  E.g., while MPICH1 allowed
\begin{verbatim}
    ./configure -cc=pgcc
\end{verbatim}
MPICH2 requires
\begin{verbatim}
    setenv CC pgcc
\end{verbatim}
(or \verb+export CC=pgcc+ for \texttt{ksh} or \verb+CC=pgcc ; export CC+
for strict \texttt{sh}) before \texttt{./configure}.  Basically, every
option to the MPICH-1 configure that does not start with
\texttt{--enable} or \texttt{--with} is not available as a configure
option in MPICH2.  Instead, environment variables must be used.  This is
consistent (and required) for use of version 2 GNU \texttt{autoconf}.

\subsection{Other Differences}
Other differences between MPICH1 and MPICH2 include the handling of
process managers and the choice of communication device.

For example, the new mpd has a new format and slightly different
semantics for the -machinefile option.  Assume that you type this
data into a file named \texttt{machfile}:
\begin{verbatim}
    bp400:2
    bp401:2
    bp402:2
    bp403:2
\end{verbatim}

If you then run a parallel job with this machinefile, you would expect
ranks 0 and 1 to run on bp400 because it says to run 2 processes there
before going on to bp401.  Ranks 2 and 3 would run on bp401, and rank
4 on bp402, e.g.:

\begin{verbatim}
    mpiexec -l -machinefile machfile -n 5 hostname
\end{verbatim}

produces:
\begin{verbatim}
    0: bp400
    1: bp400
    2: bp401
    3: bp401
    4: bp402
\end{verbatim}

\section{Choosing the Communication Device}
\label{sec:choose-device}

MPICH2 is designed to be build with many different communication devices,
allowing an implementation to be tuned for different communication fabrics.  A
simple communication device, known as ``ch3'' (for the third version of the
``channel'' interface) is provided with MPICH2 and is the default choice.

The ch3 device itself supports a variety of communication methods.  These are
specified by providing the name of the method after a colon in the
\texttt{--with-device} configure option.  For example,
\texttt{--with-device=ch3:ssm} selects the socket plus shared memory method.
The supported methods include:

\begin{description}
\item[ch3:sock]This is the default communication method.  It uses sockets for
  all communications between processes.  As of MPICH2 version 1.0.4, it is the
  only communication method that enables \texttt{MPI\_THREAD\_MULTIPLE}.
\item[ch3:ssm]This method uses sockets between nodes and shared memory within
a node.
\item[ch3:shm]This method only uses shared memory and only works within a
single SMP.  It does not support the MPI dynamic process routines such as
\texttt{MPI\_Comm\_spawn}. 
% \item[ch3:sshm]This method is a scalable version of a shared-memory method and
% only works within a single SMP.
\item[ch3:nemesis]This method is our new, high performance method.  It
supports sockets, shared memory, and Myrinet-GM at present.  It does not
support the MPI dynamic process routines in this release, but will support
them in subsequent releases.
\end{description}

Most installations should use either the \texttt{ch3:ssm} or \texttt{ch3:sock}
methods.  If you need multi-threaded MPI, you must use \texttt{ch3:sock} in
this release.  If you have a cluster of SMPs and do not need a multi-threaded
MPI, then select \texttt{ch3:ssm}.  If you are interested in trying out our
highest-performance device, and need neither threads nor the dynamic process
routines from MPI, then you can use \texttt{ch3:nemesis}.

%
% Add any configure options that are specific to each method here.  Also
% any environment variables that might apply (e.g., MPICH_INTERFACE_HOSTNAME 
% or MPICH_PORT_RANGE.
%
%\subsection{Special Features of Each Communication Device}
%\subsubsection{ch3:shm}
%\subsubsection{ch3:nemesis}

\section{Installing and Managing Process Managers}
\label{sec:process-managers}
MPICH2 has been designed to work with multiple process managers; that
is, although you can start MPICH2 jobs with \texttt{mpiexec}, there are
different mechanisms by which your processes are started.  An interface
(called PMI) isolates the MPICH2 library code from the process manager.
Currently three process managers are distributed with MPICH2
\begin{description}
\item[mpd] This is the default, and the one that is described
  in~Section~\ref{sec:steps}.  It consists of a ring of daemons.
\item[smpd] This one can be used for both Linux and Windows.  It is the
  only process manager that supports the Windows version of MPICH2.
\item[gforker] This is a simple process manager that creates all
  processes on a single machine.  It is useful for both debugging and on
  shared memory multiprocessors.
\end{description}

\subsection{mpd}
\label{sec:mpd}

\subsubsection{Configuring mpd}
\label{sec:configuring-mpd}

The \texttt{mpd} process manager can be explicitly chosen at configure
time by adding
\begin{verbatim}
    --with-pm=mpd
\end{verbatim}
to the \texttt{configure} argments.  This is not necessary, since
\texttt{mpd} is the default.


\subsubsection{System Requirements}
\label{sec:mpd-requirements}

\texttt{mpd} consists of a number of components written in Python.
The \texttt{configure} script should automatically find a version of
python in your \texttt{PATH} that has all the features needed to run
\texttt{mpd}.  If for some reason you need to pick a specific version of
Python for \texttt{mpd} to use, you can do so by adding
\begin{verbatim}
    --with-python=<fullpathname of python interpreter>
\end{verbatim}
to your \texttt{configure} arguments.  If your system doesn't have Python,
you can get the latest version from \texttt{http://www.python.org}.
Most Linux distributions include a moderately current version of Python.
MPD requires release 2.2 or later.

The \texttt{mpd} process manager supports the use of the TotalView
parallel debugger from Etnus.  If \texttt{totalview} is in your
\texttt{PATH} when MPICH2 is configured, then an interface module will
be automatically compiler, linked and installed so that you can use
TotalView to debug MPICH jobs (See the \texttt{User's Guide} under
``Debugging''.  You can also explicitly enable or disable this
capability with \texttt{--enable-totalview} or
\texttt{--disable-totalview} as arguments to \texttt{configure}.

\subsubsection{Using mpd}
\label{sec:using-mpd}

In Section~\ref{sec:steps} you installed the mpd ring.  Several commands
can be used to use, test, and manage this ring.  You can find out about
them by running \texttt{mpdhelp}, whose output looks like this:

\begin{small}
\begin{verbatim}
The following mpd commands are available.  For usage of any specific one,
invoke it with the single argument --help .

mpd           start an mpd daemon
mpdtrace      show all mpd's in ring
mpdboot       start a ring of daemons all at once
mpdringtest   test how long it takes for a message to circle the ring 
mpdallexit    take down all daemons in ring
mpdcleanup    repair local Unix socket if ring crashed badly
mpdlistjobs   list processes of jobs (-a or --all: all jobs for all users)
mpdkilljob    kill all processes of a single job
mpdsigjob     deliver a specific signal to the application processes of a job
mpiexec       start a parallel job

Each command can be invoked with the --help argument, which prints usage
information for the command without running it.
\end{verbatim}
\end{small}
So for example, to see a complete list of the possible arguments for
\texttt{mpdboot}, you would run
\begin{verbatim}
    mpdboot --help
\end{verbatim}


\subsubsection{Options for \texttt{mpd}}

\begin{description}
\item[--help] causes mpd to print a list and description of all options
\end{description}

In addition to the cmd-line options, mpd will also check for presence of
the environment variable \texttt{MPICH\_PORT\_RANGE} (note MPICH instead of MPD)
and use only the ports in that range for listening sockets.  The range
is separated by a colon, e.g., 2000:8000.


\subsubsection{Running MPD on multi-homed systems}
\label{sec:mpd-multi-homed}

If you plan to use one or more multi-homed systems, it is of course useful if
the default hostname is associated with the interface that mpd will need
to use for communications.  If not however, you can cause mpd to use a
specific interface by using the \texttt{--ifhn} (interface-hostname) option, e.g.:
\begin{verbatim}
    n1%  mpd --ifhn=192.168.1.1 &
\end{verbatim}

If you then run mpiexec on n1 connecting to that mpd, the mpiexec will
use the same ifhn for communications with remote processes that connect
back to it.  mpiexec will also accept a -ifhn option (mpiexec --help)
in the unlikely event that you wish it to use a separate interface from
the mpd.

mpdboot can also designate the ifhn to be used by both the local and remote
mpds which it starts, e.g.:
\begin{verbatim}
    n1%  mpdboot --totalnum=4 --ifhn=192.168.1.1 
\end{verbatim}
where mpd.hosts contains:
\begin{verbatim}
  n2 ifhn=192.168.1.2
  n3 ifhn=192.168.1.3
\end{verbatim}
will start one mpd locally, one on n2 and one on n3.  Each will use the
respectively designated ifhn.


\subsubsection{Running MPD as Root}
\label{sec:mpd-root}

MPD can run as root to support multiple users simultaneously.  To do this,
it easiest to simply do the ``make install'' in the mpd sub-directory as
root.  This will cause the \texttt{mpdroot} program to be installed in
the bin directory with setuid-root permissions.  Individual users then
have the option of starting and using their own mpd rings, or they can run
with a ring started by root.  To use root's ring, they must use an option
named \texttt{MPD\_USE\_ROOT\_MPD}.  This option may either be set as an environment
variable or they can set it in their own .mpd.conf file, e.g.:
\begin{verbatim}
    MPD_USE_ROOT_MPD=1
\end{verbatim}
When root starts the mpds in the ring, the procedure is the same as for a regular
user except that root's configuration file is in \texttt{/etc/mpd.conf} (note that
there is no leading dot in the file name).


\subsubsection{Running MPD on SMP's}
\label{sec:mpd-on-smps}

Typically one starts one \texttt{mpd} on each host.  When a job is
started with \texttt{mpiexec} without any particular host specification,
the processes are started on the ring of hosts one at a time, in
round-robin fashion until all the processes have been started.  Thus, if
you start a four-process job on a ring of two machines, hosta and hostb,
then you will find ranks 0 and 2 on hosta and ranks 1 and 4  on hostb.
This might not be what you want, especially if the machines are SMP's
and you would like to have consecutive ranks on the same machine as much
as possible.  If you tell the \texttt{mpd} how many cpus it has to work
with by using the \texttt{--ncpus} argument, as in 
\begin{verbatim}
    mpd --ncpus=2
\end{verbatim}
then the number of processes started the first time the startup message
circles the ring will be determined by this argument.  That is, in the
above four-process example, ranks 0 and 1 will be on hosta and ranks 2
and 3 will be on hostb.  This effect only occurs the first time around
the ring.  That is if you start a six-process job on this ring (two
mpd's, each started with \texttt{--ncpus=2}) you will get processes 0,
1, and 4 on hosta and 2, 3, and 5 on hostb.  This is for load-balancing
purposes.  (It is assumed that you do not want 0, 1, 4, and 5 on hosta
and only 2 and 3 on hostb; if that is what you \emph{do} want, you can
control process placement explicitly on the \texttt{mpiexec} command, as
described in the \textit{User's Guide}.


\subsubsection{Security Issues in MPD}
\label{sec:mpd-security}

Since the \texttt{mpd} process manager allows the remote execution of
processes, it needs to incorporate security features that prevent
inappropriate or malicious process creation by someone without
permission to do so.  In this section we describe how \texttt{mpd} deals
with a number of security issues.

The first issue is the starting of the \texttt{mpd} processes
themselves.  This is done either by the user logging into each machine
starting each \texttt{mpd} process ``by hand,'' or else by using the
\texttt{mpdboot} program.  \texttt{mpdboot} uses \texttt{ssh} by
default, although the less secure \texttt{rsh} can be used if the user
chooses.

The next issue occurs when a single new \texttt{mpd} joins a ring of
existing \texttt{mpd}'s.  This occurs during \texttt{mpdboot}, or can
occur later if the user wishes to expand an existing ring.  In this
situation a ``challenge-response'' protocol is followed.  A user must
have the same \texttt{secretword} set in his \texttt{.mpd.conf} file on
each machine.  This file must be readable only by the user starting the
\texttt{mpd}; otherwise the \texttt{mpd} will refuse to read it.  When
\texttt{mpd} A wishes to join an existing ring by connecting to
\texttt{mpd} B, which is already in the ring, at the port B is listening
on, the following sequence of events occurs:
\begin{enumerate}
\item B temporarily accepts the connection.
\item B seeds the random number generator with the high-resolution part
  of the time of day, generates a random number, and sends it to A.
\item A concatenates the random number with the value of secretword in
  \texttt{.mpd.conf}, encrypts it with md5, and sends the encrypted
  result to B.
\item Meanwhile, B encrypts the (same) random number concatenated with
  its value of \texttt{secretword}.
\item When B receives the encrypted value from A, it compares the
  encrypted value with its own.
\item If the values match, A is allowed to join the ring; otherwise the
  connection is closed.
\end{enumerate}
Note that the secret word is never sent over the connection in the clear,
and the use of the time of day means that there will be no repeating
pattern of challenges to be observed.  

The third issue concerns how \texttt{mpiexec} connects to the local
\texttt{mpd}.  This is done through a Unix socket in \texttt{/tmp} rather than
through an INET socket.  Thus security is preserved through the security
of the file system.

Once connections are established, the messages that are sent over them
are not encrypted.  However, no security information is sent over these
connections. 

\subsection{SMPD}
\label{sec:smpd}

\subsubsection{Configuration}
\label{sec:smpd_configure}

You may add the following configure options, 
\texttt{--with-pm=smpd}, 
to build and install the smpd process manager. The process manager, smpd, 
will be installed to the bin sub-directory of the installation directory 
of your choice specified by the \texttt{--prefix} option.

smpd process managers run on each node as stand-alone daemons and need to
be running on all nodes that will participate in MPI jobs.  smpd process 
managers are not connected to each other and rely on a known port to 
communicate with each other.  Note: If you want multiple users to use the 
same nodes they must each configure their smpds to use a unique port per 
user. 

smpd uses a configuration file to store settings.  The default location is 
\verb+~/.smpd+.  This file must not be readable by anyone other than 
the owner and contains at least one required option - the access passphrase.
This is stored in the configuration file as \texttt{phrase=<phrase>}. Access 
to running smpds is authenticated using this passphrase and it must 
not be your user password.

\subsubsection{Usage and administration}
\label{sec:smpd_usage}

Users will start the smpd daemons before launching mpi jobs.  To get an 
smpd running on a node, execute 
\begin{verbatim}
    smpd -s
\end{verbatim}
Executing this for the first time will prompt the user to create a 
\verb+~/.smpd+ configuration file and passphrase if one does not 
already exist.

Then users can use \texttt{mpiexec} to launch MPI jobs.

All options to \texttt{smpd}:

\begin{description}
\item[\texttt{smpd -s}]\mbox{}\\
Start the smpd service/daemon for the current user.  You can add 
\texttt{-p <port>} to specify the port to listen on.  All smpds must use
the same port and if you don't use the default then you will have to
add \texttt{-p <port>} to mpiexec or add the \texttt{port=<port>} to the 
\texttt{.smpd} configuration file.

\item[\texttt{smpd -r}]\mbox{}\\
Start the smpd service/daemon in root/multi-user mode.  This is not yet
implemented.

\item[\texttt{smpd -shutdown [host]}]\mbox{}\\
Shutdown the smpd on the local host or specified host.  Warning: this will
cause the smpd to exit and no mpiexec or smpd commands can be issued to the
host until smpd is started again.

%\item[\texttt{smpd -start}]\mbox{}\\
%Start the Windows smpd service on the local host.
%
%\item[\texttt{smpd -stop}]\mbox{}\\
%Stop the Windows smpd on the local host.
%
%\item[\texttt{smpd -console [host]}]\mbox{}\\
%Connect to a specific smpd to issue console commands.  The currently 
%supported commands are: %get, set, delete, stat, status, shutdown and validate.
%\begin{enumerate}
%\item[\texttt{get <var>}]\mbox{}
%\item[\texttt{set <var>=<value>}]\mbox{}
%\item[\texttt{delete <var>}]\mbox{}
%\item[\texttt{status}]\mbox{}
%\item[\texttt{stat <var>}]\mbox{}
%\item[\texttt{shutdown}]\mbox{}
%\item[\texttt{validate}]\mbox{}
%\end{enumerate}

\end{description}

\subsection{gforker}
\label{sec:forker}
\texttt{gforker} is a simple process manager that runs all processes on
a single node; it's version of \texttt{mpiexec} uses the system
\texttt{fork} and \texttt{exec} calls to create the new processes.
To configure with the \texttt{gforker} process manager, use
\begin{verbatim}
    configure --with-pm=gforker ...
\end{verbatim}

\section{Testing}
\label{sec:testing}
Running basic tests in the examples directory, the MPICH2 tests,
obtaining and running the assorted test suites.  


\subsection{Using the Intel Test Suite}
\label{sec:intel}

These instructions are local to our test environment at Argonne.

How to run a select set of tests from the Intel test suite:

\begin{small}
\begin{verbatim}
1) checkout the Intel test suite (cvs co IntelMPITEST) (outside users
   should access the most recent version of the test suite from the
   test suite web page).

2) create a testing directory separate from the IntelMPITEST source
directory

3) cd into that testing directory

4) make sure the process manager (e.g., mpd) is running

5) run "<ITS_SRC_DIR>/configure --with-mpich2=<MPICH2_INSTALL_DIR>", where
<ITS_SRC_DIR> is the path to the directory Intel test suite source (e.g.,
/home/toonen/Projects/MPI-Tests/IntelMPITEST) and <MPICH2_INSTALL_DIR> is
the directory containing your MPICH2 installation

6) mkdir Test; cd Test

7) find tests in <ITS_SRC_DIR>/{c,fortran} that you are interested in
running and place the test names in a file.  For example:

% ( cd /home/toonen/Projects/MPI-Tests/IntelMPITEST/Test ; \
    find {c,fortran} -name 'node.*' -print | grep 'MPI_Test' 
    | sed -e 's-/node\..*$--' ) |& tee testlist
Test/c/nonblocking/functional/MPI_Test
Test/c/nonblocking/functional/MPI_Testall
Test/c/nonblocking/functional/MPI_Testany
Test/c/nonblocking/functional/MPI_Testsome
Test/c/persist_request/functional/MPI_Test_p
Test/c/persist_request/functional/MPI_Testall_p
Test/c/persist_request/functional/MPI_Testany_p
Test/c/persist_request/functional/MPI_Testsome_p
Test/c/probe_cancel/functional/MPI_Test_cancelled_false
Test/fortran/nonblocking/functional/MPI_Test
Test/fortran/nonblocking/functional/MPI_Testall
Test/fortran/nonblocking/functional/MPI_Testany
Test/fortran/nonblocking/functional/MPI_Testsome
Test/fortran/persist_request/functional/MPI_Test_p
Test/fortran/persist_request/functional/MPI_Testall_p
Test/fortran/persist_request/functional/MPI_Testany_p
Test/fortran/persist_request/functional/MPI_Testsome_p
Test/fortran/probe_cancel/functional/MPI_Test_cancelled_false
%

8) run the tests using ../bin/mtest:

% ../bin/mtest -testlist testlist -np 6 |& tee mtest.log
%

NOTE: some programs hang if less they are run with less than 6 processes.

9) examine the summary.xml file.  look for '<STATUS>fail</STATUS>' to see if
any failures occurred.  (search for '>fail<' works as well)

\end{verbatim}
\end{small}

\section{Benchmarking}
\label{sec:benchmarking}
netpipe, mpptest, others (SkaMPI).

\section{MPE}
\label{sec:mpe}

This section describes what MPE is and its potentially separate installation.  It
includes discussion of Java-related problems.

\section{Windows Version}
\label{sec:windows}

\subsection{Binary distribution}
\label{sec:winbin}

The Windows binary distribution uses the Microsoft Installer.  Download and 
execute mpich2-1.x.xxx.msi to install the binary distribution.  The default 
installation path is \texttt{C:$\backslash$Program Files$\backslash$MPICH2}. 
You must have administrator privileges to install mpich2.msi.  The installer 
installs a Windows service to launch MPICH applications and only administrators
may install services.  This process manager is called smpd.exe.  Access to 
the process manager is passphrase protected.  The installer asks for this 
passphrase.  Do not use your user password.  The same passphrase must be 
installed on all nodes that will participate in a single MPI job.

Under the installation directory are three sub-directories: \texttt{include},
 \texttt{bin}, and \texttt{lib}.  The \texttt{include} and \texttt{lib} 
directories contain the header files and libraries necessary to compile MPI 
applications.  The \texttt{bin} directory contains the process manager, 
\texttt{smpd.exe}, and the the MPI job launcher, \texttt{mpiexec.exe}.  The
dlls that implement MPICH2 are copied to the Windows system32 directory.

You can install MPICH in unattended mode by executing 
\begin{verbatim}
    msiexec /q /I mpich2-1.x.xxx.msi
\end{verbatim}

The smpd process manager for Windows runs as a service that can launch jobs 
for multiple users.  It does not need to be started like the unix version 
does.  The service is automatically started when it is installed and when 
the machine reboots.  smpd for Windows has additional options:
\begin{description}
\item[\texttt{smpd -start}]\mbox{}\\
Start the Windows \texttt{smpd} service.
\item[\texttt{smpd -stop}]\mbox{}\\
Stop the Windows \texttt{smpd} service.
\item[\texttt{smpd -install}]\mbox{}\\
Install the \texttt{smpd} service.
\item[\texttt{smpd -remove}]\mbox{}\\
Remove the \texttt{smpd} service.
\item[\texttt{smpd -register\_spn}]\mbox{}\\
Register the Service Principal Name with the domain controller.
This command enables passwordless authentication using kerberos.  It must be 
run on each node individualy by a domain administrator.
\end{description}

\subsection{Source distribution}
\label{sec:winsrc}

In order to build MPICH2 from the source distribution under Windows, 
you must have MS Developer Studio .NET 2003 or later, perl and optionally 
Intel Fortran 8 or later.

\begin{itemize}
\item
Download \texttt{mpich2-1.x.y.tar.gz} and unzip it.
\item
Bring up a Visual Studio Command prompt with the compiler environment 
variables set.
\item
Run \texttt{winconfigure.wsf}. If you don't have a Fortran compiler add the 
\texttt{--remove-fortran} option to \texttt{winconfigure} to remove all the Fortran 
projects and dependencies.  Execute \texttt{winconfigure.wsf /?} to see all
available options.
\item 
    open \texttt{mpich2$\backslash$mpich2.sln}
%\item
%    build the \texttt{ch3sockDebug mpich2} solution
%\item
%    build the \texttt{ch3sockDebug mpich2s} project
\item
    build the ch3sockRelease mpich2 solution
\item
    build the ch3sockRelease mpich2s project
%\item
%    build the Debug mpich2 solution
\item
    build the Release mpich2 solution
%\item
%    build the fortDebug mpich2 solution
\item
    build the fortRelease mpich2 solution
%\item
%    build the gfortDebug mpich2 solution
\item
    build the gfortRelease mpich2 solution
%\item
%    build the sfortDebug mpich2 solution
\item
    build the sfortRelease mpich2 solution
\item
    build the channel of your choice.  The options are \texttt{shm}, \texttt{ssm}, \texttt{sshm}, \texttt{ib}, \texttt{essm}, and \texttt{mt}.  
The shm channel is for small numbers of processes that will run on a single 
machine using shared memory.  The shm channel should not be used for more 
than about 8 processes.  The sshm (scalable shared memory) is for use with 
more than 8 processes.  The \texttt{ssm} (sock shared memory) channel is for clusters 
of smp nodes.  This channel should not be used if you plan to over-subscribe 
the CPU's.  If you plan on launching more processes than you have processors 
you should use the default \texttt{sock} channel or the \texttt{essm} channel.  The \texttt{ssm} channel 
uses a polling progress engine that can perform poorly when multiple processes 
compete for individual processors.  The \texttt{essm} channel is derived from the ssm channel
with the addition of OS event objects to avoid spinning in the progress engine.  The \texttt{mt
channel} is the multi-threaded socket channel.  The \texttt{ib} channel is for clusters with Infiniband
interconnects from Mellanox.

\end{itemize}

\subsection{\texttt{cygwin}}
\label{sec:cygwin}

MPICH2 can also be built under \texttt{cygwin} using the source
distribution and the unix commands described in previous sections.  This
will not build the same libraries as described in this section.  It will 
build a ``unix'' distribution that runs under \texttt{cygwin}.

\section{All Configure Options}
\label{configure-options}
Here is a list of all the configure options currently recognized by the
top-level configure.  It is the MPICH-specific part of the output of 
\begin{verbatim}
    configure --help
\end{verbatim}
Not all of these options may be fully supported yet.  
% This snapshot of configure --help output was taken on Nov 3, 2005.  It
% needs to be updated frequently (or automatically) and in the long run
% to be replaced by real documentation.

\begin{small}
\begin{verbatim}
Optional Features:
  --disable-FEATURE       do not include FEATURE (same as --enable-FEATURE=no)
  --enable-FEATURE[=ARG]  include FEATURE [ARG=yes]
--enable-cache  - Turn on configure caching
--enable-echo  - Turn on strong echoing. The default is enable=no.
--enable-strict - Turn on strict debugging with gcc
--enable-coverage - Turn on coverage analysis using gcc and gcov
--enable-error-checking=level - Control the amount of error checking.
level may be
    no        - no error checking
    runtime   - error checking controllable at runtime through environment
                variables
    all       - error checking always enabled
--enable-error-messages=level - Control the amount of detail in error
  messages.  Level may be
    all       - Maximum amount of information
    generic   - Only generic messages (no information about the specific
                instance)
    class     - One message per MPI error class
    none      - No messages
--enable-timing=level - Control the amount of timing information
collected by the MPICH implementation.  level may be
    none    - Collect no data
    all     - Collect lots of data
    runtime - Runtime control of data collected
The default is none.
--enable-threads=level - Control the level of thread support in the
MPICH implementation.  The following levels are supported.
    single - No threads (MPI_THREAD_SINGLE)
    funneled - Only the main thread calls MPI (MPI_THREAD_FUNNELED)
    serialized - User serializes calls to MPI (MPI_THREAD_SERIALIZED)
    multiple:impl - Fully multi-threaded (MPI_THREAD_MULTIPLE)
The default is funneled.  If enabled and no level is specified, the
level is set to multiple.  If disabled, the level is set to single.
When the level is set to multiple, an implementation may also be
specified.  The following implementations are supported.
    global_mutex - a single global lock guards access to all MPI functions.
The default implementation is global_mutex.
--enable-g=option - Control the level of debugging support in the MPICH
implementation.  option may be a list of common separated names including
    none     - No debugging
    mem      - Memory usage tracing
    handle   - Trace handle operations
    dbg      - Add compiler -g flags
    log      - Enable debug event logging
    meminit  - Preinitialize memory associated structures and unions to
               eliminate access warnings from programs like valgrind
    all      - All of the above choices
--enable-fast - pick the appropriate options for fast execution.  This
                turns off error checking and timing collection
--enable-f77 - Enable Fortran 77 bindings
--enable-f90 - Enable Fortran 90 bindings
--enable-cxx - Enable C++ bindings
--enable-romio - Enable ROMIO MPI I/O implementation
--enable-nmpi-as-mpi - Use MPI rather than PMPI routines for MPI routines,
 such as the collectives, that may be implemented in terms of other MPI
 routines
--enable-mpe - Build the MPE (MPI Parallel Environment) routines
--enable-weak-symbols - Use weak symbols to implement PMPI routines (default)
--enable-sharedlibs=kind - Enable shared libraries.  kind may be
    gcc     - Standard gcc and GNU ld options for creating shared libraries
    osx-gcc - Special options for gcc needed only on OS/X
    solaris-cc - Solaris native (SPARC) compilers for 32 bit systems
    cygwin-gcc - Special options for gcc needed only for cygwin
    none    - same as --disable-sharedlibs
Only gcc, osx-gcc, and solaris-cc are currently supported

--enable-dependencies - Generate dependencies for sourcefiles.  This
            requires that the Makefile.in files are also created
            to support dependencies (see maint/updatefiles)
--enable-runtimevalues - Determine various parameters of the Fortran
                         environment at run time, such as the values for
                         .TRUE. and .FALSE. .  This allows a single MPICH
                         library to be used with multiple Fortran compilers
--enable-timer-type=name - Select the timer to use
for MPI_Wtime and internal timestamps.  name may be one of
    gethrtime - Solaris timer (Solaris systems only)
    clock_gettime   - Posix timer (where available)
    gettimeofday - Most Unix systems
    linux86_cycle - Linux x86; returns cycle counts, not time in seconds
    linuxalpha_cycle - Like linux86_cycle, but for Linux Alpha
    gcc_ia64_cycle - IPF ar.itc timer


Optional Packages:
  --with-PACKAGE[=ARG]    use PACKAGE [ARG=yes]
  --without-PACKAGE       do not use PACKAGE (same as --with-PACKAGE=no)
--with-device=name - Specify the communication device for MPICH.
--with-pmi=name - Specify the pmi interface for MPICH.
--with-pm=name - Specify the process manager for MPICH.
      Multiple process managers may be specified as long as they all use
      the same pmi interface by separating them with colons.  The
      mpiexec for the first named process manager will be installed.
      Example: --with-pm=forker:mpd:remshell builds the three process
      managers forker, mpd, and remshell; only the mpiexec from forker
      is installed into the bin directory.
--with-logging=name - Specify the logging library for MPICH.
--with-mpe - Build the MPE (MPI Parallel Environment) routines
--with-htmldir=dir - Specify the directory for html documentation
--with-docdir=dir - Specify the directory for documentation
--with-cross=file - Specify the values of variables that configure cannot
determine in a cross-compilation environment
--with-namepublisher=name - Choose the system that will support
                             MPI_PUBLISH_NAME and MPI_LOOKUP_NAME.  Options
                             include
                               no (no service available)
                               mpd
                               file:directory (optional directory)

--with-cxxlibname=name  - Specify name of library containing C++ interface
routines
--with-flibname=name  - Specify name of library containing Fortran interface
routines
--with-fwrapname=name  - Specify name of library containing Fortran interface
routines
--with-thread-package=package - Thread package to use.  Supported thread
packages include:
    posix or pthreads - POSIX threads
    solaris - Solaris threads (Solaris OS only)
    none - no threads
If the option is not specified, the default package is ${MPE_THREAD_DEFAULT}.
If the option is specified, but a package is not given, then the default
is posix.

Some influential environment variables:
  CC          C compiler command
  CFLAGS      C compiler flags
  LDFLAGS     linker flags, e.g. -L<lib dir> if you have libraries in a
              nonstandard directory <lib dir>
  CPPFLAGS    C/C++ preprocessor flags, e.g. -I<include dir> if you have
              headers in a nonstandard directory <include dir>
  F77         Fortran 77 compiler command
  FFLAGS      Fortran 77 compiler flags
  F90         Fortran 90 compiler command
  F90FLAGS    Fortran 90 compiler flags
  CXX         C++ compiler command
  CXXFLAGS    C++ compiler flags
  CPP         C preprocessor

Use these variables to override the choices made by `configure' or to help
it to find libraries and programs with nonstandard names/locations.


\end{verbatim}
\end{small}

\paragraph{Notes on the configure options.}
The \texttt{--with-htmldir} and \texttt{--with-docdir} options specify the 
directories into which the documentation will be installed by \texttt{make
  install}. 

\appendix


\section{Troubleshooting MPDs}
\label{app:mpd}



\subsection{Getting Started with mpd}
\label{sec:getting-started}

\texttt{mpd} stands for multi-purpose daemon.  We sometimes use the term
mpd to refer to the combination of the mpd daemon and its helper programs
that collectively form a process management system for executing parallel
jobs, including mpich jobs.  The mpd daemon must run on each host where
you wish to execute parallel programs.  The mpd daemons form a ring to
facilitate rapid process startup.  Even a single mpd on a single host
forms a loop.  Therefore, each host must be configured in such a way
that the mpds can connect to each other and pass messages via sockets.

It can be rather tricky to configure one or more hosts in such a way
that they adequately support client-server applications like mpd.
In particular, each host must not only know its own name, but must
identify itself correctly to other hosts when necessary.  Further, certain
information must be readily accessible to each host.  For example, each
host must be able to map another host's name to its IP address.  In this
section, we will walk slowly through a series of steps that will help
to ensure success in running mpds on a single host or on a large cluster.

If you can \texttt{ssh} from each machine to \emph{itself}, and from
each machine to each other machine in your set (and back), then you
probably have an adequate environment for mpd.  However, there may
still be problems.  For example, if you are blocking all ports except
the ports used by ssh/sshd, then mpd will still fail to operate correctly.

To begin using mpd, the sequence of steps that we recommend is this:
\begin{enumerate}
    \item get one mpd working alone on a first test node
    \item get one mpd working alone on a second test node
    \item get two new mpds to work together on the two test nodes
    \item boot two new mpds on the two test nodes via mpdboot
\end{enumerate}

\subsubsection{Following the steps}

\begin{enumerate}
\item Install mpich2, and thus mpd.

\item Make sure the mpich2 bin directory is in your path.
Below, we will refer to it as \texttt{MPDDIR}.

\item Kill old mpd processes.
If you are coming to this guide from elsewhere, e.g. a Quick Start guide
for mpich2, because you encountered mpd problems, you should make sure
that all mpd processes are terminated on the hosts where you have been
testing.  \texttt{mpdallexit} may assist in this, but probably not if
you were having problems.  You may need to use the Unix \texttt{kill}
command to terminate the processes.

\item Run a first mpd (alone on a first node).
As mentioned above, mpd uses client-server communications to perform
its work.  So, before running an mpd, let's run a simpler program
(\texttt{mpdcheck}) to verify that these communications are likely to be
successful.  Even on hosts where communications are well supported,
sometimes there are problems associated with hostname resolution, etc.
So, it is worth the effort to proceed a bit slowly.  Below, we assume
that you have installed mpd and have it in your path.

Select a test node, let's call it \texttt{n1}.  Login to n1.

First, we will run mpdcheck as a server and a client.  To run it as a
server, get into a window with a command-line and run this:
\begin{verbatim}
    n1 $  mpdcheck -s
\end{verbatim}
It will print something like this:
\begin{verbatim}
    server listening at INADDR_ANY on: n1 1234
\end{verbatim}

Now, run the client side (in another window if convenient) and see
if it can find the server and communicate.  Be sure to use the \emph{same}
hostname and portnumber printed by the server (above: n1 1234):
\begin{verbatim}
    n1 $  mpdcheck -c n1 1234
\end{verbatim}

If all goes well, the server will print something like:
\begin{verbatim}
    server has conn on
        <socket._socketobject object at 0x40200f2c>
            from ('192.168.1.1', 1234)
    server successfully recvd msg from client:
        hello_from_client_to_server
\end{verbatim}
and the client will print:
\begin{verbatim}
    client successfully recvd ack from server:
        ack_from_server_to_client
\end{verbatim}

If the experiment failed, you have some network or machine configuration
problem which will also be a problem later when you try to use mpd.
Even if the experiment succeeded, but the hostname printed by the server
was \emph{localhost}, then you will probably have problems later if you
try to use mpd on n1 in conjunction with other hosts.  In either case,
skip to the section ``Debugging host/network configuration problems.''

If the experiment succeeded, then you should be ready to try mpd on
this one host. To start an mpd, you will use the mpd command.  To run
parallel programs, you will use the mpiexec program.  All mpd commands
accept the -h or --help arguments, e.g.:
\begin{verbatim}
    n1 $  mpd --help 
    n1 $  mpiexec --help
\end{verbatim}

Try a few tests:
\begin{verbatim}
    n1 $  mpd &
    n1 $  mpiexec -n 1 /bin/hostname
    n1 $  mpiexec -l -n 4 /bin/hostname
    n1 $  mpiexec -n 2 PATH_TO_MPICH2_EXAMPLES/cpi
\end{verbatim}
where \texttt{PATH\_TO\_MPICH2\_EXAMPLES} is the path to the
\texttt{mpich2-1.0.3/examples} directory.

To terminate the mpd:
\begin{verbatim}
    n1 $  mpdallexit
\end{verbatim}


\item Run a second mpd (alone on a second node).
To verify that things are fine on a second host (say \emph{n2}), login
to n2 and perform the same set of tests that you did on n1.  Make sure
that you use mpdallexit to terminate the mpd so you will be ready for
further tests.


\item Run a ring of two mpds on two hosts.
Before running a ring of mpds on n1 and n2, we will again use mpdcheck,
but this time between the two machines.  We do this because the two nodes
may have trouble locating each other or communicating between them and
it is easier to check this out with the smaller program.

First, we will make sure that a server on n1 can service a client from n2.
On n1:
\begin{verbatim}
    n1 $  mpdcheck -s
\end{verbatim}
which will print a hostname (hopefully n1) and a portnumber (say 3333 here).
On n2:
\begin{verbatim}
    n2 $  mpdcheck -c n1 3333
\end{verbatim}

If this experiment fails, skip to the section ``Debugging host/network
configuration problems''.

Second, we will make sure that a server on n2 can service a client from n1.
On n2:
\begin{verbatim}
    n2 $  mpdcheck -s
\end{verbatim}
which will print a hostname (hopefully n2) and a portnumber (say 7777 here).
On n2:
\begin{verbatim}
    n2 $  mpdcheck -c n2 7777
\end{verbatim}

If this experiment fails, skip to the section ``Debugging host/network
configuration problems''.

If all went well, we are ready to try a pair of mpds on n1 and n2.
First, make sure that all mpds have terminated on both n1 and n2.  Use
mpdallexit or simply kill them with:
\begin{verbatim}
kill -9 PID_OF_MPD
\end{verbatim}
where you have obtained the \texttt{PID\_OF\_MPD} by some means such as the \texttt{ps} command.

On n1:
\begin{verbatim}
    n1 $  mpd &
    n1 $  mpdtrace -l
\end{verbatim}
This will print a list of machines in the ring, in this case just n1.
The output will be something like:
\begin{verbatim}
    n1_6789 (192.168.1.1)
\end{verbatim}
The 6789 is the port that the mpd is listeneing on for connections from other
mpds wishing to enter the ring.  We will use that port in a moment to get an mpd
from n2 into the ring.  The value in parentheses should be the IP address of n1.

On n2:
\begin{verbatim}
    n2 $  mpd -h n1 -p 6789 &
\end{verbatim}
where 6789 is the listening port on n1 (from mpdtrace above).
Now try:
\begin{verbatim}
    n2 $  mpdtrace -l
\end{verbatim}
You should see both mpds in the ring.

To run some programs in parallel:
\begin{verbatim}
    n1 $  mpiexec -n 2 /bin/hostname
    n1 $  mpiexec -n 4 /bin/hostname
    n1 $  mpiexec -l -n 4 /bin/hostname
    n1 $  mpiexec -l -n 4 PATH_TO_MPICH2_EXAMPLES/cpi
\end{verbatim}
where \texttt{PATH\_TO\_MPICH2\_EXAMPLES} is the path to the
\texttt{mpich2-1.0.4/examples} directory.

To bring down the ring of mpds:
\begin{verbatim}
    n1 $  mpdallexit
\end{verbatim}


\item Boot a ring of two mpds via mpdboot.
Please be aware that mpdboot uses ssh by default to start remote mpds.
It will expect that you can run ssh from n1 to n2 (and from n2 to n1)
without entering a password.

First, make sure that you terminate the mpd processes from any prior
tests.

On n1, create a file named \texttt{mpd.hosts} containing the name of n2:
\begin{verbatim}
n2
\end{verbatim}

Then, on n1 run:
\begin{verbatim}
    n1 $  mpdboot -n 2
    n1 $  mpdtrace -l
    n1 $  mpiexec -l -n 2 /bin/hostname
\end{verbatim}
The mpdboot command should read the mpd.hosts file created above and
run an mpd on each of the two machines.  The mpdtrace and mpiexec show
the ring up and functional.
Options that may be useful are:
\begin{itemize}
    \item \texttt{--help} use this one for extra details on all options
    \item \texttt{-v} (verbose)
    \item \texttt{--chkup} tries to verify that the hosts are up before starting mpds
    \item \texttt{--chkuponly }only performs the verify step, then ends
\end{itemize}

To bring the ring down:
\begin{verbatim}
    n1 $  mpdallexit
\end{verbatim}

If mpdboot works on the two machines n1 and n2, it will probably work on
your others as well.  But, there could be configuration problems using a
new machine on which you have not yet tested mpd.  An easy way to check,
is to gradually add them to mpd.hosts and try an mpdboot with a -n arg
that uses them all each time.  Use mpdallexit after each test.

\end{enumerate}


\subsection{Debugging host/network configuration problems}
We use mpdcheck as our first attempt to debug host or network
configuration problems.  If you run:
\begin{verbatim}
    n1 $  mpdcheck --help
\end{verbatim}
you should receive a fairly long help message describing a wide variety
of arguments which can be supplied to mpdcheck to help you debug.

The first thing to try is to simply login to a node, say n1 and run:
\begin{verbatim}
    n1 $  mpdcheck
\end{verbatim}
mpdcheck will produce no output here if it finds no problems.
If mpdcheck does find potenital problems, it will print them with *** at
the beginning of the line.  You can cause mpdcheck to be verbose by using
the -v option, e.g.:
\begin{verbatim}
    n1 $  mpdcheck -v
\end{verbatim}
Also, if mpdcheck offers comments about how you might repair certain problems,
you can get a longer version of those messages by using the -l option, e.g.:
\begin{verbatim}
    n1 $  mpdcheck -l
\end{verbatim}

If you run mpdcheck on each node and find no problems, you may still wish to
use it further to debug issues between two nodes.  For example, you might
login to n1 and create file named mpd.hosts which contains the name of another
node which is having trouble communicating with n1, e.g. n2.
Then, you may want to run:
\begin{verbatim}
    n1 $  mpdcheck -f mpd.hosts
\end{verbatim}
This test will see if n1 is having trouble discovering information about
n2.  If not, you wish to try:
\begin{verbatim}
    n1 $  mpdcheck -f mpd.hosts -ssh
\end{verbatim}
This will also try to test ssh support between n1 and n2.

If these 2 experiments go OK, you should probably try them again but
this time logged into n2 and trying to connect back to n1.  Do not 
forget to change the contents of mpd.hosts to contain the name of n1.

If none of these get you past the problems, you may need to ask for help.
If so, it will probably useful to run mpdcheck once more on each of the
nodes which are of concern:
\begin{verbatim}
    n1 $  mpdcheck -pc
    n2 $  mpdcheck -pc
\end{verbatim}
These will produce quite a bit of output which may be useful in determining
the problem.  The -pc option does not really try to offer any comemnts about
what may be wrong.  It merely prints potentially useful debugging info.


\subsection{Firewalls, etc.}

If the output from any of mpdcheck, mpd, or mpdboot leads you to believe
that one or more of your hosts are having trouble communicating due to
firewall issues, we can offer a few simple suggestions.  If the problems
are due to an ``enterprise'' firewall computer, then we can only point you
to your local network admin for assistance.

In other cases, there are a few quick things that you can try to see if
there some common protections in place which may be causing your problems.

First, you might see if iptables is active.  You will probably need to
be root to do this:
\begin{verbatim}
    n1 #  iptables -L
\end{verbatim}
This will show a set of 3 current iptables chains being applied for each
of INPUT, FORWARD, and OUTPUT.  If the chains are non-empty, then you may
have something blocked.  This could be a result of a software firewall 
package you are running (e.g. shorewall) or some home-grown set of chains.
If you are unfamiliar with iptables, you will need to get local help to 
decipher the rules and determine if any of them may be affecting you.  There
are options such as -F to iptables that will disable the chains, but that is
dangerous of course if you require them for protection.

Next, you might see if any tcp-wrappers are active.  You may need to be 
root to do this:
\begin{verbatim}
    n1 #  cat /etc/hosts.deny /etc/hosts.allow
\end{verbatim}
If there are any uncommented lines, they likely designate any (or ALL)
daemons which have their tcp communications blocked.  This can be
particularly problematic for mpdboot which uses ssh (and thus the ssh
daemon, sshd).

Next, you might wish to see if you have available ephemeral ports:
\begin{verbatim}
    n1 $  cat /proc/sys/net/ipv4/ip_local_port_range
\end{verbatim}
This should print a range something like:
\begin{verbatim}
    32768   61000
\end{verbatim}

\end{document}

%
% Comments on subclassing the document
% We can use \ifdevname ... \fi and \ifpmname ... \fi, as in
% \ifdevchiii .. \fi and \ifpmmpd ... \fi
% (these will still need to be defined)
% There should also be a way to select ``all'' in such a way that the
% document can still flow well, such as
% \ifdevall ... \else \ifdevchiii \else \ifdevmm \fi \fi \fi




