\documentclass[dvipdfm,11pt]{article}
\usepackage[dvipdfm]{hyperref} % Upgraded url package
\usepackage{../mpiman}   % Definitions for this particular version of the
                         % manual

\begin{document}

Here is a basic outline for the document

1. Structure of MPICH2 

   1a. Directory tree

   1b. Where a developer usually adds/changes

2. Adding a channel

3. Adding a device

4. Adding a pm

5. Adding a module

  5a. Collective

  5b. Topology

  5c. Others (e.g., datatypes)

8. Adding or internationalizing error messages

7. Development aids

  7a. Performance
      7ai. Logging internal state information
  
      --with-logging=<format> --enable-timing=<when>

      where format is rlog, dlog, or lwlog, and when is all, runtime (others?)

      7aii. Reading or viewing log data

   7b. Code checks
 
      7bi. --enable-strict

      7bii.  updatefiles (using the code style checker)

      7biii. --enable-g=all (includes runtime checks)
9. Troubleshooting
\end{document}
