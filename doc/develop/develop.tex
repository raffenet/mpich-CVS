\documentclass[dvipdfm,11pt]{article}
\usepackage[dvipdfm]{hyperref} % Upgraded url package
\usepackage{../mpiman}   % Definitions for this particular version of the
                         % manual
\usepackage{tpage}

\begin{document}
\markright{MPICH2 Developer's Manual}
\ANLTMTitle{MPICH2 Developer's Manual\\
Version 0.1\\
Draft of \today}{\em 
William Gropp\\
Ewing Lusk\\
David Ashton\\
Rob Ross\\
Rajeev Thakur\\
Brian Toonen\\
Mathematics and Computer Science Division\\
Argonne National Laboratory}{00}{\today}

\cleardoublepage

\pagenumbering{roman}
\tableofcontents
\clearpage

\pagenumbering{arabic}
\pagestyle{headings}

Here is a basic outline for the document

1. Structure of MPICH2 

   1a. Directory tree

   1b. Where a developer usually adds/changes files

2. Adding a channel

2a. Adding a socket

2b. Adding an RDMA

3. Adding a device

4. Adding a pm

5. Adding a module

  5a. Collective

  \input collop.tex

  5b. Topology

  5c. Others (e.g., datatypes)

8. Adding or internationalizing error messages

7. Development aids

  7a. Performance

      7ai. Logging internal state information
  
      \code{--with-logging=<format> --enable-timing=<when>}

      where format is rlog, dlog, or lwlog, and when is all, runtime (others?)

      7aii. Reading or viewing log data

   7b. Code checks
 
      7bi. \code{--enable-strict}

      7bii.  updatefiles (using the code style checker)

      7biii. \code{--enable-g=all} (includes runtime checks)

9. Troubleshooting

\end{document}
