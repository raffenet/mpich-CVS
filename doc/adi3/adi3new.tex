% 
%   This is a latex file that generates a reference manual for 
%   ADI-3 
%
\documentclass{article}
\usepackage{/home/gropp/data/share/refman}
\usepackage{/home/gropp/sowing-proj/sowing/docs/doctext/tpage}
\usepackage{url} 
\usepackage{epsf}
\usepackage{psfig}
\textheight=9in
\textwidth=6.1in
\oddsidemargin=.2in
\topmargin=-.50in
\newread\testfile

\newcommand{\mpich}{\texttt{mpich}}
\newcommand{\Mpich}{\texttt{Mpich}}

%
% For now, let findex be the same as index.  This will allow us to
% more easily separate function and nonfunction index entries later.
\let\findex=\index
%
% Modify the way titles are handled for no breaks between pages
\def\mantitle#1#2#3{\pagerule\nobreak
\ifmancontents\addcontentsline{toc}{subsection}{#1}\fi
\index{#1}}

\makeindex

\begin{document}

\markright{ADI-3 Reference Manual}

\def\nopound{\catcode`\#=13}
{\nopound\gdef#{{\tt \char`\#}}}
\catcode`\_=13
\def_{{\tt \char`\_}}
\catcode`\_=11
\def\code#1{{\tt #1}}
%\let\url=\code
\def\makeussubscript{\catcode`\_=8}
\def\makeustext{\catcode`\_=11}
%\tpageoneskip
\ANLTMTitle{MPICH Abstract Device Interface\\
Version 3.3\\
Reference Manual\\\ \\Draft of \today}{\em 
William Gropp\\
Ewing Lusk\\
Your Name Here\\
Mathematics and Computer Science Division\\
Argonne National Laboratory}{00}{\today}

\clearpage

\pagenumbering{roman}
\tableofcontents
\clearpage

\pagenumbering{arabic}
\pagestyle{headings}

%
% The first section is the design
\input design.tex

\section{Integrating a New Device into the MPICH Build Tree}
This section describes how to add a device or method into the MPICH
build tree.  This section is intended both to describe the process of
adding a device and the rationale for the design of the
device-dependent modules.

\input adiimplrules.tex

\appendix

\section{Data Structures}
\input adi3man/MPID_Request_kind.tex
\input adi3man/MPID_Request.tex
\input adi3man/MPID_Comm.tex
\input adi3man/MPID_Segment.tex
\input adi3man/MPID_Dataloop.tex
\input adi3man/MPID_Dataloop_contig.tex
\input adi3man/MPID_Dataloop_vector.tex
\input adi3man/MPID_Dataloop_blockindexed.tex
\input adi3man/MPID_Dataloop_indexed.tex
\input adi3man/MPID_Dataloop_struct.tex
\input adi3man/MPID_Datatype.tex

\section{Basic Point-to-Point}
\input adi3man/MPID_Send.tex
\input adi3man/MPID_Ssend.tex
\input adi3man/MPID_Rsend.tex
\input adi3man/MPID_Isend.tex
\input adi3man/MPID_Issend.tex
\input adi3man/MPID_Irsend.tex
\input adi3man/MPID_tBsend.tex
\input adi3man/MPID_Recv.tex
\input adi3man/MPID_Irecv.tex
\input adi3man/MPID_Request_free.tex
\input adi3man/MPID_Cancel_send.tex
\input adi3man/MPID_Cancel_recv.tex
\input adi3man/MPID_Iprobe.tex
\input adi3man/MPID_Probe.tex

\section{Persistent Point-to-Point}
\input adi3man/MPID_Send_init.tex
\input adi3man/MPID_Ssend_init.tex
\input adi3man/MPID_Rsend_init.tex
\input adi3man/MPID_Recv_init.tex
\input adi3man/MPID_Startall.tex

\section{Data Segment}
\label{sec:segment-fcns}
\input adi3man/MPID_Segment_init_pack
\input adi3man/MPID_Segment_pack
\input adi3man/MPID_Segment_init_unpack
\input adi3man/MPID_Segment_unpack
\input adi3man/MPID_Segment_free

\section{Progress Engine}
\input adi3man/MPID_Progress_start.tex
\input adi3man/MPID_Progress_end.tex
\input adi3man/MPID_Progress_test.tex
\input adi3man/MPID_Progress_wait.tex
\input adi3man/MPID_Progress_poke.tex

\section{Starting and stopping}
\input adi3man/MPID_Init.tex
\input adi3man/MPID_Finalize.tex
\input adi3man/MPID_Abort.tex

\section{RMA}
(not yet defined)
%\input adi3man/MPID_Win_put.tex
%\input adi3man/MPID_Win_get.tex
%\input adi3man/MPID_Win_accumulate.tex
%\input adi3man/MPID_Win_do.tex
%\input adi3man/MPID_Win_fence.tex
%\input adi3man/MPID_Win_begin.tex
%\input adi3man/MPID_Win_end.tex
%\input adi3man/MPID_Win_local_lock.tex
%\input adi3man/MPID_Win_local_unlock.tex

\section{Dynamic Processes}
(not yet designed)

\section{Collective Communication}
(not yet designed)

\section{Device Hooks}
\input adi3man/MPID_Dev_xxx_create_hook.tex
\input adi3man/MPID_Dev_xxx_destroy_hook.tex

% More on data structures and constants, including MPIU_Handle_obj_create
% etc.?  Or does this go into the MPICH2 document?
%
%\section{Available Utility Functions}
%\input adi3man/MPIU_Object_add_ref.tex
%\input adi3man/MPIU_Object_release_ref.tex

\let\SaveBibliography=\thebibliography
\def\thebibliography#1{\SaveBibliography{#1}\addcontentsline{toc}{section}{References}}
\bibliography{/home/MPI/allbib,/home/gropp/Update/new/gropp,/home/gropp/papers/MPI/mpich2-coding/mpich2}
\bibliographystyle{plain}

% Index
%\openin\testfile{adi3new.ind}
%\ifeof\testfile\else
\let\SaveIndex=\theindex
\long\def\theindex#1{\SaveIndex{#1}\addcontentsline{toc}{section}{Index}}
\input adi3new.ind
%\fi
%\closein\testfile

\end{document}
