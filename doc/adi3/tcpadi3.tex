%
% This file contains a discussion of a possible TCP device implementation of ADI3

\documentclass{article}
\usepackage{psfig}

\def\nopound{\catcode`\#=13}
{\nopound\gdef#{{\tt \char`\#}}}
\catcode`\_=13
\def_{{\tt \char`\_}}
\catcode`\_=11
\def\code#1{\texttt{#1}}

\begin{document}

\title{A TCP Implementation of the ADI-3}
\author{}
\maketitle

\begin{abstract}

\end{abstract}

\section{Introduction}
This document outlines an implementation of the ADI on TCP.  It defines a
specific interface to the low level OS TCP operations, and outlines a way
for at least the basic MPID_ routines to be implemented in terms of these
abstract operations.  This document is preliminary.

\section{Outline of the Implementation Structure}

\begin{verbatim}
Notes from the board


\end{verbatim}

\section{Pseudo-code for some of the \code{MPID_} routines}

\begin{verbatim}

MPID_Isend( )
{

}

MPID_Irecv( )
{

}

MPID_Wait( )
{

}
\end{verbatim}

\section{Summary}
\label{sec:tcpadi-summary}
(This section should summarize the TCP\_ routines, giving just their prototypes)

\end{document}
