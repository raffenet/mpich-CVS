%
% This file contains a discussion of a possible TCP device and a new
% channel device implementation of ADI3  

\documentclass{article}
%\usepackage{refman}
%\usepackage{tpage}
\usepackage{code}
\usepackage{url} 
\usepackage{epsf}
\usepackage{graphics}
%hyperref - do not remove this comment
\ifx\texorpdfstring\undefined
\def\texorpdfstring#1#2{#1}
\fi

\textheight=9in
\textwidth=6.1in
\oddsidemargin=.2in
\topmargin=-.50in
% \mpids{MPI_Comm}{rank}.  Prints the second argument.
\def\mpids#1#2{\code{#2}\index{#1!#2}}

\def\nopound{\catcode`\#=13}
{\nopound\gdef#{{\tt \char`\#}}}
%\catcode`\_=13
%\def_{\texttt{\char`\_}}
%\catcode`\_=11
%\def\code#1{\texttt{#1}}
%\def\code{\begingroup\makeustext\eatcode}
%\def\eatcode#1{\texttt{#1}\endgroup}
\let\file=\code

\begin{document}

\title{The CH3 Design for a Simple Implementation of ADI-3 for MPICH with a
TCP-based Implementation}
\author{David Ashton, William Gropp, Rajeev Thakur, and Brian Toonen}
\maketitle

\begin{abstract}
ADI-3 is a full featured, abstract device interface used in the MPICH
implementation of MPI to provide a portability layer that allows
access to many performance-oriented features of a wide range of
communication systems.  ADI-3 is allows research into wide range of
implementation issues in MPI. However, because it is full featured,
it contains a large number of functions that must be implemented.  To
both simplify the task of experimenting with MPI implementation
issues, a simplified ``channel'' device is described.  This device
requires the implementation of only a dozen functions but provides
many of the performance advantages of the full ADI-3 interface.  This
smaller interface, called CH3 (for third version of the channel
interface) in turn implements the full ADI-3 interface, providing a
simple way to port MPICH to a new platform.  To illustrate the
implementation issues, an implementation of CH3 using TCP sockets is
described. 
\end{abstract}

\section{Introduction}
This document outlines the CH3 ``channel'' device implementation of
the ADI and sketches an implementation of CH3 on TCP.  It defines a
specific interface to the low level OS TCP operations, and outlines a way
for at least the basic \code{MPID_} routines to be implemented in terms of
these 
abstract operations.  This document is preliminary.

The major goals of this implementation include:
\begin{enumerate}
\item Illustrate efficiency by minimizing the overhead on common cases.
For example, a send/receive of a single word should generate as few ``extra''
allocations of internal objects as possible.  In particular, this
design allows the data to be sent directly without creating a
\code{MPID_Request}. We haven't quite managed that on the receive side.

\item Provide a relatively small interface that can be used to port
MPICH to new 
platforms.  This replaces the ADI-2 ``channel'' interface.

\item Provide an example that can use remote write (put) operations for data
transfers.  
\end{enumerate}

It is \emph{not} a goal of this device to provide an optimally fast
implementation. This is  
intended to be a relatively simple but reasonably efficient implementation.

%------------------------------------------------------------------------------

\section{Outline of the Implementation Structure}
\label{sec:impl-outline}

The MPID implementation makes use of a number of layers.  These layers
can be implemented logically (without separate function calls) in
order to avoid the overhead of function calls.  In the case of the TCP
implementation, the cost of function calls is dominated by TCP network
overheads.  

%Layers: 

The MPID routines are implemented in terms of a smaller set of routines that
perform relatively simple data communication operations.  These are designed
so that they can easily be implemented with, for example, TCP, but are not
restricted to TCP.  For those familar with the ``channel device'' in ADI-2,
these routines represent the ADI-3 version of the channel device interface.
These routines are prefixed with \code{CH3_} (really \code{MPIDI_CH3_}) to
indicate that they belong to 
this interface; a complete description of the CH3 routines is presented in
Appendix~\ref{app:ch3}. 

\paragraph{Communication:}

The MPID layer communicates by sending messages consisting of a message header
(called a packet header to distinguish it from an MPI message), possibly
followed by data.  The MPID layer defines (internal to itself, so this
discussion only applies to the CH3/TCP implementation of MPID):
\begin{description}
\item[Packet Types.]An \code{enum} of types, this describes roughly a
dozen kinds of 
  message that are needed to implement MPI message-passing semantics.  The
  packet types and what they contain include:
  \begin{description}
    \item[eager\_send.] MPI envelope and optional request id; data
      immediately follows the packet (with no separate header).  An
      MPI envelope contains the data used to match MPI messages: tag,
      sender's rank, and context id, along with any MPI flow control
      (for eager messages) and error-checking features such as
      datatype signatures.  The optional request id is sent when
      message can be cancelled.
    \item[ready\_send.] Same as eager\_send but used to perform ready
    send operations.
    \item[eager\_sync\_send.] Same as eager\_send but used to perform
    synchronous send operations.  The request id is required.
    \item[eager\_sync\_ack] Request id.  An acknowledgement that the
    receiving process has posted a receive matching the send request.
    \item[rndv\_req\_to\_send.]MPI envelope and send request id.
    \item[rndv\_clr\_to\_send.]Send and receive request id
    \item[put.]Address and length, followed immediately by data
    \item[rndv\_send.]send and receive request id, followed immediately by data
    \item[cancel\_send\_req.]Send request id to cancel
    \item[cancel\_send\_resp.]Send request id and true/false (ack/nack) for was
         cancelled
    \item[flow\_cntl\_update.]Flow control for eager messages and rendezvous
      requests.  This is separate from any low-level flow control, though it
      may be coordinated with it.  For example, we may include knowledge about
      the size of the socket buffer in this level of flow control.
    \end{description}
    Additional packet types will be defined to support MPI-2 operations such
    as RMA.
\item[Packet Format.]The actual layout of a packet; for each packet type, there
  is a corresponding packet format defined by a structure.  For version zero,
  all packet layouts will have the same size in bytes; this simplifies
  the implementation.
\item[Packet Handlers.]The code to be invoked when a packet arives at its
  destination.   Each of the packet types has an associated handler.
  Viewed this way, the ch3 channel device has a straightforward
  implementation in terms of active messages, using a restricted set
  of remote functions.
\end{description}

Questions:
\begin{enumerate}
\item Why are there eager sync send and ack?  Synchronous sends can be
  accomplished by always using the rendezvous send operation.
\item Is the ``put'' operation obsolete (see the new RMA support)?
  Note that we'd like the option to use a put operation to complete a
  rendezvous send in the case that the destination buffer is
  contiguous.  If the put operation is included, how is completion
  signaled at the destination?
\end{enumerate}

\paragraph{Queues and Connections:}

(Need some discussion of these, since the requests move between queues and the 
read/write operations are ordered on a connection.)

\paragraph{CH3 routine brief summary:}

(I don't think that these are right yet, but we need to start somewhere.)
Details in Appendix~\ref{app:ch3}.

\begin{description}
\item[CH3\_Request\_create.]Create a new request. This is used with
  \code{CH3_iSend}, and for cases where a completed request is required (see
  the discussion of \code{MPID_Isend}).
\item[CH3\_Request\_add\_ref.] Add one to the request's reference count.
\item[CH3\_Request\_release\_ref.]Decrement the request's reference count
   by one.  Returns non-zero if the count is non-zero and zero if the count
   is now zero.
\item[CH3\_Request\_destroy.]Destroy an existing request. This is used
after CH3\_Request\_release\_ref indicates that the reference count is
zero.
\item[CH3\_iStartMsg.]Begin a message.  Return a request if the message has not
  been completely sent.
\item[CH3\_iStartMsgv.]Like \code{CH3_iStartmsg}, but with an \code{struct
    iovec}.
\item[CH3\_iStartRead.]Begin reading data.  Return a request if the message has
  not been completely received.
\item[CH3\_iSend.]Send data using an existing request.
\item[CH3\_iSendv.]Like \code{CH3_iSend}, but with an \code{struct iovec}.
\item[CH3\_iWrite.]Like \code{CH3_iSend}, but the data sent is within the
  request (the \mpids{MPID_Request}{active\_buf} member).
\item[CH3\_iRead.]Read data to a location specified by a request.
\item[CH3\_Progress\_xxx.]Progress functions.  \code{CH3_Progress} is
  responsible for dispatching incoming messages.
\item[CH3\_Init.]Initialize the device and setup the initial communicators.
\item[CH3\_Finalize.]Finalize the device.
\item[CH3\_InitParent.]Initialize the parent communicator (if one exists).
\item[CH3\_iPut.]Nonblocking, contiguous put into remote memory (optional,
  provided as a hook for non-TCP methods and as a placeholder for the
  routines necessary for implementing MPI-2 RMA).
\end{description}

Questions:
\begin{enumerate}
\item Why is there a separate \texttt{add\_ref} and
  \texttt{release\_ref} from the MPIU routines?  Is this because the
  channel device may be multithreaded even if MPI is supporting only
  \texttt{MPI\_THREAD\_SINGLE}?  
\item Does \texttt{CH3\_Init} really setup the initial communicators?
  What does that mean?
\item Exactly what does \texttt{CH3\_InitParent} do?  How does this
  match the original ADI3 design of \texttt{MPID\_Init} (which
  included a \texttt{parent} output argument)?
\end{enumerate}

The bindings for these routines are:
\begin{verbatim}
/* Routines that create and destroy requests */
MPID_Request * CH3_iStartMsg( MPID_VC * vc, void * header,
                              MPID_msg_sz_t header_sz )
MPID_Request * CH3_iStartMsgv( MPID_VC * vc, MPID_IOV * iov, int iov_n )
MPID_Request * CH3_Request_create( void );
void MPIDI_CH3_Request_add_ref( MPID_Request * req );
void MPIDI_CH3_Request_release_ref( MPID_Request * req, int * flag );
void CH3_request_destroy( MPID_Request * req );

/* Routines that send or receive data and use an existing request.
   These are used when incrementally processing communication, for
   example, when packing and sending the next segment from a
   non-contiguous datatype.  Each of these is nonblocking and
   indicates completion by decrementing the completion count (cc)
   field in the request. */
void CH3_iSend( MPID_VC * vc, MPID_Request * sreq, void * header,
                MPID_msg_sz_t header_sz )
void CH3_iSendv( MPID_VC * vc, MPID_Request * sreq, MPI_IOV * iov, int iov_n )
void CH3_iWrite( MPID_VC * vc, MPID_Request * sreq )
void CH3_iRead( MPID_VC * vc, MPID_Request * rreq )

/* Routines for progress */
void CH3_Progress_start( void )
void CH3_Progress_end( void )
int CH3_Progress( int is_blocking )
void CH3_Progress_poke( void )
void CH3_Progress_signal_completion( void )

/* Routines for startup/rundown */
int CH3_Init( int * has_args, int * has_env, int has_parent )
void CH3_Finalize( void )
void CH3_InitParent( MPID_Comm * parent )

/* Routines for RMA */
void CH3_iPut( MPID_VC * vc, void * buf, MPID_msg_sz_t buf_sz,
               MPID_RAint offset, MPID_RAint cmpl_flag )
MPID_Request *CH3_iStartRead( MPID_VC * vc, void * buf, MPID_msg_sz_t buf_sz )
\end{verbatim}

The use of the letter \code{i} in the names is meant to emphasize that these
are non-blocking in the MPI sense: data is not necessarily transfered before
the routine returns and the transfer may continue afterwards (this is what is
different between, for example, \code{CH3_iWrite} and a \code{write} to a
nonblocking socket).

CH3 routines do not handle datatypes; they only handle contiguous data
(buf,count) or struct iovec (viewed as bytes).  Communication routines
are nonblocking, so they must have (a) a connection to which they are
attached (so that data is correctly ordered and sent/received) and (b)
a request that allows incremental pack/unpack.  Each CH3 communication
routine, in effect, makes a callback when the communication completes.
The action may be as simple as ``decrement busy flag and dequeue
communication'' or as complex as ``pack next buffer for sending and
send it''.  However, we do not require the full generality of
arbitrary callback routines, so the action to take on completion will
be specified by an integer.  By chosing this limitation, some
operations may be inlined for greater efficiency.  The completion
actions and associated ``upcall'' functions supplied by the device will be
discussed later.

These routines must be prepared to create actual connections (e.g., establish
a socket) if there is no connection already present.  It is up to the
implementation of the CH3 routines to decide how this is accomplished.

\paragraph{Consequences.}
We can summarize the requirements for the CH3 interface as:
\begin{enumerate}
\item Nonblocking.  Correct operation of the code is not dependent on any read
  or write operation completing when first issued.  
\item Contiguous (or simple iovec) data.  For simplicity, only contiguous byte
  ranges (or Unix-style iovec) data moves are handled at the lowest level
\item Correctness and ordering.  MPI requires that the message
  envelopes are ordered; data 
  transfers must also arrive in expected order.  However, individual
data transfers are \emph{not} ordered.  Completion of a data transfer
guarantees only that all the data have arrived, not the particular
order of arrival.
\item Handshakes.  Some communication, particularly rendezvous transfers,
  requires handshakes or cooperation between sender and receiver
\item Performance.  Low latency for short messages.
\end{enumerate}
These requirements suggest that there be a single data structure that holds
the progress of all communication. 
In particular, to correctly support the first (nonblocking) requirement, there
must be a queue (a queue, not a list or a heap, because of the third
(ordering) requirement) of 
pending data transfer operations.  
Because of the second (contiguous) requirement, combined with the need to
handle general (possibly noncontiguous) MPI datatypes, we must be able to
transfer data in parts (segments).  This requires having the option of
invoking a routine when a data transfer completes (the callback described
above); further, it requires that the queue element not be automatically
removed from the data transfer queue when the current communication completes
because more data for this communication may be on the way.

Because of the fourth (handshake) requirement, once communication has been
initiated, the same data structure should be used with any communication that
requires a handshake, rather than generating a new data structure for each
individual communication operation.  

The data structure that satisfies these requirements is the
\code{MPID_Request}.  Further, once communication is started (e.g., with
\code{CH3_iSend}), further data transfers are accomplished by either placing
the request into the queue of pending data transfers or by updating a request
that is already the active request (rather than creating a new request).

To handle the communication of data, the CH3 device needs the following fields
in the \code{MPID_Request}:
\textbf{These need to be organized in a hierarchical way.  Partially done.}
\begin{description}
\item[match]Matching information: rank, tag, and context.
\item[Rendezvous send information]
  \begin{description}
  \item[user\_buf.]Pointer to the user buffer.
  \item[user\_count.]Number of 
  \item[datatype.]Datatype describing the user buffer.
  \end{description}
\item[Information for processing data buffers]
  \begin{description}
  \item[segment.]Segment used to process noncontiguous data.
  \item[segment\_size.]Size of the segment (in bytes).
  \item[segment\_first.]Current offset into the segment.
  \end{description}
\item[vc.]A pointer to the virtual connection being used to satisfy
the request.
\item[iov.]I/O vector describing the buffer to be sent or received.
\item[iov\_count.]Number of entries in the iov.
\item[ca.]Completion action.  This indicates what operation should be
performed when a data transfer is complete.
\item[tmpbuf.]Pointer to a temporary buffer
\item[tmpbuf\_sz.]Size of the temporary buffer (not necessarily the
same as the size of the occupying the buffer).  The \code{SRBuf} flag in the
state field indicates if this temporary buffer is a buffer from the
send/receiver buffer pool or an unexpected eager
\item[recv\_data\_sz.]Size of the message data.
\item[sender\_req\_id.]Handle of sender's request associated with this request.
\item[state]A series of bit fields describing the state of the request.
\item[next]A pointer to the next request in the send or receive queue.
\end{description}

Thus, a \code{ch3} struct is included in the \code{MPID_Request} structure.
\begin{verbatim}
typedef struct
{ 
    MPIDI_Message_match match;
    void * user_buf;
    int user_count;
    MPI_Datatype datatype;
    MPID_Segment segment;
    MPIDI_msg_sz_t segment_size;
    MPIDI_msg_sz_t segment_first;
    MPID_VC *connection;
    MPID_IOV iov;
    int iov_count;
    MPIDI_CA_t  ca;
    void * tmpbuf;
    MPIDI_msg_sz_t tmpbuf_sz;
    MPIDI_msg_sz_t recv_data_sz;
    MPI_Request sender_req_id;
    unsigned state;
    struct MPID_Request * next;
} ch3;
\end{verbatim}

The TCP channel also requires fields in the \code{MPID_Request} to
properly handle the communication of data:
\begin{description}
\item[iov\_offset.]Current element in the I/O vector.
\item[pkt.]Space for buffering packet headers associated with the request.
\end{description}

Like the CH3 device, the TCP channel adds a structure of its own,
namely the \code{tcp} struct, to the \code{MPID_Request} structure.
\begin{verbatim}
typdef struct
{
    int iov_offset;
    MPIDI_CH3_Pkt_t pkt;
} tcp;
\end{verbatim}

\subsection{Remote Memory Operations in the CH3 Design}
This interface includes a remote put operation \code{CH3_Put}.
By having a CH3 routine that can perform a contiguous put to remote memory, we
make it easy to experiment with RMA-capable networking (such as VIA or
Infiniband) without 
requiring a completely new device implementation.  However, this is an
optional routine (at least while we target only MPI-1) and need not be
implemented.  The packet handler descriptions given show how a put
operation can be implemented, even in a TCP/sockets environment.

In order to support MPI-2 remote memory operations, additional
routines are necessary. 

All of the routines in this section except those defined under the
asynchronous operations section are optional.  The channel implementation
will define CPP variables of the same name as the routine if an
implementation of that routine is available.

\subsubsection{Active target routines}

The following three routines may be used to perform active target
communication.  They may not be used for passive target communication as
they are not require to complete asynchronously.

\begin{verbatim}
int MPIDI_CH3_iPutc(MPIDI_VC * vc, const void * buf, int count,
                    MPID_RAint raddr, MPID_RAint rdaddr)

int MPIDI_CH3_iGetc(MPIDI_VC * vc, void * buf, int count,
                    MPID_RAint raddr, MPID_RAint rdaddr)

int MPIDI_CH3_iAccc(MPIDI_VC * vc, const void * buf, int count, MPI_Op op,
                    MPID_RAint rbaddr, MPID_RAint rdaddr)
\end{verbatim}.
where the parameters have the following meanings:
\begin{description}
\item[vc]the virtual connection over which the operation takes place

\item[buf]pointer to the local buffer

\item[count]number of bytes (both local and remote)

\item[rbaddr]address of remote buffer (may be 0 if count is zero)

\item[rdaddr]address of a remote integer counter to be decremented when
operation completes (may be 0)

\item[op] predefined \code{MPI_Reduce} operations (including
  \code{MPI_REPLACE}) 
\end{description}

Note: the original specification of \code{MPID_iPut()} in the CH3/TCP design
document called for the use of \code{MPID_RAint}.  I (Brian) continued
to use it here, but 
am not convinced it is necessary.  We said during the design meeting that
these routines would only be used on homogeneous systems.  If that is true,
\code{MPID_RAint} can be replaced with \code{MPI_Aint}.


\subsubsection{Passive target atomic operations}

The following routines are intended for passive target communication.  If
implemented, they must complete asynchronously (independent of remote process
activities).

\begin{verbatim}
int MPIDI_CH3_iPTPutc(MPID_Win * winp,  MPIDI_VC * vc,
                      MPIDI_CH3_Win_lock_type_t type,
                      MPIDI_CH3_Win_lock_state_t state,
                      const void * buf, int count,
                      MPID_RAint rbaddr, MPID_RAint rdaddr)

int MPIDI_CH3_iPTGetc(MPID_Win * winp,  MPIDI_VC * vc,
                      MPIDI_CH3_Win_lock_type_t type,
                      MPIDI_CH3_Win_lock_state_t state,
                      void * buf, int count,
                      MPID_RAint rbaddr, MPID_RAint rdaddr)

int MPIDI_CH3_iPTAccc(MPID_Win * winp,  MPIDI_VC * vc,
                      MPIDI_CH3_Win_lock_type_t type,
                      MPIDI_CH3_Win_lock_state_t state,
                      const void * buf, int count, MPI_Op op,
                      MPID_RAint rbaddr, MPID_RAint rdaddr)
\end{verbatim}
where the parameters have the following meanings:
\begin{description}
\item[winp]pointer to the window object (necessary for some implementations to
access data structures which manage exclusive access to the window).
Question: Whose window (local or remote)?  Is this a pointer or a handle or an
\code{MPID_RAint}? 

\item[vc]the virtual connection over which the operation takes place

\item[type]\code{MPIDI_CH3_WIN_LOCK_EXCLUSIVE} or
  \code{MPIDI_CH3_WIN_LOCK_SHARED}.
Question: Why not use MPI types (\code{MPI_LOCK_EXCLUSIVE} and
\code{MPI_LOCK_SHARED})? 

\item[state]\code{MPIDI_CH3_WIN_LOCK_START},
  \code{MPIDI_CH3_WIN_LOCK_CONTINUE}, or \code{MPIDI_CH3_WIN_LOCK_STOP}
Question: rather than emphasizing ``lock'', how about
\code{MPIDI_CH3_WIN_ACCESS_BEGIN}, \code{MPIDI_CH3_WIN_ACCESS_CONTINUE}, and
\code{MPIDI_CH3_WIN_ACCESS_END}.

\item[buf]pointer to the local buffer

\item[count]number of bytes (both local and remote)

\item[rbaddr]address of remote buffer (may be 0 if count is zero)

\item[rdaddr]address of a remote integer counter to be decremented when
operation completes (may be 0)

\item[op]predefined \code{MPI_Reduce} operations (including \code{MPI_REPLACE})
\end{description}


\subsubsection{Local passive target locks}

The following routines are used to control access to the local window.  If
any of the passive target atomic operations are defined, then these
functions must be defined.

\begin{verbatim}
int MPIDI_CH3_Win_local_lock(MPID_Win * winp, MPIDI_CH3_Win_lock_type_t
type)

int MPIDI_CH3_Win_local_unlock(MPID_Win * winp)
\end{verbatim}
where the parameters have the following meanings:
\begin{description}
\item[winp]pointer to the window object (necessary for some implementations to
access data structures which manage exclusive access to the window)

\item[type]\code{MPIDI_CH3_WIN_LOCK_EXCLUSIVE} or
  \code{MPIDI_CH3_WIN_LOCK_SHARED} 
Question: Why not use MPI types (\code{MPI_LOCK_EXCLUSIVE} and
\code{MPI_LOCK_SHARED})? 
\end{description}

\subsubsection{Passive target memory allocation}

The following routine allocates memory upon which the passive target
operations can be performed.  If any of the passive target atomic operations
are defined, then this function must be defined.

\begin{verbatim}
int MPIDI_CH3_alloc_mem(MPID_Win * winp)
\end{verbatim}

We also require the corresponding free routine, along with a routine to test
whether memory was allocated with this routine (for error detection).  The
ADI3 spec has these routines; I propose that CH3 simply implement them.

\subsubsection{Asynchronous operations}

The following routine sends an operation request to be performed
asynchronously by a remote process.  This routine is similar to
\code{MPIDI_CH3_iStartmsgv()} except that the operation must be
handled by a remote 
asynchronous agent.  If the send does not complete immediately, this routine
returns a request object.  When the asynchronous agent completes the send,
it calls \code{MPIDI_CH3U_Handle_async_send()} with \code{sreq->ch3.ca} set to
\code{MPIDI_CH3_CA_COMPLETE}, causing the completion counter
\code{(*sreq->cc_ptr)} to be decremented.

\begin{verbatim}
int MPIDI_CH3_arov(MPIDI_VC * vc, MPID_IOV * iov, int iov_n,
                   MPID_Request **sreqp)
\end{verbatim}
where the parameters have the following meanings:
\begin{description}
\item[vc]the virtual connection over which to send the request

\item[iov]an I/O vector the first element of which contains a packet header
(\code{MPIDI_CH3_Pkt_t}) defining the operation to be performed; if
  the send cannot 
be completed immediately, both the packet header and the I/O vector will be
copied internally, allowing both of them to be allocated on the stack.

\item[iov\_n]number of entries in the I/O vector

\item[sreqp]pointer to a send request object pointer (may be NULL if the send
completed locally before the routine returns)
\end{description}

The following routine is the same as \code{MPIDI_CH3_arov()} except it takes an
existing request allowing remote operation requests that exceed the maximum
number of I/O vector entries.  When the data described by I/O vector is
completely sent (or buffered), \code{MPIDI_CH3U_Handle_async_send()}
will be called 
by the asynchronous agent.  \code{sreq->ch3.ca} should contain a
completion action 
so that \code{MPIDI_CH3U_Handle_async_send()} can act appropriately.

\begin{verbatim}
int MPIDI_CH3_alrov(MPIDI_VC * vc,  MPID_Request * sreq,
                    MPID_IOV * iov, int iov_n)
\end{verbatim}
where the parameters have the following meanings:
\begin{description}
\item[sreq]pointer to an existing send request object
\end{description}
(other parameters the same as \code{MPIDI_CH3_arov()}).

The following routine is used to continue sending data as part of a
previously started remote operation request.  The buffers to be sent are
defined by \code{sreq->ch3.iov} and \code{sreq->ch3.iov_count}.
\code{MPIDI_CH3U_Handle_async_send()} will be called by the
asynchronous agent when 
the send is complete (or buffered).  \code{sreq->ch3.ca} should contain a
completion action so that \code{MPIDI_CH3U_Handle_async_send()} can act
appropriately.

\begin{verbatim}
int MPIDI_CH3_awrite(MPIDI_VC * vc, MPID_Request * sreq)
\end{verbatim}
where the parameters have the following meanings:
\begin{description}
\item[vc]the virtual connection over which to continue sending data

\item[sreq]pointer to the send request object
\end{description}

The following routine requests that data continue to be read by the
asynchronous agent and be placed in the buffers defined by
\code{rreq->ch3.iov} and 
\code{rreq->ch3.iov_count}.  When the data described by I/O vector is
completely 
read, \code{MPIDI_CH3U_Handle_async_recv()} will be called by the asynchronous
agent.  \code{rreq->ch3.ca} should contain a completion action so that
\code{MPIDI_CH3U_Handle_async_recv()} can act appropriately.

\begin{verbatim}
int MPIDI_CH3_aread(MPIDI_VC * vc, MPID_Request * rreq)
\end{verbatim}
where the parameters have the following meanings:
\begin{description}
\item[vc]the virtual connection over which to read data

\item[rreq]pointer to the receive request object
\end{description}

The following routine is called by the asynchronous agent (in the channel
implementation) when an asynchronous operation request arrives.  If data
follows the packet header, a receive request object must be created before
the remainder of the data can be read using \code{MPIDI_CH3_aread()}.

\begin{verbatim}
int MPIDI_CH3U_Handle_async_pkt(MPIDI_VC * vc, MPIDI_CH3_Pkt_t * pkt)
\end{verbatim}
where the parameters have the following meanings:
\begin{description}
\item[vc]the virtual connection over which to send the request

\item[pkt]the packet header
\end{description}


The following routine is called by the asynchronous agent when a previously
requested read (using \code{MPIDI_CH3_aread()}) has completed.

\begin{verbatim}
int MPIDI_CH3U_Handle_async_recv(MPIDI_VC * vc, MPID_Request * rreq)
\end{verbatim}
where the parameters have the following meanings:
\begin{description}
\item[vc]the virtual connection over which to read data

\item[rreq]pointer to the receive request object
\end{description}

The following routine is called by the asynchronous agent when a previously
requested write (using \code{MPIDI_CH3_awrite()}) has completed.

\begin{verbatim}
int MPIDI_CH3U_Handle_async_recv(MPIDI_VC * vc, MPID_Request * sreq)
\end{verbatim}
where the parameters have the following meanings:
\begin{description}
\item[vc]the virtual connection over which to read data

\item[sreq]pointer to the send request object
\end{description}

\subsection{Thread Safety}

% Who provides the queue manipulation routines (e.g., FOA)?  Is this part of the
% base set of CH3 routines, or is it an MPID routine?  Or is it
% something in the middle?  

Thread safe manipulation of the message queues requires careful
handling.  Section~\ref{sec:msg-queues} discusses the message queue
manipulation routines in depth; Section~\ref{sec:utility-routines}
sketches some of the routines that are used for the message-queue
operations.  In short, these routines performed atomic queue update
operations, such as ``find or insert'' rather
than separate ``find'' and ``insert'' operations.

%% Thread-safety of FOA (currently, inserted request must be marked as unready
%% unless the other receive parameters are passed to FOA so that the request can
%% be atomically created).  One possibility is to combine this with the request
%% state and have a state update function that ensures that any modifications are
%% written to memory (e.g., using a write barrier) before another thread might
%% access the request.

\subsection{Data Structures}
To allow the greatest efficiency, most of the data structures are
visible to all layers (below the user-layer).  However, parts of the
data structures may be defined by and used exclusively by a particular
layer.  For example, the \code{MPID_Request} contains both fields used
by the MPI implementation layer (e.g., the completion counter) and
fields used only by the channel implementation layer.  This is
essentially a subclassing approach, but implemented directly in C.

The assignment of data structures to layers is as follows.
\begin{description}
\item[MPI application layer.](User programs) Owns and allocates
  \code{MPI_Status}. 
\item[MPI implementation layer.](E.g., implementation of \code{MPI_Isend} in
  terms of \code{MPID_Isend}.) Owns and allocates communicators, datatypes,
  attributes, groups, files, window objects, keyvals, and error handlers.
\item[MPID Channel device layer.](E.g., implementation of
  \code{MPID_Isend} in terms of CH3.) Owns and allocates packets.
  Defines packet type handlers.  Owns and allocates segments.
\item[CH3 implementation layer.] Owns and allocates connections and requests.
\end{description}

% \begin{verbatim}
% Notes from the board:

%   The MPI application layer

%                                         owns/allocates MPI_Status

% ___________ MPI_Isend, MPI_IRecv, MPI_Wait _______________________

%   The MPI Layer                         owns/allocates
%   (aka MPIR layer)                          communicators
%   (mpich2.tex)                              datatypes
%                                             attributes
%                                             groups

% ____goals.tex____ MPID_Isend, MPID_Irecv, Progress routines ______

%                                         owns/allocates

%                                             packet-type handlers
%                                             data structures connecting
%                                             fds to handlers, which call
%                                             TCP layer, below 
%   packets


% ____tcpadi3.tex____CH3_Writev, CH3_Startmsgv ____________________


%      poll                                owns/allocates
%      readv                                     MPIDI_VC connections
%      writev                                    fd readahead buffer
%      connect                                connection state
%      ...                                    requests
%                                             segments (in requests)

% \end{verbatim}

\subsection{Utility Routines}
\label{sec:utility-routines}
(not written yet)
This section should describe the queue operation routines, such as the
find-or-post or find-or-allocate (FOA) operations.



\begin{description}
\item[CH3U\_Request\_FDU\_or\_AEP.]Find a request in the unexpected
message queue and dequeue it; if one is not found, create a request
and add it to the posted receive queue.
\item[CH3U\_Request\_FDP\_or\_AEU.]Like \code{CH3U\_Request\_FDU\_or\_AEP},
but first checks the posted receive queue, and if not found, adds to
the unexpected message queue.  This routine is called by a message
handler while the previous routine is called by one of the MPI receive
routines.  
\item[CH3U\_Request\_FU.]Find a matching request in the unexpected
queue.  Return a pointer to the request or NULL if a matching request
was not found.  Note: this routine does not remove the request from
the unexpected queue.
\item[CH3U\_Request\_FDU.]Given the sender's request handle and
matching information, find a matching request in the unexpected queue.
Return a pointer to the request or NULL if a matching request was not
found.
\item[CH3U\_Request\_DP.]Given a pointer to the request structure,
dequeue the request from the posted message queue.  Return true if the
request was dequeued or false if the request was not found.
\item[CH3U\_Request\_FDP.]Given message match information, find a
matching request in the posted queue.  If found, dequeue it and return
a pointer to the request; otherwise return NULL.
\item[CH3U\_Request\_create.]Initialize the \code{ch3} structure in a
MPID\_Request.  This routine should be called by
\code{CH3_Request_create}.
\item[CH3U\_Request\_destroy.]Destroy any objects attached to the
\code{ch3} structure in a \code{MPID_Request}.  This routine should be
called by \code{CH3_Request_destroy}.
\item[CH3U\_Request\_decrement\_cc.]Atomically decrement the
completion counter in the request.  If the counter reaches zero,
return zero; otherwise return a non-zero value.
\item[CH3U\_Request\_complete.]This is a convenience routine.  It
calls \code{CH3U_Request_decrement_cc} followed by
\code{MPID_Request_release} and \code{CH3_Progress_signal_completion}
if all operations associated with the request are complete.
\item[CH3U\_Request\_load\_send\_iov.](Re)load the I/O vector in a
send request.
\item[CH3U\_Request\_load\_recv\_iov.](Re)load the I/O vector in a
receive request.
\item[CH3U\_Request\_unpack\_uebuf.]Unpack data for a unexpected eager
message in the user's message buffer.
\item[CH3U\_Request\_unpack\_srbuf.]Unpack data in a send/receive
buffer into the user's message buffer.
\item[CH3U\_Buffer\_copy.]Copy the contents of a user's send buffer
into a user's receive buffer.
\item[CH3U\_SRBuf\_alloc.]Allocate a send/receive buffer and set
\code{req->ch3.tmpbuf} to point to the buffer.  Send/receive buffers are
temporary buffers used to hold a portion of the message data when
user's buffer is non-contiguous.  The desired size of the buffer is
specified when calling this routine; however the actual buffer size
may be different.  \code{req->ch3.tmpbuf} will contain the actual size
of the buffer.
\item[CH3U\_SRBuf\_free.]Free a previously allocated send/receiver buffer.
% \item[CH3U\_Request\_change\_state.]Change the state of a request.
% This is used, for example, to change a request from ``allocated by not
% yet initialized'' to ``ready to match incoming message''.
% \item[CH3U\_AllocateStorageFromEagerBuffer.]Allocate buffer space for
% saving an unexpected eager message.  This routine coordinates with the
% flow control code.
\end{description}

Bindings:
\begin{verbatim}
    request = CH3U_Request_FDU_or_AEP( source, tag, context_id, &found )
    request = CH3U_Request_FDP_or_AEU( msg_match, &found )
    request = CH3U_Request_FU( source, tag, context_id )
    request = CH3U_Request_FDU( handle, msg_match )
    flag = CH3U_Request_DP( reqptr )
    request = CH3U_Request_FDP( msg_match )
    CH3U_Request_create( reqptr )
    CH3U_Request_destroy( reqptr )
    CH3U_Request_decrement_cc( reqptr, &count)
    CH3U_Request_complete( reqptr )
    mpi_errno = CH3U_Request_load_send_iov( reqptr, iov, iov_n )
    mpi_errno = CH3U_Request_load_recv_iov( reqptr )
    mpi_errno = CH3U_Request_unpack_uebuf( reqptr )
    mpi_errno = CH3U_Request_unpack_srbuf( reqptr )
    CH3U_Buffer_copy( sbuf, scount, sdatatype, smpi_errno,
                      rbuf, rcount, rdatatype, &data_sz, rmpi_errno )
    CH3U_SRBuf_alloc( reqptr, size )
    CH3U_SRBuf_free( reqptr )
\end{verbatim}
% \begin{verbatim}
%     void CH3U_Request_change_state( MPID_Request * req, state )
%     void *CH3U_AllocateStorageFromEagerBuffer( int )
% \end{verbatim}

\subsection{NonContiguous Datatypes}
The CH3 interface supports the direct communication of either single
contiguous blocks of data or data represented by a standard iovec
structure.  However, some of the most common noncontiguous datatypes
encountered in MPI programs are not efficiently represented by an
iovec structure; these include both strided and block-indexed types.
While the ADI-3 interface allows the device to directly handle all
datatypes, the CH3 interface must pack and unpack datatypes that are
noncontiguous.  To allow arbitrarily large messages to be sent with
the CH3 device, packing and unpacking may be done incrementally.  The
routines such as \code{CH3\_iWrite} and \code{CH3\_iRead} are used for
transfering these incrementally packed buffers.

\section{Pseudo-code for some of the
  \texorpdfstring{\texttt{MPID\_}}{MPID} routines} 
\label{sec:pseudo-code}
This section outlines the pseudocode for some of the major MPID
routines.
Note that many of these operations only begin a communication.  The
completion of the communication often takes place with the
communication agent.  The routines that are implemented within the
communication agent, which are called message handlers, are described
in Section~\ref{sec:agent}.

\subsection{Sending}
\label{sec:pseudocode-sending}
The code for \code{MPID_Isend} and \code{MPID_Send} is shown below.  The code
for the other 
MPID send routines is similar, with the appropriate choice of eager
(for rsend) or rendezvous (for ssend) operations.  
% Blocking and
% nonblocking versions differ only in whether a request is returned if
% the message is complete.  
The envelopes for ready-send messages should include
a flag that indicates that they are ready-send so that the user-error of an
unmatched ready-send can be detected and reported.

Question: is this code up-to-date?  Don't we send first and allocate
the packet only if we need to?
\begin{verbatim}
MPID_Isend( )
{
    decide if eager based on message size and flow control
    if (eager) {
        create packet on stack
        fill in as eager send packet
        request = CH3_request_create()
        if (data contiguous)
            CH3_iSendv( request, iov )
        else {
            create pack buffer
            pack into buffer
            CH3_iSendv( request, iov )
            if (complete)
                free pack buffer
            else
                save location of pack buffer in request
            }
        }
    else (rendezvous) {
        create packet on stack
        request = CH3_request_create();
        fill in request
        fill in packet as rndv_req to send (include request id)
        CH3_iSend( request, packet )
    } 
    return request
}
\end{verbatim}

An alternative to creating packets on the stack is to allow the CH3 layer to
provide a way to create a new packet. The semantics would allow simple
allocation as above, but would also allow the CH3 layer to provide specially
allocated memory.  For the near term, however, we will not include this
enhancement. 

Another alternative is a ``fast send'' for short messages.  This would
allow an implementation to optimize for the low-latency case; for
example, a shared-memory implementation could use special allocation
routines to reduce the number of operations.  This may be a better
approach than trying to find a general model that achieves the
lowest-possible latency.

\code{MPID_Send} is slightly different because we want to avoid allocating a
request if possible.  

\begin{verbatim}
MPID_Send( )
{
    decide if eager based on message size and flow control
    if (eager) {
        create packet on stack
        fill in as eager send packet
        if (data contiguous) {
            request = CH3_iStartmsgv( iov )
            // note that the request will be NULL if the message was sent
            }
        else {
            create pack buffer
            pack into buffer
            request = CH3_iStartmsgv( iov )
            if (!request)
                free pack buffer
            else
                save location of pack buffer in request
          }
    else (rendezvous) {
        .. exactly like MPID_Isend
    } 
    return request
}
\end{verbatim}

\subsection{Receiving}
Both \code{MPID_Irecv} and \code{MPID_Recv} use similar code.

\begin{verbatim}
MPID_Irecv( )
{
    CH3_Progress_poke
    request = MPIDI_CH3U_Request_FPOAU( source, tag, context_id, &found )
    if (found)  {
        /* Message was found in unexpected list.  Eager data is stored
           in the request */
        if (eager) {
            copy data
            free eager buffer used for data
            mark request completed
            }
        else {
            # rendezvous
            create packet on stack
            fill in as rndv_clr_to_send
            CH3_iSend( request, packet )
            }
    }
    else {
        fill in request
        CH3U_Request_change_state( request, waiting for match )
    }
}
\end{verbatim}

Note that this code cannot avoid the allocation of a request, even in
the case of \code{MPID_Recv} and where the data is already available in the
socket, since a request must be returned to the user.  To
optimize for low latency in the case of a small, contiguous transfer,
we may want to have a version of \code{MPID_Recv} that looks something
like
\begin{verbatim}
    if (datatype is contiguous and small &&
        receive queue for this tag/context/source is empty &&
        no active receive request) {
        Try to read next packet
        if (packet read) {
            if (packet type is eager &&
                MPI envelope matches this receive) {
                transfer data to destination
                if (transfer complete) return # NULL request since done.
                else {
                    create request, make active
                    return request
                }
            }
            else {
                dispatch packet (e.g., same code as in progress engine)
            }
        } 
    }
    /* fall through in case we didn't receive the message */
    MPID_Irecv( ... )
\end{verbatim}
An advantage of this is that it avoids both the need for allocating a
request and it avoids calling the Progress routine (note that we must
still ensure that the progress routine is called sufficiently often).
The somewhat complicated tests are necessary to ensure that correct message
ordering is preserved. Note that the test ``receive queue for this tag
etc.'' need not be perfect in that false negatives (queue may be
nonempty) are allowed, since this only drops the code into the
\code{MPID_Irecv} case.  For example, a simple test to see if the
queue is empty is sufficient.

This example also serves to illustrate why the MPID interface includes
blocking receive; there is a potentially important optimization that
is otherwise not available without a blocking receive.  Also note that
this routine cannot be implemented strictly 
in terms of the \code{CH3_xxx} routines, because the ``try to read next
packet'' step requires access to the underlying message buffers.  
Question: we could provide a routine to do this (see the progress engine
discussion under ``Completion''); should we?

\subsection{Completion}
The ADI does not provide completion routines that correspond directly
to the MPI completion routines (e.g., \code{MPI_Test}).  Instead,
there are routines to make progress on communication.  To test whether
a request is complete, the \mpids{MPID_Request}{busy} flag is checked.
The implementation of the \code{CH3_Progress} routine is shown with the other
\code{CH3} routines in Section~\ref{sec:progress}.

% \begin{verbatim}

% Unresolved issue:  request queue changes/transitions in detail
%   create
%   move
%   update
% \end{verbatim}

\subsection{Persistent Requests}
MPI persistent requests allow the MPI implementation to setup the data
structures necessary for communication in advance of initiating the
communication.  This section briefly sketches these routines and their
datastructures. 

\begin{verbatim}
MPID_Send_init()
    act_request = CH3_request_create();
    request     = CH3_request_create();
    request->act_request = act_request;
    request->send_packet = MPIU_Malloc( packet );
    setup fields in request, act_request, send_packet, 
        including a free-handler for the request
    Ensure that the virtual connection is initialized
    return request   
\end{verbatim}
Rather than use \code{MPIU_Malloc}, we may want a \code{CH3} routine that can
return a packet that may not be allocated on the stack (e.g., a persistent
packet).  

The reason that a packet is allocated here rather than off the stack
is that this allows us to fill in the packet once during the
\code{MPID\_Send\_init} step, rather than during each
\code{MPID\_Start} step.  Note that \code{MPID\_Request\_free} must
free this packet and the \code{act\_request}.

% \begin{verbatim}
% MPID_Startall()
%     for each request {
%        if (send) {
%            Execute logic from MPID_Isend, MPID_Irsend, etc., using
%            request attached to persistent request and the preallocated 
%            and initialized packet. Note that in the Send case, the 
%            eager/rendezvous case cannot be precomputed, because eager
%            depends both on the message size and on the MPI flow control.
%        }
%        else if (recv) {
%            Execute logic from MPID_Irecv
%        }
%        else if (user) {
%            Ignore for now.  
%        }
%        else {
%            error
%        }
%     }
% \end{verbatim}
% One possible implementation for \code{MPID_Startall} is to invoke a
% \code{start} routine stored in the persistent request, rather than using a 
% switch on the persistent request type.  If we did that, we could consider 
% eliminating \code{MPID_Startall} from the MPID interface and make the
% \code{MPI_Startall} code invoke the \code{start} routines directly.  This has
% the advantage of taking the user-requests out of the device code.  The only
% disadvantages are eliminating an opportunity for the device to schedule the
% order in which requests are started and the costs (on some systems) of calling
% a function through a function pointer.

Persistent requests are initiated by invoking the start function in
the request.  The implementation of \code{MPI_Start} and
\code{MPI_Startall} calls the \code{start_fn} in each request.  This
provides a common approach for both user-defined requests (called
generalized requests in MPI-2) and point-to-point persistent
communication requests.

\section{Pseudo-code for Message Handlers}
\label{sec:handlers}
\label{sec:agent}
These are the routines that are called by the progress engine on receiving a
message packet.  All of these assume that the entire packet header has been 
read but that any following data may not yet have been read.  

Question: The Unix socket interface supports a ``low watermark''
setting that guarantees that at least that many bytes are available
when \code{select} or \code{poll} returns.  Should we use this in the
code?  What are the performance implications?

\subsection{EagerSend}
Action invoked by the receiver of an eagerly sent message.  The data for the
message is immediately behind the message header (eagersend packet).
\begin{verbatim}
    request = MPIDI_CH3U_Request_FPOAU( &packet->msg_match, &found )
    if (found) {
        /* Message was already posted */
        CH3_iRead( request )
    }
    else {
        /* Message is unexpected */
#       ifdef HAVE_ERROR_CHECKING
        if (message is readysend) {
             signal error (and return a message to sender)
             arrange to read and discard data 
             (simply allow the read as below; set the state to discard
             when the transfer is complete)
        }
#       endif
        request->active_buf = CH3U_AllocateStorageFromEagerBuffer( len )
        request->active_buf_len = len
        CH3_iRead( request )
    }
\end{verbatim}
This routine uses \code{CH3\_iRead} to read the data that follows the
message header into the location previously saved in the request (when
the request was posted).  We don't pass the buffer location to
\code{CH3\_iRead} because (a) the location is already present in the
request and (b) passing it as an argument to the routine unnecessarily
adds to function call overhead.


\subsection{RndvReqToSend}
Action invoked by the reciever of a rendezvous message.
\begin{verbatim}
    request = MPIDI_CH3U_Request_FPOAU( &packet->msg_match, &found )
    if (found) {
        if (dest buffer is contiguous)
            create rndv-ok-to-put packet on stack
            fill in packet
            CH3_iSend( request, packet )
        else
            create rndv-ok-to-send packet on stack
            CH3_iSend( request, packet )
    }
#ifdef HAVE_ERROR_CHECKING
    else if (ready-send) {
        return an error message to sender (error msg packet)
        tmp_request = CH3_iStartMsg( packet );
        if (tmp_request) tmp_request->ref_count--;
        MPIDI_CH3U_DeQueue_unexp( request )
    }
#endif
\end{verbatim}
Question: it may be possible to reuse the incoming packet as the
outgoing packet, thus saving some stores to the packet structure.

\subsection{RndvClrToSend}
Action invoked by the sender of a rendezvous message on receipt of an
acknowledgement from the receiver.
\begin{verbatim}
    Convert request id (in packet) to pointer to request structure
    create RndvData packet on stack
    iov[0].ptr = address of packet
    iov[0].len = sizeof(packet)
    if (data contig) {
        iov[1].ptr = data_address
        iov[1].len = data_len
        CH3_iSendv( request, iov, 2 )
    }
    else {
        Create pack buffer
        Pack first segment worth
        iov[1].ptr = address of pack buffer
        iov[1].len = lenght of packed data
        request->state = segment_sending
        CH3_iSendv( request, iov, 2 )
    }
\end{verbatim}

\subsection{Put}
Action invoked by reciever of a put packet.  
\begin{verbatim}
    Read Address from packet
    request = CH3_iStartRead( address, count )
    if (request) {
        Save flag address in request
        set request state so that on completion of data transfer, flag is
            decremented
    }
    else {
        decrement flag (address provided by packet)
    }
\end{verbatim}
Note: We do need a request for the put operation to handle incomplete
data transfers; by setting the 
request's reference count, we can ensure that the request is recovered once
the data transfer completes.  However, in the case where a put is used to
provide the data for a rendezvous receive, there is already an available
request.  Question: do we want a form of put that takes advantage of having an
existing request?  In that case, instead of the remote flag address, the
remote request id can be used.

Note that if the read completes, the returned request is NULL.

\subsection{RndvData}
Action invoked by the receiver of data sent in response to an ok to send after
a rendezvous message.
\begin{verbatim}
    Convert request id (in packet) to pointer to request structure
    /* memory location already set as active buf in request */
    CH3_iRead( request )
\end{verbatim}

\subsection{CancelSend}
Action invoked by the receiver of a request to cancel a previously sent
RndvReqToSend. 
\begin{verbatim}
    Convert request id (in packet) to pointer to request structure
    If (request is in unexpected message queue)
        if (already matched) {
            create CancelSendAck(failed) on stack
            newrequest = CH3_iStartmsg( packet )
        }
        else {
            Remove and discard request
            create CancelSendAck(succeeded) on stack
            newrequest = CH3_iStartmsg( packet )
        }
    else {
        create CancelSendAck(failed) on stack
        newrequest = CH3_iStartmsg( packet )
    }
    if (newrequest) newrequest->ref_count--;
\end{verbatim}
Note that the above code must access the message queues atomically in
the multi-threaded case in order to preserve correctness.  Question:
should the queue access part of this be an CH3U utility routine?

\subsection{CancelSendAck}
Action invoked by the receiver of a request that acknowleges a
CancelSend request.

\begin{verbatim}
    Convert request id (in packet) to pointer to request structure
    Set request to indicate whether cancel succeeded
    If succeeded, remove from pending send list
\end{verbatim}
As for the CancelSend message, this must act atomically on the message
queue.

\subsection{FlowControlUpdate}

Flow control is used at the MPI level to control the use of eager buffers and
requests for unexpected messages.  Possible choices for flow control include
IMPI-style control
(\url{http://impi.nist.gov/impi-report/impi-report-node54.html}) or an
integrated count of the number of envelopes and 
buffer space used (IMPI only counts ``packets'').  Flow control is \emph{not}
optional, though the early 
implementation can ignore this.

One possibility is to include flow control updates on all packets; this
ensures that in typical ``balanced'' communication, a separate flow control
packet is never needed.  

The original MPICH-1 flow control counted envelopes and data
separately, encoding the number of each consumed since the last
communication within a single 32-bit integer field in each message
packet.  

(need to add more details on flow control.)  
The biggest issue is the
handling of eager buffer space.  The problem is fragmentation; you
can't simply count the number of bytes.  One possibility is to
allocate space in a buddy system and count the number allocated in
each buddy pool (each pool contains identically sized blocks).  Note
that because the maximum size of an eager message is limited, we don't
really need to worry about a message being too large for a single
fragment in the eager buffer bool.

\section{Message Queues}
\label{sec:msg-queues}
This section describes the routines for handling the message queues.  For
thread safety, the queues of posted receive requests and of unexpected
messages are accessed atomically rather than through separate routines.

In addition, the match condition is based on data that is passed in the packet
and is stored in the type \code{MPID_Message_match}:
\begin{verbatim}
typedef struct {
    int32_t tag;
    int16_t source;
    int16_t context_id;
} MPID_Message_match;
\end{verbatim}
To exploit longer-word instructions where available, this is really a
union:
\begin{verbatim}
typedef union { 
    struct { 
    int32_t tag;
    int16_t source;
    int16_t context_id;
    } match;
    int32_t match32[2];
#ifdef HAVE_INT64_T
    int64_t match64;
#endif
} MPID_Message_match;
\end{verbatim}

On systems with \code{int64_t}, the match test is a single line of C code; if
the hardware has 64-bit integer operations, it is a single compare
instruction.  Otherwise, the code shown here must be expanded to test
two 32-bit fields.  

\begin{verbatim}
MPIDI_CH3U_Request_FPOAU( int source, int tag, int context_id, int *found )
   /* Look in unexpected queue for a match */
   MPID_Message_match m.match = {tag, source, context_id};
   if (tag != MPI_ANY_TAG && source != MPI_ANY_SOURCE) {
       for (r = unexpected_queue->head; r; r = r->next ) {
           if (r->match.match64 == m.match64) {
               remove r from unexpected queue
               found = 1
               return r
           }
       }
   }
   else {
       /* Must do any match */
       MPID_Message_match mask.match = {0xffffffff, 0xffff, 0xffff};
       if (tag == MPI_ANY_TAG) { mask.match.tag = 0; m.match.tag = 0; }
       if (source == MPI_ANY_SOURCE) { 
           mask.match.source = 0; m.match.source = 0; }
       /* A further optimization for no wildcards would skip the mask
          step */
       for (r = unexpected_queue->head; r; r = r->next ) {
           if ((r->match.match64 & mask.match64) == m.match64) {
               remove r from unexpected queue
               found = 1
               return r
           }
       }
   }
   found = 0;
   /* If we get here, the message was not found */
   request = CH3_request_create()
   request->match.match = { tag, source, context_id };
   request->mask.match  = { like the above };
   /* Add to tail of posted recieves */
   posted_receive->tail->next = request;
   posted_receive->tail       = request;
   return request
\end{verbatim}
The above code shows an example of optimizing for code that does not
use wildcards for the tag or source.  

The next routine is always called in response to receiving a message packet,
and hence simply passes the part of the packet that contains the tag, context,
and source values directly to the routine.  It is essentially the same as 
\code{MPIDI_CH3U_Request_FPOAU}, except that it checks first in the posted
receive queue and, if the data is not found, adds to the unexpected receive
queue.  The wildcard match testing is slightly different as well.
\begin{verbatim}
MPIDI_CH3U_Request_FUOAP( MPID_Message_match *, int *found )
   /* Look in posted queue for a match */
   for (r = posted_queue->head; r; r = r->next ) {
       if (r->match == (m & r->mask)) {
           remove r from posted queue
           found = 1
           return r
       }
   }
   found = 0;
   /* If we get here, the message was not found */
   request = CH3_request_create()
   request->match = { tag, source, context_id };
   /* Add to tail of unexpected recieves */
   unexp_receive->tail->next = request;
   unexp_receive->tail       = request;
   return request
\end{verbatim}

Cancel operations require that requests be removed from the queues.

This routine is required for cancelling sends.
\begin{verbatim}
MPIDI_CH3U_DeQueue_unexp( int handle )
    for (r = unexpected->head; r; r = r->next) {
        if (r->handle == handle) {
             remove r from list and return it
        }
    return NULL
\end{verbatim}
(Recall that the integer id of the request is called the \code{handle}.)

This routine is required for cancelling receives
\begin{verbatim}
MPIDI_CH3U_DeQueue_posted( int handle )
    for (r = posted->head; r; r = r->next) {
        if (r->handle == handle) {
             remove r from list and return it
        }
    return NULL
\end{verbatim}
The above are appropriate for rare operations such as cancel that can 
afford to search through potentially long lists and can even use locks
to guarantee exclusive access to the data structures.  Note that the
description of the cancel routines do not use these routines but they should.

\subsection{Debugger Interface}
(still to do; this section will describe how to support the message
queue operations that are part of the debugger interface.  Note that
the debugger interface provides for both a receive and send queue
interface; the debugger can get a list of the pending send and receive
operations, along with the unexpected messages.

\section{Implementing \texorpdfstring{\texttt{mpiexec}}{mpiexec}}
\label{sec:mpiexec}

(This is a temporary spot for these remarks, since they may apply to more than
just the TCP device.)

There are multiple possible implementations for \code{mpiexec} and each has
its own advantages and disadvantages.  We might implement all of them, but we
should implement the quickest-to-implement first.

\begin{description}
\item[As MPD console program.] This is ready-to-go as a minor change to
  \file{mpdcon.c}, except that the PMI interface might need to be updated to
  match 
  the current specification.  A version of \code{MPI_Info} is needed for the
  new 
  \code{PMI_Spawn}, but not for anything else, so MPI-1 routines should be OK.
  I.e, 
  the database part of PMI is already running.  All handling of \code{stdio}
  is done.
\item[As a PMI program.] This requires the above plus implementation of
  \code{PMI_Spawn}, at least for use by console.  It could also be built to
  interact 
  with a scheduler.  This (using \code{PMI_Spawn} to start the initial
  processes as 
  well as for the implementation of \code{MPI_Spawn}) was the ``original''
  plan.
\item[As an MPI program.] This is the idea in the current MPICH2 document.  It
  relies on \code{MPI_Connect}, etc.  It requires the above plus the MM
  component 
  of the PMI interface, to set up the connections.  This approach as the
  advantage of providing a ``universal'' \code{mpiexec}. 
\item[As an ``immediate scheduler''.] This makes \code{mpiexec} into a stand-in
  for the scheduler component of the Scalable System Software Project.  It is
  much like the ``MPD console'' option, but instead of using the existing
  console code to contact a local MPD, it sends the standard XML defined by
  the SSS project to the MPD, which is standing in for an arbitrary process
  startup component.  It requires hooking in an XML parser like \code{xpat}
  into the MPD and having \code{mpiexec} emit XML code.
\item[As a ``one-host-only'' process starter.] The \code{mpiexec} process could
  simply fork the application processes.  This requires a new but simple
  implementation of the put/get/fence part of the PMI interface.  The original
  \code{mpiexec} process could become the database server part after forking.
\end{description}

The first and last options seem to present the shortest paths to getting
something running that we can use to debug the coming avalanche of code.

Any of these need to contain the argument-processing code for the defined
standard arguments to \code{mpiexec}.  These are defined in Volume 1 of \emph{ MPI---The Complete Reference}, starting on page 353.  There are multiple
approaches to dealing with arguments.
\begin{description}
\item[Plain.] Use straightforward code as in \file{p4_args.c}.
\item[Fancy.] Use an ``options database'' approach, as in PETSc.
\end{description}

We will want to do both, but the first option can be implemented
immediately, especially if we postpone some of the more elaborate argument
lists and require that those be used with a file.  We have to define the
format of the file for use with the \code{-file} option.  There are three
possibilities.
\begin{description}
\item[Keyword=value pairs.] This is easy to read, and we can use the
  parsing routines from MPD, so we are practically already done.
\item[XML.] We could match the process-startup file to the format of a
  process-startup request as being defined by the Scalable Systems Software
  Project.  This would be sort of cool.  Validating XML parsers in C exist.
\item[Custom Format.] We could define our own formats, so that we could express
  anything whatsoever.  We could use multiple formats to match other software
  that we might find it useful to be compatible with, such as schedulers and
  other process managers.
\end{description}

Again, we might want to implement all three of these, since each has
advantages.  The quickest option is the first.  However, we could easily
implement an XML-style version of keyword/value pairs using a format such as
\begin{verbatim}
<MPICH keyword=value />
\end{verbatim}
and have both the first and second choices at the same time.

% \section{Summary}
% \label{sec:tcpadi-summary}
% (This section should summarize the \code{CH3_} routines, giving just their
% prototypes) 

\appendix
\section{CH3 Routines and Data Structures}
\label{app:ch3}

This section provides pseudocode for a TCP implementation of the CH3
routines.

General note: in if-else code, the most likely case should be placed first.
This is both faster and moves the most common code branch to the top, where it
makes it easier to grasp the intent of the code.

\subsection{Data Structures}
There are two primary data structures: one for virtual connections and one for
the device as a whole...\emph{needs revision}...

\paragraph{Process Specific Information (CH3 Device).}
\begin{verbatim}
typedef struct {
    MPID_Request * recv_posted_head;      /* List of posted receives */
    MPID_Request * recv_posted_tail;
    MPID_Request * recv_unexpected_head;  /* List of unexpected receives */
    MPID_Request * recv_unexpected_tail;
} MPIDI_Process_t;
\end{verbatim}
The receive lists are held on the device to simplify handling of wildcard
receives.  

\paragraph{Process Specific Information (TCP Channel).}
\begin{verbatim}
typedef struct {
    MPIDI_CH3I_Progress_group_t * pg;     /* My process group */
} MPIDI_CH3I_Process_t;
\end{verbatim}

\paragraph{Process groups.}
\begin{verbatim}
typedef struct {
    volatile int ref_count;
    char * kvs_name;                      /* name of the PMI keyval space */
    int size;                             /* number of processes */
    struct MPIDI_VC * vc_table;           /* table of virtual connections
                                             (one per process) */
} MPIDI_CH3I_Process_group_t;
\end{verbatim}

\paragraph{Connections.}
\begin{verbatim}
typedef struct { 
    int ref_count;                        /* number of comunicators using this
                                             connection */
    int lpid;                             /* local process ID for the partner
                                             of this connection (used to
                                             implement group routines)*/
    MPIDI_CH3_VC_DECL                     /* channel fields */
} MPIDI_VC;
\end{verbatim}

\begin{verbatim}
#define MPIDI_CH3_VC_DECL
typedef struct {
    MPIDI_CH3I_Process_group_t pg;        /* process group containing remote
                                             process */
    int pg_rank;                          /* rank in the process group */
    struct MPID_Request * sending_head,   /* Queue of pending sends */
    struct MPID_Request * sending_tail;
    MPIDI_CH3I_VC_state_t state;          /* TCP connection state */
    int poll_elem;                        /* element in the poll array */
    int fd;                               /* fd for the socket (cached) */
} tcp;
\end{verbatim}
The pending sends have a head and a tail pointer because it is a queue and we
want the operations of insert and delete on this queue to be fast.

\subsection{Requests}
\code{CH3_Request_create} returns a new request allocated from the
request pool.  The initial implementation uses the \code{MPIU_Handle}
module to manage this pool.  As a result, the \code{id} field in the
request structure automatically contains an integer handle for the
request object that may be used as the \code{MPI_Request} handle.
\begin{verbatim}
CH3_Request_create() 
{
    request = MPIU_Handle_obj_alloc(&MPID_Request_mem);
    request->ref_count = 1;
    return request;
}
\end{verbatim}
This routine requires that \code{MPIU_Handle_obj_alloc} be thread safe in a
multithreaded environment, preferably without using locks.

Question: What should this routine do if it cannot allocate a new
request?  Is that a fatal error?  What about the fault-tolerant case?
Should the code look something like
\begin{verbatim}
CH3_request_create()
{
    request = MPIU_Handle_obj_alloc(&MPID_Request_mem);
    if (!request) {
       int limit = 200;
       do { 
          CH3_Progress( 0 );
          request = MPIU_Handle_obj_alloc(&MPID_Request_mem);
       }
       while (!request && limit--) {
       if (!request) { 
           MPID_Abort( ) // Give up
    }
    request->ref_count = 1;
    return request;
}    
\end{verbatim}

\code{CH3_Request_destroy} releases the resources used by an existing request
back to the request pool.  
\begin{verbatim}
CH3_Request_destroy( MPID_Request * request ) 
{
    request->ref_count -= 1;
    if (request->ref_count == 0)
    { 
        MPIU_Handle_obj_free(&MPID_Request_mem, request);
    }
}
\end{verbatim}
This routine requires that \code{MPIU_Handle_obj_free} be thread safe in a
multithreaded environment, preferably without using locks.

\subsection{iStartMsg}
\code{CH3_iStartMsg} is called to send a message.  It returns a
request if the message is 
not completely sent.

\begin{verbatim}
MPID_Request *CH3_iStartMsg( MPID_VC *, void *header, int header_count )
   if there is no active request on this VC
      write the header.
      if the entire header is written, return NULL
      else /* see note below */
          create a request
          fill it in with the remaining data to write
          make this the active request
          return request               
   else
      create a request
      fill it in with the remaining data to write
      add to the pending send queue
      return request
\end{verbatim}
One possible variation is to eliminate the ``else'' branch labeled ``see note
below'' and let the code fall through into else branch that creates and
inserts the request into the pending send queue.

\begin{verbatim}
CH3_iStartMsgv( MPID_VC *, struct iovec *iov, int count )
Like CH3_iStartMsg, but for all of the data in the iov
\end{verbatim}

\subsection{iStartRead}
This is used only in the handler for a put packet type.  It reads data and
generates a new request only if not all of the requested data was
read.
It must be implemented \emph{only} if \code{CH3_Put} is implemented.

\begin{verbatim}
CH3_iStartRead( MPID_VC *, void *buf, int count )
   Try to read count bytes to buf
   if (all read) return 0;
   /* otherwise */
   create a new request
   request->active_buf = (char *)buf + bytes read
   request->count_left = count - bytes read
   return request
\end{verbatim}

\subsection{iSend}
iSend it called to send data using an existing request.  However, the data to
send is passed in separately (it is not part of the request).
\begin{verbatim}
CH3_iSend( MPID_VC *, MPID_Request *, void *buf, void *count )
    if (!vc->sending_head) { /* no active send requests */
        try to send data.
        if (all sent) {
            switch (request->comm_state)
                0: request->cc--; 
                   if (request->ref_count == 0) recover request
               >0: (request->update)(request) /* effectively "invoke 
                    MPIDI_CH3_Request_update( request )" */
         }
    else {
        make active request
        }
    }
    else {
        request->active_buf = buf; request->count_left = count;
        add request to top of pending send queue
    }
\end{verbatim}
The member \code{cc} of the request is the ``completion counter''.

Question: Can we combine the reference count and the completion count,
so that only a single flag must be tested?

\code{CH3_iSendv} is like \code{CH3_iSend}, except a \code{struct iovec} is
used instead of a \code{buf,count} pair.  The special action on a
\code{comm_state} of zero is used to avoid the overhead of a function
call in the common case (for relatively short, contiguous messages)
that no further work needs to be done.

\subsection{iRead}
iRead is used to transfer data into an already created request.  

\begin{verbatim}
CH3_iRead( MPID_VC *, MPID_Request *request )
    read upto request->count_left from fd associated with this VC
        to request->active_buf
    if (all read) {
        switch (request->state) 
            0: decrement request->cc
               remove from active list
               if (request->ref_count == 0) place request on avail list
            >0: invoke MPIDI_CH3_request_update( request )
    }
\end{verbatim}

\subsection{iWrite}
iWrite is called to continue a data transfer.  This is used by the
segment processing code to handle the incremental packing and sending
of noncontiguous datatypes.

\begin{verbatim}
CH3_iWrite( MPID_VC *, MPID_Request * )
    if request is not the active request on this VC, internal error
    try to write on fd associated with MPID_VC.  
    record amount of data written in request.
    If all data written
        switch(request->state)
            0: decrement request->cc and remove this request from
               the active list.  If ref_count is 0, return request to avail
            >0: invoke MPIDI_CH3_Request_update( request )
    return
\end{verbatim}
The request update routine handles incremental packing; typically, it
uses \code{MPIR_Segment_pack} to pack the next segment of bytes and
then calls \code{CH3_iWrite} to continue writing the data (note that
the request is left on the active list in case some other thread want
to start sending data on this connection).

IMPLEMENTATION NOTE: The final case may result in deep recursion for
large messages.  This problem must be solved eventually.  For now, we
simply verify that the call stack does not get too deep and hope for
the best.

\subsection{Progress}
\label{sec:progress}
The progress routine may be implemented on single-threaded systems as
\begin{verbatim}
CH3_Progress( int isblocking )
{
    select/poll on all fd's (wait is blocking, test is not)
    For each fd, find associated MPID_VC. (perhaps using fd_to_vc[] array)
    if (write) {
        // fd was set because data is waiting
        do {
            send data described by active request in MPID_VC.  
            If (all data sent)
                invoke request_state_method.
                if no active send requests, clear need-to-write
            } while (can write and pending writes)
    }
    if (read) {
        do {
            switch on state
                case reading pkt hdr: 
                    switch on packet headr
                        call fcn associated with pkt type
                        (These are MPID functions, not TCP functions, 
                         since pkts are in MPID layer.  Call them
                         MPIDI_CH3U_xxx) 
                    end switch
                case reading data to known address:
                     read more data.
                     if (complete)
                         invoke request_state_method
                case establishing new connection:
                     ... // this is the hard part, of course
            end switch
        } while (can read)
    }
}
\end{verbatim}
To make this more efficient, maintain the data structures needed with the
\code{poll} or \code{select} calls with the active fd's: the ones on which
there are either pending writes or on which connections for reading have been
established.  Such a structure might include
\begin{verbatim}
    typedef struct { 
       ... whatever poll needs ...
       int writes_pending;
       ...
    } MPIDI_CH3_Progress_data_t;
\end{verbatim}

The above version of \code{CH3_Progress} is for single-threaded
implementations.  Multi-threaded versions must keep track of whether any
completions occurred after \code{CH3_Progress_start} was called.

An alternative to having the \code{CH3} layer contain the progress engine is
to define some additional read and write routines at the \code{CH3} level and
then implement the progress routine in terms of those calls.  If there were
such routines, they could also be used to implement \code{MPID_Recv}.  

The routines \code{CH3_Progress_start} and \code{CH3_Progress_end} can be
no-ops in the single-threaded version.  The routine \code{CH3_Progress_poke}
can be \code{CH3_Progress( 0 )} (i.e., nonblocking call to progress),
and can be defined as a macro to avoid the function call overhead.

On multi-threaded systems, the implementation of the progress routines
may use any of the usual techniques, including condition variables.

\subsection{Startup and Rundown}
\begin{verbatim}
CH3_Init( int *has_args, int *has_env, int *has_parent )
\end{verbatim}
This routine must
\begin{itemize}
\item Make any calls to PMI
\item Set the size, rank, and \code{MPID_VC} fields in
  \mpids{MPIR\_Process}{comm\_world} and \mpids{MPIR\_Process}{comm\_self}

\item Set the return values appropriately.  \code{has_parent} should be set if
  there a parent communicator that must be initialized.
\end{itemize}

\begin{verbatim}
CH3_Finalize(void)
\end{verbatim}
This routine should free any memory and other resources.  

\begin{verbatim}
CH3_InitParent( MPID_Comm *parent )
\end{verbatim}
Initialize the \code{size}, \code{rank}, and \code{MPID_VC} fields in
the communicator \mpids{MPIR\_Process}{comm\_parent}. 
\code{CH3_InitParent} is called only if \code{CH3_Init} returned true in
\code{has_parent}.  This allows the implementation of \code{MPI_Init} to
create the parent communicator only if it is needed, and then initialize it in
a separate step.

\subsection{RMA}
\textbf{This section may be obsolete}
\begin{verbatim}
CH3_iPut( MPID_VC *, void *buf, int count, MPID_RAint offset, 
          MPID_RAint cmpl_flag )
\end{verbatim}
The remote addresses (\code{MPID_RAint}) cannot be of type \code{MPI_Aint}
because \code{MPI_Aint} 
is the size of an address on the calling system, not necessarily on
the target system.  

\section{Other Comments}
One optimization that is mentioned in the MPICH-2 coding document
\cite{mpich2} in the discussion of \code{MPI_Sendrecv} may be valuable
to the TCP implementation.  This involves piggy-backing rendezvous
acknowledgements onto data transfers where possible.  For the case
when sendrecv is used to exchange data, this can reduce the number of
separate communication steps.  The design given here provides no way
to make this optimization.  Question: do we want to allow an
extension?  What would it cost other cases?  Can we do it from within
\code{MPI_Sendrecv} and/or \code{MPI_Waitall} in such a way that we
don't penalize other communication patterns?  What is the potential
benefit?  Should we just concentrate on RMA instead?

An alternative model for the receiving end is a
\code{MPID_CH3_ReadProcess} call that reads a packet header and
processes it.  This is really just an active message or remote routine
handler, and is roughly the receiving end of the original ADI3
\code{MPID_Rhcv} (remote handler call - vector arguments) routine.
This has the advantage that fewer routines are defined, and in some
sense, the code must be structured like this (since arbitrary incoming
packets may arrive).  

\end{document}
