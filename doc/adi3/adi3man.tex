% 
%   This is a latex file that generates a reference manual for 
%   ADI-3 
%
\documentclass{article}
\usepackage{/home/gropp/data/share/refman}
\usepackage{/home/gropp/tex/inputs_local/tpage}
\textheight=9in
\textwidth=6.1in
\oddsidemargin=.2in
\topmargin=-.50in

\begin{document}

\markright{ADI-3 Reference Manual}

\def\nopound{\catcode`\#=13}
{\nopound\gdef#{{\tt \char`\#}}}
\catcode`\_=13
\def_{{\tt \char`\_}}
\catcode`\_=11
%\def\code#1{{\tt #1}}

% \ANLTitle{MPICH Model MPI Implementation\\Reference Manual\\\ \\Draft}{\em 
% William Gropp\\
% Ewing Lusk
% Mathematics and Computer Science Division\\
% Argonne National Laboratory
% }{00}{\today}

\clearpage

\pagenumbering{roman}
\tableofcontents
\clearpage

\pagenumbering{arabic}
\pagestyle{headings}

\section{Introduction}
This document contains detailed documentation on the routines that are part of
the Abstract Device Interface, version 3, used to implement the MPICH2000
model MPI implementation. 

% As an alternate to this manual, the reader should consider using the
% script \code{mpiman}; this is a script that uses \code{xman} to provide
% a X11 Window System interface to the data in this manual.

\section{ADI-3 Routines}
\input adi3func.tex

\section{ADI-3 Datastructures}
\input adi3data.tex

\end{document}









