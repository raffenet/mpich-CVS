\subsection{Steps Needed in Implementing a Device}
Here are the steps that must be taken to implement a new device.
Similar steps are required when adding to a device, such as creating a
new channel-based device.

\begin{enumerate}
\item A new device is created by adding a new directory
\file{src/mpid/<newdevice>}.  Into this directory should go all of the
files (and subdirectories) needed for the device.  These include

\begin{description}
\item[\file{setup_device}:]This script, if present, is run
\emph{before} any configure in the device directory.  It can setup any
files or communicate any data that is needed by the device.

\item[\file{configure}:]This should be a standard \code{autoconf}
script (at least, it must accept the standard \code{configure}
arguments.  All device-specific tests, such as those for header files
or libraries, should be made by the \code{configure} in the device
directory.  No changes of any kind should be made to the top-level
\code{configure} (the one in the \file{mpich2} directory).

\item[\file{localdefs}:]This file, if present, is sourced by the
top-level \code{configure} after the device's \code{configure} is run.
It should be used to communicate any  
variable values to the top-level configure.  A common use it to add
any libraries that may be required.  For example, a file
\file{localdefs.in} may be used that contains
\begin{verbatim}
LIBS="$LIBS @EXTRA_LIBS@"
\end{verbatim}
The \code{configure} in the device directory includes \file{localdefs}
in the \code{AC_OUTPUT} list, allowing the device's \code{configure} step to
create the \file{localdefs} file.
\end{description}

\item Include files.  
%% Any include files that are needed by
%% \file{mpiimpl.h} (used to build the implementation of the MPI
%% routines) should be \emph{copied} into \file{mpich/src/include} as
%% part of the device's configure step.  
Any include files that are needed by \file{mpiimpl.h} should be make
available by including the directory path in the \code{CPPFLAGS}
variable, set in the \file{setup_device} file mentioned above.  Make
sure that you append to this variable, as in 
\begin{verbatim}
    CPPFLAGS="$CPPFLAGS -I$use_top_srcdir/src/mpid/mydevice"
\end{verbatim}
Note the use of the variable \code{use_top_srcdir}; the top-level
\code{configure} sets this to the absolute path to the top-level
source directory.

Provide the files
\file{mpidpre.h} and \file{mpidpost.h}.  The implementation of all MPI
routines include files in this order:

\begin{description}
\item[mpi.h]---The standard \file{mpi.h} that all MPI users include
\item[mpidpre.h]---Any definitions needed \emph{before} the provided
definitions of the contents of the internal structures.  This can
included definitions that override parts of \file{mpiimpl.h}
\item[contents of mpiimpl.h]---The bulk of the internal definitions.
This also includes information on the timers.
\item[mpidpost.h]---Any definitions needed by the device after the rest
of the definitions in \file{mpiimpl.h}.  In many cases, this file may
be empty.
\end{description}

\item Testing codes for device-specific functions.  Place these
in a \file{test} subdirectory of the device. These tests 
should be performed through a \code{test} target in the device's
\file{Makefile.sm}. 


\item \code{mpiexec}.  Decide whether you need a special
  \code{mpiexec} program.  Many devices will be able to use any of the
\code{mpiexec} programs in the various \file{src/pm} directories, such
as \file{src/pm/mpd} and \file{src/pm/forker}.  (``pm'' is for process
manager, and MPICH2 expects to install some process manager for all
devices.)
If you need a new
\code{mpiexec}, it should be added to a new subdirectory within
\file{src/pm}.  There must be an installation target in the
\file{Makefile.sm} for any new \code{mpiexec}.  For example, if the
\code{mpiexec} program requires the multi-purpose demon
  (\code{mpd}), ensure that the \code{mpd} is installed.  For example,
  in the \file{src/pm/<my-process-manager>/Makefile.sm}, use the line
\begin{verbatim}
install_BIN = mpd
\end{verbatim}
to install the program \code{mpd} into the \file{bin} directory.  The
particular process manager to use is selected at configure time using
the \code{--with-pm} option.  Most devices should be able to use one
of the process managers provided with MPICH2.

\item Device-specific documentation, such as environment variables and
  command-line arguments used only by a particular device.  Place this
  information into the file \file{src/mpid/<yourdevice>devdoc.txt}.
  The \mpich\ 
  documentation generators will look for this file.


All of these are included by the file \file{mpiimpl.h}.
\end{enumerate}

\subsection{Directory Structure}
\label{sec:adi3-dirs}
A device should be placed in a subdirectory of \file{mpich/src/mpid/};
for example, \file{mpich/src/mpid/mm} is the multi-method ADI
delivered with \mpich.  The directory name is the same as the device
name specified to the \mpich\ \code{configure} with the
\code{--with-device} option.

\subsection{Device Configuration and Setup}
\label{sec:adi3-setup}
Each device must have a \code{configure} script.  This will be run by
the \mpich\ \code{configure} as part of the top-level configuration.
Any other commands that a device needs for setup should be run using
the \code{AC_OUTPUT_COMMANDS} \code{autoconf} macro.  Autoconf version
2.13 or later, but before 2.50, should be used; we recommend using the
macros defined in 
\file{mpich/confdb/aclocal.m4}, as they include fixes to
\code{autoconf}\footnote{There are serious bugs in the
\texttt{AC_CHECK_HEADER} macro that are still present in
\texttt{autoconf} 2.52, and version 2.57 is not backward compatible
with earlier versions of \texttt{autoconf}, including 2.52.}.
Do not modify the \mpich\ \code{configure} to support a device.

There must be a \code{echomaxprocname} target in the \file{Makefile}
in the device's directory.  This should look something like
\begin{verbatim}
echomaxprocname:
       @echo 128
\end{verbatim}
This value will be used as the value for
\code{MPI_MAX_PROCESSOR_NAME}, and must be an integer value.
(The above will be replaced by an option to provide this value through
the \file{localdefs} files.)
