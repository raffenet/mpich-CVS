% 
%   This is a latex file that generates a reference manual for the 
%   Model MPI implementation
%
\documentclass{article}
\usepackage[dvipdfm]{hyperref}
\usepackage{refman}
\usepackage{tpage}
\textheight=9in
\textwidth=6.1in
\oddsidemargin=.2in
\topmargin=-.50in

% Include each routine in the contents page
\mancontentstrue

\begin{document}

\markright{MPICH2 Reference Manual}


\def\nopound{\catcode`\#=13}
{\nopound\gdef#{{\tt \char`\#}}}
%\catcode`\_=13
%\def_{{\tt \char`\_}}
\catcode`\_=11
\def\code#1{{\tt #1}}

\ANLTitle{MPICH2 Model MPI Implementation\\Reference Manual\\\ \\Draft}{\em 
William Gropp\\
Ewing Lusk
Mathematics and Computer Science Division\\
Argonne National Laboratory}{00}{\today}

\clearpage

\pagenumbering{roman}
\tableofcontents
\clearpage

\pagenumbering{arabic}
\pagestyle{headings}

\section{Introduction}
This document contains detailed documentation on the routines that are part of
the MPICH model MPI implementation.

As an alternate to this manual, the reader should consider using the
script \code{mpiman}; this is a script that uses \code{xman} to provide
a X11 Window System interface to the data in this manual.

\section{MPI Commands}
\input refcmd.tex

\section{MPI routines}
\input refmpi.tex

%\section{MPE routines}
%\input ref4/ref.tex

%\section{ADI routines}
%\input ref5/ref.tex

\end{document}
