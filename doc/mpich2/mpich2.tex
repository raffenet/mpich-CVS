%
% Design document for MPICH2
%
% This document should be used with the ADI3 document
\documentclass{article}
\usepackage{psfig}
\usepackage{/home/gropp/data/share/refman}
\usepackage{/home/gropp/bin/share/tex/fileinclude}
\usepackage{/home/gropp/sowing-proj/sowing/docs/doctext/tpage}
\usepackage{/home/gropp/BOOKS/Beowulf/tex/url}
\usepackage{epsf}
\usepackage{mpidoc}

\makeindex

% temporary mods;  some version should go into mpidoc - RL
\setlength{\parindent}{.5in}
\setlength{\parskip}{.1in}
% end temporary mods

\begin{document}

\markright{MPICH Design Document}


%\tpageoneskip
\ANLTMTitle{MPICH2 Design Document\\
Draft of \today}{\em 
David Ashton\\
William Gropp\\
Ewing Lusk\\
Rob Ross\\
Brian Toonen\\
Mathematics and Computer Science Division\\
Argonne National Laboratory}{00}{\today}

\clearpage

\pagenumbering{roman}
\tableofcontents
\clearpage


\raggedright
% raggedright resets parindent
\parindent 1em
\pagenumbering{arabic}
\pagestyle{headings}

\section{Introduction}
This document discusses how the MPICH2 implementation is written using
the ADI-3 \cite{adi3man} for the supporting functions.  This document
contains guidelines for the MPICH2 implementation.  One important
purpose of this document is to provide common guidelines for writing
the MPICH code.

See also the Coding Standards docment \cite{coding-standards}.

To date, this document contains primarily comments on the rules for
writing the code.  Few comments on the use of ADI-3 routines have been
added yet.  No part of this document is final.

A major challenge is developing an interface that requires fewer (or
at least simpler) routines to implement.  This is particularly
difficult since the MPI standard is defined to encourage efficient
implementations.  While it is possible to meet the functional
definitions of MPI with fewer routines, achieving performance requires
something relatively close to what MPI defines.

One possibility is to consider a few classes of systems.  Pure
distributed memory is one important case.  Another is shared memory,
or at least some common shared memory.  Of course, multi-method
devices make this more difficult.  However, to be concrete, this
approach of is taken here; the details are described in
Section~\label{sec-mpi-operations}.

% Outline:
\section{General}
This section contains general recommendations are requirements for
writing the code.

\subsection{Error reporting}

MPI routines should check all possible error conditions early, before
calling any ADI routines.  The ADI routines perform little error
checking.  To allow error handling to be enabled or disabled both at compile
time 
and at runtime, the tests should be placed within the following block:
\begin{verbatim}
#ifdef HAVE_ERROR_CHECKING
BEGIN_ERROR_CHECKS
...
END_ERROR_CHECKS
#endif
\end{verbatim}
The macros \code{BEGIN_ERROR_CHECKS} and \code{END_ERROR_CHECKS} can
expand, depending on configuration settings, into either null (i.e.,
no runtime control) or 
\begin{verbatim}
#define BEGIN_ERROR_CHECK if (MPIR_Perform_Error_Tests) {
#define END_ERROR_CHECKS }
\end{verbatim}

There must be a configure option,
\cfgoption{--disable-error-checking},  
that prevents \code{HAVE_ERROR_CHECKING} from being defined.

\subsubsection{Errors to test for}
Most values should be tested to ensure that they are in-range.  For examples,
tags must be nonnegative for sending (and nonnegative or
\mpiconst{MPI_ANY_TAG} for receiving). 

Question: Code in ROMIO often includes the following test on pointers:
\begin{verbatim}
if (ptr < (Ptr_type) 0) error
\end{verbatim}
This helps catch some common invalid pointers on many systems, but isn't
correct for some other systems.  Should there be an optional
pointer-validation test?  For example, the function
\mpidfunc{MPID_Test_pointer} would return true if the pointer was valid and
false otherwise (possibly testing only for reading).  This could do anything
from test against null to changing the \code{SIGSEGV} handler, attempting to
read from the address, resetting the handler, and determining if the handler
was invoked.  It could return an error code if it finds a problem.
Note that this won't work in the multi-threaded case (unless a
thread-specific signal handler is available).

Question [BRT]: Are there systems where a pointer can be less
than zero?  I have always (perhaps incorrectly) considered pointers to
be unsigned, so the above code fragment seems bizzare to me.  

Question [BRT]: What is the purpose of performing these tests?  If the
purpose is to help us, the MPICH developers, find bugs, then I suggest
that we avoid adding these types of tests and instead make use of a
product like Insure++.  If the purpose is to keep the user's code from
core dumping, then such tests might be useful.  However, I highly
recommend that the user be able to turn off such checks as a core dump
frequently provides useful information about the source of the
problem.
Answer [WDG]: I believe that the intent was to catch user errors.  As the bug
reports have shown, users often assume that any code except theirs is at
fault, so the library code by default should be defensive.  

\subsubsection{Choosing Error Handlers and Classes}
Error handlers should be chosen using the following decision tree:
\begin{enumerate}
\item If still in the initialization step (within
\mpifunc{MPI_Init_thread}), errors return.  Question, should this be
errors abort?

\item Check if executing inside a layered routine (i.e., an MPI routine called
  within the implementation of another MPI routine).  If so, return error
  code; do not invoke the error handler.  See
  Section~\ref{sec:err-handling-nested}. 

\item If the routine has a valid \code{MPI_Comm}, \code{MPI_File}, or
  \code{MPI_Win}, use the error handler from that object.  (Question: do we
  have a common way to extract the error handler?  It isn't part of the
  communicator/file/win structures yet.)

\item If the error relates to a request, and the request refers to a valid
  communicator, use that communicator's error handler (e.g., \code{MPI_Wait}).

\item Otherwise, for everything except MPI-IO, use the error handler attached
  to \mpiconst{MPI_COMM_WORLD}  
  (see Section 7.2 in the MPI-1 Standard: ``MPI calls that are not related to
  any communicator are considered to be attached to the communicator
  \code{MPI_COMM_WORLD}.'' ).

\item For MPI-IO, the default error handler is attached to
  \code{MPI_FILE_NULL} (see Section 9.7 in the MPI-2 standard: ``The default
  file error handler can be changed by specifying \code{MPI_FILE_NULL} as the
  \code{fh} argument to \mpifunc{MPI_FILE_SET_ERRHANDLER}'').  Note that this
  requires \mpifunc{MPI_FILE_SET_ERRHANDLER} and
  \mpifunc{MPI_FILE_GET_ERRHANDLER} to handle this special case for
  \mpiconst{MPI_FILE_NULL}.  We also need either a full
\mpiconst{MPI_File} object associated with \mpiconst{MPI_FILE_NULL} or
the various associated properties.  It might be eaiser for the
implementation of \mpifunc{PMPI_File_set_errhandler} and
\mpifunc{PMPI_File_get_errhandler} to be in the same file and share a
static variable (see Section~\ref{sec:pmpi-routines}) such as
\mpidconst{MPIi_File_errhandler}. 
\end{enumerate}

%
% This file contains the error classes/codes section so that it can be
% separately printed.
\subsubsection{Error Classes and Codes}
The MPI standard defines a number of error classes and permits an
implementation to make use of additional error codes, with the proviso that
any error code belong to some error class (though this does include the
\mpiconst{MPI_ERR_OTHER} class).  
The specification of MPI error classes is rather uneven.  There are separate
classes for most of the arguments to the point-to-point communication
functions and for many of the I/O errors.  Other routines have to make due
with \mpiconst{MPI_ERR_ARG} or \mpiconst{MPI_ERR_OTHER}.

In addition, many of the errors have common subcases.  For example, most of
the errors that refer to an MPI opaque handle can indicate either a null or a
non-null but invalid handle.   To handle all of these cases, we predefine an
extended set of error codes.  Only the error classes are defined in
\file{mpi.h}; the others are defined in \file{mpiimpl.h} (Question: is this
the correct place, or should there be a separate file?).  The additional error
codes all have the form of \code{MPIi_ERR_<class>_<subclass>}.  For example,
the code for a null communicator is \mpiconst{MPIi_ERR_COMM_NULL}.

Comment [BRT]: I find that MPIi_ is painful to type.  Is there a
reason the second `i' is lowercase?

Many of these descriptions list the optional arguments.  These can be provided
(in the order and with the types specified) to the call that creates an error
code (see \mpidfunc{MPID_Err_create_code}). 

\begin{description}
\item[\mpiconst{MPI_ERR_BUFFER}]Invalid buffer pointer
    \begin{description}
    \item[\mpidconst{MPIi_ERR_BUFFER_NULL}]Null buffer pointer
    \item[\mpidconst{MPIi_ERR_BUFFER_NOSPACE}]Insufficient space in Bsend
      buffer (optional args: requested and avaliable length (int))
    \item[\mpidconst{MPIi_ERR_BUFFER_ALIAS}]Buffers must not be aliased
      (optional args: names of two arguments (string))
    \item[\mpidconst{MPIi_ERR_BUFFER_SIZE}]Invalid buffer size (optional arg:
      size (int))
    \item[\mpidconst{MPIi_ERR_BUFFER_BSEND_EXISTS}]Buffer already attached with
      \mpifunc{MPI_BUFFER_ATTACH}. 
    \item[\mpidconst{MPIi_ERR_BUFFER_BSEND_SMALL}]Buffer size is smaller than
      \mpiconst{MPI_BSEND_OVERHEAD} (optional argument: size, value of
      \code{MPI_BSEND_OVERHEAD} (int)) 
    \item[\mpidconst{MPIi_ERR_BUFFER_BSEND_NONE}]No buffer to detach.
    \end{description}
\item[\mpiconst{MPI_ERR_COUNT}]Invalid count (optional argument value (int))
    \begin{description}
    \item[\mpidconst{MPIi_ERR_COUNT_ARRAY}]Invalid count in count array
      (optional arguments: index and value (int))
    \end{description}
\item[\mpiconst{MPI_ERR_TYPE}]Invalid datatype
    \begin{description}
    \item[\mpidconst{MPIi_ERR_TYPE_NULL}]Null datatype
    \item[\mpidconst{MPIi_ERR_TYPE_ARRAY_NULL}]Null datatype in array of
      datatypes (optional arguments: name of argument (string) and index (int))
    \item[\mpidconst{MPIi_ERR_TYPE_NOT_COMMITTED}]Datatype has not been
      committed 
    \item[\mpidconst{MPIi_ERR_TYPE_FREE_PERM}]Cannot free permanent data type
      (optional argument: name (string))
    \item[\mpidconst{MPIi_ERR_TYPE_PERM_CONTENTS]}]Cannot get contents of a
      permanent or basic data type (optional argument: name (string))
    \item[\mpidconst{MPIi_ERR_TYPE_NAME}]Cannot set name in data type
    \item[\mpidconst{MPIi_ERR_TYPE_NOMATCH}]Type signatures do not
    match in communication (see \cite{gro:mpi-datatypes:pvmmpi00})
    \item[\mpidconst{MPIi_ERR_TYPE_WRONG_COMM}]Pack buffer not packed
    for this communicator.  
\cite{})
    \end{description}
\item[\mpiconst{MPI_ERR_TAG}]Invalid tag (optional argument: value (int) )
%    \begin{description}
%    \item[\mpidconst{MPIi_ERR_}]
%    \end{description}
\item[\mpiconst{MPI_ERR_COMM}]Invalid communicator
    \begin{description}
    \item[\mpidconst{MPIi_ERR_COMM_NULL}]Null communicator
    \item[\mpidconst{MPIi_ERR_COMM_INTER}]Intercommunicator is not allowed
    \item[\mpidconst{MPIi_ERR_COMM_INTRA}]Intracommunicator is not allowed
    \item[\mpidconst{MPIi_ERR_COMM_NAME}]Cannot set name in communicator
    \item[\mpidconst{MPIi_ERR_COMM_PEER}]Peer communicator is not valid
    \item[\mpidconst{MPIi_ERR_COMM_LOCAL_NULL}]Local communicator must not be
      \mpiconst{MPI_COMM_NULL}
    \end{description}
    Note that while C++ defines separate Cartesian and Graph
    communicators, errors involving improper choice of those is under
    \mpiconst{MPI_ERR_TOPOLOGY}. 
\item[\mpiconst{MPI_ERR_RANK}]Invalid rank (optional argument: value (int))
    \begin{description}
    \item[\mpidconst{MPIi_ERR_RANK_ARRAY}]Invalid rank in rank array (optional
      arguments: index, value, size-1 (int))
    \item[\mpidconst{MPIi_ERR_RANK_DUP}]Duplicate ranks in rank array
      (optional arguments: index, value, other index (int))
    \item[\mpidconst{MPIi_ERR_RANK_LOCAL}]Error specifying local_leader
      (optional arguments: value, size-1 (int))
    \item[\mpidconst{MPIi_ERR_RANK_REMOTE}]Error specifying remote_leader
      (optional arguments: value, size-1 (int))
    \end{description}
\item[\mpiconst{MPI_ERR_ROOT}]Invalid root (optional arg: value (int))
    \begin{description}
    \item[\mpidconst{MPIi_ERR_ROOT_LARGE}]Root is too large (optional
      arguments: value and size-1 (int))
    \end{description}
\item[\mpiconst{MPI_ERR_GROUP}]Invalid group
    \begin{description}
    \item[\mpidconst{MPIi_ERR_GROUP_NULL}]Null group
    \end{description}
\item[\mpiconst{MPI_ERR_OP}]Invalid \mpiconst{MPI_Op}
    \begin{description}
    \item[\mpidconst{MPIi_ERR_OP_NULL}]Null \mpiconst{MPI_Op}
    \item[\mpidconst{MPIi_ERR_OP_UNDEFINED}]\mpiconst{MPI_Op} operation not
      defined for 
      this datatype (optional argument: name of datatype (string))
    \item[\mpidconst{MPIi_ERR_OP_FREE_PERM}]Cannot free permanent
      \mpiconst{MPI_Op} 
    \end{description}
\item[\mpiconst{MPI_ERR_TOPOLOGY}]Invalid topology
    \begin{description}
    \item[\mpidconst{MPIi_ERR_TOPOLOGY_SIZE}]Topology size is greater than
      communicator size (optional arguments: topology size and communicator
      size (int))
    \item[\mpidconst{MPIi_ERR_TOPOLOGY_GRAPH_ARRAY_SIZE}]Specified edge $<$ 0
      or $>$ nnodes (optional arguments: index, value, nnodes (int))
    \end{description}
\item[\mpiconst{MPI_ERR_DIMS}]Invalid dimension argument (optional arg: value (int))
    \begin{description}
    \item[\mpidconst{MPIi_ERR_DIMS_ARRAY}]Invalid dimension argument in array
      (optional arguments: index, value (int))
    \item[\mpidconst{MPIi_ERR_DIMS_MANY}]Number of dimensions is too large
      (optional arguments: value, maxvalue (int))
    \item[\mpidconst{MPIi_ERR_DIMS_TENSOR_SIZE}]Tensor product size does not
      match nnodes (optional arguments: tensor product size and nnodes (int))
    \item[\mpidconst{MPIi_ERR_DIMS_PARTITION}]Can not partition nodes as
      requested 
    \end{description}
\item[\mpiconst{MPI_ERR_ARG}]Invalid argument (optional arg: name (string))
    \begin{description}
    \item[\mpidconst{MPIi_ERR_ARG_ERRCODE}]Invalid error code (optional arg:
      value (int))
    \item[\mpidconst{MPIi_ERR_ARG_NULL}]Invalid null parameter (optional arg:
      name of argument (string))
    \item[\mpidconst{MPIi_ERR_ARG_F77_ADDRESS}]Address of location given to
      \mpifunc{MPI_ADDRESS} does not fix in a Fortran integer (optional
      argument: address (long int))
    \item[\mpidconst{MPIi_ERR_ARG_ERRHANDLER}]Invalid errhandler
    \item[\mpidconst{MPIi_ERR_ARG_ERRHANDLER_NULL}]Null errhandler
    \item[\mpidconst{MPIi_ERR_ARG_ERRHANDLER_FREE_PERM}]Cannot free permanent
      error handler (optional argument: name (string))
    \item[\mpidconst{MPIi_ERR_ARG_STATUS_IGNORE}]Invalid use of
      \mpiconst{MPI_STATUS_IGNORE} or \mpiconst{MPI_STATUSES_IGNORE}
    \item[\mpidconst{MPIi_ERR_ARG_STRIDE}]Range does not terminate (optional
      arguments: start, end, stride (int))
    \item[\mpidconst{MPIi_ERR_ARG_STRIDE_ZERO}]Zero stride is incorrect
    \item[\mpidconst{MPIi_ERR_ARG_ARRAY_VAL}]Invalid value in array (optional
      arguments: name of variable (string), index (int), value (int))
    \item[\mpidconst{MPIi_ERR_ARG_NAMED}]Invalid argument (optional arguments:
      name (string), value (int))
    \item[\mpidconst{MPIi_ERR_ARG_NEGATIVE}]Invalid argument; must be
      nonnegative (optional arguments: name (string), value (int))
    \item[\mpidconst{MPIi_ERR_ARG_ARRAY_VAL_NEG}]Negative value in array
      (optional arguments: name of variable (string), index (int), value
      (int)) 
    \item[\mpidconst{MPIi_ERR_ARG_DARRAY_DIST_NONE}]For
      \mpiconst{MPI_DISTRIBUTE_NONE}, the number of processes in that
      dimension of the grid must be 1 (optional arguments: index of
      array_of_psizes, value (int))
    \item[\mpidconst{MPIi_ERR_ARG_DARRAY_DIST_UNKNOWN}]Unknown distribution
      type 
    \item[\mpidconst{MPIi_ERR_ARG_DARRAY_INVALID_BLOCK}]Value of m must be
      positive for block(m) distribution (optional argument: value of m (int))
    \item[\mpidconst{MPIi_ERR_ARG_DARRAY_INVALID_BLOCK2}]\code{m * nprocs} is
      $<$ \code{array_size} and is not valid for block(m) distribution
      (optional arguments: \code{m*nprocs}, \code{array_size} (int))
    \item[\mpidconst{MPIi_ERR_ARG_DARRAY_INVALID_CYCLIC}]Value of m must be
      positive for a cyclic(m) distribution (optional argument: m (int))
    \item[\mpidconst{MPIi_ERR_ARG_POSITION_NEG}]Value of position must be
      nonnegative (optional argument: value (int))
    \item[\mpidconst{MPIi_ERR_ARG_INFO_NKEY}]n is invalid (optional arguments:
      n, number of keys in info (int))
    \end{description}
\item[\mpiconst{MPI_ERR_UNKNOWN}]Unknown error.  Note that this should
  \emph{never} be used.
%    \begin{description}
%    \item[\mpidconst{MPIi_ERR_}]
%    \end{description}
\item[\mpiconst{MPI_ERR_TRUNCATE}]Message truncated (optional arguments: bytes
  received and buffer size (int))
%    \begin{description}
%    \item[\mpidconst{MPIi_ERR_}]
%    \end{description}
\item[\mpiconst{MPI_ERR_OTHER}]Other MPI error
    \begin{description}
    \item[\mpidconst{MPIi_ERR_OTHER_RESOURCE}]System resource limit exceeded
      (optional argument: name of resource (string))
    \item[\mpidconst{MPIi_ERR_OTHER_RSEND}]Ready send had no matching receive
      (optional arguments: source, destination, tag (int))
    \item[\mpidconst{MPIi_ERR_OTHER_INIT_TWICE}]Cannot call \mpifunc{MPI_INIT}
      or \mpifunc{MPI_INIT_THREAD} more than once
    \item[\mpidconst{MPIi_ERR_OTHER_INIT_BEFORE}]\mpifunc{MPI_Init} must be
      called first (optional argument: name of calling routine (string))
    \item[\mpidconst{MPIi_ERR_OTHER_STARTUP}]Error on startup, such as a
      mismatch between \code{mpiexec} and the MPI libraries (optional
      argument: text with detailed reason (string))
    \item[\mpidconst{MPIi_ERR_OTHER_NOMEM}]Out of memory (optional arguments:
      requested and available (int))
    \item[\mpidconst{MPIi_ERR_OTHER_ATTR_COPY}]User defined attribute copy
      routine returned a non-zero return code (optional argument: return code
      (int)) 
    \end{description}
\item[\mpiconst{MPI_ERR_INTERN}]Internal MPI error!  (optional argument:
  detailed text (string))
These provide English-only strings because they are for internal errors and
should never be seen by users.
%    \begin{description}
%    \item[\mpidconst{MPIi_ERR_INTERN}]
%    \end{description}
\item[\mpiconst{MPI_ERR_IN_STATUS}]See the \mpiconst{MPI_ERROR} field in
  \mpiconst{MPI_Status} for the error code
%    \begin{description}
%    \item[\mpidconst{MPIi_ERR_}]
%    \end{description}
\item[\mpiconst{MPI_ERR_PENDING}]Pending request (no error)
%    \begin{description}
%    \item[\mpidconst{MPIi_ERR_}]
%    \end{description}
\item[\mpiconst{MPI_ERR_REQUEST}]Invalid \mpiconst{MPI_Request}
    \begin{description}
    \item[\mpidconst{MPIi_ERR_REQUEST_NULL}]Null \mpiconst{MPI_Request}
    \item[\mpidconst{MPIi_ERR_REQUEST_NOT_PERSISTENT}]Request is not
      persistent in \mpifunc{MPI_Start} or \mpifunc{MPI_Startall}.
    \end{description}
\item[\mpiconst{MPI_ERR_ACCESS}]Access denied to file (optional arg: name (string)
%    \begin{description}
%    \item[\mpidconst{MPIi_ERR_}]
%    \end{description}
\item[\mpiconst{MPI_ERR_AMODE}]Invalid amode value in \mpifunc{MPI_File_open}
  (optional argument: amode (int))
    \begin{description}
    \item[\mpidconst{MPIi_ERR_AMODE_ONLY_ONE}]Exactly one of
      \mpiconst{MPI_MODE_RDONLY}, \mpiconst{MPI_MODE_WRONLY}, or
      \mpiconst{MPI_MODE_RDWR} must be specified
    \item[\mpidconst{MPIi_ERR_AMODE_RDONLY}]Cannot use
      \mpiconst{MPI_MODE_CREATE} or \mpiconst{MPI_MODE_EXCL} with
      \mpiconst{MPI_MODE_RDONLY} 
    \item[\mpidconst{MPIi_ERR_AMODE_SEQ}]Cannot specify
      \mpiconst{MPI_MODE_SEQUENTIAL} with \code{MPI_MODE_RDWR}
    \end{description}
\item[\mpiconst{MPI_ERR_BAD_FILE}]Invalid file name (optional arg: name
  (string)) 
    \begin{description}
    \item[\mpidconst{MPIi_ERR_BAD_FILE_LONG}]Pathname too long (optional
      arguments: name, length, and maximum length (string, int, int))
    \item[\mpidconst{MPIi_ERR_BAD_FILE_DIR}]Invalid or missing directory
      (optional argument: name (string))
    \end{description}
\item[\mpiconst{MPI_ERR_CONVERSION}]An error occurred in a user-defined data
  conversion function
%    \begin{description}
%    \item[\mpidconst{MPIi_ERR_}]
%    \end{description}
\item[\mpiconst{MPI_ERR_DUP_DATAREP}]The requested datarep name has already
  been specified to \mpifunc{MPI_REGISTER_DATAREP} (optional arg: name
  (string))
%    \begin{description}
%    \item[\mpidconst{MPIi_ERR_}]
%    \end{description}
\item[\mpiconst{MPI_ERR_FILE_EXISTS}]File exists (optional arg: name (string))
%    \begin{description}
%    \item[\mpidconst{MPIi_ERR_}]
%    \end{description}
\item[\mpiconst{MPI_ERR_FILE_IN_USE}]File in use by some process (optional
  arg: name (string))
%    \begin{description}
%    \item[\mpidconst{MPIi_ERR_}]
%    \end{description}
\item[\mpiconst{MPI_ERR_FILE}]Invalid \mpiconst{MPI_File}
    \begin{description}
    \item[\mpidconst{MPIi_ERR_FILE_NULL}]Null \mpiconst{MPI_File}
    \end{description}
\item[\mpiconst{MPI_ERR_INFO}]Invalid \mpiconst{MPI_Info}
    \begin{description}
    \item[\mpidconst{MPIi_ERR_INFO_NULL}]Null \mpiconst{MPI_Info}
    \end{description}
\item[\mpiconst{MPI_ERR_INFO_KEY}]Invalid key for \mpiconst{MPI_Info}
    \begin{description}
    \item[\mpidconst{MPIi_ERR_INFO_KEY_NULL}]Null key
    \item[\mpidconst{MPIi_ERR_INFO_KEY_LENGTH}]Key is too long (optional
      arguments: name, length, maxlength (string, int, int )
    \item[\mpidconst{MPIi_ERR_INFO_KEY_EMPTY}]Empty or blank key
    \end{description}
\item[\mpiconst{MPI_ERR_INFO_VALUE}]Invalid \mpiconst{MPI_Info} value
    \begin{description}
    \item[\mpidconst{MPIi_ERR_INFO_VALUE_NULL}]Null value
    \item[\mpidconst{MPIi_ERR_INFO_VALUE_LENGTH}]Value is too long (optional
      arguments: name, length, maxlength (string, int, int)
    \end{description}
\item[\mpiconst{MPI_ERR_INFO_NOKEY}]\mpiconst{MPI_Info} key is not defined
  (optional argument: keyname (string))
%    \begin{description}
%    \item[\mpidconst{MPIi_ERR_}]
%    \end{description}
\item[\mpiconst{MPI_ERR_IO}]Other I/O error (optional argument: text (string))
    \begin{description}
    \item[\mpidconst{MPIi_ERR_IO_ETYPE_FRACTIONAL}]Only an integral number of
      etypes can be accessed
    \item[\mpidconst{MPIi_ERR_IO_NO_FSTYPE}]Cannot determine filesystem type
      (optional argument: name of file (string))
    \item[\mpidconst{MPIi_ERR_IO_UNAVAILABLE_FSTYPE}]Specified filesystem is
      not available (optional argument: name of filesystem (string))
    \item[\mpidconst{MPIi_ERR_IO_MULTIPLE_SPLIT_COLL}]Only one active split
      collective I/O operation is allowed per file handle
    \item[\mpidconst{MPIi_ERR_IO_NO_SPLIT_COLL}]No split collective I/O
      operation is active
    \item[\mpidconst{MPIi_ERR_IO_ASYNC_OUTSTANDING}]There are outstanding
      nonblocking I/O operations on this file
    \item[\mpidconst{MPIi_ERR_IO_NEED_RDWR}]Read/write access is required to
      this file
    \item[\mpidconst{MPIi_ERR_IO_FILETYPE}]Filetype must be constructed out of
      one or more etypes
    \item[\mpidconst{MPIi_ERR_IO_NO_SHARED_FP}]Shared file pointers not
      supported (optional argument: name of file system (string))
    \item[\mpidconst{MPIi_ERR_IO_AMODE_SEQ}]Cannot use this function when the
      file is opened with amode \mpiconst{MPI_MODE_SEQUENTIAL} (optional
      argument: name of routine (string))
    \item[\mpidconst{MPIi_ERR_IO_MORE_WRONLY}]Cannot read from a file opened
      with amode \mpiconst{MPI_MODE_WRONLY}
    \item[\mpidconst{MPIi_ERR_IO_NO_MODE_SEQ}]\mpiconst{MPI_MODE_SEQUENTIAL}
      not supported on this file system (optional argument: name of file
      system (string))
    \end{description}
\item[\mpiconst{MPI_ERR_NAME}]Attempt to lookup an unknown service name
  (optional arg: name (string))
%    \begin{description}
%    \item[\mpidconst{MPIi_ERR_}]
%    \end{description}
\item[\mpiconst{MPI_ERR_NOMEM}]Unable to allocate memory for
  \mpifunc{MPI_Alloc_mem} (optional arguments: amount requested and amount
  available (int))
%    \begin{description}
%    \item[\mpidconst{MPIi_ERR_}]
%    \end{description}
\item[\mpiconst{MPI_ERR_NOT_SAME}]Inconsistent arguments to collective routine
(optional arguments: name of collective routine (string), name of argument
that is not consistent (string))
    \begin{description}
    \item[\mpidconst{MPIi_ERR_NOT_SAME_VALUE}]Arguments to collective routine
      must be the same (optional arguments: name of collective routine
      (string), name of argument (string), null terminated array of values
      (array of int))
    \item[\mpidconst{MPIi_ERR_NOT_SAME_ROOT}]Inconsistent root
    \item[\mpidconst{MPIi_ERR_NOT_SAME_COLLECTIVE_ORDER}]Collective routines
      called in an inconsistent order (optional arguments: null terminated
      array of names (array of string))
    \end{description}
\item[\mpiconst{MPI_ERR_NO_SPACE}]Not enough space for file (optional
  arguments: name (string), size needed (int), and size available (int))
%    \begin{description}
%    \item[\mpidconst{MPIi_ERR_}]
%    \end{description}
\item[\mpiconst{MPI_ERR_NO_SUCH_FILE}]File does not exist (optional arg: name (string))
%    \begin{description}
%    \item[\mpidconst{MPIi_ERR_}]
%    \end{description}
\item[\mpiconst{MPI_ERR_PORT}]Invalid port
    \begin{description}
    \item[\mpidconst{MPIi_ERR_PORT_EXIST}]Named port does not exist (optional
      arg: name of port (string))
    \item[\mpidconst{MPIi_ERR_PORT_TIMEOUT}]Time out attempting a
      \mpifunc{MPI_Comm_connect} to port (optional arg: name of port (string))
    \end{description}
\item[\mpiconst{MPI_ERR_QUOTA}]Quota exceeded for files (optional arg: name (string)
%    \begin{description}
%    \item[\mpidconst{MPIi_ERR_}]
%    \end{description}
\item[\mpiconst{MPI_ERR_READ_ONLY}]Read-only file or filesystem (optional arg:
  name (string))
%    \begin{description}
%    \item[\mpidconst{MPIi_ERR_}]
%    \end{description}
\item[\mpiconst{MPI_ERR_SERVICE}]Invalid service name (see
  \mpifunc{MPI_Publish_name}) (optional arg: name (string))
    \begin{description}
    \item[\mpidconst{MPIi_ERR_SERVICE_UNPUBLISH}]Attempt to unpublish an
      unknown service name (optional arg: name (string))
    \end{description}
\item[\mpiconst{MPI_ERR_SPAWN}]Error in spawn call
    \begin{description}
    \item[\mpidconst{MPIi_ERR_SPAWN_FAILED}]Could not spawn all requested processes
    \item[\mpidconst{MPIi_ERR_SPAWN_NO_PGM}]The named program could not be
      found (optional arg: name (spawn))
    \item[\mpidconst{MPIi_ERR_SPAWN_PROCESS_MNGER}]The process manager
      returned an error (optional arg: text from process manager (string))
    \end{description}
\item[\mpiconst{MPI_ERR_UNSUPPORTED_DATAREP}]Unsupported datarep passed to
  \mpifunc{MPI_File_set_view} (optional arg: name of datarep (string))
%    \begin{description}
%    \item[\mpidconst{MPIi_ERR_}]
%    \end{description}
\item[\mpiconst{MPI_ERR_UNSUPPORTED_OPERATION}]Unsupported file operation
  (optional arg: text describing specific operation (string)). 
  We may want to define subclasses for this error class.
%    \begin{description}
%    \item[\mpidconst{MPIi_ERR_}]
%    \end{description}
\item[\mpiconst{MPI_ERR_WIN}]Invalid \mpiconst{MPI_Win}
    \begin{description}
    \item[\mpidconst{MPIi_ERR_WIN_NULL}]Null \mpiconst{MPI_Win}
    \item[\mpidconst{MPIi_ERR_WIN_NAME}]Cannot set window object name
    \item[\mpidconst{MPIi_ERR_WIN_INVALID_WINDOW}]Attempt to use
    passive target access with a window not allocated with
    \mpifunc{MPI_Alloc_mem}. 
    \end{description}
\item[\mpiconst{MPI_ERR_BASE}]Invalid base address in \mpiconst{MPI_Free_mem}
%    \begin{description}
%    \item[\mpidconst{MPIi_ERR_}]
%    \end{description}
\item[\mpiconst{MPI_ERR_LOCKTYPE}]Invalid locktype
%    \begin{description}
%    \item[\mpidconst{MPIi_ERR_}]
%    \end{description}
\item[\mpiconst{MPI_ERR_KEYVAL}]Invalid keyval
    \begin{description}
    \item[\mpidconst{MPIi_ERR_KEYVAL_NULL}]Null keyval
    \item[\mpidconst{MPIi_ERR_KEYVAL_FREE_PERM}]Cannot free permanent
      attribute key
    \item[\mpidconst{MPIi_ERR_KEYVAL_NOT_IN_COMM}]Keyval is not in communicator
    \item[\mpidconst{MPIi_ERR_KEYVAL_NOT_IN_TYPE}]Keyval is not in datatype
    \item[\mpidconst{MPIi_ERR_KEYVAL_NOT_IN_WIN}]Keyval is not in window object
    \end{description}
\item[\mpiconst{MPI_ERR_RMA_CONFLICT}]Conflicting accesses to window
%    \begin{description}
%    \item[\mpidconst{MPIi_ERR_}]
%    \end{description}
\item[\mpiconst{MPI_ERR_RMA_SYNC}]Wrong synchronization of RMA calls
%    \begin{description}
%    \item[\mpidconst{MPIi_ERR_}]
%    \end{description}
\item[\mpiconst{MPI_ERR_SIZE}]Invalid size argument in RMA call (optional
  argument: size (int))
%    \begin{description}
%    \item[\mpidconst{MPIi_ERR_}]
%    \end{description}
\item[\mpiconst{MPI_ERR_DISP}]Invalid displacement argument in RMA call
%    \begin{description}
%    \item[\mpidconst{MPIi_ERR_}]
%    \end{description}
\item[\mpiconst{MPI_ERR_ASSERT}]Invalid assert argument
%    \begin{description}
%    \item[\mpidconst{MPIi_ERR_}]
%    \end{description}
%\item[\mpiconst{MPI_ERR_}]
%    \begin{description}
%    \item[\mpidconst{MPIi_ERR_}]
%    \end{description}
\end{description}

% $set 12 MPI_ERR_ARG
% 12      56      "Invalid valuelen argument; must be positive, is %d"    1
% 15      20      "Specified buffer is smaller than MPI_BSEND_OVERHEAD = %d"      1
% $set 16 MPI_ERR_INTERN
% 16      3       "Internal MPI error! Out of internal memory"
% 16      5       "Internal MPI error! Cray restriction: Either both or neither buffers must be of type character"
% 16      7       "Internal MPI error! WARNING - sender format not implemented!"
% 16      11      "Internal MPI error! Attribute in communicator is not a valid attribute\nSpecial bit pattern in attribute is incorrect."
% 16      12      "Internal MPI error! Attribute in communicator is not a valid attribute\nSpecial bit pattern %x in attribute is incorrect."     1
% 16      15      "Internal MPI error! Error in BSEND data, corruption detected"
% 16      16      "Internal MPI error! Error in BSEND data, corruption detected in %s"    1





\subsubsection{Error handling and Fault Tolerance}
\label{sec:errs-and-faults}
In order to support fault tolerance, errors should be handled as
gracefully as possible.  If it is possible to remain in a consistent
state, the process should not abort (unless, of course, the error
handler requires it, as the default \mpiconst{MPI_ERRORS_FATAL} does).  
If it is not possible to recover from an error, then the process
should call \mpidfunc{MPID_Abort} but specify \mpiconst{MPI_COMM_SELF} as the
communicator.  

\subsubsection{Error Handling for Layered Routines}
\label{sec:err-handling-nested}
In many cases, MPI routines are implemented in terms of other MPI (or
PMPI) routines.  In these cases, it is important for any error handler
(other than \mpiconst{MPI_ERRORS_RETURN}) to be invoked only by the
``top-level'' routine.  In the case of a single-threaded MPI (from the
user's perspective; that is, the user's code is single-threaded),
implementing this is not difficult.  The ``top-level'' routine needs
only to save the current error handler and then replace it with
\mpiconst{MPI_ERRORS_RETURN}.  Any errors detected in the routines
that the top level routine calls will then be returned to that
routine, which can then invoke the specified MPI error handler.

\index{thread overhead!error handlers}
In a multithreaded case, however, this approach is incorrect, since
another thread may be using the same object (think of
\mpiconst{MPI_COMM_WORLD}).  This suggests an alternative approach.
For each thread, maintain a ``current error handler.''  Most of the
time, this will be ``objects error handler''.  However, during a
layered call, this would be changed to ``errors return.''  

The only difficulty with this is that threads are not registered with
the MPI implementation; thus, depending on the details of the thread package,
there may bes no easy way for the implementation
to know a priori whether thread-private data has been created to store
the current error handler.  Instead, when an error handler must be
invoked or changed, a global table of known threads must be consulted
(by the thread id, which is unique for each thread).  This table
indicates which error handler is current for a thread.  
Note that a different thread id function is needed for each thread
package.  For example, if a system supports both pthreads and OpenMP
(where OpenMP does not use pthreads), then the MPI implementation
needs to know which thread package is being used.  This suggests that
the function that provides the thread id by well isolated and perhaps
dynamically loaded.

Comment [BRT]: I believe most threads packages provide thread-specific
storage.  If such storage is available, we should use it rather than
creating our own implementation of thread-specific storage.

Comment [BRT]: On some systems, the choice of the threads package (or
lack thereof) must be selected at compile time and be consistent for
all object files linked into the executable.  On these systems,
dynamic loading of libraries compiled with different thread packages
is likely to be problematic.

Under pthreads, the situation is somewhat easier.  We can use
\code{pthread_key_create} to create a key that can be used to access
thread-private data with \code{pthread_getspecific} in any thread.
The key needs to be a (at least file-scoped) variable such as
\mpidconst{MPID_THREAD_KEY_ERRHANDLER}. 
Since the pthread keys are common to all pthreads, this can be allocated in
\mpidfunc{MPID_Init}.  Question: is this handled as part of
\mpidfunc{MPID_Err_init}, and is there a corresponding
\mpidfunc{MPI_Err_finalize} that calls \code{pthread_key_delete}?

To implement the layered calls, the following functions may be 
used\index{MPIi_Err_set_return}\index{MPIi_Err_restore}\index{MPIi_Err_get_handler} 
\begin{verbatim}
void MPIi_Err_set_return(void);
void MPIi_Err_restore(void);
MPID_Errhandler *MPIi_Err_get_handler(ds);
\end{verbatim}
The last routine returns the error handler to use; it first checks the
global current error handler (\code{int}
\mpiconst{MPIi_QUERY_ERRORS_RETURN}); if true, then the handler is
\mpiconst{MPI_ERRORS_RETURN}.  Otherwise, it returns the handler
associated with the data structure.  In the multithreaded case, it
uses the thread id to check the table of threads for the
thread-specific version of \mpiconst{MPIi_QUERY_ERRORS_RETURN}.

This is different from the MPICH-1 version which implemented a general
push/pop of error handlers.  While push/pop of error handlers is a
nice abstract model, it is more than is needed for the layered MPI
routines, and, as noted above, is not correct for the multithreaded
case.

Comment [BRT]: push/pop would be correct for the multi-threaded case
as long as thread-specific stacks are used.

\subsection{Memory Allocation}
Avoid memory allocation wherever practical.  Use \mpidfunc{MPID_Malloc}
etc. instead of \code{malloc}, as described in the ADI-3 manual
\cite{adi3man}. 

At normal exit from the MPI library in \mpifunc{MPI_Finalize}, the
routine \mpidfunc{MPID_Trdump} must be called if the runtime parameter
\code{MPICH_TRDUMP}\eindex{MPICH_TRDUMP} is set.  

Question: Should there be separate per-thread and per-process memory
allocators? 

Question: Should we use the same calling sequence as PETSc uses for the memory
allocators?  This will make it easier to use the PETSc options database.

Question: Should there be a memory allocator that remembers allocations within
a routine so that out-of-memory errors can be cleanly handled (e.g., freeing
all memory allocated within the routine)?  This could have something like
\begin{verbatim}
#define MAX_MEM_STACK 16
typedef struct { int n_alloc; void *ptrs[MAX_MEM_STACK]; } MPID_Mem_stack;
\end{verbatim}
and a memory allocation macro that updated a routine-local version of this
with every allocation.  Then on an error, we could easily free any allocated
memory.  The memory allocator could be
\begin{verbatim}
#define MALLOC_STK(n,a) {a=MALLOC(n);\
               if (_memstack.n_alloc >= MAX_MEM_STACK) abort(implerror)\
               _memstack.ptrs[_memstack.n_alloc++] = a;}
#define FREE_STK     {int i; for (i=0;i<_memstack.n_alloc;i++) {\
               FREE(_memstack.ptrs[i]);}
\end{verbatim}

Comment [BRT]: _memstack needs to be allocated in thread specific
storage in order for the MPI implementation to be thread safe.

\subsection{MPI Opaque Objects}

Most objects in MPI (with the exception of \mpifunc{MPI_Status}) are
opaque.  In the MPICH2 implementation, opaque objects are represented
by integers.  This simplifies the implementation of the functions for
transfering handles between C/C++ and Fortran.  

Question:  One interesting idea is to encode information in the opaque
handle.  For example, the \code{sizeof} a basic datatype, and whether
a type is basic, could be part of the handle for an
\code{MPI_Datatype}. In that case, an implementation can avoid looking
up the datatype (e.g., by using the integer of the opaque type as an
array index) and instead perform a few simple operations on the handle
(e.g., mask and test and mask and shift).  Similarly, for
communicators that are dups of \code{MPI_COMM_WORLD}, the handle could
contain the \code{context_id}, again avoiding the need to look up the
communicator, since in addition the mapping from rank to local pid for 
communicators similar to \code{MPI_COMM_WORLD} is the identity
mapping.  Thus a single bit test on the opaque handle could eliminate
a number of tests and memory references for an important common case.

The questions are: do we want to make this possible?  If so, do we
want to implement it from the beginning?  Note that encoding
information into the opaque handle makes that handle conversion
functions more complex.

\subsubsection{Proposed Opaque Handle Format}

We're going to use ints as handles to datatypes and to other constructs such
as communicators.  Some of the considerations in this are:
\begin{itemize}
\item fast resolution for some number of user-created constructs
\item particularly fast resolution for built-ins
\item minimal added limitations on number of user-defined constructs 
\end{itemize}

Using the same system for storing all these constructs should be more
space-efficient.

The MPI opaque objects include \mpiconst{MPI_Comm}, \mpiconst{MPI_Group},
\mpiconst{MPI_Datatype}, \mpiconst{MPI_Errhandler}, \mpiconst{MPI_File},
\mpiconst{MPI_Info}, \mpiconst{MPI_Op}, and \mpiconst{MPI_Win}.

Nomenclature is subject to change.

Different architectures have ints of varied sizes, so we must rely on
\code{sizeof(int)} to determine what we have to work with.

We will use the 2 highest bits to encode a type of construct.  These are:
\begin{description}
\item[\code{CONSTRUCT_INVALID}]  (00)
\item[\code{CONSTRUCT_BUILTIN}]  (01)
\item[\code{CONSTRUCT_DIRECT}]   (10)
\item[\code{CONSTRUCT_INDIRECT}] (11)
\end{description}

Using 00 as invalid, especially in the two high bits, will help detect
bad parameters passed to us (although negative ints won't be caught).

\code{CONSTRUCT_BUILTIN} types include the predefined datatypes such as
\code{MPI_BYTE} and \code{MPI_DOUBLE} as well as the predefined communicators
\code{MPI_COMM_WORLD} and \code{MPI_COMM_SELF}.

The next three highest bits will encode the MPI type stored:
\begin{description}
\item[\code{MPI_Comm}]     (000)
\item[\code{MPI_Group}]    (001)
\item[\code{MPI_Datatype}] (010)
\item[\code{MPI_File}]     (011)
\item[\code{MPI_Errhandler}](100)
\item[\code{MPI_Op}]       (101)
\item[\code{MPI_Info}]     (110)
\item[\code{MPI_Win}]      (111)
\end{description}

At this point we can reasonably expect to have at least 11 bits remaining
to work with, usually 27 and for some architectures 59.

There are more than 32 but less than 64 predefined datatypes, so 6 bits are
adequate to encode all of the predefined datatypes.
That leaves us at least 5 bits
(usually 21) to specify the size.  We'll be specifying size in bytes,
which will be adequate (if long double is 16 bytes, then the long complex and
long-double-int types may be 32 bytes, but on these systems, sizeof(int) will
be at least 32 bits, leaving us enough room for the size of these longer
types). 

For predefined Groups and Communicators we simply use the remaining bits
to specify the particular group/communicator (e.g. \code{MPI_COMM_WORLD}) and 
either the rank of the process or the size (or both!).  Actually, for
small integers we  
might not have enough bits, so we should think about this a little more.

For the directly accessed values the remaining bits are used as indices into
a table of pointers or an array of preallocated objects (using the array
removes one indirect reference and ensures that storage is available at init
time).  This table should be 
a memory page or two in size, 
minus estimated \code{malloc} overhead (if we dynamically allocate).

For indirectly accessed values the remaining bits specify an allocated block
for storing pointers and an index into that block.  We'll split the bits
based on the number of pointers we can store in a block (which will need
to be statically calculable; \code{_SC_PAGE_SIZE} is a starting point, or we
can  do a configure test and define...).

\subsection{MPI and PMPI Routines}
\label{sec:pmpi-routines}
Each routine should implement the \code{PMPI} version of the routine.
Where possible, a weak symbol pragma may be used to define the
\code{MPI} version of the routine.  If weak symbol support is not
available, the \file{Makefile}s will support recompiling each file
with the definition \code{MPICH_MPI_FROM_PMPI} made.  This value
can also be used to protect code that is used by the \code{PMPI}
version of the routine.  For example, 
\begin{verbatim}
#include "mpich_impl.h"

#ifdef MPICH_MPI_FROM_PMPI
#define PMPI_Attr_get MPI_Attr_get
#else
int MPIR_Attr_get_util( int keyval, MPID_Comm *comm, void *attr )
{
...
}
#endif
int PMPI_Attr_get( ... etc ... )
\end{verbatim}

If it is necessary to create the \code{MPI} versions separately, the
object files should be renamed, allowing them to be placed into the
same library.  To handle the event that the library cannot hold over
500 files (250 for PMPI MPI 1 and 2, plus a version for the MPI
routines), the name of the library containing the profiling versions
should be separate.

Question: If we do have two libraries, then we must link with both,
even when the profiling routines are not otherwise used because any
internal functions may be defined in the PMPI versions.  We may need
to do this anyway, because any function that calls an MPI routine will
actually be calling the PMPI version.  Is this what we want to do?  Is
it the best thing to do?  Note that we do this now, but through the
confusing approach of using the MPI file names, but redefining them as PMPI
with a file containing a redefinition of every single MPI routine.

Question: The pragma code is very ugly and hard to read.  It is also hard to
update.
It would be relatively
easy to create the correct form on the fly, using configure (or
another program, run by configure).  But that
would require at least an include file for each MPI routine.
An alternate approach is to make the inclusion entirely machine
generated.  In that case, updating it requires only re-running the
update editor.  In this case, the pragmas for implementing the
profiling interface should be placed between two clear markers, such
as 
\begin{verbatim}
/* -- begin profiling interface for routine xxxx -- */
/* DO NOT EDIT.  Use the program xxx to update */
...
/* -- end profiling interface -- */
\end{verbatim}

Placing the routine name in the comment header makes it easy to
extract the correct name and generate the appropriate text.

Question: Some sets of routines need a global variable.  Rather than
make the variable truely global, we could use \code{static} to
restrict it to file scope.   Examples are handling of the bsend
buffers, the keyval used for topologies, and the error handler on
\mpiconst{MPI_FILE_NULL}.  In order to provide the profiling
interface, this can be implemented as is if weak symbols are
provided.  Otherwise, the MPI version of the routines would simply
call the PMPI routines.  Alternately the value could be accessed
through a utility routine.

\subsection{Deprecated Routines}
Question:  How should deprecated routines be handled?  At the very
least, their man pages should make clear that they are deprecated and
what functions should be used instead.

\subsection{Runtime Parameters}
\label{sec:runtime-params}
MPICH-1 suffers from having many compile-time parameters that could just as
easily be either runtime or at least initialization-time.  These parameters
include search paths and buffer sizes.  These should have a compile-time
default (particularly the search paths) but have an easy way to override at
initialization and/or run time.  While environment variables are one way to do
this, we should not rely on them, since not all environments guarantee that
environment variables are propagated to all processes.

Question: Should we use the PETSc options database code or something similar
to provide uniform access to runtime parameters?  

Question: What should the routines be?  For 
example,\index{MPIi_Param_init}\index{MPIi_Param_get_int}\index{MPIi_Param_get_string}\index{MPIi_Param_finalize}
\begin{verbatim}
int MPIi_Param_init( int argc, char *argv[] );
int MPIi_Param_get_int( const char name[], int default_val, int *value );
int MPIi_Param_get_string( const char name[], const char *default val,
                           char **value );
void MPIi_Param_finalize( void );
\end{verbatim}
The simplest implementation of these might be
\begin{verbatim}
int MPIi_Param_init( int argc, char *argv[] ) {return 0}
int MPIi_Param_get_int( const char name[], int default_val, int *value ) 
    { char *tmp = getenv{name); 
      if (tmp) { 
          *value = atoi(name); /* if name not an integer, return 1 */
          return 0; }
      *value = default_val; 
      return 0; }
int MPIi_Param_get_string( const char name[], const char *default_val,
                           char **value ) 
    { char *tmp = getenv(name); 
      if (!tmp) tmp = default_val; 
      else { *value = (char *)malloc( sizeof(tmp) + 1 ); 
             strcpy( *value, tmp );
             return 0; }
     return 0; }
void MPIi_Param_finalize( void ){}
\end{verbatim}
The \mpidfunc{MPIi_Param_init} and \mpidfunc{MPIi_Param_free} allows
values to also be passed via the command line.  BNR routines could be
used instead of \code{getenv} to get values.  Even more complex
routines could read a \file{.mpichrc} file and remember parameter
values in a table, looking them up when requested (this is closest to
the PETSc options database).
In fact, we may want to enforce the following order:
\begin{enumerate}
\item Check for an override value (e.g., a priority environment variable)
\item Use any info or attribute value
\item Use environment value
\item Use configure file value
\item Use default (compile-time) value
\end{enumerate}
The coinfiguration file is read once, by one process (at
\code{MPI_Init} time).  

Note that errors are indicated by a non-zero return from the routine.
For example, if \mpidfunc{MPIi_Param_get_int} is called but the value
is \code{"big"}, it will return one.

Question: should parameters be registered?  This allows the runtime
system to find them (e.g., with BNR put/fence/get), and simplifies the
generation of help text.  It 
would also allow text describing the parameter to be provided from
within the code.


\subsection{Threads}
\label{sec:threads}
All thread-related operations must not assume a particular thread
package.  At least three different packages are of interest:
\begin{enumerate}
\item pthreads.  This provides a powerful model with reasonable
portability to most Unix platforms, including Linux.  
\item Windows threads.  
\item OpenMP threads.  OpenMP has a small set of thread runtime
routines (such as lock/unlock).  
\end{enumerate}
There are some operations, such as condition variables, monitors, and
thread-shcheduling control, that may not be available (e.g., OpenMP
has no condition variables)  

Question: do we just want to emulate the missing operations, or (and?)
do we want to know that they are missing so that an alternatie
approach can be taken?

\subsection{Coding Practices}
\label{sec:coding-practices}
This section reviews some coding practices for the MPICH code.

\begin{description}
\item[Function prototypes.]
All routines should be prototyped.  The prototypes should be in the
smallest scope possible.  For example, if the routine is used only
within the files in a subdirectory, the prototype should be in an
include file within that directory.
Question: do we really want this?

\item[Static functions.]
Functions used entirely within a single file should be declared
\code{static}.  Functions that are not static must follow the naming
convention of starting with \code{MPI_} or \code{PMPI_} (for routines
implementing the MPI Standard) or \code{MPIR_} for internal routines.  
Global symbols, such as keyvals, should use \code{MPICH_} as the
prefix.  Also consider the use of \code{inline}, along with the
autoconf macro to test for this feature.

\item[Parameter declarations.]
Parameters (with the exception of the MPI routines defined by the
standard) should follow the guidelines in \code{coding}.  In
particular, \code{const} and \code{restrict} should be used.
Parameters that are semantically arrays should be declared as arrays
(using \code{[]}).  

\item[Indentation style.]
Question: should we specify an indentation style?  If so, we need to
specify it at the top of each file (as an emacs command) so that emacs
will use the correct style for users with other styles or projects.

Comment [BRT]: Yes.  Having a style standard makes it much easier to
read and modify code written by others.

\item[File header.]
There should be a standard file header, containing the copyright and
standardized includes (e.g., something like "\file{mpiimpl.h}").  This header
template should be part of the development environment.  Question:
what should this header be?

\item[Function Name.]
There should be a macro, \code{__FUNCTION__}, containing the name of the
function.  This name is chosen because some versions of \code{gcc} will set it
for you.  The statement \code{SET_FUNCTION_NAME("name");} should be included
in the declarations part of the program \emph{or} we should include the
appropriate \code{\#define} line (as PETSc does) before the function
declaration. 

Alternately, we could use \code{FCNAME} and include optional code to
set it to either \code{__FUNCTION__} for systems that define that or
explicitly to the name.

Comment [BRT]: Names beginning with an underscore are reserved for the
compiler and runtime system.  Use of \code{__FUNCTION__} could affect
portability.  So, I prefer the use of \code{FCNAME}.

\item[Boolean Flags.]
Data structures that contain a number of flags should store them
together in a flag vector and use special macros to check and set
them.  

Question: should the flag values be defined or enums that are assigned
special values?

Question: Are the following acceptable for working with flags?  These
assume that several data structures use flags.
\begin{verbatim}
#define MPID_FLAG_DECL             int _flag
#define MPID_IS_FLAG_SET(ds,field) ((ds)->_flag & (field))
#define MPID_SET_FLAG(ds,field)    (ds)->_flag |= (field)
#define MPID_CLR_FLAG(ds,field)    (ds)->_flag &= ~(field)
\end{verbatim}
\mpiconst{MPI_Datatype}s need multiple boolean flags (e.g., has a
sticky upper bound marker, lower bound marker).  Window objects also
have multiple flags.

\end{description}

\subsection{Other Subsystems}
In MPICH, there is code that is not directly part of the MPI
implementation, such as the MPE and test suite code.  For MPICH2,
these should be cleanly separated.  To simplify the construction of a
full MPICH distribution, there should be a Makefile (and configure
options) that know how to build MPICH2 with MPE, perftest, the test
suite, and other options.  These should \emph{not} be part of the base
MPICH2 project (as far as CVS is concerned).  This will encourage
better separation of the projects.

Question: What about Fortran and C++?  These are required for full MPI support
(unlike MPE or the performance tests), but have many unique requirements
during configurations.  I believe that these should continue to have a
separate configure, though it is not necessary to make them standalone (e.g.,
work with other implementations of MPI).

\section{Include Files}
The include file \file{mpi.h} should not require any \code{-Dxxx}
definitions by the compiler.  This will require generating the
\file{mpi.h} from another file (and probably not using \code{configure} and an
\file{mpi.h.in}).  It should not include any other
files, with the (possible) exception of \file{mpi++.h} for C++
support.

The \file{mpif.h} (or \file{mpif.h.in}) should be created from
\file{mpi.h} so that the various integer values (e.g., error classes,
datatypes, etc.) are guaranteed consistent.  This can be done prior to
distribution, similar to the way \file{configure} is generated from
\file{configure.in}.  The point is to automate this and ensure that,
at least in the development Makefiles, the \file{mpif.h.in} file
should be created from the \file{mpi.h} automatically, at least with
respect to any values. 

% Scenarios (move to ADI doc?)
\newif\ifcodefirst
%\codefirsttrue
\codefirstfalse

\section{Scenarios}
\label{sec:pt-2-pt-scenarios}

To best understand how the point-to-point communication routines work, we will
describe several scenarios that illustrate how various communication methods
may implement communication.  We start with one of the more complex cases and
then discuss optimizations for special cases such as sending and receiving
contiguous messages.

\subsection{Nonblocking Send and Receive with Complex Datatypes.}
This represents one of the more complex cases; most communication will offer
some opportunities for exploiting special cases such as simple datatypes
representing contiguous data or blocking communication.  

Question: this discussion assumes that only contiguous data can be handled.
We may want to consider the more general \code{iovec} (array of structures
contain a pointer and length in bytes for each member).  However, this is not
an efficient way to handle common vector datatypes or fixed blocksize hindexed
datatypes.  How important is it to handle the case where a modest-sized
\code{iovec} array can be used?  (Rob, Brian, and David feel that this can be
a big 
win, particularly for master-slave program containing a header plus a
block of significant data.)

(Still to do: modify code samples to allow for the short iovec case.)

The datatype is assumed to be complex enough that the device cannot handle it
directly; for example, it may be an hindexed type with a large number (e.g.,
10000) of entries, or a simple resized struct datatype and a large count.
The code for this scenario may look something like this:
\begin{verbatim}
req = MPI_REQUEST_NULL;
if (rank == dest) {
    MPI_Irecv( buffer, count1, datatype1, tag, source, comm, &req );
}
else if (rank == source) {
    MPI_Isend( buffer, count2, datatype2, tag, dest, comm, &req );
}
MPI_Wait( &req, &status );
\end{verbatim}

% In the following, a \emph{segment} is a description of part of the data, for
% example, the first 32k bytes.  A \emph{stream} is communication made up of
% segments.

%   Question: there is no explicit stream here; that is, the stream handling is
%   done through calls to \mpidfunc{MPID_Rhcv} and the way in which the
%   communication agent responds to those messages.  Does it make sense to have
%   an explicit notion of a stream here?  Note that we do want something more
%   that we have here for some of the collective routines such as
%   \mpifunc{MPI_Bcast} (resend data before unpacking) and
%   \mpifunc{MPI_Allreduce} (operate on data before resending).

%\subsubsection{Implementation of Point-to-Point}
%\label{sec:pt2pt-implementation}
%This section is a placeholder for pseudocode for the various routines.  They
%are organized by method.

The following sample implementation assumes a homogeneous system (all
processes use the same data representation).  

\subsubsection{\tcpname.}
This method is a prototype for a TCP method.  

The basic communication is provided by
\begin{enumerate}
\item Buffered read and write.  These are roughly like the Unix \code{fread}
  and \code{fwrite}.  Just as for those routines, using buffered routines can
  reduce the number of system calls and data motion.  
%  One major difference is
%  that these are nonblocking in the MPI sense.  
  To avoid copies, however, these are slightly different.
  That is, the buffered read
  returns a pointer into the internal buffer rather than copying the
  data.
  (This requires releasing the data when it is no longer needed, but
  does eliminate a message copy.)
  The buffered write accepts
  pointers without making a temporary copy, and is nonblocking (in the
  MPI sense).  The write variant sets an integer
  flag when the data is written.

  Note that this functionality is provided (I believe) by the current active
  queue code.

\item Stream read and write.  These provide a way to receive and send data that
  may not be in contiguous locations.  Roughly, a stream used for
  sending data:
    \begin{enumerate}
    \item packs some number of bytes into a contiguous
      buffer\label{stream:step1} 
    \item writes those bytes to a socket
    \item when all of those bytes are written, returns to
      step~\ref{stream:step1} to 
    get the next group of bytes
    \item when all bytes are written, a completion routine is called
    that may, for example, free the buffer used to pack data.
    \end{enumerate}

    All of this happens without any further action by the upper levels
    of the code.  A special case of a stream is one in which the data
    is contiguous; in this case, no intermediate copy of the data is
    made.

    A stream is initialized by providing a \code{segment}.  A
    \code{segment} is used to handle packing and unpacking of general 
    datatypes, and is initialized with the usual MPI tuple of
    \code{buffer} address, \code{count}, and \code{datatype}.  The
    actual \code{segment} data structure is always part of a larger 
    structure, such as an \mpidconst{MPID_Request}.  In addition, 
    a local buffer is allocated (if necessary) to be used for packing
    or unpacking.  (In other methods, such as the \shmemname, this
    memory may be special, such as memory shared between processes or
    registered to a network driver.)  

    A stream is sent by specifying a destination process.  All data, including
    streams and data sent as a single contiguous buffer, 
    starts with a \mpidconst{MPID_Hid_data} packet that contains a
    matching request number.

    A stream used for reading data is similar, with data read into a
    contiguous buffer and then unpacked instead.  Bytes in
    the read buffer are used first; then one or more calls to \code{read} are
    used for the remainder of the data.  A further refinement is to allow the
    use of \code{readv} once the read buffer is flushed for
    noncontiguous messages.  

    To implement these stream operations, the current active queue code may
    need to be modified.

    Question: We could use \code{MSG_PEEK} and \code{recv} instead of
    \code{read} to see how much data is available without reading it.  Can we
    use this, or is it better to just read some of the data (one system call)?
    Rob tells us that there is a way to get the number of bytes in the
    buffer.  On Linux, see \code{man 7 tcp}; the \code{ioctl} options
    are \code{FIONREAD} (amount of unread data in buffer) and
    \code{TIOCOUTQ} (amount of unsent data in send buffer).


  Question: Where is the buffer to pack into allocated?  Is this part of the 
  \mpidconst{MPID_Segment} structure?  Part of the \mpidconst{MPID_Stream}
  structure? How do we make sure that the choice is
  good for this particular method (e.g., are segments allocated with a source
  or destination rank and a group/communicator)?  Note that the ADI-3 provides
  a \mpidfunc{MPID_Segment_init_pack} (and an unpack) that takes the
  communicator and rank for the destination These are needed for the
  heterogeneous case, since this may affect how the data is
  packed. The communicator and rank of the destination may also be
  needed if special method-dependent memory is allocated for the pack
  buffer (e.g., shared or pinned memory). In a multi-method device, each
  device must provide this function (unless it is happy with generic
  memory).
  One reason to leave the allocation to the segment routines is that
  the size may depend on the particular method or even datatype
  layout; some methods may have particularly efficient methods for
  handling vectors (e.g., shared memory); others may not.  

  Note that it is very useful to know if a datatype represents contiguous
  data; a \mpids{MPI_Datatype}{is_contig} flag value may be helpful.  Note
  that in the case of resized datatypes, a type that is contiguous with a
  \code{count} of $1$ may not be contiguous with a \code{count} greater than
  one (e.g., when the MPI UB (upperbound) is past the last data byte).  Do we
  want a separate flag for that?  Or should we optimize only for 
  the case of a datatype that is contiguous regardless of
  \code{count}?  Rob has suggested a refinement that looks at, for
  example, the ratio of the overall size to the \code{count}.

    A refinement that is not discussed in detail here is to allow for
    \emph{double 
    buffering}; in the case of a stream used for sending, the next
    group of bytes to be sent is packed before the first is entirely
    sent, overlapping the communication with the packing operation.  In fact,
    this optimization can be implemented within the stream operations that are
    discussed below without changing any of the sample code here.

  The stream routines look roughly like:
\begin{description}
\item[\mpidfunc{MPID_Stream_send_init}]Initialize the
  \mpids{MPID_Request}{stream} structure (usually within a
  \mpidconst{MPID_Request} for sending.
\item[\mpidfunc{MPID_Stream_recv_init}]Likewise, for receiving.
\item[\mpidfunc{MPID_Stream_isend_tcp}]Send a stream without waiting for
  (local) completion.  The arguments for this function are a \code{rank} and
  \code{communicator}, indicating the destination process, a packet type and 
  \code{iovec} (similar to \mpidfunc{MPID_Rhcv}), a previously initialized
  \code{stream}, and two arguments that describe what to do when the operation
  completes (locally).
  When all of the data has been sent, invoke the
  specified completion routine for the stream.  The \code{NULL} function has a
  special meaning and is used as a short-hand for ``set a location (flag) to
  one''.  The \code{_tcp} suffix indicates that this is the TCP-specific
  routine.  
\item[\mpidfunc{MPID_Stream_irecv_tcp}]Likewise, for receiving a stream.
\item[\mpidfunc{MPID_Stream_discard_tcp}]Like
  \mpidfunc{MPID_Stream_irecv_tcp}, but the data is discarded.  This is used,
  for example, for unexpected \mpifunc{MPI_Rsend}s.
\end{description}
  Note that there is no separate ``wait'' or ``test'' routine for a stream.
  This is different from what is currently in the ADI3 document, and reflects
  the need for this interface to support nonblocking operations (the stream
  discussion in the ADI3 document has been optimized for implementation of
  collective operations, all of which (in MPI-1) are blocking).
    
\item Direct read and write.  These are roughly like the Unix
  \code{read} and \code{write}.  They are not used directly, but are a special
  case of the stream read and write.  For example, a direct read may
  be used for a transfering a large message directly to a contiguous
  user's buffer.  This can be generalized to a \code{readv} for
  user-datatypes with acceptable size/count ratios.  Similar rules
  would hold for \code{write} and \code{writev}.  

\item Socket agent.  This can be thought of as a \code{select}
    or \code{poll} loop that advances the communication.  In the case
    of buffered write, for example, it sends more data whenever the
    socket becomes available for writing.

    ``Whenever'' should be interpreted loosely.  If the socket
    agent is running in a separate thread, this may happen when the
    thread is scheduled.  If a polling implementation is used, this
    will happen when the agent is called.

    Question: Can we use \code{poll} always for Unix?  Should we?  If not, how
    do we want to choose between \code{poll} and \code{select}?
    Note that the reason for using \code{poll} is that newer versions of Unix
    allow you to configure the number of fds to be larger than that supported
    by \code{FD_SET}, rendering \code{select} nearly useless.  Is there a 
    simple interface that we could use?  For example, should we use the
    \code{struct pollfd} structure to hold the fds used by a socket-based
    method?  Alternately, should we define an interface that allows use to 
    declare, set, and check a collection of fds that works for both
    \code{select} and \code{poll} (so that we don't need to extract the fds
    everytime we need to do a \code{poll} or \code{select}?

\end{enumerate}

These are the low-level communication services that are used to
implemement a TCP version of a two-sided communication method.
These can be implemented with an ``active queue'' system.  In this
approach, each I/O operation adds an element to a FIFO queue.  Each
file descriptor has separate read and write queues (note that only
active connections need queues; we should not preestablish all
possible connections on large systems).  As operations
complete, the queue element is removed.  Special operations, such as
streams, stay in their position in the queue until all data has been
transfered.  This allows us to use the underlying flow control in TCP,
rather than adding an extra layer of our own, when moving large
amounts of data.

When a low-level communication queue element is removed, a special handler is
invoked.  The 
most common handler is the do-nothing handler (actually a \code{NULL}, so no
function call is made).  The second most common
sets a flag to one; this is normally used to set the
\code{request_ptr->xfer_completed} flag.  For efficiency, this could be a
special case of the \code{NULL} handler with a non-null argument.
For a stream involving a datatype that was packed, it could call a true
routine to free up any temporary buffers (this is, in fact, part of the
interface proposed here).

Errors must be handled gracefully.  If at all possible, it should be possible
to continue beyond an error.  For example, in the TCP case, if a socket closes
(e.g., due to congestion or a timeout), the lowest-level code should try to
reestablish the connection, without generating an error event\footnote{Note
  that recovering from a lost connection requires that we receive an ack
  before releasing control over the send buffer, and we must be careful to
  drain the old socket before reading from the new one.  We should consider a
  design for an unreliable communication device that exploits the MPI
  messaging semantics, particularly for non-blocking communication which allow
  acks to be deferred or aggregated with other communication}.  (It may wish
to record the fact that the socket closed, particularly in a
performance-diagnostic mode.)  Only if a connection cannot be reestablished
within a reasonable time (defined by a runtime parameter) should an error be
propagated up to the MPI levels.  The procedure for passing an error will be
discussed later.

Question: we could use an attribute value on the communicator to allow
the user to give us some guidence on what to do if a TCP connection
fails.  Should we do that?  What should it look like? (It is too bad
that the communicator construction don't have an info hint.)

\implementation{{General Notes}}
The basic structures include the following:
\begin{verbatim}
typedef struct { 
     volatile void *ptr;
     void *tempptr;
     MPID_Msg_format msg_format;   /* Only for heterogeneous systems */
     } MPID_Eager_info;   /* Info for eager delivery */
typedef struct { 
     int sender_id;
     int receiver_id;
} MPID_Rndv_info;         /* Info for rendezvous delivery */
typedef struct {
     void          *ptr;
     int           count;
     MPID_Datatype *dtype;
     MPID_Comm     *comm_ptr;  /* Needed for heterogeneous case */
} MPID_Buf_info;          /* Info describing the user's buffer */

#define MPID_REQ_REGULAR_MASK 0x1
#define MPID_REQ_PERSIST_MASK 0x2
#define MPID_REQ_USER_MASK    0x4
typedef enum { MPID_REQ_ISEND=1, MPID_REQ_IRECV=3, MPID_REQ_PERSIST_ISEND=0,
               MPID_REQ_PERSIST_IRECV=2, MPID_REQ_USER=4 } MPID_Request_kind;

/* All packet types start with the same values */
typedef struct {
  int16_t kind;
  int16_t len;
} MPID_Hid_general_t;

typedef union {
  MPID_Hid_general_t         general;
  MPID_Hid_data_t            data;
  MPID_Hid_eager_t           eager;
  MPID_Hid_request_to_send_t rtosend;
  ... many other types ...
} MPID_Packet;

typedef struct _MPID_Segment {
  /* Remember the source (send) or destination (recv) buffer */
  void             *user_buffer;
  /* Describe the (count,datatype) */
  MPID_Dataloop    dataloop[MAX_DATALOOP];
  int              count, cur_sp;
  /* Used to hold contig form of data */
  void             *tmp_buf;  
} MPID_Segment;

typedef struct _MPID_Stream {
  MPID_Segment segment;
  /* Completion routine */
  void (*completion_fn)( struct _MPID_Stream *, void * );
  /* and argument */
  void *completion_arg;
  ... other stuff ...
} MPID_Stream;

typedef struct {
     MPID_Request_kind kind;
     volatile int      busy;
     volatile int      xfer_completed;
     int               self;    /* integer id of request */
     int               context_id;
     MPI_Status        status;  /* Stores tag, source, error, and n bytes */
     ...
     MPID_Eager_info   eager;
     MPID_Rndv_info    rndv;
     MPID_Buf_info     buf;
     ...
     MPID_Stream       stream;
     ...
     MPID_Packet       packet;
     } MPID_Request;
\end{verbatim}

Notes:
\begin{enumerate}
\item \code{volatile} is required on \code{ptr} in the \code{MPID_Eager_info}
  structure only for multithreaded versions, and then only because the code
  below looks at \code{ptr} to see if data is available after an eager send.
  If a separate flag indicated this, only the flag would have to be
  \code{volatile}. 

\item The \code{comm_ptr} is part of the \code{MPID_Buf_info} because it may
  be necessary to decrement the reference count on a communicator once a
  message has been received, just as the reference count on the datatype must
  be decremented.  

\item The values on the \code{MPID_Request_kind} are chosen to allow single
  bit tests to distinguish between persistant and non-persistant requests.

\item The \code{busy} and \code{xfer_completed} flags in \code{MPID_Request}
  need be 
  \code{volatile} only when either (a) the process is multithreaded or (b) the
  device shares memory, particularly requests, among the processes.

\item Packets are used to send information from one process to another.
  Packets have two components: a packet header and a packet payload.  The
  payload contains user data; many packets consist only of a header. 
  The packet type \code{MPID_Hid_eager_t} is one of the few packet types that
  contain both a header and a payload. 

  Question: some things might be simpler if all packet headers were the same
  size.  For example, they could all be the same number of cache lines.
  This has some costs but may make other things simpler.  Note that it can
  make packet headers \emph{shorter}: there is no need to record the length of
  the packet header since they are all the same.  We probably should do this
  unless there is a wide variation in header size.

\item Should the \code{MPID_Eager_info} and \code{MPID_Rndv_info} in
  the \code{MPID_Request} be in a
  union?  That would help reduce the overall size of an \code{MPID_Request}.
  One goal should be to keep an \code{MPID_Request} to a single cache-line;
  that will improve the performance, particular for the shared-memory
  devices.
  Rob and Brian vote yes, in a union (must be a named union because we're
  writing in C).

\item Should there be an entire \code{MPI_Status} in the request, or should
  this contain less information?  For example, a send request has no need for
  any status field (except for \code{MPI_ERROR}, which is used only for
  multiple completions such as \code{MPI_Waitall}).  
\end{enumerate}

In the case where there are multiple threads, it is necessary to ensure that
only one thread (usually) modifies a request at a time.  This can be
accomplished using (thread) locks or using flags (e.g., a \code{busy} flag).
However, if flags are used, it is necessary to ensure that all pending writes
to memory on the process are completed before the flag is updated.  The macro
\mpidfunc{MPID_MemWrite_ordered} can be used for this.  
For example, for an Alpha processor using \code{gcc},
\mpidfunc{MPID_MemWrite_ordered} might look like this \cite{alpha-asm}:
\begin{verbatim}
#define MPID_MemWrite_ordered(var,value) { \
      asm volatile ("wmb":/*no output*/:/*no input*/); *var = value ; }
\end{verbatim}

Note that in OpenMP, this can use the \code{flush} directive, although that
may require a routine call, since you can't include a \code{\#pragma} within a
\code{\#define} (unless you run \code{cpp} multiple times, which we don't want
to do).  In this case, \mpidfunc{MPID_MemWrite_ordered} would be a 
function, with the definition:
\begin{verbatim}
void MPID_MemWrite_ordered( int *var, int value )
{
    #pragma omp flush
    *var = value;
}
\end{verbatim}

Question: are there any systems where it helps to flush only
specific variables?  (OpenMP allows this.)

A single-threaded implementation might simply use
\begin{verbatim}
#define MPID_MemWrite_ordered( var, value ) 
\end{verbatim}
That is, the entire operation would be eliminated.

Brian has proposed a set of macros and commands for memory modification that
can be used for a variety of systems with different kinds of shared memory
access support.  The general structure of these are shown in the following
example:

\begin{verbatim}
Thread_fast_lock(&foo->mutex);
Thread_invalidate_cache(&foo->state, sizeof(foo->state));

if (foo->state == STATE_SOBER)
{
    foo->state = STATE_SEEK_BEER;
    foo->dest = "Founder's Hill";

    Thread_flush_cache(&foo->state, sizeof(foo->state));
    Thread_flush_cache(&foo->dest, sizeof(foo->dest));
    Thread_fast_unlock(&foo->mutex);
}
else
{
     Thread_fast_clear(&foo->mutex);
}
\end{verbatim}

Possible implementations of these follow:
\begin{description}
\item[RC (SMP) implementation]

\begin{verbatim}
#define Thread_fast_set(M)   {*M = 1;}
#define Thread_fast_clear(M) {*M = 0;}

#define Threadfast_lock(M)                      \
{                                               \
    while(!test_and_set(*M));                   \
}

#define Thread_fast_unlock(M)                   \
{                                               \
    asm volatile ("wmb"::);                     \
    *M = 0;                                     \
}

#define Thread_invalidate_cache(P, S)
#define Thread_flush_cache(P, S)
\end{verbatim}

\item[Eager/Lazy RC implementation]

\begin{verbatim}
#define Thread_fast_set(M)    LRC_acquire(M)
#define Thread_fast_clear(M)  LRC_release(M)
#define Thread_fast_lock(M)   LRC_acquire(M)
#define Thread_fast_unlock(M) LRC_release(M)
#define Thread_invalidate_cache(P, S)
#define Thread_flush_cache(P, S)
\end{verbatim}

\item[non-CC implementation]

Note: This assumes threads from one process run across multiple
processors in a non-CC system.  If that is not true, then the cache
bypass, flushing, and invalidation can be eliminated.

Note: I (Brian) separated the notion of inter-thread mutex and memory
operations 
from inter-process oeprations so that they could independently
optimized.  One can  imagine creating a similar set of routines for
inter-process operations.

\begin{verbatim}
#define Thread_fast_set(M)   /* cache bypass set(M) */
#define Thread_fast_clear(M) /* cache bypass clear(M) */
#define Thread_fast_lock(M)   /* cache bypass T-and-S(M) */
#define Thread_fast_unlock(M) /* cache bypass clear(M) */
#define Thread_invalidate_cache(P, S) /* invalidate cache lines */
#define Thread_flush_cache(P, S)  /* flush cache lines */
\end{verbatim}

\end{description}

Question: For systems that can only invalidate or flush the entire cache, how
do we efficiently keep them from issuing multiple invalidations/flushes?
I wouldn't expect too many invalidatte or flush calls, so perhaps a flag
on a mutex structure would be alright.  This would require passing the
mutex to teh cache functions as well.

Request allocation:

Question: In a multithreaded case, the request id's could be partitioned
among the threads, avoiding the need for a thread lock.  Do we want to
consider that?  This is complicated by the fact that threads are not
required to register themselves with MPI (e.g., with a call to something
like \code{MPI_Thread_initialize}).\index{thread overhead!request allocation}
This probably isn't worth it, but it is a potential bottleneck on systems with
large numbers of CPUs per node and a single MPI process per node.  For
such systems, we could dynamically partition the request ids by
placing this information (and other per-thread information) into
per-thread data.  Most thread packages provide a way to access such
data; some systems even make it efficient.  For now, we won't worry
about it.


\implementation{{MPI_Irecv}}

\ifcodefirst
\fileinclude{samples/irecv_tcp.c}
\fi

Notes:
\begin{enumerate}
\item \mpidfunc{MPID_Request_recv_FOA} always returns a request.  If the
  request was found (and not \code{busy}), it is removed from the queue.  If
  one was not found, a new one is inserted and marked as busy.
  Marking it as busy 
  keeps another thread from finding the request, through another call to
  \mpidfunc{MPID_Request_recv_FOA}, of course, and attempting to use it before
  all of the fields have been set.  In a single threaded implementation, the
  \code{busy} flag is not needed.  
  Whether the request was found or inserted, 
  the \code{status} fields are filled in, along with the \code{context_id}.
  This ensures that another thread that calls \mpidfunc{MPID_Request_recv_FOA}
  will match this inserted request, but, because of the
  \code{busy}\index{thread overhead!request use} field,
  will wait until the request has been fully filled in.  An alternative is to
  use a thread lock on the request, but that is really unnecessary since only
  two threads (the one that inserts and the one the removes) will ever need to
  synchronize access to the request.

  Question: the following assumes that the fields in
  \mpids{MPID_Request}{status} are already filled in.  Is that what we want?

  Question: Should the \code{busy} flag be an integer or just a bit in the
  request flags?  Making it a bit makes atomic update more difficult, though
  for \code{busy}, it may not matter (no other thread can do anything until
  the \code{busy} bit is cleared.  The reason to consider a bit instead of an
  integer is to keep the size of a \mpidconst{MPID_Request} down.  

\item If a matching request was found, there are two cases.  Either the
  message was delivered eagerly or a rendezvous message was sent. Note that
  in either case the request has been removed from the receive queue already 
  by \mpidfunc{MPID_Request_recv_FOA}.

\item The first check is against the total size of the user's receive buffer.
  In the homogenous case, the size of the total message (saved in the
  \mpids{MPID_Request}{status.count} field of \mpidconst{MPID_Request}) is
  compared with \code{count} times the size in bytes of the user's
  \code{datatype}.  The original size is saved in \code{msg_size} so that the
  eager buffer can be freed and flow control updated.
  If the user's buffer is too small, invokes \mpidfunc{MPID_Err_create_code}
  to return a convenient message (using predefined text).

\item The next test is for the special case of a contiguous buffer.  In this
  case, we can use \code{memcpy} directly.  (Note: this should be
  generalized to any buffer that does not require an intermediate
  copy.  We need a term for that kind of data.)

  Question: do we want to use a private \code{memcpy} that might use, for 
  example, quadword load and store instructions that some system libraries
  might not use?  Since we are likely to want this, we've used
  \mpidfunc{MPID_Memcpy} instead of \code{memcpy}.  Note that \code{mpptest}
  has a \code{-memcpy} switch; \code{-memcpy -int} uses \code{int}s for the
  moves instead of \code{memcpy} and \code{-memcpy -double} uses
  \code{double}s for the moves.  Some quick tests with a version that used
  \code{long long} instead of \code{int} and that set the options
  \code{-Msafeptr=auto,arg -Munroll=n:8} for \code{pgcc} showed that the
  \code{memcpy} and \code{long long} loop version had similar performance at
  1024 bytes but the (inlined) \code{long long} version was twice as fast at
  128 bytes.  \code{double}s are \emph{terrible} on Ix86 (17 to 30
  \emph{times} slower).

\item Otherwise, the user's buffer is not contiguous, and the data must be
  unpacked.  This is a simple case since all of the data is present, and 
  can be handled with \mpidfunc{MPID_Unpack}.

  Question: note that \mpidfunc{MPID_Unpack} has slightly different arguments
  from the one in the ADI-3 document.  Are these the correct ones?  In a
  homogeneous system, the \code{msg_format} value is never needed.
  \code{msg_format} is used to indicate the format used by the sender; this
  might be ``senders format'' or XDR or external32 etc.  Homogeneous systems
  will always use ``senders format.''

\item Once the data has been moved from the eager buffer, the eager buffer 
 can be freed.  This call also updates any flow control information (e.g.,
 updating the number of free eager memory buffers available).  The request is
 now marked as complete.

  Question: Are there other values that we want to mark here?  For example, 
  a one in \code{xfer_completed} may mean ``data received but cleanup not done'' and
  a two might mean ``finished''.

\item If the request was found but the data was not available, we need to ask
  the sender to send us the data.  We do this with the \code{packet} that is
  within the \mpidconst{MPID_Request}.  We use the integer index of the
  request rather than its address because it is both addressing independent
  (e.g., on heterogeneous systems, we don't need to worry about mixes of 32
  and 64 bit pointers) and may even be shorter (e.g., we could use a 16 bit
  \code{short} for the value).  We send the request id of the receiving
  request back to the sender so that the data can be properly stored when it
  arrives (see the discussion of \mpidconst{MPID_Hid_data} in the
  communication agent).

  Question: if the message is truncated (user's buffer is too small), do 
  we want to tell the sender?  We aren't required to, and it could be
  difficult in the case of an eager send (the sender's call has long since
  completed).  Brian votes no.

  Question: do we want to remember the source (rank in communicator), rather
  than having to look up 
  the source again through the communicator?  (This can be saved in the
  \code{status} field in the request.)

  Question: in the case that the receive data is contiguous, we could just
  send the sender the memory location of where to deposit the data.  Do we
  want to do that here or leave that for the remote memory case?  Note that
  we still need to have the \code{xfer_completed} flag set once the data has
  arrived.  

  Note that \code{MPID_Rhcv_tcp} must remember \code{vector} if it can't send
  the data immediately (e.g., because the socket is full).  However, if it can
  send it immediately, no copy needs ever be made.

\item If no matching request is found, the information needed to store the data
  when it does arrive must be saved.  We must also increment the reference
  counters on the datatype and the communicator to ensure that they aren't
  deleted before the communication completes.
  The \mpifunc{MPI_Irecv} can now return.  Completion of the message transfer
  is now the responsibility of the communication agent (see below). 

  Question: Should we go ahead and create the segment for the unpack, at least
  for non-contiguous datatypes?

  Question: Should the segment init increment the reference count for the
  datatype to ensure that no \mpifunc{MPI_Type_free} frees it before the
  segment code is done using it?  No, if the request has incremented the
  reference count, but yes, if the request relies on the datatype handler
  (e.g., the segment init) to decide whether the reference count/datatype is
  needed later.

  Question: In the \mpifunc{MPI_Recv} case, we never need to increment the
  datatype and communicator reference counters.  Is there any significant 
  benefit to avoiding that step, particularly for very short messages?
  Probably not if the message is found; if the message is not found, the
  situation is murkier.  We should avoid the update to the permanent
  communicators and datatypes (writes are expensive).

\item This code doesn't handle the case where only some of the data is
  available.  It needs to be modified to record the amount of data received,
  the current location in the buffer (more generally, in the data unpack), and
  it needs to have a clear way to finish receiving the data (the poll/progress
  engine?).
\end{enumerate}

\ifcodefirst
\else
\fileinclude{samples/irecv_tcp.c}
\fi

\implementation{{MPI_Isend}}

\ifcodefirst
\fileinclude{samples/isend_tcp.c}
\fi

Notes:
\begin{enumerate}
\item For now, we assume that there are no ``speculative receives,'' that is,
  a receive that sends a message to the designated source of the message
  indicating that the receiver is prepared to receive a message.  Thus, 
  \mpidfunc{MPID_Request_send_FOA} always allocates, never finds, a request.

\item There are two cases.  Short messages are sent eagerly if flow control
  allows it; all other messages are sent by rendezvous.  
  The function \mpidfunc{MPID_Flow_limit} both tests the message size against
  the flow control limits, and if the limits allows it, updates the flow
  control limits to reserve that space.  This is needed for multithreaded
  codes to ensure that two threads do not both reserve the same eager buffer
  space.  Note that flow limits are used to limit the amount of space consumed
  at the destination; this is different from trying to limit the amount of
  data within the socket buffer.  For example, consider the code
\begin{verbatim}
    if (rank == 0) {
        for (i=0; i<1000; i++) MPI_Iprobe( ... );
    } 
    else if (rank == 1) {
        for (i=0; i<1000; i++) MPI_Isend( buf, 100, MPI_INT, ... );
    }
\end{verbatim}
  Each message sent by process 1 will fit within an empty socket buffer.  
  On process 0, some number of messages will be received eagerly.  However, at
  some point, process 0 will stop accepting messages from process 1 because
  the space for buffering them is exhausted.  Using \mpidfunc{MPID_Flow_limit}
  causes the \mpifunc{MPI_Isend} to shift to rendezvous mode even for short
  messages.  (This assumes that we can accept 1000 envelopes).

\item In the first case, the message is too large to send.  In this case,
  information on the message data is saved and a request-to-send packet is
  sent using \code{MPID_Rhcv_tcp}.  Note that the reference counts for the
  communicator and datatype are incremented to ensure that they are not
  deleted until the message is sent.

  If speculative receives are implemented, \mpidfunc{MPID_MemWrite_ordered}
  should be called at the end of this code to clear the \code{busy} flag.

  Question: do we really need to save the communicator?  We don't need the
  \code{context_id}, and only need the communicator to convert a rank in to a
  specific connection.  If we save the connection, we don't need to increment
  the reference count on the communicator.  We don't really need it if there
  is a failure in communication, because that requires us to mark \emph{all}
  communicators that include that process to be marked as broken (this does
  suggest another operation that the ADI should provide, specifically, a way
  to find all communicators that share a given process).

  Question: Do we want to create the segment now instead of just saving the 
  \code{buffer, count, datatype}?  One reason not to is to try and inject the 
  message into the network as early as possible (i.e., before creating the
  segment).  We could, of course, send the request-to-send and then create the
  segment; this attempts to overlap the creation of the segment data structure
  with the transmission of the request-to-send packet to the destination.

\item In the other case, the message may be sent eagerly.  There are two
  subcases here.  In the first, the message is either contiguous or small
  enough to be packed into a single temporary buffer.  The other case is a
  larger message.  We could ignore the larger message case by setting the
  eager limit to a small value, but we still would want to handle the large
  message case for \code{MPI_Rsend}, so we might as well do it here.

%   An important generalization is the case of a short \code{iovec} datatype;
%   that is, one that is not contiguous but is described by an \code{struct
%   iovec} array with a small number (e.g., 7) of elements.  This is handled in
%   the same way as the contiguous case.

\item We can generalize the contiguous case to datatypes that are described by
  short arrays of \code{struct iovec}, for example, seven elements or less.
  This allows the header to use one element and the datatype the remaining
  seven.  Longer arrays may not be handled efficiently by the operating
  system. 

\item In the case where the data is contiguous and is sent directly with
  \code{MPID_Rhcv_tcp}, the request is marked complete only when the data has
  been sent.  In the case where data is packed up, the request is complete as
  soon as the \code{MPID_Rhcv_tcp} call is made since the user's buffer is now
  available for reuse.  

\item In the case where the data is longer and noncontiguous, we must send it
  using incremental packing.  This uses a special data transfer routine,
  \mpidfunc{MPID_Stream_isend}.  The stream description is saved in
  the \mpids{MPID_Request}{stream} field in the \mpidconst{MPID_Request}.

\item We need some way to indicate that the \code{MPID_Request} itself 
  is complete, not just that the transfer is complete.  Setting the
  \code{xfer_completed} flag indicates that the user's 
  buffer 
  can be reused.  Do we want to replace the \code{NULL} with something like
  \code{\&request_ptr->available} or \code{\&request_ptr->complete}?

\item Question:  Where does the memory allocated with \code{MPID_SendAlloc} in
  the short case get freed?  
  
\end{enumerate}

\ifcodefirst
\else
\fileinclude{samples/isend_tcp.c}
\fi

\implementation{{Communication Agent}}
The communication agent in MPICH2 replaces the \code{MPID_Device_check} in
MPICH-1.  In a version of MPICH2 that uses polling, there will be a similar
routine that executes the communication agent code.  Other implementations may
execute the same code in a separate thread; in those implementations, there
will be no separate polling routine.

In a multi-method device, it is often easiest if each method's communication 
agent runs in a separate thread.  That allows the operating system to  
schedule the threads.  If the agents are running in the same thread (and 
particularly, if they are running in the main thread in a polling-mode 
implementation), each agent must normally be executed in the
\code{NONBLOCKING} (or \code{EXPECTING}) mode.
However, using separate threads for each method can significantly increase
latency.  A hybrid approach that uses separate threads for high-latency
methods and a polling loop in main thread for low-latency methods may be
best.  The frequency of polling should be weighted so that slow methods are
polled less frequently than fast methods.  A refinement of this could adapt
the polling frequency based on message history.

\ifcodefirst
\fileinclude{samples/agent_tcp.c}
\fi

Notes:

\begin{enumerate}
\item The routine \code{GetNextPacketHeader} returns a pointer to the next
  packet header, along with the \code{source} of the message.  This routine
  guarantees 
  that the entire packet header (the first \code{vector} element sent with
  \mpidfunc{MPID_Rhcv_tcp}) has been delivered.  
  One possible implementation (for variable-length packet headers) is for
  \code{GetNextPacketHeader} to look at the 
  first two shorts; the second of these is the length of the packet header.
  If less than a full packet has been delivered, \code{GetNextPacketHeader}
  does not 
  return a packet.  This simplifies the handling in the rest of the
  communication agent without requiring that \code{GetNextPacketHeader}
  understand the different packet types.

  The code for \code{GetNextPacketHeader} might look roughly like
\begin{verbatim}
while (1) {
    if (any buffer has at least 4 bytes &&
        buffer state is ``waiting for packet header'') {
        short *p = <head of buffer>
        if (p[1] >= <data in buffer>) {
            <data in buffer> -= p[1];
            <head of buffer> += p[1];
            (handle circular buffer)
            if (packet has payload)
                set buffer state to ``waiting for payload''
            *source = <id of sending process, not rank in comm>;
            return (MPID_Packet_t *)p;
        }
    }
    if (blocking == MPID_NONBLOCKING) return 0;
    <wait for data using poll or select on ALL fds for either 0 seconds
    (MPID_NONBLOCKING) or 2*round-trip> 
    if (still no data and blocking == MPID_EXPECTING) return 0;
}
\end{verbatim}

  Question: what is the source value?  Is it the local process id (e.g., the
  connection number)?

  Question: If we piggy-back flow-control information in the packets,
  \code{GetNextPacketHeader} could process that, eliminating that step from the
  communication agent code.  This flow control information would contain
  information such as ``2k of eager buffers released''.

\item The blocking parameter to \code{GetNextPacketHeader} has the following
  meanings:
  \begin{description}
  \item[\code{MPID_BLOCKING}:]Block until a packet is available.
  \item[\code{MPID_NONBLOCKING}:]Return a packet if one is available, otherwise
    return \code{NULL}.
  \item[\code{MPID_EXPECTING}:]Like \code{MPID_NONBLOCKING}, except
    \code{GetNextPacketHeader} may wait for a short time before returning.  For
    example, it may wait the time it takes a message to make a round trip in a
    case where a response is expected from another process.
  \end{description}

\item When \code{GetNextPacketHeader} returns with a packet, the data that
  \code{packet} points at is valid until the next time that
  \code{GetNextPacketHeader} is called.  This allows
  \code{GetNextPacketHeader} to read 
  into an internal buffer and return a pointer into that buffer, avoiding an
  extra memory copy.  This works only if only one thread at a time calls the
  communication agent.  However, the restriction that only one thread call the
  communication agent at a time is fairly natural.  In fact, the MPI thread
  mode \mpiconst{MPI_THREAD_SERIALIZED} expresses this; in such a case, the
  communication agent does not need acquire a thread lock in the case where
  many threads may call the communication agent (e.g., in a polling mode
  implementation where there is no separate thread running the communication
  agent).  

  Question: is there one buffer or one per link (e.g., source process, not
  source rank in communicator)?  At least for TCP, the most likely
  implementation is one per link (e.g., one per fd).

\item \code{GetNextPacketHeader} must also keep track of the total message
  size that 
  a packet starts, so that subsequent calls to \code{GetNextPacketHeader} will
  not 
  interpret data as a packet header.  This is covered in more detail under
  \mpidconst{MPID_Hid_data}.  A consequence of this is that
  \code{GetNextPacketHeader} might process additional data before returning a
  new packet.

  Note: an alternative is for \code{GetNextPacketHeader} to return a special
  kind of packet for streaming data, allowing the communication agent to
  process data as it is read.  The reason that we don't do this is that for
  some kinds of messages, we want to bypass any buffering, and that requires
  putting some message handling into the lower level communication functions.
  Question: Do we want to change the name to something that indicates that it
  does more than just read a packet?  Brian votes yes.

  Note also that the processing implicit in \mpidfunc{MPID_Stream_isend} or
  \mpidfunc{MPID_Stream_irecv} may happen inside of
  \code{GetNextPacketHeader}.  Continuing the code fragment describing
  \code{GetNextPacketHeader} from above, at the top we might have
\begin{verbatim}
   foreach (socket with an active send stream) {
       if (stream buffer empty) {
           pack data into stream buffer
       }
       <try to write current stream buffer>
       if (success and at end of stream) {
           mark stream as complete
           invoke completion routine
       }
   }
   foreach (socket with an active recv stream) {
       <try to receive data to fill stream buffer>
       if (stream buffer full) {
           unpack data from stream buffer
           if (success and at end of stream) {
               mark stream as complete
               invoke completion routine
           }
       }
   }
\end{verbatim}


\item Question: Is \code{GetNextPacketHeader} fair?  How do we ensure that
  \mpifunc{MPI_Testsome} is efficiently implemented?

\item Under Linux, very short messages are often delayed, even when
  \code{TCP_NODELAY} is set (see
  \url{http://www.icase.edu/coral/LinuxTCP.html}).  Loncaric found that it is 
  advantageous 
  to always send at least 100 bytes.  We could accomodate this by either
  padding all of the packet types out to 100 bytes or always sending at least
  100 bytes with \code{MPID_Rhcv_tcp} (Brian and Bill prefer the latter).  In
  that case, we must ensure that 
  \code{GetNextPacketHeader} knows to skip over the ``extra'' bytes, possibly
  by 
  changing the \code{len} field in the \code{packet} and sending the extra
  bytes.  

\item Note that \code{GetNextPacketHeader} guarantees only that the packet
  header is available.  The payload in packet types that contain payload, such
  as \mpidconst{MPID_Hid_eager_t} or \mpidconst{MPID_Hid_data_t}, may not be
  available.  

  Brian votes against \code{GetNextPacketHeader} and would prefer that its
  functionality was rolled into the state machine for processing all message
  data.  

\item If we use a test on \mpidconst{MPID_THREAD_LEVEL} to check to see if we
  need to lock the agent, we also need to use that same mutex in any routine
  that we might provide to change the thread level.  If we only permit the
  choice of thread level within \mpifunc{MPI_Thread_init}, then we don't need
  to worry about this.  Of course, we'll have a configure option
  \cfgoption{--enable-single_threaded} that eliminates all thread locks needed
  to support multiple user threads at compile time.  In the documentation, we
  also need to distinguish between single-threaded and single-threaded
  implementations (run-time vs. compile-time).

\item The sample code shows inlined code for each packet type.  An
  implementation is more likely to use functions to handle each packet type,
  at least for the less common cases (e.g., \mpidconst{MPID_Hid_control} and
  error return on a ready-send).

\item In the multithreaded case but polling case, we use a single mutex on the
  agent.  Instead, we could have \code{GetNextPacketHeader} set a mutex on that
  source (e.g., socket), which would then be released by the agent when the
  packet was fully handled.  This would allow multiple threads to run the
  agent simultaneously on different sources.  This matches completion ports in
  Windows.  
\end{enumerate}

\mpidconst{MPID_Hid_eager}:
\begin{enumerate}
\item This is a message followed by data.  The \code{lpacket} points to the
  packet header.  Note that while the data is immediately behind the packet,
  there is no guarantee that it has arrived yet (more on this below).

\item We use an array reference on the \code{context_id} to find the
  corresponding communicator.  Another approach would be to store the
  communicator pointer in the request, and allow the matching logic in
  \mpidfunc{MPID_Request_recv_FOA} find the corresponding request.  
  If \mpidfunc{MPID_Request_recv_FOA} needs the communicator (e.g., in the
  multimethod case to find the appropriate method queue), it could do the
  lookup on the \code{context_id} on its own.  Brian votes for the latter
  (pass the \code{context_id} to \mpidfunc{MPID_Request_recv_FOA}.  Note that
  this would introduce another asymmetry between sends and receives: A send
  starts with a communicator pointer while an incoming message starts with a
  context id.

\item There are two cases here depending on whether a matching request was
  found by \mpidfunc{MPID_Request_recv_FOA}.  

\item If a request was found, there is an available user buffer.  We need to
  tell the socket layer to transfer data to that buffer.  This is relatively
  easy if the datatype is contiguous (and only slightly more difficult if it
  is easily described by an \code{iovec}).  Handling the more general datatype
  case requires a little more care.  In either case, we need to tell the
  socket layer where the data should be put, how many bytes to process, and
  how to move them.  This is a \emph{stream}; the low-level code is
  responsible for handling this.

%             /* Initiate an unpack.  Note that the data may not all 
%                have arrived yet.  This must:
%                1. Process all available data (up to lpacket->msg_bytes)
%                2. If not all data is available, indicate to low-level
%                   communication that the data is stream.
%                3. Indicate to stream what field to set in the request
%                   when all data is read and unpacked.
%             */

\item If a request was not found, then one was allocated and returned by
  \mpidfunc{MPID_Request_recv_FOA}.  

\item Check to see if this is a ready-send.  If so, then there is an error
  because a matching request was not found.  Return an error indication to the
  sending process and tell the socket layer to discard the data.
  Note that this process must read the data since it is being sent.  Since
  this is an error case, there is no reason to try and optimize this case by
  asking the sender to stop sending the data.

  Question: do we also want to generate an error message?  Do we want to
  invoke the error handler on the intended communicator on the receiving
  process?  Brian votes yes for generating an error message.  Bill votes for
  using the error handler on the intended communicator.  

\item If it isn't a ready-send, then we need to allocate space to hold the
  data and copy it in.  We allocate the space with
  \mpidfunc{MPID_EagerAlloc}. 

  We cannot
  use \code{malloc} instead of \mpidfunc{MPID_EagerAlloc}, at least on all
  platforms (and we must enforce 
  resource limits) because the \code{malloc} might fail, and our flow-control
  promise is that there is space available.  If we depend on \code{malloc}, we
  need to provide for a negative acknowledgement on eager messages.

  Note: we could leave the data in the internal buffer (i.e., the buffer used
  by \code{GetNextPacketHeader} or by \code{bsocket}, not the socket buffer
  managed by the OS) at least briefly,
  avoiding one copy.  The added complexity of remembering where to move the
  data when we needed to probably outweighs the small gain in efficiency.

\item Once the data is allocated (the allocation always succeeds as long as
  the flow control algorithm is correct), a stream is set up to receive the
  data as bytes into the allocated buffer.  

\item Once all of the bytes of the stream have been read, the function
  \code{MPID_Eager_complete_func}, which was passed to
  \mpidfunc{MPID_Stream_irecv_tcp}, is called with the fifth argument of
  \mpidfunc{MPID_Stream_irecv_tcp}, which is the pointer to the request.  This
  allows the communication agent to 
  \begin{enumerate}
  \item Indicate that the data is available by setting the eager buffer
    pointer
  \item Clear the busy flag
  \item For multithreaded processes, where one thread is waiting on this
    request, signal the waiting thread.
  \end{enumerate}
  The implementation of \code{MPID_Eager_complete_func} here uses
  \code{pthread_cond_signal} to release a thread waiting for this transfer to
  complete.  Pthreads are used here only as an example; for the
  implementation, we will use a thread abstraction layer (as yet undefined) to
  implement condition variables and the like.

\end{enumerate}

\mpidconst{MPID_Hid_request_to_send}:
\begin{enumerate}
\item This starts in the same way as \mpidconst{MPID_Hid_eager}, with a call
  to \mpidfunc{MPID_Request_recv_FOA}.  

\item If the request is found, this executes the same steps as in the
  \mpifunc{MPI_Irecv} case for found and not an eager message (e.g., it
  returns an \mpidconst{MPID_Hid_ok_to_send} packet).

\item Otherwise, it saves the information needed to request the data at a
  later time. Note
  that the \mpids{MPID_Hid_request_to_send}{length} field is needed to
  implement \mpifunc{MPI_Iprobe}. 
\end{enumerate}

\mpidconst{MPID_Hid_ok_to_send}:
\begin{enumerate}
\item First, the request is identified from the request id in the packet.

\item The data will be sent with a packet header of
  \mpidconst{MPID_Hid_data}.  There are two subcases:
  \begin{enumerate}
  \item The data is contiguous.  In this case, a simple \code{MPID_Rhcv_tcp}
    call can send all of the data.

  \item The data is not contiguous.  In this case, we create a message
    stream.  

   Question: do we need to call a routine to release any stream or segment
   resources when we are done?  Or do we make this automatic with the stream
   routines? 

   Note that because data can be sent with \mpidfunc{MPID_Rhcv} (because it
   was contiguous or a simple \code{iovec} but received with a stream (since
   the receiving datatype was a complex noncontiguous datatype), the
   implementation of stream cannot insert any extra bytes into the data
   stream.  

  \end{enumerate}
\end{enumerate}

\mpidconst{MPID_Hid_data}:
\begin{enumerate}
\item This indicates data that matches a request.  The request id is part of
  the packet header.  The data must be received with a stream since all of the
  data may not be available yet.  

  Question: Do we need to call a routine at completion to release any stream
  or segment resources when we are done? Or do we make this automatic with the
  stream routines?

\end{enumerate}

\mpidconst{MPID_Hid_control}:
\begin{enumerate}
\item This case is for miscellaneous control messages.  The only control
  message defined so far is the ready-send error.  Others might include abort,
  exit, are-you-alive, etc.
\end{enumerate}

\ifcodefirst
\else
\fileinclude{samples/agent_tcp.c}
\fi

\implementation{{MPI_Wait}}

\ifcodefirst
\fileinclude{samples/wait_tcp.c}
\fi

Notes:
\begin{enumerate}
\item In the single-threaded case, we need only check the
  \mpids{MPID_Request}{xfer_completed} flag. 
  If set, then copy out the \mpiconst{MPI_Status} data (if the \code{status}
  pointer is not \mpiconst{MPI_STATUS_NULL}) and free the request.  Otherwise, 
  call the communication agent in blocking mode (wait until something happens)
  and then check again.

\item In the multi-threaded case, if the request is not complete, we want to
  transfer control to another thread, particularly to the thread running the
  communication agent if there is one, so that the message can be completed.
  This allows the implementation to avoid spinning on the
  \mpids{MPID_Request}{xfer_completed} flag.  However\index{thread overhead!request
    completion}, it does force the use of a thread mutex to guard the
  \code{xfer_completed} flag and a thread condition variable to release the
  particular request.  (We are probably missing guards around \code{xfer_completed}
  elsewhere as well.)

  The sample code shows the use of condition variables with pthreads.
  The routine \code{pthread_cond_wait} atomically releases the
  \code{request_mutex} and 
  waits for the condition variable \code{cond}.  The condition variable is set
  when another thread executes \code{pthread_cond_broadcast} (or
  \code{pthread_cond_signal} to this particular thread).
  Note that this algorithm requires either a mutex per request or (less
  scalably) a mutex 
  on all requests (but local to the calling MPI process).  
  Note that a single mutex may be more appropriate for the multiple completion
  routines such as \mpifunc{MPI_Waitsome} or \mpifunc{MPI_Waitall}.
  The above approach also requires that the communication agent execute a
  \code{pthread_cond_signal} to release this waiting thread.  Note that by
  storing the thread id of the waiting process, we can avoid calling
  \code{pthread_cond_signal} when no process is waiting.  Question: is
  \code{pthread_cond_signal} expensive enough to make it worth implementing
  this optimization?  (Note that at most one process will be waiting on this
  condition variable.)

  Another alternative is to simply busy wait on
  \mpids{MPI_Request}{xfer_completed}. 
  
  Note that pthreads are not available on all platforms; we may need to
  implement other approaches as well.
  
  Question: How well do pthread condition variables work on our important Unix
  platforms (e.g., Linux, Solaris, AIX, etc.)?  Is there a similar Windows NT
  approach, or is something different required?  Brian notes that Windows
  supports semaphores, which can be used to implement condition variables.

\item Missing from this routine is any processing that may be required after
  the data transfer completes.  Note that, particularly in the case of
  \mpidfunc{MPID_Rhcv}, the \code{xfer_completed} field is set once data is
  transfered.  However, if some temporary buffer was allocated for sending or
  a datatype reference count was incremented, some further steps must be taken
  when finishing the request off.
\end{enumerate}

\ifcodefirst
\else
\fileinclude{samples/wait_tcp.c}
\fi

%\pagerule\par
%The remaining text is leftover from a previous version and will be mined as
%appropriate for new text.  Read at your own peril.
%
%\pagerule\par

\subsubsection{\shmemname}

This method is very similar to the \tcpname\ method.  In fact, a simple
implementation could use the very same code, changing only the implementation
of the \code{GetNextPacketHeader}, \mpidfunc{MPID_Rhcv}, and the various
Stream routines.  The method must implement all of the control messages (the
\mpidconst{MPID_Hid_xxx}), but arrange for data exchanges to use shared
memory.  One refinement is to place the message queues (which are the same as
the request queues; that is, these are the \code{MPID_Request_xxx_FOA} queues)
in shared memory, 
allowing any process to update the queues.  Note that this does change the
\mpidfunc{MPID_Isend} implementation, because the sender may in fact discover
the matching receive.  More on this below.

\paragraph{Implementing Streams in Shared Memory.}
Streams are implemented by allocating \emph{two} (contiguous) regions in
shared memory.  Call these region 1 and region 2.  The (send) stream starts by
placing data in region 1.  It then informs the destination process that data
is available in that region by providing the address and length of that
region (probably by sending it a message).  The sending process then fills in
region 2 with the next section of data and informs the destionation that that
region is ready.  On the receiving side, as each region is read, the receiver
indicates that to the sender.  Because two buffers are used, one process can
be filling the one while the other process is reading the other buffer.  This
can effectively double the bandwidth of a transfer that make use of a single
buffer.

% We still need streams here, but since we're responsible for
% everything, we may need to send messages back and forth for each
% segment.  This may add an additional message type,
% \mpidconst{MPID_Hid_next_data_t}.  However, this could be hidden within the
% lowest level code, making streams a good abstraction in this case as well.

Issues:
\begin{enumerate}
\item How does the sending process indicate that the next buffer is ready?
  Does it set a flag at the end of the buffer?  Does it send a message?  The
  advantage of sending a message is that it allows the destination process to
  check one location (or queue) for input, regardless of how many streams and
  other operations are taking place.  Since the message is short, it could be
  sent in a single cache line, making the cost very low (particular in an
  implementation that used lock-free queues for messages).  

\end{enumerate}

\paragraph{Allocating Eager Buffer Space.}
In this device, the message queues are kept in shared memory, allowing a
sending process to immediately know whether a posted receive exists.  In this
case, an eager message can be delivered by allocating the eager buffer from
shared memory.  The sending process then fills the eager buffer with the data,
sets the eager buffer fields in the request (e.g., \code{MPID_Eager_info}),
and clears the \code{busy} flag in the request.  When the eager message is
finally received, the eager buffer is returned to shared memory.

\implementation{{MPI_Irecv}}

The initial steps are the same as in the \tcpname\ case.  A major difference
is caused by the fact that the requests can be (and we will assume are)
present in shared memory.  In this case, each process can check for a matching
queue element directly. The two cases are
\begin{enumerate}
\item The request was not found.  This is the same as the \tcpname\ case.
\item The request was found.  This is close to the \tcpname\ case,
  particularly for the complex-datatype case.  The particular cases are
  \begin{enumerate}
  \item The data has already been delivered.  It is either within the request
    (very short) or in a shared-data buffer used for eager messages.  As in
    the \tcpname\ case, copy the data into the user's buffer using
    \mpidfunc{MPID_Unpack}.  Note that in the multi-method case, if the device
    is heterogeneous, we still need the \mpidconst{recved_format} etc. fields
    since the data may have been packed for a communicator containing
    processes with different data represenations.

    Question: who frees the eager buffer space?  Does the receiver free it
    after unpacking the data, or do we ask the sender to free it?  We might
    want the sender to free it if the sender allocated it.  Brian votes for
    the receiver freeing it.  

  \item The data has not been delivered.  This follows the \tcpname\ case,
    since the data must be delivered by having the sender place some of it in
    shared memory, then signal the receiving process, and so on.  The
    receiving process creates the corresponding segment (including one or more
    buffers in shared memory?) and sends the address and size of that buffer
    to the sender using \mpidfunc{MPID_Rhcv} to deliver a message to the
    sender's incoming message queue.  Further processing is handled by the
    communication agent (including any double buffering of the transfer).
  \end{enumerate}
\end{enumerate}


\implementation{{MPI_Isend}}

Check remote queue for matching receive request (using
\mpidfunc{MPID_Request_send_FOA}).  
\begin{enumerate}
\item If not found, insert (an unexpected receive).  There are two cases: a
  short message that uses an eager delivery, and a long message that requires
  a rendezvous.  If an eager delivery, allocate the space in shared memory,
  copy into the eager buffer, and set the corresponding eager fields in the
  request.  If a rendezvous, set the matching data
  including the sender's 
  request id, use \mpidfunc{MPID_MemWrite_ordered} to clear the busy flag, and
  return.  
\item If found, match (remove from queue) and begin transfer.  There are
  several cases:
    \begin{enumerate}
    \item The destination buffer is in shared memory and the datatype is
      simple (not the case in this example, but one that must be considered). 
      In this case, copy to the destination buffer (basically
      \mpidfunc{MPID_Pack}), followed by \mpidfunc{MPID_MemWrite_ordered} to
      set 
      the \code{xfer_completed} flag in the receive request.  Also mark the
      send request as (transfer) completed.
    \item The destination buffer is not in shared memory, but the total
      message length is small.  In this case, use \mpidfunc{MPID_Pack} to pack
      the message into a shared-memory buffer (Question: who allocates this?
      The sender?  The receiver?) and mark the send request as complete.
      Use \mpidfunc{MPID_MemWrite_ordered} to mark the receive request as
      having 
      the data but not yet complete (e.g., a final step is needed to move the
      data from shared memory into the user's buffer).  Question: how do we
      indicate this in the request?
    \item The destination buffer is not in shared memory and the message
      length is large.  In this case, the message must be delivered in
      segments, using shared memory to effect the transfer.  One option is to
      place the first segment into a designated shared-memory buffer (who
      allocates it?) and indicate its location in the receive request.  All
      further transfers will be accomplished with the communication agent.
    \end{enumerate}
    Subsequent transfers will be handled by the communication agent.
\end{enumerate}

\implementation{{MPI_Wait}}

Again, this is much like the \tcpname\ case.  Note, however, that the
\mpids{MPID_Request}{xfer_completed} field can, in some cases, be set by the
sending process (particularly for contiguous receive buffers that are in
shared memory), so this field must be marked \code{volatile}, even in the
single-threaded process case.

Also note the case that all of the data (or the last segment) has been
delivered (by being placed 
into shared memory), but the final \mpidconst{MPID_Unpack} to move that data
into the user's buffer has not been performed.  How is this indicated?  
In the \shmemname\ case, do we indicate
this with a bit or field in the \mpidconst{MPID_Request}?  Do we use \mpids{MPID_Request}{data_available} field that is used to indicate
that memory is available?  This field could be used instead of an explicit
message to manage the communication.  

Question: If we use a \code{data_available} field, how do we avoid
busy-waiting?  Do we simply poll and yield?

\implementation{{Communication agent}}

The communication agent is responsible for making progress on a stream (the
sending of segments).  There is no special action on the receiving end.  On
the sending end, when the \mpidconst{MPID_Hid_ok_to_send} is received, the
send is started.  Note that a handshake is required on the transfer of data
from the sender to the receiver because the buffer that is used for the
transfer must be explicitly filled by the sender and emptied by the receiver.
Question: Do we want to double buffer?  How does that change the messages
exchanged between the sender and receiver?

Brian suggests a variation of this could use three buffers (why three?).  If
no buffers are available, the 
sender assumes that the receiver is busy with other work and simply sends a
message saying that more data is available.  

If the communication agent is in a separate thread, we can wake up, spin a
little (checking the incoming message queue), and yield.  If single-threaded,
we should spin for roughly the round-trip message time (an internal message,
not an MPI message) and then yield (on Linux, with \code{sched_yield}).
Question: do we want the option of using something like the System V
semaphores to implement a condition variable for the communication agent, so
that updates to the incoming message queues would wake up the agent?

Question: How do we notify the agent that we are ready for another segment?
Does the communication agent wait on something?  What?  What about the case
where there are 100 pending requests?
An easy way may simply be to send a message, following the approach in
\tcpname.  A more complex approach, introduced in the discussion of
\mpifunc{MPI_Wait} above, is to use a field within the request (or even in the
shared-memory transfer area itself) to indicate that a segment has been copied
into or out of shared memory.  

Note that waiting on fields in individual request items is ok in this simple
example, but becomes less suitable when there are a significant number of
pending requests.  Question: we could establish a bit vector of pending
requests (each bit represents a request and the position matches the index);
this could be atomically updated into indicate that a process's 
request was ready for operation.  However, this may not be much better than
sending a message to the communication agent.

Another approach is to both send a message and set a flag.  Then the receiver
can continue receiving if data is available but is guaranteed to continue
later if he has gone off to do other work (without having to check every
pending request). 

The major role of the communication agent is to respond to messages, and to
ensure that the stream operations make progress.

\subsubsection{\shmemallname}
This is a variation of the \shmemname\ case where all memory is visible to all
processes.  Note that in this case, the simple transfers can be
made directly, once the send and receive requests are matched.  Also simple
are the cases where either the sender or the receiver provides a contiguous
datatype; in that case, the other process executes either
\mpidfunc{MPID_Unpack} or \mpidfunc{MPID_Pack}.  Only in the case where both
processes specify complex datatypes may it be necessary to transfer data
through a cannonical, contiguous representation.

\subsubsection{\vianame}

This method is very similar to the \tcpname\ method, with the difference that
data can be written into and read from remote memory that has been registered.

The principle difference between this method and the \tcpname\ method is in
the way in which messages are delivered and the ability to deliver contiguous
(and perhaps simple \code{iovec} structures) directly.  Because of this, there
are two different types of rendezvous:
\begin{enumerate}
\item A contiguous to contiguous (or simple \code{iovec} to simple
  \code{iovec}) transfer 
\item A stream transfer, requiring a copy into an intermediate buffer.
  This has two subcases:
  \begin{enumerate}
  \item One process has contiguous data
  This case is much like the \shmemname\ case, since the data in contiguous
  form can be moved directly with a remote write or read, and the process with
  the non-contiguous data either packs or unpacks it as appropriate.
  \item Neither process has contiguous data.  This is similar to the
  stream implementation for the \tcpname\ case (data is packed into a
  contiguous buffer, copied to a remote contiguous buffer, and then unpacked).
  \end{enumerate}
\end{enumerate}
These suggest special cases that can be optimized.

\implementation{{MPI_Irecv}}

This follow the same approach as in \tcpname.  
If a message must be sent (e.g., a
\mpidconst{MPID_Hid_ok_to_send}), \mpidfunc{MPID_Rhcv} sends it by writing to
a pre-registered location.  Flow control provides information on where to
write (e.g., multiple incoming locations can be allocated, and either
information on each message updates what is free or separate flow control
messages are sent).   

Note that because the data is noncontiguous, it must be delivered into some
intermediate memory in contiguous form\footnote{We assume that either the
  method cannot handle any noncontiguous data or that the noncontiguous
  datatype in this example cannot be directly handled.}.  We further assume
that memory to be used for such transfers has already been registered; the
location of that memory can be communicated back to the sender in the
\mpidconst{MPID_Hid_ok_to_send} message.  

Question: who manages this pool of pre-registered memory?  Note that since
target memory must be registered, the receiver must be the one that specifies
this.  

Further processing is handled by the communication agent.

\implementation{{MPI_Isend}}

This also follows the approach in \tcpname.

Note that since we don't know whether the destination is going to provide a
contiguous, registered-memory buffer or not, we shouldn't take the step of
transfering the first segment into an internal registered-memory buffer.

\implementation{{MPI_Wait}}

The same as for \tcpname.  

\implementation{{Communication agent}}

The agent must respond to messages in much the same way as \tcpname.  However,
the process of sending a stream of segments may be different. 

Question: how do we want to tell the process at the other end of the stream to
read the current block of data?

\subsection{Nonblocking Send and Receive with Contiguous Data.}

The major difference in this case is for the \shmemname\ and \vianame\ cases,
where it may be possible to send data directly from one user buffer to
another.  

\begin{via}
  Question: How do we indicate that a message has been transfered?  Do we want
  to use a remote write to set the \mpids{MPID_Request}{xfer_completed} field
  directly, rather than sending a message?  Is it better to receive a
  completion message from the sender or to spin-wait on
  \mpids{MPID_Request}{xfer_completed}?   The latter is important for systems,
  like LAPI, that can increment a remote counter when the transfer is
  complete.  

\end{via}

\subsection{Blocking Send and Receive with Noncontiguous Data.}
\label{sec:blocking-optimization}
(not complete)

Once a transfer is initiated, particularly in the \shmemname\ and 
\vianame\ cases, instead of sending a separate handshake message, we could set
a flag 
value in the transfer buffer itself, particularly for the cases involving
non-contiguous data where the transfer buffer is not the same as the user's
buffer.  

By initiated here, we mean once the first block has actually been
transferred.  This ensures that the sender has received the
\mpidconst{MPID_Hid_ok_to_send} and has started to act on it.  If the sender
knows (or has been told in the \mpidconst{MPID_Hid_ok_to_send} message) that
the receiver is blocking, it can switch to this alternate method for
indicating that data has been transferred.

Note also that in this case the reference counts for the datatype and
communicator in the operation do not need to be changed.

\subsection{Cancel}
Cancelling a receive is relatively easy (unless speculative receives
are implemented).   This simply removes a posted but unmatched receive
from the receive queue.  This needs a
\mpidfunc{MPID_Request_recv_remove} call.

Cancelling a send requires more effort, particularly when the receive
queue is not directly accessible to the sender (as it is in the
\shmemname\ method).  

Note that once a send has been matched, it cannot be cancelled.  Only sends
that have not been matched may be cancelled. (For these purposes,
\mpifunc{MPI_Prove} and \mpifunc{MPI_Iprobe} do not count as matching a
message.  See 3.2.9 in the MPI-2 Standard.).

\subsubsection{\tcpname}
\ifcodefirst
\fileinclude{samples/cancel_tcp.c}
\fi

Notes:
\begin{enumerate}
\item A request is identified by the pair of sender request id and
source process.  Question: should this be the local process id or the
pair of context id and rank in communicator?

\item Where are the packets allocated for the cancel acknowledgement?
How are they recovered once 
they are sent? One possibility is to first link the packets onto a
pending list and include a flag in the packet that is used as the
completion flag by the \code{MPID_Rhcv_tcp} call.  Then this list can
be occassionally checked for complete items, particularly when such a
packet must be allocated.  This list then become a list of ``items to
be freed soon''.

\item The code shows a request \code{state}.  Do we really need this?
If we do, we need to check this elsewhere.  We could use this to
remember other states, such as ``delete temporary buffers when
complete''.

\end{enumerate}

\ifcodefirst
\else
\fileinclude{samples/cancel_tcp.c}
\fi

\subsubsection{\shmemname}
(not done)
(direct access to receiver's queue may make this relatively easy)

Question: Do we just want to have \mpidfunc{MPID_Request_recv_cancel}
and \mpidfunc{MPID_Request_send_cancel} (instead of the current
\mpidfunc{MPID_Request_cancel})? 

\subsubsection{\shmemallname}
(not done)

\subsubsection{\vianame}
(not done)
(Is this like \tcpname?)

\subsection{Multiple Completion}
(not done)

(This section needs to examine testsome and similar routines)



\section{Special Issues}
\label{sec:special-issues}

\subsection{Heterogenity}
\label{sec:hetero-issues}

Handling communication between systems with (potentially) different data
representations is difficult, particularly when the differences are more than
just differences in the lengths of datatypes (non-IEEE floating point formats
are particularly painful).

Some issues that have come up:
\begin{enumerate}
\item When using XDR\index{XDR}, the assignment of native types to XDR types
  is not as easy as it appears.  For example, the external representation for
  a C \code{long} provided by the \code{xdr_long} actually moves 32 bits
  even for systems where a \code{long} is 64 bits.  I.e., the XDR types (e.g.,
  in \code{xdr_long} match a specific set of sizes, not the particular sizes
  chosen by the C compiler.  Thus, when choosing the XDR routines to use, the
  sizes of the datatypes need to be considered, as well as whether the local
  processor provides \code{xdr_longlong} or \code{xdr_hyper} (note that the
  XDR type ``hyper'' is defined as an 8-byte integer (see RFC1014) and should
  be available everywhere).

\end{enumerate}

\section{MPI Operations}
\label{sec:mpi-operations}
This section describes the implementation of the MPI operations.  The
descriptions may include discussion of some implementation issues.
These are split up according to function, and roughly (but not
exactly) match the MPI standard.  Each of these has a corresponding
directory in the MPI source tree.  Note that this means that the
directory structure does not exactly match the chapter structure of
the MPI Standard.

For each routine, the description may be text, usually indicating that no ADI
routines are involved.  For example, many of the process topology and other
informational routines fit into this category.
For routines that make use of the ADI, the description will often be presented
as follows:
\begin{adi3}The ADI3 routine(s) that are called
\begin{mmadi}One possible implementation of the above routine(s), using
  lower-level ADI3 routines.
\begin{core}One possible implementation of the MMADI routines, using just the
  \code{MPID_CORE}. 
\end{core}
\end{mmadi}
\end{adi3}
The purpose of this description is to evaluate the ADI-3 design (at all
levels, not just to core and the most general, top level) for the
implementation of MPI.  

In cases where the implementation depends on the properties of the
underlying communication layer, the different implementations are
shown.  For the purposes of illustration in this document, there are
four different classes of communication layers:
\begin{description}
\item[\tcpname.]Conventional, two-sided (e.g., send/receive)
messaging.  Examples are TCP, UDP, and MPI-1.  

\item[\shmemname.]Shared memory.  This assumes separate processes that
can share some memory.  Operations on this shared memory are
accomplished using the usual language-defined methods for accessing
memory.  However, most user memory (e.g., user-declared
variables) is not shared.  Examples are Unix System V shared memory
segments and shared \code{mmap} regions.  

\item[\shmemallname.]All memory is shared.  This assumes that separate
  processes have some way to access all of the memory in another process.
  Under Linux, this can be accomplished using \code{ptrace} and the
  \file{/dev/proc} filesystem.  Under IRIX, processes created with
  \code{sproc} may use \code{prctl} with \code{PR_ATTACHADDR} to share
  memory. Another possibility is a system where MPI 
  processes are really just separate threads in a single OS process; this
  requires that the compiler make all global variables thread-private (the NEC
  SX-4 offered such an option).

\item[\vianame.]Distributed memory.  This assumes that there is a
method for remotely accessing memory in another process.  Operations
on remote memory are \emph{not} accomplished with language features;
instead, routine calls implement remote memory operations.  Examples
of this include VIA, Cray SHMEM, and IBM LAPI.
\end{description}

\subsection{Attributes}

Error classes defined for keyvals and attributes:
\mpiconst{MPI_ERR_KEYVAL} (note that this is new in MPI-2).


\paragraph{MPI-1 Attribute Functions.}
The following five functions are deprecated.  These are implemented in
terms of MPI-2 functions.  In case an error is encountered, they must 
ensure that the original routine name is reported in any error
message.  For example, if \code{MPI_ATTR_DELETE} is called by the
user and an error occurs when that routine calls
\code{MPI_COMM_DELETE_ATTR}, then the error message returned to the
user will indicate \code{MPI_ATTR_DELETE}, not
\code{MPI_COMM_DELETE_ATTR}.

\subsubsection{\mpifunc{MPI_ATTR_DELETE}}
\begin{adi3}
Set error handler to return on error (see
Section~\ref{sec:err-handling-nested}).
Call \mpifunc{PMPI_COMM_DELETE_ATTR}.
Restore error handler and if an error, invoke the correct handler.
All layered calls manage the error handler in this way.
\end{adi3}

\subsubsection{\mpifunc{MPI_ATTR_GET}}
Calls \mpifunc{PMPI_COMM_GET_ATTR}.

\subsubsection{\mpifunc{MPI_ATTR_PUT}}
Calls \mpifunc{PMPI_COMM_SET_ATTR}.

\subsubsection{\mpifunc{MPI_KEYVAL_CREATE}}
Calls \mpifunc{PMPI_COMM_CREATE_KEYVAL}.

\subsubsection{\mpifunc{MPI_KEYVAL_FREE}}
Calls \mpifunc{PMPI_COMM_FREE_KEYVAL}.

\subsubsection{\mpifunc{MPI_COMM_CREATE_KEYVAL}}
Calls \mpidfunc{MPID_Keyval_create} with object type
\mpidconst{MPID_Comm_t}. 

Question: To help catch user errors, should this try not to reuse the
same keyval (i.e., the integer corresponding to the keyval structure)
that was recently freed by \mpifunc{MPI_COMM_FREE_KEYVAL}?

\subsubsection{\mpifunc{MPI_COMM_FREE_KEYVAL}}
Test that the keyval belongs to communicators.  Call
\mpidfunc{MPID_Keyval_free}. 

%\subsubsection{\mpifunc{MPI_COMM_NULL_COPY_FN}}

%\subsubsection{\mpifunc{MPI_COMM_DUP_FN}}

\subsubsection{\mpifunc{MPI_COMM_GET_ATTR}}
\begin{adi3}
\mpidfunc{MPID_Attr_find}, followed by access to the value.

Question: does this require a thread lock, or is that burden on the user?  Do
we want to have the ability to detect thread races (e.g., a flag to indicate
that another thread has touched the list)?  Note that a valid MPI
program could have a (deliberate) race condition here, so any checking
for thread races should be an optional debugging feature.

This suggests the following
\begin{algorithm}
lock (attribute list)
find 
access value
unlock
return value
\end{algorithm}

The \code{lock} and \code{unlock} may be defined as follows:
\begin{verbatim}
#if defined(MPICH_ENABLE_USER_THREADS}
#define DS_LOCK(comm,ds) \
    if (comm->multi_threaded) mutex_lock(&ds->mutex);
#define DS_UNLOCK(comm,ds) \
    if (comm->multi_threaded) mutex_unlock(&ds->mutex);
#else
#define DS_LOCK(comm,ds)
#define DS_UNLOCK(comm,ds)
#endif
\end{verbatim}
This allows each communicator support different levels of
threadedness, providing more modular control of threadedness.
\end{adi3}


\subsubsection{\mpifunc{MPI_COMM_SET_ATTR}}
\begin{adi3}
\mpidfunc{MPID_Attr_find}, followed by access to the value:
\begin{algorithm}
lock
find
if not present, add
set
unlock
\end{algorithm}
Attributes are also used to control special characteristics.  Within
the lock, it must also call \mpidfunc{MPID_Attr_extension} with the
communicator, keyval, and attribute.  

Question: should the device be responsible for setting fields such as
\mpids{MPI_Comm}{multi_threaded} or should the MPI routine first check
the attribute keys?

Question: should \mpidfunc{MPID_Attr_extension} be called for
attributes set on windows and datatypes?

Note: Lists need a separate head and list element; the head contains
the lock and other helpful items such as a count of the number of
elements.  Should we have a \mpidconst{MPID_List_head_t} and
\mpidconst{MPID_List_elm_t} for the head and elements of a list?  Are
the attribute lists special enough that there should be a separate
attribute list type (\mpidconst{MPID_Attr_list_head_t} and
\mpidconst{MPID_Attr_list_elm_t})?   Is the list head itself a member
of the structure, or is a pointer to the list head used?
\end{adi3}

\subsubsection{\mpifunc{MPI_COMM_DELETE_ATTR}}
\begin{adi3}
\mpidfunc{MPID_Attr_delete}.
\begin{algorithm}
lock
find
remove
unlock
\end{algorithm}
\end{adi3}

\subsubsection{\mpifunc{MPI_TYPE_GET_ATTR}}
See \mpifunc{MPI_COMM_GET_ATTR}.

\subsubsection{\mpifunc{MPI_TYPE_SET_ATTR}}
See \mpifunc{MPI_COMM_SET_ATTR}.

\subsubsection{\mpifunc{MPI_TYPE_DELETE_ATTR}}
See \mpifunc{MPI_COMM_DELETE_ATTR}.

\subsubsection{\mpifunc{MPI_TYPE_CREATE_KEYVAL}}
See \mpifunc{MPI_COMM_CREATE_KEYVAL} with object type
\mpidconst{MPID_Datatype_t}. 

\subsubsection{\mpifunc{MPI_TYPE_FREE_KEYVAL}}
See \mpifunc{MPI_COMM_FREE_KEYVAL}.

%\subsubsection{\mpifunc{MPI_TYPE_NULL_COPY_FN}}
%\subsubsection{\mpifunc{MPI_TYPE_DUP_FN}}

\subsubsection{\mpifunc{MPI_WIN_CREATE_KEYVAL}}
See \mpifunc{MPI_COMM_CREATE_KEYVAL}.

\subsubsection{\mpifunc{MPI_WIN_FREE_KEYVAL}}
See \mpifunc{MPI_COMM_FREE_KEYVAL}.

\subsubsection{\mpifunc{MPI_WIN_SET_ATTR}}
See \mpifunc{MPI_COMM_SET_ATTR}.

\subsubsection{\mpifunc{MPI_WIN_GET_ATTR}}
See \mpifunc{MPI_COMM_GET_ATTR}.

\subsubsection{\mpifunc{MPI_WIN_DELETE_ATTR}}
See \mpifunc{MPI_COMM_DELETE_ATTR}.
 
%\subsubsection{\mpifunc{MPI_WIN_NULL_COPY_FN}}

\subsection{Info}
\paragraph{Info.}

Error values defined for info:
\mpiconst{MPI_ERR_INFO_KEY},
\mpiconst{MPI_ERR_INFO_VALUE},
\mpiconst{MPI_ERR_INFO_NOKEY}.

Constants defined for info: 
\mpiconst{MPI_MAX_INFO_KEY},
\mpiconst{MPI_MAX_INFO_VAL}.
Note that the MPI standard sets limits on the ranges that these can take.

Predefined info keys are:
\mpiconst{access_style}\index{MPI_Info!keys!access_style}, 
\mpiconst{appnum}\index{MPI_Info!keys!appnum}, 
\mpiconst{arch}\index{MPI_Info!keys!arch}, 
\mpiconst{cb_block_size}\index{MPI_Info!keys!cb_block_size}, 
\mpiconst{cb_buffer_size}\index{MPI_Info!keys!cb_buffer_size}, 
\mpiconst{cb_nodes}\index{MPI_Info!keys!cb_nodes}, 
\mpiconst{chunked}\index{MPI_Info!keys!chunked}, 
\mpiconst{chunked_item}\index{MPI_Info!keys!chunked_item}, 
\mpiconst{chunked_size}\index{MPI_Info!keys!chunked_size}, 
\mpiconst{collective_buffering}\index{MPI_Info!keys!collective_buffering}, 
\mpiconst{external32}\index{MPI_Info!keys!external32}, 
\mpiconst{false}\index{MPI_Info!keys!false}, 
\mpiconst{file}\index{MPI_Info!keys!file}, 
\mpiconst{file_perm}\index{MPI_Info!keys!file_perm}, 
\mpiconst{filename}\index{MPI_Info!keys!filename}, 
\mpiconst{host}\index{MPI_Info!keys!host}, 
\mpiconst{internal}\index{MPI_Info!keys!internal}, 
\mpiconst{io_node_list}\index{MPI_Info!keys!io_node_list}, 
\mpiconst{ip_address}\index{MPI_Info!keys!ip_address}, 
\mpiconst{ip_port}\index{MPI_Info!keys!ip_port}, 
\mpiconst{native}\index{MPI_Info!keys!native}, 
\mpiconst{nb_proc}\index{MPI_Info!keys!nb_proc}, 
\mpiconst{no_locks}\index{MPI_Info!keys!no_locks}, 
\mpiconst{num_io_nodes}\index{MPI_Info!keys!num_io_nodes}, 
\mpiconst{path}\index{MPI_Info!keys!path}, 
\mpiconst{random}\index{MPI_Info!keys!random}, 
\mpiconst{read_mostly}\index{MPI_Info!keys!read_mostly}, 
\mpiconst{read_once}\index{MPI_Info!keys!read_once}, 
\mpiconst{reverse_sequential}\index{MPI_Info!keys!reverse_sequential}, 
\mpiconst{sequential}\index{MPI_Info!keys!sequential}, 
\mpiconst{soft}\index{MPI_Info!keys!soft}, 
\mpiconst{striping_factor}\index{MPI_Info!keys!striping_factor}, 
\mpiconst{striping_unit}\index{MPI_Info!keys!striping_unit}, 
\mpiconst{true}\index{MPI_Info!keys!true}, 
\mpiconst{wdir}\index{MPI_Info!keys!wdir}, 
\mpiconst{write_mostly}\index{MPI_Info!keys!write_mostly}, 
\mpiconst{write_once}.

\paragraph{Defining New Keys.}
The MPI Forum has not resolved an ambiguity in the definition of
info.  While it was clear during many of the discussions that the
expectation was that \code{MPI_Info}\index{MPI_Info} could be used by
layered implementations of parts of MPI, particularly the I/O part,
IBM did not interpret the standard this way and their interpretation
is both consistent with the standard and offers a feature not
otherwise available (specifically, the ability to determine what info
keys are recognized by the implementation).  Question: we don't want
to do this, do we?

Note also that the interface to access the info values is not
thread-safe, since it has the implicit assumption that the number of
keys in an info object does not change unless the same thread changes it.
For example, consider this sequence:
\begin{verbatim}
    MPI_Info_get_nkeys( info, &nkeys );
    MPI_Info_get_nthkey( info, nkeys-1, keystring );
\end{verbatim}
In a multi-threaded environment, another thread may have called
\begin{verbatim}
   MPI_Info_delete( info, "any-key-in-info");
\end{verbatim}
after \mpifunc{MPI_Info_get_nkeys} but before
\mpifunc{MPI_Info_get_nthkey}.  There's no way to really fix this, but
we can at least raise the issue in the manual pages and generate
helpful error messages in this case.  We may also want to add an
extension that raises a special error code if a different thread
modifies an \code{MPI_Info} while any of the routines with a notion of
the ``current'' state of info are operating on it.

\paragraph{Implementing Info.}
There are two ways to handle \code{MPI_Info}.  One is to implement a general
mechanism for handling key/value pairs, much like the code that is part of
\file{mpich/src/misc2}.  The other is to implement only the defined keys that
MPICH2 needs.  This is the approach used by IBM; in this model, the known keys
are not stored; instead, the keys are mapped to a predefined set of values
(e.g., to an enum).  This makes it easy to extract a value from a particular
info object (the values can be preconverted into booleans or integers and
stored in a small array); further, it provides a way (which is otherwise
lacking) to 
indicate which key values are known to the implementation.  
The problem with restricting keys to those known to the implementation is it
prevents using \code{MPI_Info} to pass information to another subsystem, such
as to the process manager and allocator (through \code{MPI_COMM_SPAWN}). 
Since \code{MPI_Info} is not used by any performance-critical functions
(\code{MPI_Info} is only used in \code{MPI_Alloc_mem}, \code{MPI_Comm_accept},
\code{MPI_Comm_connect}, 
\code{MPI_Comm_spawn}, \code{MPI_Comm_spawn_multiple}, \code{MPI_File_delete},
\code{MPI_File_open}, \code{MPI_File_set_view}, \code{MPI_Lookup_name},
\code{MPI_Open_port}, \code{MPI_Publish_name}, 
\code{MPI_Unpublish_name}, as well as the functions with \code{INFO} in their
name), speed is not critical for the Info functions.

Question: Since many of the predefined Info values encode either booleans or
integers, do we want an internal routine such as \mpidfunc{MPID_Info_get_int}
that returns an integer value if the key is found and has either an integer
value?  Similarly, should there be an \mpidfunc{MPID_Info_set_int}?
If we do this, do we want to cache the result in the Info item
structure?
There are a few info keys (e.g., \mpiconst{chuncked} or
\mpiconst{io_node_list}) that are lists of integers and at least one
(\mpiconst{access_style}) that is a list of strings.  \mpiconst{soft}
is a list of triplets; an info routine that returned triplets would be
needed for this.
We also need \mpidfunc{MPID_Info_get_bool} that returns 1, 0, or -1
(for error).  
Do we want these to return an error code instead, and return the value
through an argument?

Question: Do we want to make the key part of the structure, and set
\mpiconst{MPI_MAX_KEY_VALUE} to a small value such as 32 (the minimum
allowed)?  Doing so slightly simplifies the code to set and delete info
values. 

Question: Do we want the list to be sorted by key name?  The current
implementation uses a linear list, which is probably ok for most uses.

\subsubsection{\mpifunc{MPI_INFO_CREATE}}
Call \mpidfunc{MPID_Info_create}.

\subsubsection{\mpifunc{MPI_INFO_DELETE}}
Remove a key from an info object:
\begin{algorithm}
lock info
check that object is valid
find key and remove key and associated value
decrement count.
unlock info
\end{algorithm}
Note that this lock must look at the global threadedness to decide if
the lock is necessary.  

The ``check that object is valid'' looks at the
\mpids{MPI_Info}{cookie} to make sure that the object is a valid (and
not a deleted) \mpiconst{MPI_Info} object.  The check happens within
the lock to ensure that the object is never deleted by another thread
between the check and acquring the lock.

\subsubsection{\mpifunc{MPI_INFO_DUP}}
Note that this routine must be thread-safe.  This requires info routines that
modify the info structure or the list of key/value pairs to operate safely.
Because none of the info routines are performance critical, and because none
of these operations is very complex, using a single lock per \code{MPI_Info}
or even a single lock for \emph{all} info objects is probably adequate.
\begin{algorithm}
Create a new info object with MPID_Info_create
lock
check that object is valid
walk list, copying each entry
unlock
\end{algorithm}
Note that we don't use a shallow copy because this is a relatively
rare operation, and because implementing a shallow copy (e.g., with
reference counts) is tricky because any before any changes are made,
a full copy must be performed (or even trickier versioning must be
used).

\subsubsection{\mpifunc{MPI_INFO_FREE}}
Call \mpidfunc{MPID_Info_free}.  We may want to lock the info object
while removing the individual entries.  While inside the lock, we
could also mark the object as invalid by clearing the objects
``\mpids{MPI_Info}{cookie}''.\index{MPID_Info_free}
\begin{algorithm}
lock
check that object is valid.  Mark as invalid
free all entries
unlock
free with MPID_Info_free
\end{algorithm}
The ``mark as invalid'' sets the object's \mpids{MPI_Info}{cookie}
to indicate that the object is no longer valid.  
This serves as a
useful check that the user is not accessing an already deleted
object.  

\subsubsection{\mpifunc{MPI_INFO_GET}}
\begin{algorithm}
lock info
check that object is valid
find key and return associated value (by copying to designated location)
unlock info
\end{algorithm}


\subsubsection{\mpifunc{MPI_INFO_GET_NKEYS}}
Question:  Should we keep a count of the number of keys, or just count them?
(Yes, I think so).  If instead we count them, then
\begin{algorithm}
lock info
check that object is valid
run through list to count all keys
unlock info
\end{algorithm}

\subsubsection{\mpifunc{MPI_INFO_GET_NTHKEY}}
\begin{algorithm}
lock info
check that object is valid
find indicated key and return name
unlock info
\end{algorithm}

Question:  Do we want to have this return an error if the info list is 
modified by another thread?  Is there any way to actually do this?  For
example, the self-id of a thread could record itself in the info object when
ever the object is modified.  This routine could (optionally) return an error
if the list is modified by another thread after \code{MPI_INFO_GET_NKEYS}.

Since the most likely use of this routine is to search for all keys,
should we remember the index and location in the list of the last key
returned (or modified)?  That converts the process of extracting every
key from $n^2$ in the length of the list to $n$.  Naturally, we put
this information into the list header.  Inorder to get the
\emph{value} that matches the key, we may want to check that
particular entry first when searching for a \code{key}.

\subsubsection{\mpifunc{MPI_INFO_GET_VALUELEN}}
\begin{algorithm}
lock info
check that object is valid
find indicated key and return length of the associated value.
unlock info
\end{algorithm}
Note that the returned length does not include the end-of-string character.

\subsubsection{\mpifunc{MPI_INFO_SET}}
\begin{algorithm}
lock info
check that object is valid
find indicated key.  
If found, set the value,
else add the key and value; also increment count of keys
unlock info
\end{algorithm}


\subsection{Datatypes}
\label{sec:datatypes}

Error classes include \mpiconst{MPI_ERR_TYPE},
\mpiconst{MPI_ERR_ARG}, and \mpiconst{MPI_ERR_OTHER} (for
out-of-memory allocating internal fields).

The ADI defines a datatype structure that is believed to be a good
choice for implementing operatoins that involve datatypes, such as
\mpifunc{MPI_Pack} and \mpifunc{MPI_Unpack}.

All datatypes should call a function
(\mpidfunc{MPIi_Type_compute_extent} to compute the extent because the
computation is a bit complex.  Here is the rule for the extent:
\begin{algorithm}
/* Compute the \mpids{MPI_Datatype}{ub} */
If a sticky ub exists for the old datatype (datatypes for struct), then
    use the \mpids{MPI_Datatype}{sticky_ub} and set the sticky ub flag
    (\mpids{MPI_Datatype}{MPID_TYPE_STICKY_UB}).
else
    use the \mpids{MPI_Datatype}{true_ub}
/* Similar for the \mpids{MPI_Datatype}{lb},
   \mpids{MPI_Datatype}{sticky_lb},
   \mpids{MPI_Datatype}{MPID_TYPE_STICKY_LB}   and
   \mpids{MPI_Datatype}{true_lb}. */
\mpids{MPI_Datatype}{extent} = \mpids{MPI_Datatype}{ub} - \mpids{MPI_Datatype}{lb} + PAD.
where PAD is determined by alignment rules.  
/* Similar for \mpids{MPI_Datatype}{true_extent} */

For the alignment rules, each datatype keeps track of the largest
alignment obect in \mpids{MPI_Datatype}{alignment_size}; these are the
sizes of the predefined language datatypes such as \code{char} and
\code{long}.  The PAD is chosen to force the extent to be an integral
multiple of the alignment size.
\end{algorithm}
Note that the choice of alignment rule is made at runtime, using the
routines in Section~\ref{sec:runtime-params}.  The default alignment
rule is determined by configure using \code{PAC_C_STRUCT_ALIGNMENT}.

Questions about the implementation of datatypes:

\begin{enumerate}
\item Should we require alignment of data when packing/unpacking?  The
   problem is in the heterogeneous case, where we'd need to communicate
   the alignment rules, along with byte ordering and data lengths.

\item For nested datatypes, should we allow loop interchange (as NEC did
   in their ``flattening on the fly'' paper)?  We can implement this
   within the current representation by creating new dataloop
   structures for the re-ordered loops.  

\item We need to provide for the important special cases of aligned moves
   of sizes 1, 2, 4, 8, and perhaps 16.

\item We could even compile code to pack and unpack the given datatype
   and dynamically load the code.  PETSc has code for this for some
   user-interface convenience functions.  In general, we could
   consider allowing the pack and unpack functions to be specified as
   part of the datatype, with defaults based on the dataloop
   structures.  A datatype attribute could be used to decide when to
   create a datatype-specific routine.

\item We need to include instrumentation on the pack/unpack functions
   themselves so that we can gather information about the performance
   of the pack/unpack.  Should this be stored by datatype instance?
   Datatype kind (e.g., vector, indexed, struct)?  pack/unpack?

\item Do we need separate pack and unpack descriptions (e.g., if we
   optimize for the transfers by reordering loops, will we want
   different versions for each direction)?

\item In dataloop, kind should include information on basic alignment
   and/or length (to allow fast loops using wide moves based on long
   or double instead of char).

\item For types that do not contain struct, we can preload the entire
   processing stack, since the elements never change (just the
   position on the stack).  This is close to creating a simple nested
   loop structure for an interpreter.  We may want the datatype to
   have a field indicating that it has this feature; alternately, we
   might encode this by specifying a different pack/unpack routine,
   one that preloads the stack and eliminates any code to fill the
   stack during processing.  Another approach that would apply to the
   more general case would be to cause datatypes that have simple
   nested structure to load the entire stack and switch the stack
   interpreter into a mode that knew that the stack had been loaded.

\item Struct alignment (pad) should have optional rules.  That is, we
   need to support at runtime all alignment options that a compiler might pick
   (we currently test for this in the MPICH configure).  For systems
   where different padding rules can be specified (e.g., IBM's xlc has 4
   different choices), we should allow an environment variable to
   select a different padding rule.  We might implement this by using
   a separate routine for each type of padding, and call a routine to
   compute the padding towards the end of creating a struct datatype.
   See Sections~\ref{sec:configure} and~\ref{sec:cross-compile} for
   how the default alignment is determined or specified.

\item For pack and unpack code, we need to handle the tests for sizes of
    the output buffers efficiently, hoisting the tests out of the
    loops where possible.  

\item For the homogeneous case, some struct types (those that contain
    only basic datatypes) can be changed into indexed types (as if
    they were all \code{MPI_BYTE}).

\item Structs with no gaps (except at the ends, possibly because of 
    structure padding, an \code{MPI_UB}, or an explicit resize,
    should be replaced with a strided type.  In the heterogeneous
    case, this can only be done when the struct contains a single
    basic type.

\item In the heterogeneous case, we may want two different
    representations: one for homogenous communication and one for
    heterogeneous communication.  Thus the datatype structure needs
    several dataloop entries, at least in the heterogeneous case.
    There may be multiple heterogeneous representations.  For example, 
    most communicators may use reader-makes-right (RMR) \cite{Zhou:1995:RMR}
    but any IMPI (interoperable MPI) communicators need a different
    representation.  Communicators that connect an unusual system
    (e.g., one using a non-IEEE floating point format) may need to use
    XDR.  

\item Should all datatype creation routines call a routine to compute
the representative type signature?  This could be
\mpidfunc{MPIi_type_signature}.  The configure option
\cfgoption{--disable-type-signature} could turn this off (and remove
the overhead from the communication, since the signature value must be
communicated to the destination to allow it to be checked).
\end{enumerate}

\subsubsection{\mpifunc{MPI_ADDRESS}}
Deprecated.  Use \mpifunc{MPI_GET_ADDRESS}.

\subsubsection{\mpifunc{MPI_GET_COUNT}}
Uses the \mpids{MPI_Datatype}{size} field of the datatype 
and the
\mpids{MPI_Status}{count} field of \mpiconst{MPI_Status}.  
Note the
special case of a 
datatype of size zero and a message of size zero; this should return a count
of zero (see the MPI errata discussion). 

Question: should this call an MPID function for this for systems that want to
manage the count field differently, that is, not use it to store the number of
bytes in the message?  

\subsubsection{\mpifunc{MPI_GET_ELEMENTS}}
This requires some care.  This should return the number of basic datatypes in
a message.  So, to start with, each datatype should keep track of the number
of basic datatypes.  Then a quick version of this is:
\begin{algorithm}
\code{sizeof_datatype} = \mpids{MPI_datatype}{size} field of \mpiconst{MPI_Datatype}
\code{n_bytes}         = \mpids{MPI_Status}{count} field of \mpiconst{MPI_Status}
If \code{sizeof_datatype} is zero, then
    If \code{n_bytes} is zero, return zero
    else return \mpiconst{MPI_UNDEFINED}
\code{m_count} = \code{n_bytes} / \code{sizeof_datatype}.  
\code{m_rem}   = \code{n_bytes} \% \code{sizeof_datatype}.
If \code{m_rem} is zero, then 
    the number of elements is this \code{m_count *}
    \mpids{MPI_Datatype}{elements_per_datatype}. 
Else if all elements in the datatype are the same size then
    (e.g., an indexed case)
    the number of elements is \code{n_bytes} /
    \mpids{MPI_Datatype}{sizeof_each_element}
    Use \mpids{MPI_Datatype}{_flags} with
    \mpids{MPI_Datatype}{MPID_ELEMENTS_SAME_SIZE} for this test, along
    with \mpids{MPI_Datatype}{element_size}.
Else 
    /* This is the difficult case */
    \code{element_count} = \code{m_count * elements_per_datatype}
    Process \code{m_rem} recursively as follows:
    Two cases:
    If the datatype has a single old type (e.g., everything except 
    a structure type), recursively apply the algorithm with \code{m_rem} 
    instead of \code{n_bytes} to the old type.
    Else 
       (the struct case).
       Apply the above algorithm to each datatype component of the
       struct in turn (there is only one instance of the struct
       datatype to worry about)
\end{algorithm}
To implement this, we should have a utility routine
\mpidfunc{MPIi_Type_get_elements} that takes just a byte count and a
datatype and returns the number of basic elements.  This routine can
then be called recursively.

\subsubsection{\mpifunc{MPI_STATUS_SET_ELEMENTS}}
Questions: Where are the values defined for the count field(s) in the status?
Is this just the \mpids{MPI_Status}{count} field?  Is there an
\mpidfunc{MPID_Status_set_elements} routine?

\subsubsection{\mpifunc{MPI_TYPE_HINDEXED}}
Deprecated.  However, we can't easily use \mpifunc{MPI_TYPE_CREATE_HINDEXED}
because that could (if \code{MPI_Aint} is longer than \code{int}) require
making a copy of an array argument.  Thus, this code should copy most
of \mpifunc{MPI_TYPE_CREATE_HINDEXED}.  Note that the combiner name for this
is \mpiconst{MPI_COMBINER_HINDEXED_INTEGER}.

\subsubsection{\mpifunc{MPI_TYPE_HVECTOR}}
Deprecated.  Use \mpifunc{MPI_TYPE_CREATE_HVECTOR}.  Note that the combiner
name is \mpiconst{MPI_COMBINER_HVECTOR_INTEGER}.

\subsubsection{\mpifunc{MPI_TYPE_STRUCT}}
Deprecated.  For reasons similar to \mpifunc{MPI_TYPE_HINDEXED}, we do
not want to call the new function.  Note that the combiner name is
\mpiconst{MPI_COMBINER_STRUCT_INTEGER}. 

\subsubsection{\mpifunc{MPI_GET_ADDRESS}}
See the MPICH-1 \mpifunc{MPI_ADDRESS} in \file{mpich/src/pt2pt/address.c}.
We may want a configure test that the integer value of a \code{char *} pointer
is the address in bytes.

\subsubsection{\mpifunc{MPI_TYPE_CONTIGUOUS}}
Create a new datatype with \mpidfunc{MPID_Type_create} and fill in the dataloop
with type \mpidconst{MPID_Contig}.  

\subsubsection{\mpifunc{MPI_TYPE_INDEXED}}
Create a new datatype with \mpidfunc{MPID_Type_create} and fill in the dataloop
with type \mpidconst{MPID_Indexed}.
While copying index values, check for monotone increasing or
decreasing.  Note that a datatype used to specify a file type must be
monotonically nondecreasing (MPI Section 9.3, ``File Views'').

\subsubsection{\mpifunc{MPI_TYPE_VECTOR}}
Create a new datatype with \mpidfunc{MPID_Type_create} and fill in the dataloop
with type \mpidconst{MPID_Vector}.

\subsubsection{\mpifunc{MPI_TYPE_CREATE_DARRAY}}
Create a new datatype with \mpidfunc{MPID_Type_create} and fill in the
dataloops (the number depends on the dimension of the darray) 
with type \mpidconst{MPID_Vector}.

\subsubsection{\mpifunc{MPI_TYPE_CREATE_HINDEXED}}
Create a new datatype with \mpidfunc{MPID_Type_create} and fill in the dataloop
with type \mpidconst{MPID_Indexed}.  While copying index values, check
for monotone increasing values.

\subsubsection{\mpifunc{MPI_TYPE_CREATE_HVECTOR}}
Create a new datatype with \mpidfunc{MPID_Type_create} and fill in the dataloop
with type \mpidconst{MPID_Vector}.

\subsubsection{\mpifunc{MPI_TYPE_CREATE_INDEXED_BLOCK}}
Create a new datatype with \mpidfunc{MPID_Type_create} and fill in the dataloop
with type \mpidconst{MPID_BlockIndexed}.

\subsubsection{\mpifunc{MPI_TYPE_CREATE_STRUCT}}
Create a new datatype with \mpidfunc{MPID_Type_create} and fill in the dataloop
with type \mpidconst{MPID_Struct}.  Check for contiguous elements
while setting up arrays.

\subsubsection{\mpifunc{MPI_TYPE_CREATE_SUBARRAY}}
Create a new datatype with \mpidfunc{MPID_Type_create} and fill in the
dataloops (the number depends on the dimension of the subarray) 
with type \mpidconst{MPID_Vector}.

\subsubsection{\mpifunc{MPI_TYPE_CREATE_RESIZED}}
Create a new datatype with \mpidfunc{MPID_Type_create} and copy in the
dataloop from the old type.  Then change the
\mpids{MPI_Datatype}{extent} of the type as specified by the
\code{extent} argument and the lowerbound by the \code{lb} argument. 

\subsubsection{\mpifunc{MPI_TYPE_COMMIT}}
Optimize the datatype for communication.  

Question:  How do we want to organize the optimization code for datatypes?  We
shouldn't embed it within the \code{MPI_TYPE_COMMIT} function.  Should each of
the dataloop types (e.g., \code{MPID_Vector}) have a corresponding routine
that is called with the entire datatype (not just the specific dataloop)?

One special case is to identify datatypes that are contiguous.  
Question: should this set a field in the datatype, a bit in
\mpids{MPI_Datatype}{_flags}, or a bit in the id for the
\mpiconst{MPI_Datatype}? 

\subsubsection{\mpifunc{MPI_TYPE_DUP}}
Duplicate a datatype.  Invoke the common attribute copy code (e.g., the same
that is used for \mpifunc{MPI_COMM_DUP}) for the attribute
list (\mpids{MPI_Datatype}{attributes}) on this datatype.
We don't need to lock around this because a user that deletes a type
at the same type that \mpifunc{MPI_TYPE_DUP} is executed for it is
writing an erroneous program.  If we do want to protect against
erroneous user programs, we can use the same strategy as used for the
\mpiconst{MPI_Info} routines.

\subsubsection{\mpifunc{MPI_TYPE_FREE}}
This first calls \mpidfunc{MPID_Datatype_incr} with an increment of $-1$.  If
the returned value is zero, it should free the contents of \code{dataloop} and
then call \mpidfunc{MPID_Datatype_free}.  Invoke the common attribute list
delete code.

\subsubsection{\mpifunc{MPI_TYPE_EXTENT}}
Simply uses the \mpids{MPI_Datatype}{extent} field in the structure.

\subsubsection{\mpifunc{MPI_TYPE_LB}}
Simply uses the \mpids{MPI_Datatype}{lb} field in the structure.

\subsubsection{\mpifunc{MPI_TYPE_SIZE}}
Simply uses the \mpids{MPI_Datatype}{size} field in the structure.

\subsubsection{\mpifunc{MPI_TYPE_UB}}
Simply uses the \mpids{MPI_Datatype}{ub} field in the structure.

\subsubsection{\mpifunc{MPI_TYPE_GET_TRUE_EXTENT}}
Simply uses the \mpids{MPI_Datatype}{true_extent} field in the structure.

\subsubsection{\mpifunc{MPI_TYPE_GET_CONTENTS}}
Uses the \code{dataloop} field to access the data uses to create the datatype.
Actually, must use a separate dataloop field so that the arguments that the
user provided are returned, rather than an optimized form).

\subsubsection{\mpifunc{MPI_TYPE_GET_ENVELOPE}}
Uses \code{dataloop} field to identify how the datatype was constructed.
The \code{combiner} type must be one of 

\noindent\mpiconst{MPI_COMBINER_NAMED}\\
\mpiconst{MPI_COMBINER_DUP}\\
\mpiconst{MPI_COMBINER_CONTIGUOUS}\\
\mpiconst{MPI_COMBINER_VECTOR}\\
\mpiconst{MPI_COMBINER_HVECTOR_INTEGER}\\
\mpiconst{MPI_COMBINER_HVECTOR}\\
\mpiconst{MPI_COMBINER_INDEXED}\\
\mpiconst{MPI_COMBINER_HINDEXED_INTEGER}\\
\mpiconst{MPI_COMBINER_HINDEXED}\\
\mpiconst{MPI_COMBINER_INDEXED_BLOCK}\\
\mpiconst{MPI_COMBINER_STRUCT_INTEGER}\\
\mpiconst{MPI_COMBINER_STRUCT}\\
\mpiconst{MPI_COMBINER_SUBARRAY}\\
\mpiconst{MPI_COMBINER_DARRAY}\\
\mpiconst{MPI_COMBINER_F90_REAL}\\
\mpiconst{MPI_COMBINER_F90_COMPLEX}\\
\mpiconst{MPI_COMBINER_F90_INTEGER}\\
\mpiconst{MPI_COMBINER_RESIZED}\\

\subsubsection{\mpifunc{MPI_TYPE_GET_EXTENT}}
Simply uses \mpids{MPI_Datatype}{extent} field.

\subsubsection{\mpifunc{MPI_TYPE_MATCH_SIZE}}
This function returns the MPI Datatype corresponding to a specified type class
(one of \mpiconst{MPI_TYPECLASS_INTEGER}, \mpiconst{MPI_TYPECLASS_REAL}, or
\mpiconst{MPI_TYPECLASS_COMPLEX}) and size.  This will need to be implemented
by using the datatype sizes determined by configure, and then mapped into 
the actual types.  For example
\begin{verbatim}
switch (typeclass) {
    case MPI_TYPECLASS_REAL:
        switch (size) {
        case 4:  *rtype = MPI_REAL;
        case 8:  *rtype = MPI_DOUBLE_PRECISION;
        case 16: *rtype = MPI_REAL16
        default: (invoke error handler from MPI_COMM_WORLD)
        }
    case MPI_TYPECLASS_INTEGER:
        ...
\end{verbatim}


\subsubsection{\mpifunc{MPI_TYPE_GET_NAME}}
Uses the \mpids{MPI_Datatype}{name} field.  Note that the Fortran
versions must be careful to 
blank-pad the value rather than null-terminating it.

Question: We should have a common routine to handle the Fortran version of all
of the get/set name routines.  I propose \mpidfunc{MPIi_C2F_get_name} and
\mpidfunc{MPIi_F2C_set_name}; these can be used for any character string and
are not limited to the datatype, communicator, or window name.  Names are
stored in C style (null terminated) in all MPI data structures.

\subsubsection{\mpifunc{MPI_TYPE_SET_NAME}}
Sets the \mpids{MPI_Datatype}{name} field.  Returns error if supplied
name is too long. 
Note that the name may be set for all datatypes, including the predefined
names.

\subsubsection{\mpifunc{MPI_PACK}}
Call \mpidfunc{MPID_Pack} with a \code{rank} of \code{MPI_ANY_SOURCE}.
\begin{adi3}
Native/Homogeneous case: Simply execute the dataloop

Heterogeneous case: If reader-makes-right (RMR) is used, then this is the same
as the native case.  If XDR or external32 is used, then each basic type must
be identified and processed appropriately.
\end{adi3}

\subsubsection{\mpifunc{MPI_PACK_SIZE}}
Call \mpidfunc{MPID_Pack_size} with a \code{rank} of \code{MPI_ANY_SOURCE}.

Question: One issue is with IMPI \cite{impi}, which requires that there be no
header on any pack buffers.  Do we want to say something about a header on a
pack buffer?  Note that implementing the datatype signature
\cite{gro:mpi-datatypes:pvmmpi00} requires a header. 

\begin{adi3}
Native/Homogenous case: \mpids{MPI_Datatype}{size} field of
\mpiconst{MPI_Datatype}. 

Heterogeneous case: Except for the RMR case, this is more awkward.  On
possibility is to compute a 
\mpids{MPI_Datatype}{pack_size} and \mpids{MPI_Datatype}{pack_alignment} for
each datatype and use that to compute the final size, at least for choices
that are not dependent on the rank in the communicator.  Or, if there
are only a few choices, one for each hoice.
\end{adi3}

\subsubsection{\mpifunc{MPI_UNPACK}}
\label{sec:mpi-unpack}
Call \mpidfunc{MPID_Unpack} with a \code{rank} of \code{MPI_ANY_SOURCE}.

\begin{adi3}
Native/Homogeneous case: Simple use the dataloop to unpack the data.
Special case: As described below, packed buffers may have a header; if the
implementation requires them, even the native case must first check and skip
over the header.

Heterogeneous case:  In all cases (RMR and XDR/external32/etc.), each basic
datatype must be identified and processed.  Further, for RMR, we need to know
the origin of the data so the the receiver can figure out what to do.

Question: some communicators may require a symmetric format, such as XDR or
external32.  An example is any communicator that involves a process connected
through IMPI \cite{impi}.  Do communicators need a structure that contains
information on heterogeneity (e.g., a \mpids{MPI_Comm}{data_rep})?

Question:  The MPI-FT project (no papers available) has proposed reordering
the data so that data of each time is placed together.  For example, instead
of sending char-int-char-int, it might send int-int-char-char, and rely on the
datatype at the destination to receive it correctly.  Do we want to make this
an option?  How do we handle the case that less than one complete instance of
a datatype is sent (e.g., in the above case, only char-int-char is sent as
int-char-char)?  Note that it is possible but difficult.
\end{adi3}

\paragraph{Using \mpifunc{MPI_PACK}, \mpifunc{MPI_UNPACK},
  \mpidfunc{MPID_Pack} and \mpidfunc{MPID_Unpack} in the ADI.}
Data that is sent with \mpiconst{MPI_PACKED} as the datatype may either be
received as \mpiconst{MPI_PACKED} \emph{or} with any datatype that matches the
type signature of the types used to pack the data on the sending end.  In
homongeneous systems, this doesn't matter, but in systems where different data
formats may be used depending on the source and destination of a message,
along with the communicator connecting them, there are many issues.
Consider the following cases of sending between two processes:
\begin{enumerate}
\item Source process uses MPI datatypes (not including \mpiconst{MPI_PACKED})
  to send the data.  In this case, a particular destination is known, and the
  sending process can check to see if the destination process uses the same
  data representation as the source process.  If so, it can send the data as
  native.  However, it needs to indicate that the data is in native format to
  the destination.

  Question: do we want datatypes to contain information on what basic types
  they contain?  How about the optimization for the case of a single basic
  type?  Type signature?

\item Source process uses \mpifunc{MPI_Pack} and sends using type
  \mpiconst{MPI_PACKED}.  Since \mpifunc{MPI_Pack} does not specify a
  destination rank, the representation format must be chosen based on the
  communicator, not the destination rank.  At the destination, one of two
  things happens:
  \begin{enumerate}
  \item The receive type is not \mpiconst{MPI_PACKED}.  The data is converted
    from the packed format into the user's buffer.  There must be some
    indication that the message is in a particular format, whether it is
    RMR, native, XDR, external32, etc.  This must be part of the
    envelope, not the data.

  \item The receive type is \mpiconst{MPI_PACKED}.  The data must be copied
    (almost) as is, except that enough information must be saved so that
    \mpifunc{MPI_UNPACK} can unpack it later.  This may include the message
    format \emph{and source}, stored in the header (see case 3).

    For IMPI communicators, the format is fixed for all communication within
    the communicator \emph{and} no header is permitted on packed data (at
    least in the parts of the code visible to IMPI).
  \end{enumerate}
\item Source process use \mpifunc{MPI_Pack} and sends using type
  \mpiconst{MPI_PACKED}.  Receiving process receives as \mpiconst{MPI_PACKED}
  and then resends the message to another process in the same communicator.
  The recipient of that message then unpacks it.

  This last case makes it clear that the rank in the communicator of the
  process that packed the message must be retained; the rank of the sender is
  not sufficient.
\end{enumerate}

For better error checking, packed data could contain the communicator
(actually, context id) that it was packed for in the header, and an
error signaled for use in a different communicator.

\subsubsection{\mpifunc{MPI_PACK_EXTERNAL}}
This is like \mpifunc{MPI_PACK}, but in the ``\mpiconst{external32}'' format
defined by MPI-2.  There is no header; I believe that this exactly matches the
IMPI format.

\subsubsection{\mpifunc{MPI_PACK_EXTERNAL_SIZE}}
Like \mpifunc{MPI_PACK_SIZE}, but for ``external32''.

\subsubsection{\mpifunc{MPI_UNPACK_EXTERNAL}}
Like \mpifunc{MPI_UNPACK}, but for ``external32''.  Actually, this is
slightly simpler, since the incoming format is specified.

\subsubsection{\mpifunc{MPI_REGISTER_DATAREP}}
Specify a set of user data conversion functions.  The data representation
defined by this routine may be used by \mpifunc{MPI_FILE_SET_VIEW}.  The error
handler used is that defined on \mpiconst{MPI_FILE_NULL}.  Note that
\mpifunc{MPI_PACK_EXTERNAL} and \mpifunc{MPI_UNPACK_EXTERNAL} take a
\code{datarep} as an argument; if possible, the implementation of those
routines should accept a general data representation defined by this routine
so that they may be used in an MPI I/O implementation.

Where are the list of datareps stored?  We need a list of datareps,
containing functions.  This list needs a lock so that multiple threads
can define new datareps.  Note that there is no deregister for
datareps, but we need one for \mpifunc{MPI_Finalize}.
\mpidfunc{MPIi_Datarep_finalize}? 

\subsubsection{Heterogeneity}
\label{sec:hetero}
Optimizing for the common case of machines or clusters with a common
data representation is important.  

In MPICH, macros were used to include code that handled heterogeneous
systems.  For MPICH2, I'd prefer to use clearer blocks of code rather
than special macros.  For example,
\begin{verbatim}
#define MPICH_IS_HETERO
...
#else
...
#endif
\end{verbatim}
even is some code is duplicated as a result.

\subsection{Groups}
\label{sec:groups}

Error classes include \mpiconst{MPI_ERR_GROUP}, \mpiconst{MPI_ERR_RANK}, 
\mpiconst{MPI_ERR_ARG}, and \mpiconst{MPI_ERR_OTHER} (memory allocation). 

Groups are simple.  The key point here is to make sure that the use of
groups to map from ranks in a communicator to a particular destination
process is fast.
To provide a fast implementation of \mpifunc{MPI_COMM_GROUP}, groups must
have reference counts (\mpids{MPI_Group}{ref_count})

Question: Do we want a special case for the groups of self and comm
world?  This would eliminate a lookup in the group table for the a very
common case.  Should there be an \code{MPID_GROUP_WORLD}?  

Question: What data structure should be used to represent groups?
MPICH-1 used a simple array that mapped rank in a group to rank in the
group of \code{MPI_COMM_WORLD}.  In MPI-2, we can't use
\code{MPI_COMM_WORLD}.  ADI-3 defines ``local process ids,'' which
simply refers to the processes known to the current process.  The code below
suggests the use of an array mapping ranks to the ADI-3 local process
ids.  Local 
process ids are not related to Unix process ids (perhaps we need a new name);
rather, they are local \emph{MPI} process ids.  The MPI processes in
\mpiconst{MPI_COMM_WORLD} have local process ids that range from zero to
\code{size} of \mpiconst{MPI_COMM_WORLD} $-1$.  Local process ids
indicate a ``connection'' or link to other proceeses.
Processes that are added to a
running MPI process (e.g., by \mpifunc{MPI_COMM_SPAWN}) have local process ids
of at least \code{size}.

Actually, groups are simple as long as we don't require a scalable
representation of group membership.  An interesting question is what
sort of representation should be used for truly massively parallel
systems such as Blue Gene.  

Plan: At least initially, the group implementation in ADI-3 will \emph{not} be
scalable.  Each group will have an array that maps ranks to local process ids.

Note: the MPICH-1 implementation of \mpifunc{MPI_Group_difference},
\mpifunc{MPI_Group_union}, and \mpifunc{MPI_Group_intersection} has complexity
that is the product of the sizes of the groups (!).

\subsubsection{\mpifunc{MPI_GROUP_RANK}}
Simply return \mpids{MPI_Group}{rank} field.

\subsubsection{\mpifunc{MPI_GROUP_SIZE}}
Simply return \mpids{MPI_Group}{size} field.

\subsubsection{\mpifunc{MPI_GROUP_TRANSLATE_RANKS}}
We may want to detect the special case of a group that is a subset of
\code{MPI_COMM_WORLD} (does this imply a flag in the \mpidfunc{MPID_Group}
structure?).  Such a flag might be
\mpids{MPI_Group}{MPID_GROUP_SUBSET_WORLD}. 

\begin{algorithm}
If either group is not a subset of \code{MPI_COMM_WORLD}, then 
\begin{enumerate}
\item For \code{group2}, if necessary, create a new array containing the pairs
  \code{local process id} (\mpids{lpid_to_lrank}{lpid}), \code{local rank}
  (\mpids{lpid_to_lrank}{lrank}), sorted by \code{local process 
    id} (\code{lpid}). Call this array
\mpids{MPI_Group}{lpid_to_lrank}; the elements are 
  structures of type
\mpidconst{MPID_Group_pmap_t}\index{MPID_Group_pmap_t!lpid}\index{MPID_Group_pmap_t!lrank} 
  (Note that \mpifunc{MPI_GROUP_FREE} needs to free this array.)
\item For each rank in \code{ranks1}, find the corresponding \code{local
    process id} 
  using \mpids{MPI_Group}{lrank_to_lpid} and then search for that \code{lpid}
  in 
  the sorted array \code{lpid_to_lrank} in \code{group2}.
\end{enumerate}
\end{algorithm}

Question: should there be a \code{MINIMUM_MEMORY} option that controls when
memory is left allocated to speed subsequent operations?  In this case, we
leave the memory allocated so as to speed subsequent translations.  Normally,
this routine is only used in tracing libraries to convert relative ranks into
absolute ranks; in that case, it is likely to be called frequently.
This should/could also be a runtime parameter.

\subsubsection{\mpifunc{MPI_GROUP_FREE}}
Free all internal fields (e.g., \mpids{MPI_Group}{lpid_to_lrank}) and then
call \mpidfunc{MPID_Group_free}. 

\subsubsection{\mpifunc{MPI_GROUP_COMPARE}}
\begin{enumerate}
\item Check that sizes are the same.  If not, set \code{result} to
  \mpiconst{MPI_UNEQUAL} and return.
\item Check that the elements of \code{lrank_to_lpid} are the same.  If so,
set \code{result} to \mpiconst{MPI_IDENT} and return.
\item Check that the \mpids{MPI_Group}{lrank_to_lpid} arrays
  contain the same values, but 
  in a different order.  We could use the same array needed by
  \code{MPI_GROUP_TRANSLATE_RANKS} here.  If those two arrays have the same
  local process ids (they'll be in the same order), return
  \mpiconst{MPI_SIMILAR}, otherwise return 
  \mpiconst{MPI_UNEQUAL}.  
  Question: should we free the new arrays if we allocate them?  See
  the discussion in \mpifunc{MPI_GROUP_TRANSLATE_RANKS}.
\end{enumerate}

Question: what routines should we use to manage the marker arrays?

\subsubsection{\mpifunc{MPI_GROUP_EXCL}}
\begin{adi3}
  Construct new group from the designated subset of
  \mpids{MPI_Group}{lrank_to_lpid} field of input group.

  To conform to Intel test suite error test for duplicate ranks, check for 
  duplicate ranks in the exclusion list.

  We can arrange to perform both the group creation and the test for
  duplicates through the use of a marker array; this can be used for many of
  the group creation routines.

  The \mpidconst{marker_array} is an integer array whose size is the size of
  the group.  For \mpifunc{MPI_GROUP_EXCL}, initialize all entries to one.
  For each rank in the exclusion list, decrement the corresponding entry in
  the marker array.  Any entry that is less than zero indicates an error
  (duplicate in the exclusion list).  Any entry that is still one indicates
  that the corresponding process in the original group is to be retained in
  the new group.

  We may want an \mpidfunc{MPIi_Group_create_from_marker} for this and many of
  the other group creation routines.
\end{adi3}

\subsubsection{\mpifunc{MPI_GROUP_INCL}}
\begin{adi3}
  Construct new group from subset of \mpids{MPI_Group}{lrank_to_lpid} field of
  input group.

  Make sure to check for duplicate input ranges (invalid input).  Use the
  \mpidconst{marker_array} approach from \mpifunc{MPI_GROUP_EXCL}, but start
  with zero in each element; any element over one indicates an error.
\end{adi3}

\subsubsection{\mpifunc{MPI_GROUP_RANGE_EXCL}}
  Check that the ranges terminate.  
\begin{adi3}
  Construct new group from subset of \mpids{MPI_Group}{lrank_to_lpid} field of
  input group.  This is a little tricky because multiple ranges can be
  specified and they can exclude overlapping ranges of ranks.  
  Use the \mpidconst{marker_array} with each element initialized to one; zero
  out each rank specified by each range.  Create the new group from the
  corresponding processes that have a positive entry.
\end{adi3}

\subsubsection{\mpifunc{MPI_GROUP_RANGE_INCL}}
  Check that the ranges terminate.
\begin{adi3}
  Construct new group from subset of \mpids{MPI_Group}{lrank_to_lpid} field of
  input group.  Like \mpifunc{MPI_GROUP_RANGE_EXCL}, but start with zero in
  each element of the \mpidconst{marker_array}.
\end{adi3}

\subsubsection{\mpifunc{MPI_GROUP_DIFFERENCE}}

This should use the local process id and rank array
(\mpids{MPI_Group}{lpid_to_lrank}) to identify the different
processes to include.  Note that this includes only the elements of the first
group that are not in the second group, ordered as in the first group.  
This can first save the indexes of entries that will be used; that array of
indices in then sorted by the rank-in-group field, and the new group can be
created by calling \mpifunc{MPI_GROUP_INCL} (or an internal version that works
directly with the non-opaque data structures and assumes valid inputs).

A better solution might be to use the \code{marker_array} of size of
\code{group1}, where \code{group2} is used to zero elements of the
\code{marker_array}.  

\subsubsection{\mpifunc{MPI_GROUP_INTERSECTION}}

This is implemented similarly to \mpifunc{MPI_GROUP_DIFFERENCE}.

\subsubsection{\mpifunc{MPI_GROUP_UNION}}

Start with all of \code{group1}.  For each process in \code{group2} that is
not in \code{group1} (check the \mpids{MPI_Group}{lpid_to_lrank} array for
\code{group1}), add that local process id to the union.

\subsection{Communicators}
\label{sec:communicators}

Communicators have two main features: a context id and a group.  In
addition, communicators that are created with \mpifunc{MPI_Comm_dup} must
copy attributes (where requested) from the old communicator.

Question: do we want a ``hidden'' communicator for implementing the
collective routines?  Just a hidden context?  One concern when there
are two communicators: which do you lock (e.g., with
\mpidfunc{MPID_Comm_thread_lock})?  How do you avoid deadly embraces?  How do
you 
ensure that the expected communicator is acted on?  An alternative to two full
communicators is to have two context values.  If there is a hidden
communicator, should there be a flag that can be used to indicate that the
communicator is, in fact, a hidden one?

If a hidden communicator is used, it should have its error handler permanently
set to \mpifunc{MPI_ERRORS_RETURN} so that an error action that is
appropriate to 
the collective routine, not the routine called, may be used.

What utility routines do we wish to define?  There are a number of
routines that create communicators, including the topology routines.
Note that attributes are only copied by \mpifunc{MPI_Comm_dup}.

\subsubsection{\mpifunc{MPI_COMM_COMPARE}}
This compares first the \mpids{MPI_Comm}{context_id} values (Question:
this assumes that 
we don't use the same \mpids{MPI_Comm}{context_id} for communicators
with disjoint 
groups).  If the same, return \mpiconst{MPI_IDENT}.  Other wise, call
\mpifunc{MPI_GROUP_COMPARE} for the remote group (and if both are
intercommunicators, local group).  If the group comparision(s) return
\mpifunc{MPI_IDENT}, then return \mpifunc{MPI_CONGRUENT}.  Otherwise, return
the same value as given by \mpifunc{MPI_GROUP_COMPARE}.  If the
communicator is an intercommunicator, return the lowest value returned
by \mpifunc{MPI_GROUP_COMPARE}.


\subsubsection{\mpifunc{MPI_COMM_CREATE}}
We need a basic communicator creation routine for this.  In
particular, many of the communicator construction routines can create
a group and then use that to specify the communicator.  We may want a
varient that takes a group and does not make a duplicate or copy; this
would allow us to create the group and then provide it to the
communicator creation routine.

\paragraph{Allocating Context Ids.}
\index{thread overhead!context ids}
The single threaded case is relatively easy: a global variable can be used
that contains a list of available context ids; there can also be a way to
generate new context ids if a large number of communicators are in use.  The
``list'' could, in fact, be a bit vector, with the bits indicating whether or
not the context id was in use.  Communicator creation routines could find a
context id by performing an \mpifunc{MPI_Allreduce} with the appropriate bit
operator (\mpiconst{MPI_BAND}).  The position of the lowest set bit can be
used.  

The multithreaded case is more difficult.  You cannot do
\begin{algorithm}
  lock
  \mpifunc{MPI_Allreduce}
  unlock
\end{algorithm}
because different threads in the same process might lock the data structure,
causing a deadly embrace.
One possible solution is to use a lock/read/unlock on the bit vector, followed
by an \mpifunc{MPI_Allreduce}, followed by an \mpifunc{MPI_Allreduce} on
whether the bit vector has been changed by another thread.  If not, then the
value for the context id can be accepted; otherwise, start over.  (need to
make this more precise)

Here is an algorithm that will work in the multithreaded case.  It
uses a bit mask of context ids (each bit set indicates a context id
available; 32 32-bit integers covers 1024 context ids.  This mask,
along with a queue containing the context ids of communicators that
are requesting a new context id and a variable that indicates that
some thread has acquired the rights to the mask, are stored in
thread-shared memory.
\begin{algorithm}
while (no context id found) {
    local_mask = 0
    lock 
    if mask_in_use return a mask of zero
    else if (first_time) add to this context id to queue of pending
    requests in order of context id value
    if at the head of queue (lowest numbered context), 
        mask_in_use = 1
        local_mask = mask
    unlock
    MPI_Allreduce( local_mask, MPI_BAND )
    If a set bit is found in mask,
        lock
        unset corresponding bit in mask
        mask_in_use = 0
        remove this context_id from queue
        if queue non-empty, release condition variable
        unlock
        return the new context_id.
    else if had low context value (i.e., this thread set mask_in_use)
        lock
        mask_in_use = 0
        release condition variable
        unlock
    else
        wait on condition variable
    } /* end while */
\end{algorithm}
The use of \code{mask_in_use} ensures that only one thread per process
is accessing the mask of available context ids at any time; thus a
success in the \mpifunc{MPI_Allreduce} step guarantees that the found
value is in fact available.  If that step fails, that means that some
thread was unable to access the mask and contributed an all-zero bit
vector as a result.  
The key to handling this case is to use the
\mpids{MPI_Comm}{context_id} value to break ties when several threads
in the same process are attempting to find a context id.  Note that a
correct program cannot have two collective routine on the same
communicator active in the same process at the same time.  If that
happens, the program is erroneous.  Note that this algorithm can
detect that by detecting two identical \code{context_id}s in the
queue.

Note that this algorithm involves no extra communication in the
single-threaded case; even in the multi-threaded case, no extra
communication is require in most circumstances.

Question: do we want to fix the number of context ids?  Note that
these are not globally unique; they are only unique among a collection
of processes.  1024 might be enough.  Is this a compile time or
runtime parameter?  Is the \cfgoption{--with-maxcomm=n} the configure
control for this?

\paragraph{Caching context ids.} 
The performance of operations such as \mpifunc{MPI_Comm_split} and
\mpifunc{MPI_Comm_dup} can be improved if there is a preallocated cache of
context ids, at least in the single-threaded case.  In the above
algorithm, more than one id may be extracted from the mask following a
successful call to \mpifunc{MPI_Allreduce} (success defined as
returning a mask with at least one bit set).  Following the rules that
require ordering of collective calls, even in the multi-threaded case,
context ids can be extracted from this cache with no communication.

Question: Do we want to support context id caching?  If so, how many?
Is there an attribute to provide runtime control?  Is
\cfgoption{--enable-commid-cache=n} the configure option?

\subsubsection{\mpifunc{MPI_COMM_DUP}}
One approach is to extract the group from the incoming communicator,
invoke \mpifunc{MPI_Comm_create}, and then invoke the attribute
copying step.

Question: Who (if anyone) guarantees that two threads don't run the same
attribute copy functions at the same time?  The standard is silent here, but
some examples use code where the attribute is a pointer to storage that holds
an integer (e.g., a private tag) and the copy routine performs (without
locking) a fetch and increment.  Do we want to allow/force the attribute copy
functions to behave like Java synchronized methods?

This question was posed to the MPI Forum and the answer was that
because any operation (including communication) is permitted, it isn't
permissible to lock around the attribute copy routines.  Thus, we can
only warn the user on the man page.  Or provide a test as an option.

\subsubsection{\mpifunc{MPI_COMM_FREE}}
Decrement the \mpids{MPI_Comm}{ref_count}.  If zero, free the communicator.
This routine must invoke the attribute delete functions for each
attribute, then free the groups, then any per-communicator
structures.  Calls \mpidfunc{MPID_Comm_free}.


\subsubsection{\mpifunc{MPI_COMM_GROUP}}
\begin{adi3}
  Returns a duplicate (shallow copy) of the local group.  Calls
  \mpidfunc{MPID_Group_incr} to do this. 
\end{adi3}

\subsubsection{\mpifunc{MPI_COMM_RANK}}
Return the \mpids{MPI_Comm}{rank} field.

\subsubsection{\mpifunc{MPI_COMM_REMOTE_GROUP}}
\begin{adi3}
  Return a duplicate (shallow copy) of the remote group.  Call
  \mpidfunc{MPID_Group_incr}. 
\end{adi3}

\subsubsection{\mpifunc{MPI_COMM_REMOTE_SIZE}}
See \code{MPI_COMM_SIZE}.  This is actually not an unusual operation,
since all point-to-point operations need to check the rank of the
sender or destination against this size, not the size of the local
group.  

\subsubsection{\mpifunc{MPI_COMM_SIZE}}
Return the \mpids{MPI_Comm}{size} field.  This is really the size of the local
group, which for an intercommunicator may be different from the remote
size.  Note that for point-to-point communication, error checking for
destination or source ranks must look at the remote size.  Should
there be \mpids{MPI_Comm}{local_size} and \mpids{MPI_Comm}{remote_size} fields
in the communicator? 

\subsubsection{\mpifunc{MPI_COMM_SPLIT}}
Perform an Allgather on the color and key.  Processes with the same color are
in the same new communicator.  Count the number with the same
\code{color}.  Allocate an array of that size and fill in with the
ranks of the current communicator and the keys (from the
\mpifunc{MPI_Allgather}.  Sort the ranks according to the \code{key}.
Create the new communicator by passing that list of ranks to an
internal form of \mpifunc{MPI_Comm_create} (the same routine is needed
by \mpifunc{MPI_Comm_create} and \mpifunc{MPI_Intercomm_merge}).  Or
we can create the group explicitly with \mpifunc{MPI_Group_incl} and
use \mpifunc{MPI_Comm_create} as is.

\subsubsection{\mpifunc{MPI_COMM_TEST_INTER}}
Should this simply test to see if the remote group and the local groups are
the same, or should there be a separate communicator ``kind'' field?  Note
that the C++ binding defines 4 kinds of communicators: intercomm, intracomm,
graphcomm, and cartcomm. 

Suggestion: Either no separate field (using
\mpids{MPI_Comm}{local_group} \code{==}
\mpids{MPI_Comm}{remote_group}), or a single bit in a boolean flags field.  
Note that this test does need to be performed for every collective operation,
since the inter- and intra-communicator algorithms are different.

\subsubsection{\mpifunc{MPI_INTERCOMM_CREATE}}
The local leaders exchange messages with the remote leaders to gather the
information on the two groups and to agree on a context id.  (Question: should
a debugging version ensure 
that consistent local and remote leader ranks are specified by first
performing an allgather of the root values?)  
Leaders then broadcast information containing process identifiers to
their respective groups (using 
\mpiconst{PMPI_Bcast}). Note that this requires communication wholely
within the local group.  Question: Should intercommunicators contain a
private intracommunicator representing the local group?

Question:  Should intercommunicators contain a private, internal
intracommunicator for use by intercommunicator collective operations including
\mpifunc{MPI_Comm_dup}?

Question: What are the process identifiers?  In the case of MPI-1, 
these can simply be the rank in \mpiconst{MPI_COMM_WORLD}.  For the
purposes of building the local process ids, we need to convert a local
process id on one process into a identifier that can be used by
another process.  There are two cases:
\begin{enumerate}
\item The (remote) process is already known to the local process.
This is similar to the case of \mpiconst{MPI_COMM_WORLD}.
\item The (remote) process is not known to the local process.  That
is, there is no connection between the two processes.  This is the
case for groups created by \mpifunc{MPI_COMM_SPAWN} or
\mpifunc{MPI_COMM_CONNECT}. 
\end{enumerate}
In the first case, we only need to establish the correspondence
between the local process ids.  
We can convert the second case into the first case by forming a
correspondence when processes are joined to an MPI process with 
\mpifunc{MPI_COMM_SPAWN}, \mpifunc{MPI_COMM_CONNECT} and related
routines.  This means that there must be a function that converts local
process numbers into the ``global'' numbering and back.  The global
numbering provides enough information to connect to another process.
In turn, this implies that these routines must ensure atomic access,
since the ``global'' numbers change when another group of processes
connects or is spawned (particularly connects, since both previously
disjoint process sets may have picked the same global numbers).
This could have the following
interface:\index{MPIi_Gprocmap_lock}\index{MPIi_Gprocmap_update}\index{MPIi_Gprocman_gtol}\index{MPIi_Gprocmap_update} 
\begin{verbatim}
int MPIi_Gprocmap_lock( int flag ) /* lock/unlock */
void MPIi_Gprocmap_ltog( int n, const int lpid_array[], int gpid_array[] )
void MPIi_Gprocmap_gtol( int n, const int gpid_array[], int lpid_array[] )
void MPIi_Gprocmap_update( int n, int (*gpid_array)[] )
void MPIi_Gprocmap_get( int *n, int (*gpid_array)[] )
\end{verbatim}
\index{MPIi_Gprocmap_get}

Still to do: make this consistent with the discussion of intercommunicator
creation in \mpifunc{MPI_Comm_spawn}.  An alternative to the
\mpidfunc{MPIi_Gprocmap_xxx} routines is a local process id and a table that
maps this to a global identifier made up of a BNR group id and a rank in that
group.  For a job with a single \code{MPI_COMM_WORLD}, this would map to a BNR
group id of zero and the rank in \code{MPI_COMM_WORLD}.  More complex jobs
(such as that in Figure~\ref{fig:spawn-ic}) would have multiple BNR groups.

\begin{figure}
\begin{verbatim}
(place holder for nice figure containing the following
(a)         g1             g2            (separate MPI jobs)
(b)         g1 ----------- g2            (connect/attach)
(c)         g1 ----------- g2            (g1 spawns)
            |
            g3
(d)         g1 ----------- g2            (g2 spawns)
            |              |
            g3             g4
(e)         (intercomm_merge using (g1-g2) as peer comm, and 
            (g1+g3) and (g2+g4) as the local groups.
\end{verbatim}
\caption{Example of MPI process creation and
\mpifunc{MPI_Intercomm_merge}}\label{fig:spawn-ic} 
\end{figure}

The intercommunicator routines should have robust error checking because they
require care and understanding in use and errors are hard to diagnose.  
Errors to check for include inconsistent leaders (all members of the local
group should agree) and overlapping groups (remote and local groups
must not overlap).

Note that this routine is \emph{not} collective in the peer
communicator; that is why a \code{tag} value is required.  That makes
it more difficult to check for consistent leaders between the two
groups, though it could be done tbrough some sort of central registry.
 
\subsubsection{\mpifunc{MPI_INTERCOMM_MERGE}}
Create a new group from the union of the local and remote groups.  Rank 0 in
the group with \code{high == true} communicates with rank 0 in the group with
\code{high == false}.  Once this group is created, call
\mpifunc{MPI_COMM_CREATE} with this group.  

Note that all processes in an intercommunicator are already known to the
local process (see \mpiconst{MPI_INTERCOMM_CREATE}).

\subsubsection{\mpifunc{MPI_COMM_CLONE}}
This is a special C++ function; it behaves similarly to
\mpifunc{MPI_COMM_DUP}, but returns a reference (pointer) to the created
communicator, rather than the communicator itself.  This is necessary because
the C++ binding makes a \code{Comm} an abstract base class, and since
you cannot return an instance of an abstract base class, you can't use
\mpifunc{MPI::Dup} (which returns an instance).  \mpifunc{MPI::Dup}
may only be used on one of the four derived classes.
\mpifunc{MPI::Clone} was provided to give C++ programmers a way to
create a reference to a duplicate of an arbitrary communicator.
Question: should we design an internal dup function so that both
\mpifunc{MPI_COMM_DUP} and the C++ \code{MPI::Comm::Clone} function can use it?

\subsubsection{\mpifunc{MPI_COMM_GET_NAME}}
Return a copy of the \mpids{MPI_Comm}{name} field.

\subsubsection{\mpifunc{MPI_COMM_SET_NAME}}
Set the \mpids{MPI_Comm}{name} field.  Check for valid length.

\subsection{Point to Point Communication}
\label{sec:pt-2-pt}

The ADI provides a relatively close match to the point-to-point
communication routines.  

Questions that remain:  Handling of persistent requests.  The ADI
contains memory registration.  Is anything else needed for persistent
requests? 

Since requests are allocated by the \mpidfunc{MPID_Request_xxx_FOA} routines,
should a persistent request simply have a pointer to the active request?  The
request pointer could be null to indicate an inactive persistent request.

Question:  There are a number of flags that we may want to check in order to
drop into a special optimized case.  Should we set things up so that a single
int of flags, where the ``good'' case is a zero bit for each flag, can be
tested with a single compare against zero?

\subsubsection{\mpidfunc{MPID_Rhcv}}
This function is not part of MPI but is a critical part of the ADI.
The choice of implementation depends on the properties of the ADI;
some are reviewed below.

\paragraph{Threadedness.}\index{thread overhead!handler
invocatoin}\index{thread overhead!MPID_Rhcv}In a single-threaded ADI
implementation, 
\mpidfunc{MPID_Rhcv} can simply call the appropriate routine to act on
the object (e.g., deliver a message or inspect a remote queue in
shared memory).  In an implementation that allows multiple-user
threads to invoke the ADI routine, \mpidfunc{MPID_Rhcv} must ensure
that it is thread-safe.  One easy way to do this is to use a lock;
that is, consider \mpidfunc{MPID_Rhcv} a \emph{synchronized}
function.  An alternative, particularly when there is a separate
communication agent (See Section~\ref{sec:comm-agent}), is to have
\mpidfunc{MPID_Rhcv} enqueue operations on a pending work queue.  This
operation can sometime be done without a lock (using memory atomic
operations or load-reservation/store-conditional split operations);
removing items from the queue can also be done without a lock.

Questions: do we want a separate send and receive agent in this case?
The send agent could wait on a condition variable that is set by any
thread that adds to the queue.  The receiving agent needs to respond
to events coming from other processes, making the situation somewhat
asymmetric.  Do we want to have a virtual split, allowing either zero,
one, or two agents?  

\subsubsection{\mpifunc{MPI_PROBE}}
\begin{adi3}
Call \mpidfunc{MPID_Request_probe}.
\begin{mmadi}
Look for a match in the unexpected receive queue. 
If no match is found, then wait until another message is receive and check
again.  In a polling, single-threaded implementation, this can simply invoke a
blocking call to wait for incoming messages. 

This routine immediately brings up the problem of how to structure code that
uses multiple threads to achieve good performance when a blocking call is used
by one or more user threads.

Note that in a multithreaded MPI implementation, this must watch for the race
condition of
\begin{verbatim}
     Thread 1                    Thread 2
 Check queue, no match found
                                 Handle incoming unexpected message
 Wait for a message to arrive
\end{verbatim}
Handling this is device-implementation specific.

Consider the following cases:
\begin{enumerate}
\item There is only one thread (e.g., the current \code{ch_p4} case).  In this
  case, after checking the queue, a call that polls the communication agent
  and waits for something to arrive (e.g., with \code{select} for TCP-only
  devices) may be used.  A multimethod device might briefly spin on all
  ``fast'' devices (e.g., shared-memory queues) and then yield the time slice.
\item There is a single thread that acts as the communication agent (see
  Section~\ref{sec:comm-agent}) that is different from the user's thread.
  In this case, the user's thread could pass control to the communication
  agent.  For example, it could use \code{pthread_cond_wait} to wait on the
  communication agent.  The communication agent can use
  \code{pthread_cond_broadcast} to release any user thread that is waiting on
  the communication agent (\code{pthread_cond_signal} may be better if only
  one user thread ever holds the condition variable).
\item There is one communication agent for each method.  For example,
  a TCP method that waits in select for activity on an fd and a thread that
  handles shared memory and may use (in Linux) \code{sched_yield} after
  spinwaiting (or, in systems that support condition variables shared between
  processes, may use that to wait for an incoming event or message).
  A similar approach may be used here: \code{pthread_cond_broadcast} can be
  used from any method's communication agent thread to release all waiting
  threads. 
\end{enumerate}

%Question: how efficient are the pthread condition wait routines?  Do they spin
%or do they yield the processor to other threads?  This is not specified in the
%POSIX standard; the question is more about the quality of implementation.

The POSIX \code{pthread_cond_wait} has two arguments: a mutex and a condition
variable.  The routine atomically unlocks the mutex and waits for the
condition variable. This suggests, for pthreads, the following solution for
\code{MPID_Request_probe} in the multithreaded cases:
\begin{verbatim}
    while (1) {
        pthread_mutex_lock( &queue_mutex );
        <look through queue>
        if (found) {
            pthread_mutex_unlock( &queue_mutex );
            return;
            }
        else 
            pthread_cond_wait( &queue_mutex, &cond );
    }
\end{verbatim}
In the case of a single user thread, you might want to try
\begin{verbatim}
    <look through queue>
    if (found) return;
    <same code as above>
\end{verbatim}
This is ok as long as the communication agent can't remove items from the
queue.  Unfortunately, cancel does just that.  

In fact, we may want to consider a higher-level abstraction, such as monitors,
which could (usually would) be implemented using locks and condition variables.
\end{mmadi}
\end{adi3}

\paragraph{Buffered send.}
The buffered send (both blocking and nonblocking) should first use
\mpidfunc{MPID_tBsend} to attempt 
and send the message before implementing the buffer-copying strategy.

Question: should the request needed to implement a buffered send be allocated
from the user-supplied buffer?  The standard suggests so, though it isn't
strictly required.  For example, only the information needed to create
the request could be saved; if no request is available, the bsend
handler can wait until later.

What utility routine should be defined to allocate buffer space from the
user-specified buffer?  How will it be made thread-safe?  What is the
interface to \mpifunc{MPI_REQUEST_FREE}?  How do we ensure that we wait on
pending bsend operations in a polling implementation?  See the file
\file{mpich/src/util/bsendutil2.c} in MPICH-1.  Note that while the buffer is
a global (! thread safety warning), it can be stored as a local
\code{static} variable in the file the implements the buffer
management utility routines.  

We need a \mpidfunc{MPIi_Bsend_init} and
\mpidfunc{MPIi_Bsend_finalize} to control the initialization and
finalization of the bsend buffer.  The initialization include
initializing the thread lock used to guard access to the rest of the
structures.   Finalization must ensure that any pending operations
complete (locally); thus the \mpidfunc{MPIi_Bsend_finalize} needs to
be called before any of the routines that free any data structures or state.

The bsend operations are roughly:
\begin{adi3}
Try \mpidfunc{MPID_tBsend}; if it succeeds, done.
Otherwise, find the first block in the buffer that is large enough for
the data, stored as packed with \mpifunc{MPI_Pack}.  
Save: the data, the type of the data (e.g., \code{MPI_PACKED} or some
contiguous type), the count, and the \code{MPI_Request} used to start
an \mpifunc{MPI_Isend} on the data.

As a slight refinement, we should be prepared to save the
communicator, tag, and rank, so that if no request is currently
available, the communication can be deferred until later.  SGI doesn't
do this currently, and as a result, their implementation fails on some
valid MPI programs.

If some bsend operations are deferred pending the availability of a
request, there needs to be some way for the communication agent to
know that it needs to try to send messages once requests become
available.

\end{adi3}
We take advantage of \mpiconst{MPI_BSEND_OVERHEAD} to ensure that each
block is aligned on a \code{double} (question: should we make it a
cacheline?)

The fields in the bsend buffer element include
\mpids{MPIi_Bsend_elm}{tag},
\mpids{MPIi_Bsend_elm}{communicator},
\mpids{MPIi_Bsend_elm}{rank},
\mpids{MPIi_Bsend_elm}{datatype},
\mpids{MPIi_Bsend_elm}{count}, and 
\mpids{MPIi_Bsend_elm}{request}, as well as \mpids{MPIi_Bsend_elm}{next}.

The bsend buffer itself is described by
\mpids{MPIi_Bsend_buffer}{buffer}, \mpids{MPIi_Bsend_buffer}{size},
\mpids{MPIi_Bsend_buffer}{head}, \mpids{MPIi_Bsend_buffer}{tail}, and
\mpids{MPIi_Bsend_buffer}{pending}.  The value of \code{tail} points
to the first free byte and \code{head} points to the first used byte,
or to \code{tail} if the buffer is empty.

Question: for the nonblocking versions, in principle, we could wait to
copy into the buffer until the wait/test.  Maybe in 2008.

\begin{mmadi}
MPID_tBsend:
Check message size.  
If not within eager limit, return false
If flow control allows, send and return true, 
else return false

Note that the flow control check may require a lock so that another
thread doesn't also try to send and change the flow control state.
\end{mmadi}


\subsubsection{\mpifunc{MPI_IBSEND}}
See Buffered send.

\subsubsection{\mpifunc{MPI_BSEND}}
See Buffered send.

\subsubsection{\mpifunc{MPI_BSEND_INIT}}
Question:  Should this reserve space in the buffer that is used as
necessary by \code{MPI_Start}?  No, that doesn't conform to the
standard's model implementation.  This should setup a persistant
request in the same way as the other persistent routines.

\subsubsection{\mpifunc{MPI_BUFFER_ATTACH}}
\begin{adi3}
If a buffer is already attached, return error.
Otherwise, attach the designated buffer and initialize the buffer as empty.
The buffer is organized as a circular buffer as described in the model
implementation of buffered mode in the MPI standard.
\end{adi3}

\subsubsection{\mpifunc{MPI_BUFFER_DETACH}}
Buffer detach must first wait for all operations to complete before
returning.  
In order to catch race conditions in a multi-threaded environment, 
\code{MPI_Buffer_detach} should set a flag on the buffer on entrance; all
buffered send operations should check this value before proceeding, generating
a \mpiconst{MPI_ERR_OTHER} class of type \code{THREAD_RACE} if the flag is
set.

\subsubsection{\mpifunc{MPI_CANCEL}}
\begin{adi3}\mpidfunc{MPID_Request_cancel}
\begin{mmadi}
Start by separating the cases:
\begin{itemize}
\item Inactive persistent.  Return error.
\item Active persistent.  Handle as a non-persistent of the same type.
\item Receive.  If already complete or in progess, cancel fails.  
\begin{core} 
Otherwise, remove request from receive queue (atomically).  In many
ADIs, this is a local operation.  However, some systems may require
logic similar to the send branch (this is the speculative receive
case, where information on a receive with a designated source is sent
directly to the sender).

Question: The check-and-remove must be done atomically.  This needs a
variation of the find-or-allocate that performs a find-and-remove.

\end{core}
\item Send. If already complete or in progress, cancel fails.
\begin{core} 
Otherwise use \mpidfunc{MPID_Rhcv} to send a cancel
request (\mpidconst{MPID_Hid_cancel}).  The cancel operation must wait for a
acknowledgement that 
indicates whether the cancel succeeded or failed.  

Question: How is this wait managed?  For a polling implementation, should this
use some (possibly higher latency) alternate communication path (e.g., a
\code{SIGIO} handler for TCP or \code{SIGUSR1} for shared memory)?

Again, the check and send if not in progress must be atomic.  
Essentially, the cancel matches the send request in the queue and
marks it as satisfied.

How is the send identified?  By the same request id that is used by
the \mpidconst{MPID_Hid_ok_to_send}?  

Note that if we receive an ok-to-send on this request, it means that
the send cancel has failed.  No separate negative-ack is required.
That is, in the communication agent, if the send is in progress, the
cancel request may be discarded without further action, at least in
cases where a separate acknowledgement is required to initiate the
data transfer described by a send.
\end{core}
\end{itemize}
\end{mmadi}
\end{adi3}

Question: do we want to have an attribute that indicates that there
are no send-cancels?  Handling send-cancel is the only part of MPI-1
that requires that a communication agent runs even if no MPI calls are
made by a process.

Question: there was some discussion that send-cancel required more
care that described here (including a time-stamp on requests to ensure
that the correct request was cancelled.  Here's the situation
\begin{verbatim}
     process 0                             process 1
   thread 0    thread 1                
   isend 
   cancel                                   irecv (matches isend)
             <------- ok-to-send  -------<
   send matched, removed
               isend(same request)
                                            isend arrives
                                            cancel arrives (late)
\end{verbatim}
At the end of this, the cancel matches the second isend even though it
should only match the first.  To avoid this, we only need a sequence
number on the messages; we need this anyway for supporting profiling.
Another alternative is to manage the list of available requests as a
FIFO rather than LIFO queue; in that case, there's very little chance
that the same request woul be used soon enough to cause any problems.  
Other solutions can be used; if the communication is handled by a
single thread, the fact that the request is not made available again
until the cancel is known to either have succeeded or failed
guarantees that no mismatch can occur.

\subsubsection{\mpifunc{MPI_IPROBE}}
\begin{adi3}\mpidfunc{MPID_Request_iprobe}
\begin{mmadi}
This is the same code as \mpidfunc{MPID_Request_recv_FOA}, but without
the allocation if no matching request is found.  See
\mpifunc{MPI_Probe} for additional discussion; \mpifunc{MPI_Iprobe},
of course, does not block.
\end{mmadi}
\end{adi3}

\subsubsection{\mpifunc{MPI_IRECV}}
\begin{adi3}
\mpidfunc{MPID_Irecv}
\begin{mmadi}
\mpidfunc{MPID_Segment}\\
\mpidfunc{MPID_Request_recv_FOA}\\
\mpidfunc{MPID_Stream_irecv}
\begin{core}
This should ensure that any polling call is made.  Question: what is
the interface with the communication agent?  Note that the agent will
arrange the actual data transfer.  In a multithreaded case, the
calling (user) thread only enqueues the request.

The implementation of \mpidfunc{MPID_Request_recv_FOA} might include:
\begin{mmadi}
Determine source.  If a specific source, find or add (atomically) to
the queue for that method. 
If \mpiconst{MPI_ANY_SOURCE}, then 
    lock common queue
    find or add (atomially) to each method's queue; stop if there is a
    match
    add to common queue
    unlock common queue.
\end{mmadi}

\end{core}
\end{mmadi}
\end{adi3}

\subsubsection{\mpifunc{MPI_IRSEND}}
Is this the same as \code{MPI_ISEND}, but with the ready mode set?  Or is
there a separate routine that exploits the ready nature of the operation?
Special note: we should be able to report an error when a ready send is not
matched by a posted receive.  This requires a bit for ready mode in the
\mpidconst{MPID_Hid_short} and \mpidconst{MPID_Hid_request_to_send}; perhaps a
\mpids{MPID_Hid_short}{is_ready}\index{MPID_Hid_request_to_send!is_ready}. 

\subsubsection{\mpifunc{MPI_ISEND}}
\begin{adi3}
\mpidfunc{MPID_Isend}
\begin{mmadi}
\mpidfunc{MPID_Segment}\\
\mpidfunc{MPID_Request_send_FOA}\\
\mpidfunc{MPID_Stream_isend}
\begin{core}
If the message is small in total size and flow control allows it, send
the data with \mpidconst{MPID_Hid_short} using \mpidfunc{MPID_Rhcv}.
Otherwise, send a \mpidconst{MPID_Hid_request_to_send} with
\mpidfunc{MPID_Rhcv}.  
\end{core}
\begin{via}
Attempt to register the source memory so that direct memory operations may be
used on it.  If this is not possible (e.g., no more memory can be pinned),
then copy to some preallocated space.
\end{via}
\end{mmadi}
\end{adi3}

Question: do we want a special case for short messages that reduces
overhead by indicating that the message is completed?  Note that for
any operation that is not complete, we must ensure that the related
objects, such as datatypes and communicators, are marked as in use (by
incrementing their \mpids{MPI_Comm}{ref_count}.  This is not needed if
the operation is already complete, and eliminating these steps may be
important for reducing the latency of short messages (particularly
since, in the multi-threaded case, updating the reference count
requires a write and, for objects being used by several threads, a
cache miss.

Another special case is the predefined, permanent objects.  Should the
\mpids{MPI_Datatype}{ref_count} be updated for those objects?  Is the
branch (we've already loaded the object identifier) faster than the
load/increment/store?  

\subsubsection{\mpifunc{MPI_ISSEND}}
Same as \code{MPI_ISEND}, but with the synchronous mode set.  

\begin{core}
For all sizes, use \mpidconst{MPID_Hid_request_to_send} with
\mpidfunc{MPID_Rhcv}.  Note that this includes messages of size zero.
In the absence of a speculative receive, this will never complete
before the routine an exit.

\end{core}
Note that an alternate approach is possible for short messages that uses
\mpidconst{MPID_Hid_short} but requires an acknowledgement when the message is
matched at the destination.  At least for now, we don't plan to implement this
approach (it was used in MPICH/ADI-1, but was not really worth the effort). 

\subsubsection{\mpifunc{MPI_RECV}}
This can simply be \code{MPI_IRECV} followed by an \code{MPI_WAIT}.  However,
we might want to let the partner (matching sender) know that this is blocking
a thread or process.  

Note that an early version of MPICH waiting for a specific tag,
context-id, and source tuple.  This sped up the process of matching
against the message queue (incoming messages were first matched
against that tuple), but it isn't correct if the queue already
contains entries with wild cards (e.g., \mpiconst{MPI_ANY_SOURCE}).

\subsubsection{\mpifunc{MPI_RECV_INIT}}
\begin{adi3}
Create a persistent request and save the message
parameters (\mpids{MPI_Request(persistent)}{communicator}, 
\mpids{MPI_Request(persistent)}{tag}, \mpids{MPI_Request(persistent)}{source_rank}, 
\mpids{MPI_Request(persistent)}{datatype},
\mpids{MPI_Request(persistent)}{buffer}, and
\mpids{MPI_Request(persistent)}{count}). 
Call \code{MPID_Memory_register} for either the buffer or a created segment
that will be used for receiving the data.  

Note that a persistent request is not like an
\mpidconst{MPID_Request}; rather, it only contains enough information
to identify it as a persistent request and a pointer to a normal
\mpidconst{MPID_Request}.  In fact, the pointer to the request can be
used to indicate whether the persistent request is active, rather than
using a separate field.  This field could be
\mpids{MPI_Request}{active_request}.
\end{adi3}

Note that \mpidfunc{MPID_Memory_register} must fix both the physical
memory and the virtual to physical mapping, so that both any
peripheral device (such as a network card that implements VIA) and the
process will both access the same locations.  Apparently, Linux
doesn't provide this service, and it must be emulated by restricting
the behavior of \code{malloc}!

\subsubsection{\mpifunc{MPI_REQUEST_GET_STATUS}}
Extract the status data from the request.
This requires either an \mpidfunc{MPID_Request_get_status} or clearly defined
status elements in the \mpidfunc{MPID_Request}.

Question: do we want to use something besides status?  If we do use
status, it should be the \mpids{MPI_Request}{status} field.

\subsubsection{\mpifunc{MPI_REQUEST_FREE}}
\begin{adi3}\mpidfunc{MPID_Request_free}
%\begin{mmadi}
%\begin{core}
%\end{core}
%\end{mmadi}
\end{adi3}
Who handles persistent or user-defined requests?

Note that this is not the same as cancelling a request.  
A request that is freed must still complete.  Thus, the request needs
a reference count (\mpids{MPI_Request}{ref_count}) that must be
checked when the related communication 
completes.  If the count is zero, there will be no \code{MPI_Wait}
etc. call, and the request data strucutre must be recovered.  A
polling device that expects a wait or test call may need to maintain a
list of freed but not completed requests and effectively call
\mpidfunc{MPID_Testsome} on that list.  See Section~\ref{sec:comm-agent}.

\subsubsection{\mpifunc{MPI_RSEND}}
Like \mpifunc{MPI_Send}, but with the ready mode.  This is a bit in
the message header (at least when full debugging is enabled) that can
be checked at the destination to detect an erroneous use of
\mpifunc{MPI_RSEND} (no matching receive).

\subsubsection{\mpifunc{MPI_RSEND_INIT}}
See \mpifunc{MPI_RECV_INIT}.

\subsubsection{\mpifunc{MPI_SEND}}
See \mpifunc{MPI_ISEND}.  This could simply be isend followed by wait.

Question: do we want to indicate to the receiver that this is a
blocking send?  For example, that would suggest a higher priority in
handling the operation, since the source process may be blocked on
this operation.

\subsubsection{\mpifunc{MPI_SENDRECV}}
Question: If the source and destination are the same, is there anything
special that we want to do?  Note that this routine matches communication from
any point-to-point operation, not just other sendrecv calls.  
Simply use \mpifunc{MPI_Isend}, \mpifunc{MPI_Irecv}, and
\mpifunc{MPI_Waitall}. 

Note that in the case of a rendezvous exchange where the data is sent
in a number of blocks, an exchange can be handled more efficiently
that two independent isend/irecv pairs, as shown in Figure~\ref{fig:sendrecv}.
\begin{figure}
\begin{verbatim}
(temporary figure)
    (a) isend/irecv                           (b) combined
  p0                  p1                p0                   p1
  data ------->                        data -------------->
       <-----------   ack                   <-------------- data+ack
       <-----------  data              data+ack ------------>
  data ------------>
  ack  ------------>
\end{verbatim}
\caption{Two sendrecv scenarios}\label{fig:sendrecv}
\end{figure}
Question: do we want to allow ok-to-send acks to be piggy-backed onto
an ongoing data stream?  Once a data stream starts, we can record that
it is ongoing; any ack to that partner can be added to the ongoing
stream.  If no communication is pending, the ack can be sent
immediately.

Note that since \mpifunc{MPI_Sendrecv} can match other MPI
communication calls, such as \mpifunc{MPI_Send} and
\mpifunc{MPI_Irecv}, we cannot depend on \mpifunc{MPI_Sendrecv} as
giving us enough information to decide whether to piggyback acks on a
data stream.
\subsubsection{\mpifunc{MPI_SENDRECV_REPLACE}}
Question: Should this use the segment code to bound the memory buffer that
must be allocated for the replacement?
\begin{adi3}
\mpidfunc{MPID_Segment}, \mpidfunc{MPID_Stream_irecv}, and
\mpidfunc{MPID_Stream_isend}. 
\end{adi3}

\subsubsection{\mpifunc{MPI_SEND_INIT}}
See \mpifunc{MPI_RECV_INIT}.

\subsubsection{\mpifunc{MPI_SSEND}}
Like \mpifunc{MPI_Send}, but with the synchronous mode.

\subsubsection{\mpifunc{MPI_SSEND_INIT}}
See \mpifunc{MPI_RECV_INIT}.

\subsubsection{\mpifunc{MPI_START}}
This just calls \mpifunc{MPI_STARTALL} with a single request.  Note that
errors must indicate that the error occurred in \code{MPI_START}, not
\code{MPI_STARTALL}.  

\subsubsection{\mpifunc{MPI_STARTALL}}
Do we want startall to allow for some scheduling of the operations?  For
example, it could start the ``furthest away'' first.  It could also batch
operations.
% (see the paper on improving TCP performance \cite{??}).  
If so, we need an \mpidfunc{MPID_Startall}.

Should each of the individual persistent routines provide an internal
routine that is used to start the operation?
These can simply call the related non-persistent routine using the fields from
the persistent request (e.g., \mpids{MPI_Request(persistent)}{communicator})
and storing the new request in the \mpids{MPI_Request(persistent)}{request}
field. 

Note that generalized requests are not started with \mpifunc{MPI_START}; i.e.,
there is no persistent generalized request.

\subsubsection{\mpifunc{MPI_STATUS_SET_CANCELLED}}
Where are the values defined for indicating cancelled message?
A \mpidconst{MPID_COUNT_MSG_CANCELLED} for the count field in the
status?
(We shouldn't use the \mpids{MPI_Status}{MPI_TAG} field of
\mpiconst{MPI_Status} because that field is visible to the user.)

\subsubsection{Point-to-point completion functions}
There are several special kinds of requests that require special handling by
all of the completion (e.g., \code{MPI_Test} and \code{MPI_Waitany})
functions. 

Generalized requests:  On completion, invoke the
\mpids{MPI_Request(generalized)}{free_fn}. 

Persistent requests: The actual request is the
\mpids{MPI_Request(persistent)}{active_request} in the
\mpidconst{MPID_Request} structure.

The two most basic routines are \mpifunc{MPI_Testany} and
\mpifunc{MPI_Waitany} in the sense that all of the other operations
can be built from these.  For example:
\begin{description}
\item[\mpifunc{MPI_Wait}]\mpifunc{MPI_Waitany}
\item[\mpifunc{MPI_Waitsome}]\mpifunc{MPI_Waitany} followed by
\mpifunc{MPI_Testsome} (or \mpifunc{MPI_Testany} until flag is false).
Note that without the \mpifunc{MPI_Testsome} 
call, the requirements of \mpifunc{MPI_Waitsome} won't be met; in
particular, the \mpifunc{MPI_Testsome} is needed to allow
\mpifunc{MPI_Waitsome} to provide fairness (indicate \emph{all}
requests that are ready).
\item[\mpifunc{MPI_Waitall}]\mpifunc{MPI_Waitany} until all non-null
requests have completed.
\item[\mpifunc{MPI_Test}]\mpifunc{MPI_Testany}
\item[\mpifunc{MPI_Testsome}]\mpifunc{MPI_Testany} until flag returns
false.
\item[\mpifunc{MPI_Testall}]\mpifunc{MPI_Test} for each request.
\end{description}
These are not necessarily the best implementations of the eight
completion functions, but they do provide reasonable implementations
as long as the number of requests provided to the completion function
is not too large (since some of the algorithms above have complexity
proportional to the square of the number of requests).

Question:  Should we define \mpidfunc{MPID_Testany} and
\mpidfunc{MPID_Waitany} instead of the \mpidfunc{MPID_Testsome} and
\mpidfunc{MPID_Waitsome} routines?

\subsubsection{\mpifunc{MPI_TEST}}
Call \mpifunc{MPI_Testany}.

\subsubsection{\mpifunc{MPI_TESTALL}}
See \mpifunc{MPI_TESTANY}.

\subsubsection{\mpifunc{MPI_TESTANY}}
Call \mpidfunc{MPID_Testsome} on each request, one at a time.

Question: do we want something better?  Can we subdivide
\mpidfunc{MPID_Testsome} into smaller building blocks from which we
can build the four test routines without dropping down to an
\mpidfunc{MPID_Test}?  For example, the closest match to \code{select} is to
implement xxxsome, for advancing the state/completing operations, but decide
how to update/record the requests that have completed based on whether the
call is implementing the any, all, or some version.

\subsubsection{\mpifunc{MPI_TESTSOME}}
Call \mpidfunc{MPID_Testsome}.

\subsubsection{\mpifunc{MPI_TEST_CANCELLED}}
Does this use the status field in a request to test for a cancelled message?

\subsubsection{\mpifunc{MPI_WAIT}}
Call \mpifunc{MPI_Waitany}.

\subsubsection{\mpifunc{MPI_WAITALL}}
See \mpifunc{MPI_Testall}.

\subsubsection{\mpifunc{MPI_WAITANY}}
See \mpifunc{MPI_Testany}.

\subsubsection{\mpifunc{MPI_WAITSOME}}
Call \mpidfunc{MPID_Waitsome}.

\subsection{Communication Agent}
\label{sec:comm-agent}
All implementations require some sort of communication agent.  This agent
handles the delivery of data as described by \mpifunc{MPI_Recv} and
\mpifunc{MPI_Irecv}, RMA operations that require action at the target (such as
handling complex datatypes and for two-sided communication layers),
progress for nonblocking sends, and more
subtle operations such as cancelling of nonblocking sends.  This agent may be
invoked explicitly (a polling interface) or implicitly (e.g., in response to
an I/O interrupt or a thread-schedule event).  

\begin{tcp}
Receive a message header.  This is a message sent with \mpidfunc{MPID_Rhcv},
and the header is one of the \mpidconst{MPID_Hid_xxx_t} structures.
Based on the header type, invoke the appropriate processing routine (there
should be one for each header type, such as
\mpidfunc{MPID_Hid_xxx_method}). 

Still to do:  List the detailed behavior of head message type.
\begin{description}
\item[\mpidconst{MPID_Hid_short}.]
Call \mpidfunc{MPID_Request_recv_FOA}.  If found, transfer data using
\mpidfunc{MPID_Unpack}.  Otherwise, save data (transfer with \code{memcpy}) in
message buffer area.  Update flow control.
Special case: If ready-message bit is set and no matching message
exists, use \mpidconst{MPID_Hid_control} to return a remote error
indication.
Fields include \mpids{MPID_Hid_short}{size},
\mpids{MPID_Hid_short}{tag}, \mpids{MPID_Hid_short}{context_id},
\mpids{MPID_Hid_short}{source_rank}, and
\mpids{MPID_Hid_short}{data}.  It may also include
\mpids{MPID_Hid_short}{type_sig}, containing the shortened type
signagure, and \mpids{MPID_Hid_short}{flags}, containing various flags
(such as \mpidconst{MPID_READY_SEND}).  Heterogeneous systems must
also contain \mpids{MPID_Hid_short}{data_format}, which may include
the rank of the process that packed the data for data sent with
\mpiconst{MPI_PACKED} (see Section~\ref{sec:mpi-unpack}).

\item[\mpidconst{MPID_Hid_request_to_send}.]
Call \mpidfunc{MPID_Request_recv_FOA}.  If found, reply with
\mpidconst{MPID_Hid_ok_to_send} sent with \mpidfunc{MPID_Rhcv}.  Otherwise,
store message id in newly allocated request and mark request as ready
for use.  
Fields include \mpids{MPID_Hid_request_to_send}{size},
\mpids{MPID_Hid_request_to_send}{tag},
\mpids{MPID_Hid_request_to_send}{context_id}, 
\mpids{MPID_Hid_request_to_send}{source_rank}, and a
\mpids{MPID_Hid_request_to_send}{request_id}, which is used to
identify this operation to the sender when it is acknowledged.  
It may also include
\mpids{MPID_Hid_request_to_send}{type_sig}, containing the
request_to_sendened type 
signagure, and \mpids{MPID_Hid_request_to_send}{flags}, containing
various flags 
(such as \mpidconst{MPID_READY_SEND}).  Heterogeneous systems must
also contain \mpids{MPID_Hid_request_to_send}{data_format}, which may include
the rank of the process that packed the data for data sent with
\mpiconst{MPI_PACKED} (see Section~\ref{sec:mpi-unpack}).

Question: In order to implement the store and forward stream
operation, should there be a ``disposition'' field that indicates what
to do with the raw (pre-unpacked) data?

\item[\mpidconst{MPID_Hid_ok_to_send}.]Send the requested data back
with a header of \mpidconst{MPID_Hid_data}.  Note that this may be
only a subset of the full data.  That is, a long message may be sent
as multiple pieces.  Each piece requires an \mpidconst{MPID_Hid_data}
header.
Fields include \mpids{MPID_Hid_ok_to_send}{request_id} and
\mpids{MPID_Hid_ok_to_send}{recv_id}, which is used to identify the
data in \mpidconst{MPID_Hid_data}.  

Question: do we also want to return a maximum data length to indicate
the maximum amount of data that should be returned?

\item[\mpidconst{MPID_Hid_data}.]Is this just \mpidconst{MPID_Hid_put}
where the address has been sent with \mpidconst{MPID_Hid_ok_to_send}?
The fields in this structure include \mpids{MPID_Hid_data}{recv_id},
\mpids{MPID_Hid_data}{size}, and \mpids{MPID_Hid_data}{data}.  We may
also want to include the corresponding
\mpids{MPID_Hid_data}{request_id} to simplify the process of handling
the case wheere the data is delivered in several parts.

\item[\mpidconst{MPID_Hid_cancel}.]Attempt to access a previous
request.  If unmatched, remove it and return a success
\mpidconst{MPID_Hid_cancel_ack} using \mpidfunc{MPID_Rhcv}.
Otherwise, return a failure and leave the request as unchanged.
The fields here include \mpids{MPID_Hid_cancel}{request_id} (matching
a \mpidconst{MPID_Hid_request_to_send}).

\item[\mpidconst{MPID_Hid_cancel_ack}.]Set the cancel success/failure
field of the indicated request.  Question: Do requests need a separate
field to indicate failed cancel, or can \mpifunc{MPI_Test_cancelled}
return failure if it doesn't find success (e.g., do we need to
distinquish between cancel attempted and failed and never attempted?
Question: Should there be a single cancel type with an operation
subfield that combines \mpidconst{MPID_Hid_cancel} and
\mpidconst{MPID_Hid_cancel_ack}?
The fields here include \mpids{MPID_Hid_cancel_ack}{request_id} and
\mpids{MPID_Hid_cancel_ack}{flag}, indicating success or failure.

Question: do we ever need the failure, since that will mean that a
\mpidconst{MPID_Hid_ok_to_send} was generated for the request?

\item[\mpidconst{MPID_Hid_lock_op}.]Manipulate a lock on the indicated
window object.  The operation are lock, lock-exclusive, unlock, and
lock-grant (we don't need an unlock-grant).  The window object is
specified by the window object id (previous set as part of the
\mpifunc{MPI_Win_create} step).
Fields include \mpids{MPID_Hid_lock_op}{source_rank},
\mpids{MPID_Hid_lock_op}{window_id}, and
\mpids{MPID_Hid_lock_op}{flags}, which indicate whether this is a lock
or unlock, and whether it is exclusive or not.

\item[\mpidconst{MPID_Hid_win_count_op}.]Modify one of the local window
start/complete counters (see \mpifunc{MPI_Win_start}).  

\item[\mpidconst{MPID_Hid_put}.]Emulate a put operation.  The handler
contains the offset of the destination and window object id, as well
as the size.  
Question: should we have the sender convert the offset to an address?
Question: does this assume contiguous data?  If it does, do we want a
strided access version?
Fields include \mpids{MPID_Hid_put}{offset},
\mpids{MPID_Hid_put}{window_id}, \mpids{MPID_Hid_put}{datatype_id},
and \mpids{MPID_Hid_put}{count}.

Question: do we really want a stream version of this instead?  How do
we implement streams without it?

\item[\mpidconst{MPID_Hid_accumulate}.]Emulate an accumulate
operation.  Much like \mpidconst{MPID_Hid_put}.

\item[\mpidconst{MPID_Hid_get}.]Like \mpidconst{MPID_Hid_put}, but for
the get operation.  Question: Should there be a single RMA type, with
put, get, and accumulate as separate subtypes?

\item[\mpidconst{MPID_Hid_flow}.]Flow control.  This sends an update
about the free resources of the sender.  Should a variation of this
request an update?  Also, do we want to a version that request more or
less resources for buffering?  Do we want to piggyback flow control
information on all headers?  E.g., we could use a single signed char
(byte) to update the number of data blocks in use; this would reduce
the number of flow-control only messages.

\item[\mpidconst{MPID_Hid_datatype_desc}.]Cache a datatype description
to be used in RMA operations.  This is similar to a
\mpidconst{MPID_Hid_request_to_send}, including the need to use a
rendezvous to send complex and lengthy datatypes (e.g., indexed
types).
Question: As for \mpidconst{MPID_Hid_lock_op}, should there be a
single type with subfields?  

\item[\mpidconst{MPID_Hid_control}.]This is a general-purpose control
message.  For example, this could be used to implement an abort,
disconnect, deadlock detection, remote error indication (e.g., a ready
send that was not matched), and orderly exit.
\end{description}

Questions and comments:

The communication agent for methods based on network read/write
operations such as TCP should probably use a buffered read to reduce
the number of system 
calls.  The buffering code should be able to switch to direct read when a
large amount of data is being moved (by emptying the buffer and then switching
to direct read).  

How is flow control handled?  Should flow control be bundled with the buffered
read/write logic?

Are there separate stream operations?  Are there stream operations
instead of the \mpidconst{MPID_Hid_request_to_send} etc. operations?
\end{tcp}

\begin{shmem}
Is this the same as \tcpname, but with slightly different message
types?
Where are there differences?  For example, should datatypes be stored
in shared memory, thus eliminating the
\mpidconst{MPID_Hid_datatype_desc}?  If so, how does the device ensure
that the datatype code places everything in shared memory (e.g.,
\mpifunc{MPI_TYPE_CREATE_INDEXED} cannot use \code{malloc} to save the
index arrays)?  Should the lock/unlock operations
be direct rather than using \mpidconst{MPID_Hid_lock_op}?
If the message queues are stored in shared memory, how does that
change the various routines?  For example, does a \shmemname\ device not
use \mpidconst{MPID_Hid_short}, using instead direct access to shared
memory so as to reduce the latency of short messages?

Note that since the assumption is that only some memory is shared, it
isn't possible to implement most message operations without using a
communication agent at the destination to implement the transfer of
data into user-defined buffers that are not in shared memory.

Question:  Do we want a separate case for systems that can share all
of process memory (Windows and Linux)?
\end{shmem}

\begin{via}
Like \tcpname, except some data can be transfered without the agent
(remote read and write (get/put) operations).
\end{via}

% subsection collective communication and computation; contains the
% discussion.  The description of each individual routine remains in the
% mpich2.tex file for now.
\subsection{Collective Communication and Computation}
\label{sec:collective-comm}

One of the major changes in MPICH2 is in the implementation of the
collective routines.  The MPICH2 implementation will exploit
pipelining and store and forward algorithms; these are supported by
the XFER interface. 
%\code{MPID_Stream_xxx} routines.  

Since each system may have some feature that provides for even faster
implementation of the collective routines, it will be possible to
substitute a system-specific implementation for any of the collective
routines.  The purpose of the implementations provided with MPICH is
to provide a level of performance that will be adequate
for many users.  

The $\alpha$-tree approach described in
\cite{bern:mpi-collective:hpcn99} should be considered; this is a
simple variation on the binomial tree approach used in the MPICH
implementations of many of the collective routines.  We will consider
combining this with the pipelining and scatter/gather approaches
championed by van de Geijn (\cite{vandegeijn} isn't quite the right
reference but it will do for now).

\subsubsection{Reduction functions}
The reduction functions must use the \code{restrict}\index{restrict} qualifier.

Each reduction operation (e.g., \code{MPI_SUM}) has a corresponding
implementation (e.g., \code{MPIR_Sum}) and is placed in a separate file (e.g.,
\file{opsum.c}.  Each of these must be careful to conditionally include the
Fortran datatypes and Fortran logical operations (see
Section~\ref{sec:fortran}).

Some reduction functions are not defined on a particular datatype.  To
indicate errors, the routine \code{MPID_Op_set_error} is called.
The MPI reduction routine (reduce, allreduce, scan, exscan, and reducescatter)
checks this with \code{MPID_Op_get_error}.  In order to ensure that
separate threads manage their own error flags for reductions, there is
an \mpids{MPIR_PerThread}{op_error} field in the per-thread data
structure.

% An implementation of this might be 
% \begin{verbatim}
% #ifdef MPID_HAS_THREADS
% /* Use thread private storage for the error value.  Allocate on demand */
% extern int MPID_Op_error_key;
% void MPID_Op_error_delete( void *val ) { if (val) free(val); }
% #define MPID_Op_error_init \
%     pthread_key_create( &MPID_Op_error_key, MPID_Op_error_delete )
% #define MPID_Op_error_finalize \
%     pthread_key_delete( MPID_Op_error_key )
% #define MPID_Op_set_error(err) {\
%  int *e = (int*)pthread_get_specific(MPID_Op_error_key); \
%  if (!e) { e = (int*)malloc(sizeof(int));\
%            pthread_set_specific( MPID_Op_error_key, e );}\
%  *e = err; }
% #define MPID_Op_get_error(err_p) \
%     *err_p = *(int*)pthread_get_specific(MPID_Op_error_key)
% #else
% #extern int MPID_Op_error;
% #define MPID_Op_error_init
% #define MPID_Op_error_finalize
% #define MPID_Op_set_error(err) MPID_Op_error = err
% #define MPID_Op_get_error(err_p)  *(err_p)=  MPID_Op_error
% #endif
% \end{verbatim}


\subsubsection{Code Structure for the Implementation of the Collective
  functions} 

The MPICH code uses one gigantic file, \file{intraops.c}, to provide a
generic implementation of each collective operation.  Each
communicator has a structure of pointers to functions.  Unless
otherwise set, each communicator points to the predefined structure
\code{MPIR_intra_collops}\index{MPIR_intra_collops} which is
initialized to point to \emph{all} of 
these functions.  

For MPICH2, each of the MPI functions (in its own file) contains the generic
implementation of the collective operation, based initially on the
point-to-point code similar to that in MPICH-1 and eventually on the
stream-oriented operations.  This will simplify the process of tuning each
operation; it will also reduce the size of (unshared) executables since few if
any programs use all of the collective operations.
See the discussion of the implementation of PMPI.

Question: Now that MPI-2 defines intercommunicator collective
routines, do we want these in the same file as the intracommunicator
routines, or in an alternate file.  E.g., should \file{bcast.c} contain
the intracommunicator implementation of \mpifunc{MPI_Bcast} and
\file{icbcast.c} contain the intercommunicator implementation.
Also, we may select no intercommunicator collectives at configure (or run?)
time to reduce the size of libraries and code.

We may want to have multiple ``generic'' implementations and an easy
way, say with the runtime parameter routines, to select among them at runtime.

We may want to compute and save things like the neighbors for each
collective communication pattern; this can be done either when the
collective operation is first encountered or at communicator creation
time (the descision could be a runtime attribute).  Question: how do
we modularize this?  Is there a ``collective'' 
attribute?



\subsubsection{Collective Computation}

\subsubsection{\mpifunc{MPI_OP_CREATE}}
Create the object and set the \mpids{MPI_Op}{kind},
\mpids{MPI_Op}{language}, and \mpids{MPI_Op}{function}.  Use an 
internal routine that can be shared with the init routine to create
the predefined operations.

In the multithreaded case, \mpiconst{MPI_Op} needs to have reference counts.
Since operations on \mpiconst{MPI_Op} are infrequent, we should have a
\mpids{MPI_Op}{ref_count} field for all cases (even single threaded).

How do we handle the predefined types?  Who creates them?  Do we want an 
\mpidfunc{MPIi_Op_init} and \mpidfunc{MPIi_Op_finalize}?

\subsubsection{\mpifunc{MPI_OP_FREE}}
If predefined and not in finalize, indicate error.
Otherwise, decrement reference count and free if zero.

%Question: do we need a reference count?  Not in a single threaded
%case; in a multithreaded case, you could argue that a user that
%frees an \mpiconst{MPI_Op} while a collective routine is using it in
%another thread has written an erroneous program.

\subsubsection{Intracommunicator Collective Operations}
The following section (will) briefly describe the algorithms used to implement
the intracommunicator collective operations.

Many functions support \mpiconst{MPI_IN_PLACE} as an argument.  These need to
be prepared for that case.

One important check is to test for mismatched collective operations.
MPICH uses a different tag value for communication for each collective
operation, but has no way to test for a mismatch (because the
communication selects on tag and, without preceeding all communication
with an \mpifunc{MPI_Iprobe} call to check that the ``next'' message
has the right tag.  We might want a routine that returns an
``unexpected message'' when it finds a message with a different tag
from a particular source and communicator.
That is, if the communication is a virtual stream (virtual in the
sense of being separate for each communicator/rank pair, stream as
being ordered), then it is
an error to see a message with an different tag value.  

Question: do we need an MPID routine to implement this? Is it a
(optional) feature of the stream routines?

We also want to provide the option to check other parameters in
collective calls, for example, that the message sizes conform or that
all processes agree on the root.  One approach is, when an error is
detected locally, to send the usual header but no data and with an
error indication in the header.

General question:  A number of the algorithms make send data destined
for several processes to an intermediate process.  For example, a
\mpifunc{MPI_Scatter} might send the data destined for processes $p/2$
to $p-1$ to process $p/2$; that process in effect becomes the root for
a smaller broadcast.  However, this works easily only if (a) the data
is contiguous and (b) the subset of processes is also contiguous in
rank.  If we exploit topology information to determine a better
communication pattern, the contiguity of ranks in the process subsets
may be broken.  Do we want to handle this by rearranging the data to
match the ordering of the topology?  If so, we need to keep a flag
with the communicator topology information that indicates whether a copy is
necessary or not. 

We also need some common routines for collective argument checking.  These
fall into a few cases:
\begin{enumerate}
\item All must have the same value.  For example, the \code{root} value in an
  intracommunicator broadcast or the operation in an allreduce.
\item All must specify the same type signature (number of items and types).
  For example, the \code{count} and \code{datatype} in an intracommunicator
  broadcast.
\end{enumerate}

\subsubsection{\mpifunc{MPI_ALLGATHER}}
For short data, use recursive doubling algorithm.  For long data, consider the
bucket brigade algorithm.

Question:  For heterogeneous systems, the decision as to whether the
data is ``short'' needs to be made relative to some cannonical
representation, such as XDR or external32.  What is the routine to
determine cannonical size?  Also, do we want to have a separate routine for
the heterogeneous case?

\subsubsection{\mpifunc{MPI_ALLGATHERV}}
Same as \mpifunc{MPI_ALLGATHER}, since all processes can make the
short/long determination.

Question: Is there a role for a common routine to compute total message
lengths from a count array and a datatype, and to precompute the locations in
the send and/or receive buffer for communicating?

\subsubsection{\mpifunc{MPI_ALLREDUCE}}
Consider recursive doubling algorithm, with care taken to ensure that all
results are the same, particularly in the presence of extended registers
(e.g., 80 bit intermediate quantities on Intel).

Note that \mpiconst{MPI_IN_PLACE} is valid for this routine. 

Question: Do we want to implement this using a modification of the accumulate
operation?  That might be a simpler way to handle \mpiconst{MPI_IN_PLACE}.

Question: What should be used in the heterogeneous case?  Should that
reduce to \mpifunc{MPI_Reduce} followed by \mpifunc{MPI_Bcast}?

\subsubsection{\mpifunc{MPI_ALLTOALL}}
For short data and for all data on completely connected networks, use a
hypercube algorithm: each process exchanges with its partner in that dimension
the data needed by the partner and that partner's subsequent partners (in the
remaining dimensions).

For long data on less capable networks, use a bucket brigade algorithm.

\subsubsection{\mpifunc{MPI_ALLTOALLV}}
This is similar to \code{MPI_ALLTOALL}, but the decision on size of data is
more complicated.  

\subsubsection{\mpifunc{MPI_ALLTOALLW}}
Like \code{MPI_ALLTOALLV}.

\subsubsection{\mpifunc{MPI_BARRIER}}
Barrier is a special case of allreduce with no operation or data.

\subsubsection{\mpifunc{MPI_BCAST}}
Broadcast will be implemented by a scatter followed by an allgather.  These
will use an ordering of nodes from \mpidfunc{MPID_Topology_xxx}, rather than
the rank ordering of the communicator.
Data that is very short (e.g., a single int) should use a MST (minimal
spanning tree).  The tree itself should be defined by
\mpidfunc{MPID_Topo_cluster_info} or something similar (e.g., a
\mpidfunc{MPID_Topo_MST} function).  
Since the tree should be defined by the topology rather than computed, the
algorithm should look something like (this is the simple MST, not the
scatter/allgather approach)
\begin{algorithm}
Get_MST( \&parent, \&nchildren, \&children );
if (*parent) 
    Recv( from *parent )
for (i=0; i<nchildren; i++)
    Send( to (*children)[i] )
\end{algorithm}

\subsubsection{\mpifunc{MPI_EXSCAN}}
Use the same approach as \code{MPI_SCAN} but do not include the local
contribution in the local result.

\subsubsection{\mpifunc{MPI_GATHER}}
This will use a MST for short gathers to reduce the impact of latency.  

\subsubsection{\mpifunc{MPI_GATHERV}}
Like \mpifunc{MPI_Gather}, but the amount of data sent can be different in
each process.  The root process knows what is coming from each other process,
but the other processes can't tell how much data is being moved.  Thus, it
can't easily choose to use different algorithms for short and long messages.  

\subsubsection{\mpifunc{MPI_REDUCE}}
Use a spanning tree.  Pipeline for long vectors.

\subsubsection{\mpifunc{MPI_REDUCE_SCATTER}}
For short data, this can use \mpifunc{MPI_Reduce} followed by
\mpifunc{MPI_Scatterv}.  On 
complete networks, it is possible to implement this by using hypercube-like
exchange algorithms.

For long data, this should use the bucket brigade algorithm.

\subsubsection{\mpifunc{MPI_SCAN}}
Use reflection.  At step $k$, processes with rank $r$ exchange their current
result with the process at $r+2^k$ or a $r-2^k$, where the sign is positive if
the $k$th bit of $r$ is not set, and negative otherwise (see
Figure~\ref{fig:scan-pattern}).  This allows the scan 
to be computed in $\log p$ steps.  

Note that towards the end of this process, some of the exchanges are not
needed; the data needs to flow only to the processes with higher rank, not
lower rank.  Do we want to do this or does it complicate the code?

Note also that this algorithm must use the rank order of the communicator, not
a reordering for a better fit to the topology of the system.  If there is
strong clustering in the underlying interconnect topology, a different
algorithm will be needed.

\begin{figure}
\centerline{\psfig{file=scanpattern.eps}}
\caption{Communication pattern for \code{MPI_Scan}}\label{fig:scan-pattern}
\end{figure}

\subsubsection{\mpifunc{MPI_SCATTER}}
For short data, use an MST. On store and forward networks, MST should be used
for long data as well.  To avoid excessive memory consumption, the
\mpidfunc{MPID_Stream_iforward} routines should be used.

On a switched network, an MST may not be optimal for the long case.  Do we
want to provide a simple send-to-each in that case?

\subsubsection{\mpifunc{MPI_SCATTERV}}
For short data, use an MST. 

\subsection{Intercommunicator Collective Operations}
(Not yet done)
\mpiconst{MPI_ROOT} and \mpiconst{MPI_PROC_NULL} are used by the group
containing the root process; the other group refers to the rank of the root.
% \subsubsection{\mpifunc{MPI_ALLGATHER}}
% \subsubsection{\mpifunc{MPI_ALLGATHERV}}
% \subsubsection{\mpifunc{MPI_ALLTOALL}}
% \subsubsection{\mpifunc{MPI_ALLTOALLV}}
% \subsubsection{\mpifunc{MPI_ALLTOALLW}}
% \subsubsection{\mpifunc{MPI_BARRIER}}
% \subsubsection{\mpifunc{MPI_BCAST}}
% \subsubsection{\mpifunc{MPI_ALLREDUCE}}
% \subsubsection{\mpifunc{MPI_REDUCE}}
% \subsubsection{\mpifunc{MPI_EXSCAN}}
% \subsubsection{\mpifunc{MPI_GATHER}}
% \subsubsection{\mpifunc{MPI_GATHERV}}
% \subsubsection{\mpifunc{MPI_SCAN}}
% \subsubsection{\mpifunc{MPI_SCATTER}}
% \subsubsection{\mpifunc{MPI_SCATTERV}}

\subsection{Topology}

How do we implement \mpifunc{MPI_Cart_create} and \mpifunc{MPI_Dims_create}
with the MPID routines?   
Do we need an \mpidfunc{MPID_Topology_cart} and
\mpidfunc{MPID_Topology_cart_dims}?  Constructing a mesh from the 
hierarchical description that we've included can only be done
approximately.

The MPICH implementation uses private attributes to hold this information 
within a communicator.  Do we still want to do that?  Do we want to make the
topology routines a separate module so that, at the MPI level, it is easy to
substitute for these routines?  How do we define the routine called by
\mpifunc{MPI_Init} to initialize these routines (e.g., acquire attribute keys
for the topology information)?

One advantage to using attributes (or equivalently a pointer to a
structure) is it allows any information to be saved in with the
communicator, not just some predefined fields.  Note that
\mpifunc{MPI_Comm_dup} is required to copy both attributes and
topologies, so it makes sense to implement topologies as an attribute.

If attributes are used for holding topology information, we need to
\emph{add} a routine to initialize the corresponding keyval; this will
allow the topology routines to be replaced as a module.

Question: how do we define routines needed to support the MPI (not
MPID) calls?  Should we have \code{MPIi} routines?  For topology, we
could have \mpidfunc{MPIi_Topo_init} (and the corresponding
\mpidfunc{MPIi_Topo_finalize}).   Note that we need this to allocate the
keyval used to store the topology attributes.  We also need a
\mpidconst{MPID_Topo_graph_t}, \mpidconst{MPID_Topo_cart_t}, and
\mpidconst{MPID_Topo_common_t} structure that holds the information for each
topology (the common type is a subset of the other two that provides access
only to the topology type).

\subsubsection{\mpifunc{MPI_CARTDIM_GET}}
Access the topology description and return the number of dimensions of
Cartesian topology (if defined) from the \mpids{MPID_Topo_cart_t}{ndims}
field. 

\subsubsection{\mpifunc{MPI_CART_CREATE}}
This routine argues for a corresponding MPID routine, along with one for dims
create. Alternately, as suggested by the MPI standard, this could call
\mpifunc{MPI_CART_MAP} followed by \mpifunc{MPI_COMM_SPLIT}.

\subsubsection{\mpifunc{MPI_CART_GET}}
Access the topology description and return the associated fields
(\mpids{MPID_Topo_cart_t}{dims}, \mpids{MPID_Topo_cart_t}{periods}, and
\mpids{MPID_Topo_cart_t}{coords}). 

\subsubsection{\mpifunc{MPI_CART_MAP}}
This routine should call \mpidfunc{MPID_Cart_map}.  A trivial implementation
of this routine (as described in the MPI standard) is to simply return the
rank of the process in the input communicator.

\subsubsection{\mpifunc{MPI_CART_RANK}}
Access the topology description and convert the specified Cartesian
coordinates into a rank.  This uses \mpids{MPID_Topo_cart_t}{dims} and
\mpids{MPID_Topo_cart_t}{ndims} to compute the rank; note that is must also
handle the case of periodic coordinates (\mpids{MPID_Topo_cart_t}{periods}).

\subsubsection{\mpifunc{MPI_CART_SHIFT}}
This routine accesses the topology description and computes the requested
shifted rank.  This is roughly \mpifunc{MPI_Cart_coords}, followed by an
update to the coordinates, followed by \mpifunc{MPI_Cart_rank}.  

\subsubsection{\mpifunc{MPI_CART_SUB}}
This routine can be implemented with \mpifunc{MPI_COMM_SPLIT} (see the MPI-1
standard, section 6.5.7 ``Low-level topology functions'').  It may also
want to call \mpidfunc{MPID_Cart_map} to allow subdimensions to be reordered
when requested.

\subsubsection{\mpifunc{MPI_DIMS_CREATE}}
This routine calls \mpidfunc{MPID_Dims_create}, which trys to return a
``good'' set of dimensions.  It could use \mpidfunc{MPID_Topo_cluster_info} to
provide a good match to a cluster; otherwise, it should strive to create a
decomposition that is as even as possible.

\subsubsection{\mpifunc{MPI_GRAPHDIMS_GET}}
Access the topology description and return the number of dimensions of
the nodes and edges of a graph topology (if defined).

\subsubsection{\mpifunc{MPI_GRAPH_CREATE}}
This is implemented using \mpifunc{MPI_GRAPH_MAP} and
\mpifunc{MPI_COMM_SPLIT}. 

\subsubsection{\mpifunc{MPI_GRAPH_GET}}
Access the topology description and return the associated fields, which
include \mpids{MPID_Topo_graph_t}{index} and
\mpids{MPID_Topo_graph_t}{edges}. 

\subsubsection{\mpifunc{MPI_GRAPH_MAP}}
This should eventually have an MPID routine, but not in ADI-3.  It simply
returns the \code{rank} of the input communicator.

Question: Should this try to detect special patterns for which good mappings
are known?  For example, if we provide routines that are used by the
collective to determine good minimal spanning tree mappings, can
\mpifunc{MPI_GRAPH_MAP} take advantage of them?

\subsubsection{\mpifunc{MPI_GRAPH_NEIGHBORS}}
Access the topology description and return the associated fields by using
\mpids{MPID_Topo_graph_t}{index} and \mpids{MPID_Topo_graph_t}{edges}.

\subsubsection{\mpifunc{MPI_GRAPH_NEIGHBORS_COUNT}}
Access the topology description and return the associated fields.

\subsubsection{\mpifunc{MPI_TOPO_TEST}}
Return \mpiconst{MPI_GRAPH} for graph topology, \mpiconst{MPI_CART} for
Cartesian topology, and \mpiconst{MPI_UNDEFINED} otherwise.  This uses the
\mpids{MPID_Topo_common_t}{topo_kind} field.

\subsection{RMA}
\label{sec:rma}
My original plan was to implement this using the \code{Segment},
\code{Rhcv}, \code{Put_contig} and \code{Get_contig} routines.  We
will need code to support datatype caching at the destination process.
We may want to provide a way to define datatypes in globally shared
memory for systems like large SMPs that provide global access to at
least some memory.  Currently, there is no ADI interface for that.
I have since added additional put/get for the case where the origin
and target datatypes are the same.  

Question:  Should there be a model of remotely-defined datatypes that
would allow processes to avoid caching the description?  How would
this work in the multi-method case where some processes might have
shared memory and others might not?

For systems with ordered delivery, we may want a simpler completion
model, one that has completion per destination process (or per process
per window) rather than per RMA operation.  This is a further reason
to require that completion flags be created, and that this creation
contain both destination process and window.  Where operations are
ordered, this flag can simply count the number of started but not
completed operations, or it could contain a sequence number of some
sort for the most recent operation.  

Question:  For this to work with the waitflags and testflags, we
really need a flag set for the RMA window, which each RMA operation
takes (instead of a separate flag address).  How should the API for
both the flag set creation, reference, and completion work?  

The current ADI-3 interface defines put and get operations for both
contiguous data (at both origin and target) and for the case where the
same datatype is used at both origin and target.  Who is responisble
for the other cases?  The MPICH code or the ADI code?

The completion flags for the \code{MPID_Put_contig} etc. operations
have not be throughly thought out.  For example, there is no explicit
support for the group-based window completion (\code{MPI_Win_post}
etc.), nor is there simple support for systems like the Cray T3E that
have (roughly) hardware support for \mpifunc{MPI_Win_fence}.

Question: The MPI RMA design is actually pretty lean and general, and
without further constraints or properties, it is hard to create a
simpler interface.  However, we might be able to simplify by
considering three important cases:
\begin{description}
\item[Shared Memory.]This is not fully shared, but shared memory
segments or shared \code{mmap} regions. There may need to be special
calls to enforce memory ordering and coherency.
\item[Distributed Memory with DMA.]This is for systems that support
some one-sided data delivery, such as VIA or LAPI.
\item[Distributed Memory with no DMA.]This is for simple
network-connected processes, such as Unix processes connected by TCP.
\end{description}
Question: are these sufficient?  Should we put these classes into the
method-based interface instead?

For example, where shared memory is available, the synchronization and
lock operations can act directly on the shared memory area that is
allocated as part of the window object.  For example, the
start/post/complete/wait can use counters and flags in shared memory.
Locks can be acquired directly and quickly in shared memory, and (for
the passive target operations), the RMA operations can then be done
directly in shared memory.

In contrast, in the distributed memory case, particular with high
latency interconnects, deferred synchronization can be used.  For
example, a \code{MPI_Win_lock} in that case could return immediately.
At the first RMA operation, particularly if the amount of data is
small, the request for a lock can be piggy-backed on the RMA request.
In fact, following the BSP style, all of the RMA operations could be
held until the \code{MPI_Win_unlock}.

Clearly, the choice of immediate or deferred locks depends on the kind
of communication between processes.

Question: are there any special values for window objects similar to the ones
considered for datatypes and communicators?  For example, one bit could
indicate whether all windows of the window object are in shared memory.

To remove the complexity of datatypes, we might want a
\mpidfunc{MPID_Stream_put} that acts on a segment, rather than using several
special-case versions of put.  It would still need to work on \emph{two}
segments; that is, both the origin and targets.

Still needed: a discussion of the completion of one-sided operations.  Do we
want to use the flags (e.g., \mpidfunc{MPID_Flags_waitall})?

\subsubsection{\mpifunc{MPI_ACCUMULATE}}

Note that the target address is computed as base address of target window +
\code{target_offset * target_window_displacement_unit}.

Among the errors to check for is offset out of range.  This is
\mpiconst{MPI_ERR_DISP}; are there any subcases?

\begin{tcp}
The assumption here is that the target process must perform the operation.

Determine if destination datatype is known at target.  If not, send it using
\mpidfunc{MPID_Rhcv} with a type of \mpidconst{MPID_Hid_datatype_desc}.  Note
that if this is a complex datatype, this operation may require a rendezvous.

In that case, do we want a \mpidconst{MPID_Hid_datatype_desc_rts} (rts for
request to send)?  
\mpidconst{MPID_Hid_datatype_desc} needs either to specify an id for this
datatype (for future use) or an acknowledgment needs to indicate what id to
use.  Note that the ids are not global; beyond the predefined types, the ids
are valid only between the particular pair of processes that established them.

This operation should have an MPID routine since Put and Get also require it. 

Once the datatype is known at the target, the data can be sent with
\mpidfunc{MPID_Stream_isend}.  Using a stream allows the receiver to receive
part of the data and combine that with the target buffer without either
allocating a temporary buffer the size of the full message or waiting for the
all of data to be delivered.  In fact, a double-buffer arrangement can be used
(the Stream operations should support this).  

Question: how is this stream matched between the sender and receiver?  Should
this use a ``\code{Stream_put}'' instead, where the target sends the buffer
address 
to use, or do we just use a context from a communicator within in the Window
object, combined with a unique tag value?  How is completion handled?

Note that high-latency systems may want to defer any communication until the
access epoch completes (i.e., the closing \mpifunc{MPI_Win_fence},
\mpifunc{MPI_Win_complete}, or \mpifunc{MPI_Win_unlock}).  Exploiting that
requires merging messages into a 
single message (as seen by the OS).  The above description doesn't handle this
case. 

Question: do we want to use offsets or should we provide the address?  Is
there a special window object that exposes all of a processes memory, for use
by the MPI implementation in delivering messages?
\end{tcp}


\begin{shmem}
\begin{enumerate}
\item If the target window is in shared memory, 
If both origin and target datatypes are simple, then the origin process simply
applies the 
operation (e.g., both contiguous or both vectors).  Otherwise, move through an
intermediate form (e.g., contiguous).  

An alternate implementation would have the target process rather than the
origin process perform the operation.  The difficulty with this is that the
origin buffer need not be in shared memory, so it is less likely that this
single-move form can be carried out.  

[BRT] The act of communicating the need for the target process to do
work on behalf of the origin process introduces extra overhead.  I
fail to see how passing the work onto the target process will result
in a performance gain that outweighs the extra overhead.  Perhaps I am
missing something important...

Question: how is completion handled?  

\item Otherwise (If the target window is not in shared memory),
use the \tcpname\ code (send datatype description, use stream to deposit
data).

\end{enumerate}
\end{shmem}
\begin{via}
Use the \tcpname\ code.  
\end{via}

One question is whether active and passive target operations should be handled
separately.  For example, a TCP device could establish two sockets for each
communication path; one to be used for active target operations and one for
passive.  The passive socket could be handled by a separate thread while the
active socket could be handled by routines invoked by the main thread, thus
eliminating a context switch on active-target operations (active target could
include MPI-1 communication, particular blocking calls).

Also note that the case of either the origin or target datatype is
contiguous can be handled with a simple call to either
\mpidfunc{MPID_Pack} or \mpidfunc{MPID_Unpack}; the only complex case
is where both datatypes are not contiguous or the same, requiring a
copy to an intermediate form.

Another possible implementation would have a thread per window object,
or a thread for all window objects that allow locks\index{thread
overhead!passive RMA}.

[BRT] Alternatively, passive target operations could be communicated
over the same socket as all other operations, but a message handling
thread could be used to periodically check for new messages when other
threads were busy with non-MPI related computations.  This avoids a
context switch whenever a message fragment is received, but insures
that passive operations are processed in a timely fashion.  For the
non-threaded implementation, a similar solution could be used,
replacing the message handling thread with a SIGALRM signal handler.

\subsubsection{\mpifunc{MPI_PUT}}
\begin{adi3}\mpidfunc{MPID_Put}
\begin{mmadi}If target and origin datatype are
\mpids{MPI_Datatype}{contiguous}, use 
  \mpidfunc{MPID_Put_contig}.  Otherwise, if they are the same (and system is
  homogeneous?), use \mpidfunc{MPID_Put_sametype}.  
  Otherwise, what?

\begin{tcp}
Like \code{MPI_ACCUMULATE}, with \mpiconst{MPI_REPLACE} as the operation. 
This should be optimized for this case.  
\end{tcp}

\begin{shmem}
Like \code{MPI_ACCUMULATE}, with \mpiconst{MPI_REPLACE} as the operation.  
This should be optimized for this case.  
\end{shmem}

\begin{via}
If the target datatype is supported (e.g., contiguous) and the target window
is registered, then use \code{MPID_Put_contig}.  Question: do we want a
special version that works only on registered memory?  If the origin datatype
is \emph{not} simple, this will require copying the data to cannonical form.
Question: do we want to define an \code{MPID_Put_contig_stream} that would
allow an overlap of packing and sending?

Otherwise, like \code{MPI_ACCUMULATE}, with \mpiconst{MPI_REPLACE} as the
operation.  

Question: how is completion handled?
\end{via}

\end{mmadi}
\end{adi3}

\subsubsection{\mpifunc{MPI_GET}}
\begin{tcp}
This is roughly like \code{MPI_PUT}, except the target is requested to send
the data.  
\end{tcp}
\begin{shmem}
\begin{enumerate}
\item If the target window is in shared memory, 
If both origin and target datatypes are simple, then the origin process simply
reads the data from the target window and stores it in the origin buffer.
Otherwise, move through an intermediate form (e.g., contiguous).  

\item Otherwise (the target window is not in shared memory), 
use \tcpname\ approach.
\end{enumerate}
\end{shmem}
\begin{via}
If the target datatype is supported (e.g., contiguous) and the target window
is registered, then use \code{MPID_Get_contig}.  The destination on the origin
process is either the origin buffer (if registered) or a temporary registered
buffer.  

Question: If a temporary buffer is used, we must signal completion to the
origin somehow.  How?

Otherwise, use \tcpname\ approach.
\end{via}

\subsubsection{\mpifunc{MPI_WIN_FENCE}}
\begin{tcp}
This can be viewed as a special case of the post/start/complete/wait
synchronization, with a carefully chosen set of neighbors (e.g., the usual
barrier tree).  Or just use \mpifunc{MPI_BARRIER}.  

Question: do VIA-like remote memory access require any cache flush operations?
\end{tcp}

\begin{shmem}
As for \tcpname, this can be viewed as a special case of the
post/start/complete/wait synchronization. 
\end{shmem}

\begin{via}
As for \tcpname, this can be viewed as a special case of the
post/start/complete/wait synchronization. 
\end{via}

In all cases, any pending RMA operations must complete first before
\mpifunc{MPI_WIN_FENCE} may return.

Question:  There are four possible \code{assert} values for
\mpifunc{MPI_Win_fence}.  Are the following correct?
\begin{description}
\item[\mpiconst{MPI_MODE_NOSTORE}]No write barrier is required.
\item[\mpiconst{MPI_MODE_NOPUT}]No action.
\item[\mpiconst{MPI_MODE_NOPRECEDE}]All processes must specify this if any do;
  it 
  indicates that no process will initiate an RMA call.  No barrier is required
  in this case.
\item[\mpiconst{MPI_MODE_NOSUCCEED}]All processes must specify this if any do;
  it indicates that no process will initiate an RMA call.  No action.
\end{description}

\subsubsection{\mpifunc{MPI_ALLOC_MEM}}
Call \mpidfunc{MPID_Mem_alloc}.  We also need a routine that
\mpifunc{MPI_WIN_CREATE} can call to determine if memory was allocated with
this (or a similar) routine.

Note [BRT]: The performance of point-to-point and collective communication
could be improved in some situations if the user buffers were
allocated using \mpifunc{MPI_Mem_alloc}.  The info argument could be
used to express the intended use of the space, alllowing
\mpifunc{MPI_Mem_alloc} to select an appropriate memory pool.

Question: Should this routine be \mpidfunc{MPID_Mem_isalloc}\code{( int
size, void *ptr )}?  ([BRT] isalloc???)

\subsubsection{\mpifunc{MPI_FREE_MEM}}
Call \mpidfunc{MPID_Mem_free}.

For error reporting, we may want to keep a reference count so that a
\mpifunc{MPI_Free_mem} applied to a window that is currently part of a window
object generates an error message.

\subsubsection{\mpifunc{MPI_WIN_CREATE}}
Allocate a new window object.  Call \mpifunc{MPI_Comm_dup} to create a
private \mpids{MPI_Win}{communicator} that can be used as necessary; this also stores
the group of the window object.  Save the \mpids{MPI_Win}{base},
\mpids{MPI_Win}{size}, and \mpids{MPI_Win}{displ}. 
Setup the default attributes (\mpiconst{MPI_WIN_BASE},
\mpiconst{MPI_WIN_SIZE}, and \mpiconst{MPI_WIN_DISP_UNIT}).  Note that these
attributes could return pointers to the corresponding fields in the window
object, but for safety against users storing through those pointers, they
should use a separate area of memory.  Question: should they be in the same
struct (e.g., fields \mpids{MPID_Win}{user_base}, \mpids{MPID_Win}{user_size},
and \mpids{MPID_Win}{user_disp}) or far way where a mistake by the user is
less likely to cause trouble?

Use the private communicator to perform an \mpifunc{MPI_Allgather}
collect all of the window base addresses, sizes, and displacement
units from all of the processes using \mpifunc{MPI_Allgather}, along
with a flag that indicates if the local window is in shared memory.
If all of the base addresses are the same, set \mpids{MPI_Win}{_flags}
with \mpidconst{MPID_WIN_CONST_BASE}; otherwise save the base
addresses in an array \mpids{MPI_Win}{bases}.  Likewise, either set
\mpids{MPI_Win}{_flags} with \mpidconst{MPID_WIN_CONST_SIZE} or save
the sizes in an array \mpids{MPI_Win}{sizes}, and either set
\mpids{MPI_Win}{_flags} with \mpidconst{MPID_WIN_CONST_DISPL} or save
the displacement units in an array \mpids{MPI_Win}{displs}.

In the case of a device that supports \emph{only} \tcpname, it isn't
necessary to collect the displacement units or the window bases,
because the target process can apply these adjustments to the
address.  However, for any one-sided operation performed by the device, it is
necessary to 
have this information.  Further, knowing whether the target window is
in shared (or registered for \vianame) memory is necessary when
implementing the RMA operations.

Comment [BRT]: \tcpname can benefit from collecting the sizes and
displacement units, as it allows the origin to identify out-of-bounds
errors prior to sending requests to the target.

If the info key \mpiconst{nolocks}\index{MPI_Info!keys!nolocks} is \code{true},
then no provision needs to be made for either passive target access or for
\mpifunc{MPI_Win_lock} and \mpifunc{MPI_Win_unlock} calls.  Save this
fact as \mpiconst{MPID_WIN_NO_LOCKS} in \mpids{MPI_Win}{_flags}.

For shared memory, we may want the window object to be in shared
memory itself.  Even if the window object is not in shared memory,
some things, like the local window locks, may need to be.  Question:
how is the window object allocated?  If there is an MPID routine for
it, does it need to know the group of the window (e.g., in a
multimethod device, a window object whose group contains no processes
that shares memory should not consume limited shared memory space).

Question: how are pending (not yet completed) RMA operations
remembered?  Do we need to keep a list of requests (or streams) on which we
must 
wait at the end of an access epoch?  For efficiency and low-latency
with short data transfers (ones that are completed immediately,
e.g. by sending a short message), do we want to have those indicate
that they are complete (e.g., by returning a null handle to wait on)?
Do we only need to use the flags array and \mpidfunc{MPID_Flags_waitall}?

\subsubsection{\mpifunc{MPI_WIN_FREE}}
Call \mpifunc{MPI_Barrier} on the internal \mpids{MPI_Win}{communicator}.
Check for errors, such as unreleased locks, pending RMA operations, or
incomplete 
post/start/complete/wait synchronization.
Free the internal communicator.  Execute any attribute delete functions.

\subsubsection{\mpifunc{MPI_WIN_GET_GROUP}}
Access the group of the related communicator (Question: does this increment the
reference count for the group?)

\subsubsection{\mpifunc{MPI_WIN_GET_NAME}}
Uses the \mpids{MPI_Win}{name} field.  Note that the Fortran versions
must be careful to blank-pad the value rather than null-terminating it.

\subsubsection{\mpifunc{MPI_WIN_SET_NAME}}
Sets the \mpids{MPI_Win}{name} field.  Returns error if the supplied name
is too long. 

\subsubsection{\mpifunc{MPI_WIN_LOCK} and \mpifunc{MPI_WIN_UNLOCK}}
There are two types of lock and unlock implementations.  In the most
obvious, based on the name, \code{MPI_WIN_LOCK} waits until the
indicated process acknowledges the lock. This may be appropriate when the
window is in memory that is shared among the processes in the window
object, such as a fully shared-memory implementation or a distributed
shared memory implementation.

For systems without direct access to the memory, an alternate but
equally valid approach is to make the lock a local operation, and wait
to issue it until the first RMA operation.  This is particularly
appropriate when the RMA operation (e.g., the put or accumulate)
involves a small amount of data and the interprocess communications
have high latency.  In fact, in the high-latency case, we may prefer
to hold all operations until the \code{MPI_WIN_UNLOCK} and then issue
them in a single communication.  I believe this is similar to what BSP
does, but for fence operations (I need the same discussion under fence).

Question.  For the nonblocking lock case, should we have an info key
for \mpifunc{MPI_WIN_CREATE} that asks for the blocking lock?  

Another alternative is to combine the lock with the first operation
request, particularly in the \tcpname\ case.  This is simpler than
queueing up a long list of operations.  In this case, at the second
RMA request, issue both operations.  This allows sequences such as
\begin{verbatim}
MPI_Win_lock( 0, rank, 0, win );
MPI_Put( buf, 1, MPI_INT, rank, 0, 1, MPI_INT, win );
MPI_Win_unlock( rank, win );
\end{verbatim}
to turn into a single \mpidfunc{MPID_Rhcv} call, issued at the
\mpifunc{MPI_Win_unlock} operation.  To implement this, the window
object could store a single \mpidconst{MPID_Hid_rma_op} structure and
issue it as soon as either a second operation is defined or an access
epoch ends (perhaps restricted to the passive target case).  We could
even use an info value, specified at window creation time, to guide
whether operations are started as soon as possible or as late as possible.

Question: Can we optimize for the nonexclusive lock (read)?  

Question: In the case where the operation is lock-put-unlock or
lock-accumulate-unlock, we could avoid serialization in access to the window
by only locking the byte range defined by the operation.  This would guarantee
the MPI semantics while providing for a higher degree of parallelism in
access.  Should we do something like this?  Note the a lock for the local
window must lock the entire window since access may be through local
load and store operations.  Alternately, if all operations are serialized
through the local communication agent, then we don't need to do this at all.
Even in the local access case, if we specified through the \code{assert}
argument that no local stores were used, it would be possible to allow
disjoint put operations to take place concurrently.  We could do this through
the \code{MPI_MODE_NOCHECK} assert value, or through a new
\code{MPIX_MODE_NO_LOCAL_STORE} value.

Question: Do we want a predefined window attribute that can select between
different lock approaches (early versus lazy) instead of the info value?
The advantage is that info applies only at window creation time, while the
attribute can be changed after the window is created.

In the shared memory case, we may prefer acquiring the lock early if that is a
simple operation.  However, it may still be advantageous to ask the target to
perform the operation so as to maintain memory locality for the lock
variables.

\begin{tcp}
Send a message of type \mpidconst{MPID_Hid_lock_op} with
\mpidfunc{MPID_Rhcv}.  The message indicates whether the lock is
exclusive or not.  If this needs to wait for an acknowledgement,
either wait for the lock-granted flag to be set (spin loop!?) or make
this into a (internal generalized) request and use
\mpidfunc{MPID_Waitsome} to wait for the acknowledgement.
\end{tcp}
\begin{shmem}
The locks are allocated in shared memory as part of the window object
creation.  Access the lock directly.  Note that, in the multithreaded
case, if an OS lock is used, that lock must not block any other threads.
\end{shmem}
\begin{via}
Like \tcpname.  Note that some distributed memory systems provide some
support for remote locks; use them if they are available.
\end{via}

The \code{assert} value \mpiconst{MPI_MODE_NOCHECK} can be used to eliminate
the need to wait for the lock to be acquired.  This allows
\mpifunc{MPI_Win_lock} and \mpifunc{MPI_Win_unlock} to be used soley to begin
and end RMA operations.  This suggests that the RMA handler operations
(e.g., \mpidconst{MPID_Hid_put}) may want a few bits to specify
whether a lock should first be acquired and whether a lock is needed
at all (the \mpiconst{MPI_MODE_NOCHECK} case).  

Question: Does the window object have two bits that indicate whether it is
currently within an access epoch and/or an exposure epoch?  This could be used
for error checking (e.g., \mpidconst{MPIi_ERR_WIN_NOACCESS} or
\mpidconst{MPIi_ERR_NOEXPOSURE}).

\subsubsection{Scalable Active Target Synchronization}
The scalable active target synchronization routines (\mpifunc{MPI_WIN_POST},
\mpifunc{MPI_WIN_START}, \mpifunc{MPI_WIN_COMPLETE}, \mpifunc{MPI_WIN_WAIT})
can be implemented by keeping two counts at each process.  One count is
incremented by \mpifunc{MPI_WIN_POST} for each process in the group.  
The other is incremented by \mpifunc{MPI_WIN_WAIT} for each process in the
group specified by \mpifunc{MPI_WIN_POST}.  These counts are zeroed by
\mpifunc{MPI_WIN_START} and \mpifunc{MPI_WIN_COMPLETE} respectively once all
processes have checked in.

This approach is a conmpromise between letting each target process check in
separately (allowing some RMA operations to proceed even before all processes
in the group are ready) and the simplicity of waiting until all are ready to
proceed.  This approach is scalable since the time is independent on the size
of the group of the window object and scales linearly with the size of the
group in the post and start calls.  

A better approach may be to follow the same approach recommended above for
lock/unlock: defer until an RMA operation is going to each designated
neighbor. This might lead to an approach that involved no extra messages, at
least in the \tcpname\ case:
\begin{tcp}
No messages are exchanged for start, post, complete, or wait.
(the fact that they have been called may be remembered)
When the first RMA operation (i.e., put, get, or accumulate) arrives, it is 
applied (if the exposure epoch has started) or is queued (if not).  This only
requires that, at least until the first ack, a long RMA must not assume that
the exposure epoch has started.

Question: is a message needed to indicate that an exposure epoch has ended (I
don't think so)?
\end{tcp} 

Question: If only one group is ever used for scalable synchronization on this
window, is there anything that we can take advantage of?  Do we indicate this
with an info key \mpidconst{onegroup}\index{MPI_Info!keys!onegroup}?
\subsubsection{\mpifunc{MPI_WIN_POST}}
Begin an exposure epoch for the local window.

For each member of the group, use \mpidfunc{MPID_Rhcv} with type
\mpidconst{MPID_Hid_access_cnt} to increment the start counter of that
process.  (Each window object has a separate start and complete counter for
each process.)
Save the group (increment reference count and save in the window object's data structure).

% \begin{tcp}
% \end{tcp}
% \begin{shmem}
% \end{shmem}
% \begin{via}
% \end{via}

\begin{description}
\item[\mpiconst{MPI_MODE_NOCHECK}]This matches the same assert value for
  \mpifunc{MPI_WIN_START}.  If set, no \mpidfunc{MPID_Rhcv} calls are made.
\item[\mpiconst{MPI_MODE_NOSTORE}]No write barrier/flush.  THis refers to a
  memory operation needed in some architectures to ensure that writes to
  memory have completed.
\item[\mpiconst{MPI_MODE_NOPUT}]No action.
\end{description}

\subsubsection{\mpifunc{MPI_WIN_START}}
Start creates an access epoch for the processes in the specified group.  The
implementations here block until the matching \mpifunc{MPI_WIN_POST} calls are
made (implementations that defer communicating can proceed through
\mpifunc{MPI_WIN_START} as long as the matching post occurs before and RMA
actions are taken).

The \mpiconst{MPI_MODE_NOCHECK} assert value is similar to the ready-send
mode.  If this is set, \mpifunc{MPI_WIN_START} does not block, since the
assumption is that the matching \mpifunc{MPI_WIN_POST}s have already been
made; further, the effect of \mpifunc{MPI_WIN_POST} (i.e., incrementing the
start counter) is performed by this routine.

Note that it is incorrect to spinwait on the counter.  Consider the following
correct MPI program:
\begin{verbatim}
    Process 0                       Process 1
---------------------        ----------------------
                              MPI_Irecv (...)
                              MPI_Win_post(...)
                              MPI_Win_start(...)
MPI_Ssend( to 1 )
MPI_Win_post(...)
MPI_Win_start(...)
\end{verbatim}
In the above, time runs down the page.  In other words, process 1 posts an
irecv, then performs the win post step, followed by the
\mpifunc{MPI_Win_start}.  If \mpifunc{MPI_Win_start} enters a tight spin loop
on the counter, the \mpifunc{MPI_Ssend} started by process 0 will be unable to
match with the \mpifunc{MPI_Irecv} in process 1, and this correct code would
hang.  

Implementation:

See \mpifunc{MPI_WIN_POST}.  Wait for the start counter to reach the size of
the group provided to this function.  When it is reached, set it back to zero
and return.  
% \begin{tcp}
% \end{tcp}
% \begin{shmem}

% \end{shmem}
% \begin{via}
% \end{via}

\subsubsection{\mpifunc{MPI_WIN_COMPLETE}}
Complete ends an access epoch for the processes in the group specified with
\mpifunc{MPI_Win_start}.

Like \mpifunc{MPI_WIN_START}, but for the complete counter.
% \begin{tcp}
% \end{tcp}
% \begin{shmem}
% \end{shmem}
% \begin{via}
% \end{via}

\subsubsection{\mpifunc{MPI_WIN_WAIT}}
End an exposure epoch for the local window.

Like \mpifunc{MPI_WIN_POST}, but for the complete counter.


% \begin{tcp}
% \end{tcp}
% \begin{shmem}
% \end{shmem}
% \begin{via}
% \end{via}

\subsection{Starting and Ending MPI}

\subsubsection{\mpifunc{MPI_ABORT}}
\begin{adi3}
\mpidfunc{MPID_Abort}.  This should abort only the specified communicator.  If
no communicator is specified, abort all.  

Question: What is the BNR call for aborting processes?  Is there one for
subsets?  
\end{adi3}

\subsubsection{\mpifunc{MPI_INIT_THREAD}}

One complication to the \mpifunc{MPI_Init} and \mpifunc{MPI_Init_thread} is
handling the case where this process is created by \mpifunc{MPI_Comm_spawn} or
\mpifunc{MPI_Comm_spawn_multiple}.  This part of the code is shown below:
\begin{verbatim}
   BNR_Port_t port;

   BNR_Init( &port );
   if (BNR_Is_spawned( &port )) {
       <Wait until all processes are ready>
       if (my rank == 0) {
           BNR_Connect( port );
       }
   <construct intercommunicator for parent>
   MPID_COMM_PARENT = <that intercommunicator>
   }
   else {
       MPID_COMM_PARENT = MPI_COMM_NULL;
   }
\end{verbatim}
An alternative to the \mpidfunc{BNR_Is_spawned} is to have \mpidfunc{BNR_Init}
return a pointer to a \mpidfunc{BNR_Port_t}; if the pointer is null, then the
process was not spawned.

The initialization of the processes in (the local) \mpiconst{MPI_COMM_WORLD}
are carried out with \mpidfunc{MPID_Init}.

\begin{adi3}
\mpidfunc{MPID_Init}
Each device and method in a device will also require initialization.  
\begin{tcp}
Acquire enough information to establish a connection with each process
in \mpiconst{MPI_COMM_WORLD}.  This may include host and port for
TCP.  This is likely to use something like the following:
\begin{verbatim}
   <Get contact port>
   sprintf( key, "%d:%d:contact", gid, lrank );
   sprintf( value, "%s:%d", hostname, port );
   BNR_Put( key, value );
   BNR_Fence();
\end{verbatim}
\end{tcp}
\begin{shmem}
Create a shared memory area for \mpidconst{MPID_Request}s, short
messages, and an area for streams (used to move long messages).
Exchange information on the addresses (or ensure that all 
processes have mapped the shared area into the same local addresses).
Create a shared memory area for use by \mpidfunc{MPI_Mem_alloc}.
\end{shmem}
\begin{via}
Acquire enough information to establish connections (this is
particularly critical because connections may be a scarce resource).  
Consider pre-establishing some connections based on runtime parameter
values (e.g., a \mpidconst{MPICH_NBRLIST} value that, for each
process, contains the ranks in \mpiconst{MPI_COMM_WORLD} that that
process should start connected to).  

For each connection, register some memory that can be used for
communication in the event that a message buffer cannot be registered.
Initialize the list of registered memory.
\end{via}
\end{adi3}
This should also set the value \mpidfunc{MPID_THREAD_PROVIDED}.
Note that for processes that were spawned from another MPI process, we will
want to limit the level of thread support to what that in the spawning
process. 

This must also invoke the various init functions for the different
subsystems and predefined objects.  These include keyvals, topology, datatypes,
groups, communicators, reduction operations (\code{MPI_Op}), timers, and error
handlers.  Each of these should be handled by calling an
\mpidfunc{MPIi_xxx_init} or \mpidfunc{MPID_xxx_init}.  We may also
want to have a similar initialization routine for Fortran, Fortran 90,
and C++.  
Also setup information for the debugger (process tables, etc.)

Question: Do we want to support the special case of a single language?  I.e.,
only C or C++?  Do we do that by dynamically loading the Fortran, Fortran 90,
and C++ initialization routines as required?

Question:  We need to describe here how connections are established, even if
they are established lazily.  That is, we shold describe here, even if the
connections are not established until needed, how connections are
established.  For example, for \tcpname, the code might look like
\begin{verbatim}
   sprintf( key, "%d:%d:contact", gid_of_process, lrank_of_process );
   BNR_Get( key, value );
   <use value as hostname:port to contact>
\end{verbatim}\index{BNR_Get}

\subsubsection{\mpifunc{MPI_QUERY_THREAD}}
This returns the level of thread support provided.  Should there be an
\mpidfunc{MPID_THREAD_PROVIDED} variable?  This could be a static
variable within the file that implements the (PMPI version of the)
thread-related functions. 

\subsubsection{\mpifunc{MPI_IS_THREAD_MAIN}}
This needs a thread-private variable that is set only by the thread that calls
\mpifunc{MPI_INIT_THREAD}.  Alternately, we can use a public variable that
holds the thread id of the thread that called \mpifunc{MPI_INIT_THREAD}.  For
example, 
\begin{verbatim}
static pthread_t main_thread_id;
...
is_main_thread = pthread_equal( main_thread_id, pthread_self() );
\end{verbatim}
This does require that the thread library used by the user is the same as the
one that the MPICH library is built for.  We may want to put this routine in a
separate library, allowing several different thread libraries to be used with
MPICH. 

\subsubsection{\mpifunc{MPI_FINALIZED}}
See \code{MPI_INITIALIZED}

\subsubsection{\mpifunc{MPI_INIT}}
Call \mpifunc{MPI_INIT_THREAD} with \code{MPI_THREAD_MULTIPLE} as the
requested level of thread support.

\subsubsection{\mpifunc{MPI_INITIALIZED}}
As part of the error checking code, each routine should check the
state of the \code{is_initialized} flag.  Should there be an 
\begin{verbatim}
    enum { MPICH_PRE_INIT=0, MPICH_IS_INITIALIZED=1,
           MPICH_POST_FINALIZED=2 } MPIR_Initialized;
\end{verbatim}
variable?  This can be used by the \mpifunc{MPI_INITIALIZED} and
\mpifunc{MPI_FINALIZED}
calls.\index{MPICH_PRE_INIT}\index{MPICH_IS_INITIALIZED}
\index{MPICH_POST_FINALIZED}

\subsubsection{\mpifunc{MPI_FINALIZE}}
The MPI-2 standard requires that \code{MPI_Finalize} first delete the
attributes associated with \mpiconst{MPI_COMM_SELF}, even before
\mpifunc{MPI_FINALIZED} would return true.  This allows any number of
modules to attach ``end-of-job'' actions to \code{MPI_Finalize}.

Just as \mpifunc{MPI_INIT_THREAD} invokes initialization routines for
the various subsystems, \mpifunc{MPI_FINALIZE} should invoke
\mpidfunc{MPI_xxx_finalize} for those systems, in reverse order.

Question: instead of having particular \code{MPI_xxx_finalize}
routines, an alternative for the less-used subsystems, such as
topologies and name servers, is to allow those subsystems to register
routines to be called when \mpifunc{MPI_Finalize} is called.  For
example, we could include a file containing
\begin{verbatim}
typedef struct {
    static int (*f)( void * );
    void *extra_data;
} Finalize_func_t;

#define MAX_FINALIZE_FUNC 16
Finalize_func_t fstack[MAX_FINALIZE_FUNC];
fstack_sp = 0;  /* First free entry */

void MPIR_Add_finalize( static int (*f)( void * ), void *extra_data )
{
    if (fstack_sp >= MAX_FINALIZE_FUNC) {
        /* panic ! */
        }
    fstack[fstack_sp].f = f;
    fstack[fstack_sp++].extra_data = extra_data;
}
void MPIR_Call_finalize( void )
{
    int i;
    for (i=fstack_sp-1; i>=0; i--) {
        if (fstack[i].f) fstack[i].f( fstack[i].extra_data );
    }
    fstack_sp = 0;
}
\end{verbatim}
and have \mpifunc{MPI_Finalize} call \mpidfunc{MPIR_Call_finalize}.

The advantage to this is that applications that do not use parts of
MPI that require additional libraries (such as \code{ldap} for the
name server) do not need to load those libraries just to resolve
symbols that appear only in the functions that appear in code called
during \mpifunc{MPI_Finalize}.  

Question:
Since storing function pointers is vunerable to user-errors that
overwrite memory, do we want to add sentinals, either on each side of
the function stack or around each entry in the stack?

Question: 
Do we need to provide an ordering to these callbacks?  We could add a
third argument that specified a phase; the callbacks would be called
in phase order; within a phase, the order would be arbitrary.

Question:
Another approach is to add these as internal attributes on
\mpiconst{MPI_COMM_SELF}, with the delete function corresponding to
the callback defined above. The major problem with this is defining
and communicating the private keyvals without losing the separation
of the module from the rest of the code.  The other problem with
relying on the attribute is that the order of invocation is not defined.

A partial list of subsystems that we might handle with these finalize
callbacks include 
\begin{enumerate}
\item Bsend 
\item Name service
\item Topologies
\item Generalized requests
\item Datareps
\item Fortran 90 types created with \mpifunc{MPI_Type_create_f90_int} etc.
%\item RMA (e.g., free any allocated shared memory)
%\item I/O (e.g., close any open files)
\end{enumerate}

\subsection{Dynamic Processes}
\label{sec:spawn}

The MPI dynamic process management functions require more interaction with the
operating environment than the rest of MPI does.  In particular, we assume
that there is an external mechanism for starting new processes, which we call
the {\em process manager}, and which may in turn require interaction with a
job scheduler or resource manager.  In order that MPICH be capable of
operating in a variety of environments, we isolate the interaction of the MPI
library with a process manager in an API we call BNR, described here.
Multiple implementations of the BNR interface are possible;  indeed, a
design goal for the BNR interface definition is to provide the functionality
required by a parallel library like MPICH without constraining the
implementation.  Although we intend to provide at least one implementation of
BNR (MPD), we will encourage other process manager suppliers to implement is
as well.

\subsubsection{The BNR Process Manager Interface}
\label{sec:bnr}

The primitive concepts of the BNR interface are the {\em group}, the {\em
  database},
and the {\em domain}.  Each of these was chosen in order to
provide a simple, MPI-independent interface that would be straightforward for
process managers to implement.

A BNR {\em group\/} (or process group) is a set of processes started by the
process manager ``at the same time.''  It is designed to fit the process
manager's own concept of a related set of parallel processes belonging to a
single parallel job.  A process belongs to only one process group and can be
identified by (group, rank) 

({\em Rationale:\/}  This approach in discussion was called ``big groups.''
An alternative approach is to have BNR process groups correspond to MPI groups
(the ``little groups'' approach).  While this has some appeal, it requires an
assortment of group construction and manipulation routines and imposes a new
concept on the process manager.)

A BNR {\em database\/} is a collection of key=value pairs associated with a
job and database name.  Its purpose is to provide certain services to the
library linked with the application.
One type of service required by the library from the process manger might be
called ``precommunication.''  Since only the process manager knows where other
processes have been started, it may be necessary to ask at run time how to
communicate with other processes.  We allow other processes to deposit their
``business cards'' into a database accessible to other processes in the same
job, with information on who they may be contacted (shmem keys, IP host/ports,
switch ports, etc.)  Thus precommunication takes place through this database.
Databases are identified by name, names are assigned by the process manger;
no process is allowed access to the database of another job (maybe ``job''
needs a definition).  Each process group does automatically have access to a
database, but some databases may be shared among process groups.

({\em Rationale:\/}  An alternate approach is to attach databases to process
groups.  This requires too much duplication of data in multiple databases in
environments where it is easy for multiply groups to share the same database.)

A BNR {\em domain\/} is an environment managed by a single instance of a
process manager.  Thus within a domain databases may be shared among multiple
process groups.  In order to support distributed computing applications,
multiple domains are allowed, in which case databases may need to be copied
rather than shared.  The mechanisms for doing so are included in the BNR
interface. 

({\em Rationale:\/}  In our initial design, we said that a database would be
local to a process group.  All process in the process group would be able to
insert (put) or extract (get) information from the database.  If a member of
the process group desired to share the information contained within one of its
databases, it could extract the information using the iterator functions, and
communicate the key-value pairs to a non-member using a mechanism external to
BNR.  If the recipient of this information was also using BNR, it could create
a new database using BNR and populate the database with the key-value pairs it
received, thus making the information available to all members of its process
group.

Rusty pointed out that the recipient process group might very well be in the
same database domain, and that extracting and communicating all of the data
within a database would be unnecessary and costly.  We realized that if we
associated a database to some notion of a domain rather than a particular
process group then we might be able to avoid this potentially costly
replication of information when process groups needed to exchange information.

To make this practical, we restrict a process group to a single BNR domain.  A
database is accessible to any process in the domain so long as that process
knows the name of the database.

BNR-Spawn-multiple may instantiate the set of requested processes in a domain
outside of the one its calling process belongs to so long as the entire process
group resides within a single domain.)


\subsubsection{The BNR Basic Functions}
\label{sec:bnr-basic}

These functions implicitly refer to the process group to which the calling
process belongs. 

\begin{verbatim}
  BNR_Init( )              - initialize BNR for this process group         
  BNR_Get_size( *int )     - get size of process group
  BNR_Get_rank( *int )     - get rank in process group
  BNR_Barrier( )           - barrier across processes in process group
  BNR_Finalize( )          - finalize BNR for this process group.
\end{verbatim}

({\em Rationale:\/}  Note that there is no access to an identifier for the
process group itself.  Con:  This means that a process cannot identify itself,
which might be helpful for debugging, nor can it send its identifier in the
form of (group, rank) to another process, which might be handy.  Pro:  There
doesn't seem to be a compelling need for this, and if process group id's don't
appear in the interface, we don't have to worry about their type;  the concept
belongs entirely to the BNR implementation and to the matching process
manager.) 


\subsubsection{The BNR Database Functions}
\label{sec:bnr-database}

These functions are the interface to BNR databases.  Some implementations might
be integrated with the process manager; other implementations could be
independent of the process manager (e.g. separate server). 

\begin{verbatim}
  int BNR_DB_Get_my_name(char *dbname)        - get name of database
  int BNR_DB_Get_name_length_max( )           - needed since might be sent
  int BNR_DB_Create(char *dbname [OUT})       - make a new one, get name 
  int BNR_DB_Destroy(char *dbname [IN])       - finish with one 
  int BNR_Put(char *dbname,
              char key[BNR_MAX_KEY_LEN],
              char value[BNR_MAX_VALUE_LEN]); - put data
  int BNR_Commit(char *dbname)                - block until all pending put
                                                operations from this process
                                                are complete.  This is a process
                                                local operation.
  int BNR_Get(char *dbname,
              char *key,
              char value[BNR_MAX_VALUE_LEN]); - get value associated with key
  int BNR_DB_iter_first(char *dbname, char *key, char *val)  - loop through the
  int BNR_DB_iter_next(char *dbname, char *key, char *val)      pairs in the db
\end{verbatim}
  
One a \code{BNR_Put}, if a pair with the same key is already in the database,
the value is overwritten.  On a \code{BNR_Get}, if there is no pair with
matching key, the return value is -1.  

({\em Rationale:\/}  Note that there is no fence operation, since process
groups are separate from databases.  Since \code{BNR_Put}s and \code{BNR_Get}s
are globally asynchronous, it is the responsibility of the user to ensure that
the data sought by a get operation has been placed in the database by a put
operation.  The \code{BNR_Commit} ensures that the put has taken place
``locally'' (synchronization between a process and the database), and then
another mechanism is required for synchronization across processes.  Within
a process group this can be accomplished by the \code{BNR_Barrier}; across
process groups it can be accomplished by message passing (See the {\tt MM}
interface below.)

({\em Rationale:\/}  The iteration scheme for extracting the total contents
of a database is obviously not thread safe.  This is not viewed as a problem.

\subsubsection{The BNR Process Creation Functions}
\label{sec:bnr-spawn}

In this section are the process creation routines.

\begin{verbatim}

int BNR_Spawn_multiple(count, cmds, argvs, maxprocs, infos, errors,
                       [OUT] bool_t same_domain);
\end{verbatim}

({\em Rationale:\/}  The \code{same_domain} argument lets the process manager
tell us whether the database associated with the new process group is shared,
or whether we will need to receive the contents of the new group's database
and add it to our own.

Note that there is no new process group identifier returned.  The only real
need for it would be to implement a \code{BNR_Kill} function, which is not
really necessary for implementing MPI.  See also the comments above about the
advantages of keeping process group id's out of the interface.  As a result,
there is no \code{BNR_Kill}.  This might be difficult for a process manager to
implement, anyway. 

\subsubsection{Utility Functions}

We postulate the existence of some low-level communication routines.  The {\tt
  MM} stands for ``multi-method.''

\begin{verbatim}
MM_get_port_name
MM_connect
MM_accept
MM_send
MM_receive
MM_???
\end{verbatim}

One use of this form of communication is to copy databases between domains.
Here are functions to carry this out:

\begin{verbatim}
SendDatabases([IN] conn, [IN] comm)
{
    MM_send(conn, Ndb)
    foreach dbname (used in a vc in comm)
    {
        MM_send(conn, dbname);
        done = BNR_Key_iter_first(dbname, key, val);
        while (!done)
        {
            MM_send(conn, (key,val));
            done = BNR_Key_iter_next(dbname, key, val);
        }
        MM_Send(conn, (dbdone));
    }
}

RecvDatabases([IN] conn, [OUT] dbmap)
{
    MM_recv(conn, Ndb);
    dbmap = malloc(Ndb * sizeof(int));
    
    for (i = 0; i < Ndb; i++)
    {
        MM_recv(conn, (remote_dbname));

        BNR_Create_DB(dbname);
        strcpy(dbmap[i], dbname);
        
        while(1)
        {
            MM_recv(conn, (key, val));
            if (key == (dbdone)) break;
            BNR_Put(dbname, key, val);
        }

        BNR_Commit(dbname);
    }
}
\end{verbatim}

\subsubsection{Implementation of MPI on BNR Plus Utility Functions}

In this and the following sections, we describe the implementation of MPI
routines in terms of the BNR functions defined above, together with the
MM utility communication functions.  We will also assume that certain
MPI functions have been implemented.  (Note:  we need to explain why we are
not in an infinite loop here.)

\begin{verbatim}
mpiexec::main()
{
    MPI_Init();

    BNR_Convert_args_to_info(argc, argv, infos);
    
    /* generate command lines */

    /* pack job configuration information into MPI info structures */

    /* use info parameter to tell spawn process group to notify mpiexec when
    spawn group has reached MPI_Finalize().  this can be accomplished by
    sending a message from one of the spawned processes to the mpiexec process
    over the intercomm created during the spawn.  without this info parameter
    mpiexec will return "immediately". */
    
    MPI_Comm_spawn_multiple(count, cmds, argvs, maxprocs, infos, 0,
                            MPI_COMM_WORLD, &intercomm, errors);

    /* communicate some things here */

    /* wait for spawned job to finish ??? */

    MPI_Finalize();
}


int MPI_Init()
{
    BNR_Init();

    /* initialize methods, device, etc. */
    
    BNR_Get_my_DB_name(my_dbname);
    BNR_Put(my_dbname, ..., ...);
    BNR_Commit(my_dbname);

    BNR_Barrier();
    
    /* Various initializations like datatypes, COMM_WORLD, etc. */

    /* get parent_port_name from environment */
    if (parent_port_name is defined)
    {
        MPI_Comm_connect(parent_port_name, info, 0, MPI_COMM_WORLD,
                         &intercomm);
    }
    else
    {
        intercomm = MPI_COMM_NULL;
    }
}

int MPI_Comm_Spawn_multiple(count, cmds, argvs, maxprocs, infos, root, comm,
                            intercomm, errors)
{
    if (root)
    {
        port_info = NULL; /* ??? */
        MPI_Open_port(port_info, port_name);

        /* add port_name to all infos */

        BNR_Spawn_multiple(count, cmds, argvs, maxprocs, infos, errors,
                           &bnr_same_domain, &pg);
        if (bnr_same_domain)
        {
            MPI_Info_set(accept_info, "BNR_SAME_DOMAIN", "");
        }
    }

    MPI_Comm_accept(port_name, accept_info, root, comm, intercomm);
    
}

int MPI_Open_port(info, port_name)
{
    port_name = MM_get_port_name();  /* query_descriptor() ??? */
}

int MPI_Comm_accept(port_name, info, root, comm, intercomm)
{
    if (root)
    {
        conn = MM_accept(port_name);

        MPI_Info_get(info, "BNR_SAME_DOMAIN", 0, NULL, &bnr_same_domain);

        MM_send(conn, bnr_same_domain);
        MM_send(conn, comm->size);
        MM_recv(conn, remote_size);
        
        MPID_Intercomm_alloc(INTER, intercomm, comm->size, remote_size);
        intercomm->local_VCTable = comm->VCTable;
        intercomm->local_size = comm->size;
        
        SendCommVCTable(conn, comm->VCTable, comm->size);
        if (!bnr_same_domain)
        {
            SendDatabases(conn, comm);
        }
        
        RecvCommVCTable(conn, intercomm->remote_VCTable,
                        intercomm->remote_size);
        if (!bnr_same_domain)
        {
            RecvDatabases(conn, dbmap);
            FixVCTable(dbmap, intercomm->remote_VCTable);
        }

        MPI_Bcast(remote_size, root, comm);
        MPI_Bcast(intercomm->remote_VCTable, root, comm)
    }
    else
    {
        MPI_Bcast(remote_size, root, comm);
        
        MPID_Intercomm_alloc(INTER, intercomm, comm->size, remote_size);
        intercomm->local_VCTable = comm->VCTable;
        intercomm->local_size = comm->size;
        
        MPI_Bcast(intercomm->remote_VCTable, root, comm)
    }
}

int MPI_Comm_connect(port_name, info, root, comm, intercomm)
{
    if (root)
    {
        conn = MM_connect(port_name);
        
        MM_recv(conn, bnr_same_domain);
        MM_recv(conn, remote_size);
        MM_send(conn, comm->size);

        MPID_Intercomm_alloc(INTER, intercomm, comm->size, remote_size);
        intercomm->local_VCTable = comm->VCTable;
        intercomm->local_size = comm->size;
        
        RecvCommVCTable(conn, intercomm->remote_VCTable,
                        intercomm->remote_size);
        if (!bnr_same_domain)
        {
            RecvDatabases(conn, dbmap);
            FixVCTable(dbmap, intercomm->remote_VCTable);
        }
        
        SendCommVCTable(conn, comm->VCTable, comm->size);
        if (!bnr_same_domain)
        {
            SendDatabases(conn, comm);
        }
        
        MPI_Bcast(remote_size, root, comm);
        MPI_Bcast(intercomm->remote_VCTable, root, comm)
    }
    else
    {
        MPI_Bcast(remote_size, root, comm);
        
        MPID_Intercomm_alloc(INTER, intercomm, comm->size, remote_size);
        intercomm->local_VCTable = comm->VCTable;
        intercomm->local_size = comm->size;
        
        MPI_Bcast(intercomm->remote_VCTable, root, comm)
    }
}
\end{verbatim}


\subsubsection{Previous Dynamic Processes Section }
\label{sec:spawn-impl}

{\em This is intended to be merged with the previous section}.


This section has not been written.  A big question is how or if the BNR
interface to a parallel process manager is used in the implementation
of the MPICH routines, or whether another layer of abstraction is
used, and BNR is just one of the ways in which to implement that abstraction.

\subsubsection{\mpifunc{MPI_COMM_CONNECT}}

\begin{verbatim}
  BNR_Connect( &port, &gid )
\end{verbatim}

\subsubsection{\mpifunc{MPI_COMM_DISCONNECT}}
This function is like \mpifunc{MPI_Comm_free}, except that it also guarantees
that all communication has completed before it returns and it affects the
status of ``connected'' processes.



\subsubsection{\mpifunc{MPI_COMM_GET_PARENT}}
Return \mpidconst{MPID_COMM_PARENT}.  This value is set by
\mpifunc{MPI_Init_thread}. 

\subsubsection{\mpifunc{MPI_COMM_JOIN}}
\mpifunc{MPI_Comm_join} creates an intercommunicator for two (and only two)
MPI processes that have an established socket between them.  It is permissible
for an MPI implementation to refuse to create the intercommunicator; for
example, an MPI implementation that only implements a shared-memory device can
return a failure for this routine.  The standard requires all implementations
to document any limitations on \mpifunc{MPI_Comm_join}; to handle this, each
device must also provide a file \file{join_limits.txt} (similar to
\file{signal_limits.txt}).  If no file is present, then there are no limits
and \mpifunc{MPI_Comm_join} will always succeed (in the absence of other
problems, like out-of-memory when updating internal tables).

This routine argues for a routine that exchanges the necessary connection
information between two processes, perhaps by formatting the data to and from
a string.  Then \mpifunc{MPI_Comm_join} can use \code{read} and \code{write}
to send this data; \mpifunc{MPI_Comm_connect} and \mpifunc{MPI_Comm_accept}
(or \mpidfunc{BNR_Connect} and \mpidfunc{BNR_Accept}) can use the same data
representation but different methods for communicating the data between
processes.  

\subsubsection{\mpifunc{MPI_COMM_SPAWN}}

Notes:
\begin{itemize}
\item \code{BNR_Port_t} is a BNR-defined data structure that is set by
  \mpidfunc{BNR_Spawn} and used by \mpidfunc{BNR_Accept}.   In a TCP-based
  implementation, \code{BNR_Spawn} would get a port, communicate that port
  (probably through an environment variable but possibly through an LDAP
  service) to the new processes, which would use that name to as an argument
  to \mpidfunc{BNR_Connect}.  A shared-memory-only implementation could use a
  SYSV segment id.

\item \mpidfunc{BNR_Accept} does not return until all of the requested
  processes have started and completed their initialization (see
  \code{MPI_Init_thread}).

\item \mpidfunc{MPID_New_connections} informs the device that there are
  \code{count} new processes with BNR group id \code{gid}.  This routine
  returns the local process ids (\code{lpid}s).

\item As used here, \mpidfunc{BNR_Spawn} could return immediately; only after
  the \mpidfunc{BNR_Accept} call returns does this code need to know if 
  \mpidfunc{BNR_Spawn} has succeeded, or, in the case of a ``soft'' spawn, how
  many processes were successfully created.  This allows \mpidfunc{BNR_Spawn}
  to be ``nonblocking'' in the MPI sense, with completion handled by
  \mpidfunc{BNR_Accept} on the ``port'' returned by \mpidfunc{BNR_Spawn}.

\end{itemize}

\begin{verbatim}
    BNR_Port_t port;
    int        err_stat[count];
    int        lpid[count];
    if (rank == root) {
        BNR_Spawn( count, command, args, envp, info, &port, err_stat );
        BNR_Accept( &port, &gid );
    }
    PMPI_Bcast( &gid, 1, MPI_INT, root, comm->private_comm );
    MPID_New_connections( gid, count, lpid );

    /* Now we need to create the intercommuicator for the new processes */
    <Use the intercomm create code, but with lpid as the local process ids
     of the remote group rather than using an existing MPI group>
\end{verbatim}

Question:  Should this be a special case of spawn multiple?

\subsubsection{\mpifunc{MPI_COMM_SPAWN_MULTIPLE}}

This is very similar to \mpifunc{MPI_COMM_SPAWN}, with the change that
multiple calls to \mpidfunc{BNR_Spawn} and \mpidfunc{BNR_Accept} are made.
All of the calls to \mpidfunc{BNR_Spawn} are made before any call to
\mpidfunc{BNR_Accept} in order to start any time-consuming process scheduling
and creation operations before waiting for any to complete (see the discussion
on \mpidfunc{BNR_Spawn} in \mpifunc{MPI_Comm_spawn}.
\begin{verbatim}
    BNR_Port_t port[count];
    int        err_stat[max_procs][count];
    int        lpid[max_procs][count];
    if (rank == root) {
        for (i=0; i<count; i++) {
            BNR_Spawn( array_of_maxprocs[i], array_of_commands[i], 
                       array_of_argv[i], envp, array_of_info[i], &port[i], 
                       err_stat[i] );
        }
        for (i=0; i<count; i++) {
            BNR_Accept( &port[i], &gids[i] );
        }
    }
    PMPI_Bcast( gids, count, MPI_INT, root, comm->private_comm );
    for (i=0; i<count; i++) {
        MPID_New_connections( gids[i], array_of_maxprocs[i], lpid[i] );
    }

    /* Now we need to create the intercommuicator for the new processes */
    ... not done ...
\end{verbatim}

\subsubsection{\mpifunc{MPI_LOOKUP_NAME}}
Question:  Do we want to define an API or a wire protocol for the name
service?  Perhaps both?

An alternative is to use OpenLDAP \cite{openldap}.  LDAP stands for
``Lightweight Directory Access Protocol,'' and implements the X.500
directory services.  For environments where TCP is available, LDAP
provides all of the services needed by the MPI-2 name service.  The
OpenLDAP project includes both a client library and a simple LDAP
server (\code{slapd}).  

A typical implementation using OpenLDAP might look something like
this (this is incomplete and only includes the name of the ldap
routines that can be used):
\begin{verbatim}
#include <lber.h>
#include <ldap.h>
static LDAP ldap_handle = 0;
LDAPMessage *res;
char        **value_ptr;
if (!ldap_handle) {
    ? how do we get the server name and port (LDAP_PORT is the default)?
    ? Use ldap_url_parse or ldap_is_ldap_url?
    ldap_handle = ldap_open( host, port );
    /* method can be LDAP_AUTH_SIMPLE, LDAP_AUTH_KRBV41 or 
       LDAP_AUTH_KRBV42 */
    ldap_bind_s( ldap_handle, who, cred, method );
    /* Or ldap_simple_bind_s( handle, who, passwd) 
       or ldap_kerberos_bind_s( handled, who ) */
    MPIR_Add_finalize( MPID_Nameserver_finalize &ldap_handle );  
          /* See MPI_Finalize */
    }
/* Synchronous call because MPI_LOOKUP_NAME is blocking */
/* This isn't correct yet */
ldap_search_s( ldap_handle, jobname, LDAP_SCOPE_BASE, 
               NULL, NULL, portname, &res );

/* Use ldap_search_st to search with a timeout */
res = ldap_first_entry( ldap_handle, res );
/* attr is the LDAP attribute to return the value for. */
value_ptr = ldap_get_values( ldap_handle, res, attr );

/* ?? ldap_parse_result(); */
/* Free result data with msgfree */
ldap_msgfree( res );
...
static int MPID_Nameserver_finalize( void *ptr )
{
    ldap_handle = (LDAP *) ptr;
    ldap_unbind_s( ldap_handle );
#ifdef __WIN32
    /* See man -s 3 ldap */
    ldap_memfree( );
#endif
    return 0;
}
...
/* if an error seen, use */
str = ldap_err2string( err );
MPIR_Err_setcode( "Error from ldap library during %s operation: %s", 
                  routine_name, str );
\end{verbatim}

\subsubsection{\mpifunc{MPI_PUBLISH_NAME}}
\begin{verbatim}
/* attrs is an LDAPMod *attrs[] type */
attrs[0]->mod_type = ?;
attrs[0]->mod_values[0] = ?;
attrs[0]->mod_op = LDAP_MOD_ADD; /* or LDAP_MOD_REPLACE */
attrs[1] = NULL:
if (ldap_add_s( ldap_handle, name, attrs) == -1) {
    <error; see ldap_handle->ld_errno>
}
\end{verbatim}

\subsubsection{\mpifunc{MPI_UNPUBLISH_NAME}}
\begin{verbatim}
/* like add, but with LDAP_MOD_DELETE as the mod_op */
ldap_modify_s( ldap_handle, name, attrs );
/* Use ldap_delete_s( ldap_handle, name ) to completely remove a name */
\end{verbatim}

\subsubsection{\mpifunc{MPI_OPEN_PORT}}
An MPI ``port'' is just a string that is used in \mpifunc{MPI_Comm_attach} and
\mpifunc{MPI_Comm_connect}.  Since these ports are used to establish
connections between groups or processes that are defined by BNR, 

\begin{verbatim}
    BNR_Open_port( );
\end{verbatim}

\subsubsection{\mpifunc{MPI_CLOSE_PORT}}
\begin{verbatim}
    BNR_Close_port( );
\end{verbatim}

\subsection{User-Defined Requests}
\label{sec:grequest}

Question:  What ADI support is required for these?  Note that the
request is under the control of the device, so many of the fields
aren't defined yet.

Note that if \mpidfunc{MPID_Waitsome} is implemented in the ADI, then the
ADI must understand these requests (or at least be able to ignore
them).
Note that a user-defined request is started with callbacks (functions
to call for query, cancel, and free); these need to be associated with
the request.  

\subsubsection{\mpifunc{MPI_GREQUEST_START}}
Where are the typedefs for the query, free, and cancel function defined? 
Are they in \file{mpi.h}?  Where are they stored for the generalized request?
For example, \mpids{MPI_Request(generalized)}{query_fn},
\mpids{MPI_Request(generalized)}{free_fn},
\mpids{MPI_Request(generalized)}{cancel_fn}, and
\mpids{MPI_Request(generalized)}{extra_state}. 

\subsubsection{\mpifunc{MPI_GREQUEST_COMPLETE}}

(Need to describe the user-request handling)

\subsection{Error Handlers}

\subsubsection{\mpifunc{MPI_ERRHANDLER_FREE}}
Calls \mpidfunc{MPID_Errhandler_free}

\subsubsection{\mpifunc{MPI_ERRHANDLER_CREATE}}
Deprecated.  Call \mpifunc{MPI_COMM_CREATE_ERRHANDLER}.

\subsubsection{\mpifunc{MPI_ERRHANDLER_GET}}
Deprecated.  Call \mpifunc{MPI_COMM_GET_ERRHANDLER}.

\subsubsection{\mpifunc{MPI_ERRHANDLER_SET}}
Deprecated.  Call \mpifunc{MPI_COMM_SET_ERRHANDLER}.

\subsubsection{\mpifunc{MPI_ERROR_CLASS}}
Return the error class of an error code.  We may either allow the
error reporting module to do this, or define the mask used to extract
the class field from the error code.  I prefer defining a routine in
order to modularize the error reporting.  In this case, the routine
might be \mpidfunc{MPID_Err_code_to_class}.

\subsubsection{\mpifunc{MPI_ERROR_STRING}}
Calls \mpidfunc{MPID_Err_get_string} to return the error string
associated with an error code. 

\subsubsection{\mpifunc{MPI_ADD_ERROR_CLASS}}
Call \mpidfunc{MPID_Err_add_class} with a null string for the
\code{instance_msg_string}. 

\subsubsection{\mpifunc{MPI_ADD_ERROR_CODE}}
Call \mpidfunc{MPID_Err_add_code} with a null string for the
\code{instance_msg_string}. 

\subsubsection{\mpifunc{MPI_ADD_ERROR_STRING}}
Call \mpidfunc{MPID_Err_set_msg}.

\subsubsection{\mpifunc{MPI_COMM_CALL_ERRHANDLER}}
All error handler calls use the common error handler structure
\mpidfunc{MPID_Errhandler} structure.  There should be a common
routine to invoke the error handler from an object.  We could do this
is objects with error handlers have the same header layout;
alternately, we have something like\index{MPIi_Call_errhandler}
\begin{verbatim}
int MPIi_Call_errhandler( void *obj, MPI_Errhandler errhander, ... )
\end{verbatim}
and invoke it as
\begin{verbatim}
MPIi_Call_errhandler( comm_ptr, comm_ptr->errhandler, ... )
\end{verbatim}

\subsubsection{\mpifunc{MPI_COMM_CREATE_ERRHANDLER}}
Create an \mpidconst{MPID_Errhandler}, set the kind to
\mpidconst{MPID_COMM_OBJ}, set the language to \mpidconst{MPID_LANG_C}, and
save the function.  

Question: do we want to define a generic error handler
creation function that could be used from C, Fortran, and C++, as well as from
communicators, windows, and files?  Or is it simple enough to inline?

\subsubsection{\mpifunc{MPI_COMM_GET_ERRHANDLER}}
Return the errhandler from the \code{err_handler} field.  Increment the 
reference count for the error handler.

\subsubsection{\mpifunc{MPI_COMM_SET_ERRHANDLER}}
Error checking: ensure that the error handler is of the correct type.  
Question: do we need a special case for \mpifunc{MPI_ERRORS_RETURN} etc?

Free (reduce the \code{ref_count} and free if zero) the current error handler
and set the error handler field to the specified error handler.

\subsubsection{\mpifunc{MPI_WIN_CREATE_ERRHANDLER}}
Similar to the communicator versions.

\subsubsection{\mpifunc{MPI_WIN_CALL_ERRHANDLER}}
Similar to the communicator versions.

\subsubsection{\mpifunc{MPI_WIN_GET_ERRHANDLER}}
Similar to the communicator versions.

\subsubsection{\mpifunc{MPI_WIN_SET_ERRHANDLER}}
Similar to the communicator versions.

\paragraph{MPI I/O Error Handlers.}
We need to ensure that ROMIO's error handlers are the same as the MPICH-2
handlers.

\subsection{Handle Transfers}
These provide for the conversion of handles to and from the C and
Fortran representations.  C++ is handled as a descendant of C (that
is, there is no Fortran representation of a C++ handle directly, but
C++ can use C handles.  
\begin{figure}
\begin{verbatim}
     +-----+         +---------+
     |  C  | < --- > | Fortran |
     +-----+         +---------+
        ^
        |
        v
     +-----+ 
     | C++ | 
     +-----+ 
\end{verbatim}
\caption{Relationship of handle conversion functions.  The C to/from
C++ are part of the C++ binding of MPI.}\label{fig:handle-transfers}
\end{figure}
These should normally (i.e., unless
\cfgoption{--disable-mpi-macros} is
specified to configure) be implemented as macros.  

Unresolved question (raised by Barry Smith):  What should happen if
the handle is invalid?  Should there even be a check?  Raise the error
on \mpiconst{MPI_COMM_SELF} or \mpiconst{MPI_COMM_WORLD}?

The current plan is that all handle transfers will be handled by casting;
the handle transfer routines should all be available as macros, as allowed by
the MPI standard.  The handle transfer routines are:
\mpifunc{MPI_COMM_C2F},
\mpifunc{MPI_COMM_F2C},
\mpifunc{MPI_ERRHANDLER_F2C},
\mpifunc{MPI_ERRHANDLER_C2F},
\mpifunc{MPI_FILE_C2F},
\mpifunc{MPI_FILE_F2C},
\mpifunc{MPI_GROUP_F2C},
\mpifunc{MPI_GROUP_C2F},
\mpifunc{MPI_INFO_F2C},
\mpifunc{MPI_INFO_C2F},
\mpifunc{MPI_OP_F2C},
\mpifunc{MPI_OP_C2F},
\mpifunc{MPI_REQUEST_F2C},
\mpifunc{MPI_REQUEST_C2F},
\mpifunc{MPI_TYPE_C2F},
\mpifunc{MPI_TYPE_F2C},
\mpifunc{MPI_WIN_C2F}, and
\mpifunc{MPI_WIN_F2C}.

% \subsubsection{\mpifunc{MPI_COMM_C2F}}
% \subsubsection{\mpifunc{MPI_COMM_F2C}}
% \subsubsection{\mpifunc{MPI_ERRHANDLER_F2C}}
% \subsubsection{\mpifunc{MPI_ERRHANDLER_C2F}}
% \subsubsection{\mpifunc{MPI_FILE_C2F}}
% \subsubsection{\mpifunc{MPI_FILE_F2C}}
% \subsubsection{\mpifunc{MPI_GROUP_F2C}}
% \subsubsection{\mpifunc{MPI_GROUP_C2F}}
% \subsubsection{\mpifunc{MPI_INFO_F2C}}
% \subsubsection{\mpifunc{MPI_INFO_C2F}}
% \subsubsection{\mpifunc{MPI_OP_F2C}}
% \subsubsection{\mpifunc{MPI_OP_C2F}}
% \subsubsection{\mpifunc{MPI_REQUEST_F2C}}
% \subsubsection{\mpifunc{MPI_REQUEST_C2F}}
% \subsubsection{\mpifunc{MPI_TYPE_C2F}}
% \subsubsection{\mpifunc{MPI_TYPE_F2C}}
% \subsubsection{\mpifunc{MPI_WIN_C2F}}
% \subsubsection{\mpifunc{MPI_WIN_F2C}}

\subsubsection{\mpifunc{MPI_STATUS_F2C}}
This needs to recognize the constants \mpiconst{MPI_F_STATUS_IGNORE} and
\mpiconst{MPI_F_STATUSES_IGNORE} (which must be declared in \file{mpi.h}; see
Section 4.12.5 ``Status'' in MPI-2).  

\subsubsection{\mpifunc{MPI_STATUS_C2F}}
Like \mpifunc{MPI_STATUS_F2C}, but must handle the C constants
\mpiconst{MPI_STATUS_IGNORE} and \mpiconst{MPI_STATUSES_IGNORE}.

\subsection{Timers}
The MPI standard allows \code{MPI_Wtime} and \code{MPI_Wtick} to be
implemented as macros; we should also allow that, at least as an
option.  Question:  should there be a
\cfgoption{--enable-mpi-macros} feature in
configure?

Eventually, we should allow for a synchronized timer.  That is, even
if the underlying hardware does not provide a global timer, we should
provide one as an option.

Question: Since we have \mpidfunc{MPID_Gwtime}, should we make that available?

\subsubsection{\mpifunc{MPI_WTICK}}
Call \mpidfunc{MPID_Wtick}.

\subsubsection{\mpifunc{MPI_WTIME}}
Call \mpidfunc{MPID_Wtime}.  Use \mpidfunc{MPID_Wtime_diff} to convert this to
a double and return that value.
This requires that \mpifunc{MPI_Init} cause \mpidfunc{MPID_Wtime_init} to be
called, and that an initial time stamp is stored.

\subsection{Runtime Environment}
\subsubsection{\mpifunc{MPI_GET_PROCESSOR_NAME}}
Call \mpidfunc{MPID_Get_processor_name}.  

Question: How is the value of \mpiconst{MPI_MAX_PROCESSOR_NAME} specified?
Should configure run something to get this value from the chosen device?

\subsubsection{\mpifunc{MPI_GET_VERSION}}
Return the values of \mpiconst{MPI_VERSION} and \mpiconst{MPI_SUBVERSION}. 
Note that this routine can be called anytime, even before \code{MPI_Init} or
after \code{MPI_Finalize}.

\subsection{Profiling}

\subsubsection{\mpifunc{MPI_PCONTROL}}
This is a simple stub and performs no action other than returning
\mpiconst{MPI_SUCCESS} as the result.  See the MPICH-1 implementation.

\subsection{I/O}
MPI I/O will be provided by ROMIO.  We expect to update ROMIO to exploit both
MPI-2 functions and to make use of MPID functions (such as the Stream and
Segment modules) when ROMIO is part of MPICH.

There are a few places where we need to improve the current code to provide
better integration:
\begin{itemize}
\item I/O requests must be integrated with all other MPI requests.  This
  eliminates the need for ROMIO's \code{MPIO_Wait} routine.
\item Error handler should consistent with the rest of MPI-2.  Error reporting
  should follow the rest of MPICH-2.
\end{itemize}

% \subsubsection{\mpifunc{MPI_FILE_CALL_ERRHANDLER}}
% \subsubsection{\mpifunc{MPI_FILE_CLOSE}}
% \subsubsection{\mpifunc{MPI_FILE_CREATE_ERRHANDLER}}
% \subsubsection{\mpifunc{MPI_FILE_DELETE}}
% \subsubsection{\mpifunc{MPI_FILE_GET_AMODE}}
% \subsubsection{\mpifunc{MPI_FILE_GET_ATOMICITY}}
% \subsubsection{\mpifunc{MPI_FILE_GET_BYTE_OFFSET}}
% \subsubsection{\mpifunc{MPI_FILE_GET_ERRHANDLER}}
% \subsubsection{\mpifunc{MPI_FILE_GET_GROUP}}
% \subsubsection{\mpifunc{MPI_FILE_GET_INFO}}
% \subsubsection{\mpifunc{MPI_FILE_GET_POSITION}}
% \subsubsection{\mpifunc{MPI_FILE_GET_POSITION_SHARED}}
% \subsubsection{\mpifunc{MPI_FILE_GET_SIZE}}
% \subsubsection{\mpifunc{MPI_FILE_GET_TYPE_EXTENT}}
% \subsubsection{\mpifunc{MPI_FILE_GET_VIEW}}
% \subsubsection{\mpifunc{MPI_FILE_IREAD}}
% \subsubsection{\mpifunc{MPI_FILE_IREAD_AT}}
% \subsubsection{\mpifunc{MPI_FILE_IREAD_SHARED}}
% \subsubsection{\mpifunc{MPI_FILE_IWRITE}}
% \subsubsection{\mpifunc{MPI_FILE_IWRITE_AT}}
% \subsubsection{\mpifunc{MPI_FILE_IWRITE_SHARED}}
% \subsubsection{\mpifunc{MPI_FILE_OPEN}}
% \subsubsection{\mpifunc{MPI_FILE_PREALLOCATE}}
% \subsubsection{\mpifunc{MPI_FILE_READ}}
% \subsubsection{\mpifunc{MPI_FILE_READ_ALL_BEGIN}}
% \subsubsection{\mpifunc{MPI_FILE_READ_AT}}
% \subsubsection{\mpifunc{MPI_FILE_READ_AT_ALL_BEGIN}}
% \subsubsection{\mpifunc{MPI_FILE_READ_ORDERED}}
% \subsubsection{\mpifunc{MPI_FILE_READ_ORDERED_BEGIN}}
% \subsubsection{\mpifunc{MPI_FILE_READ_SHARED}}
% \subsubsection{\mpifunc{MPI_FILE_SEEK}}
% \subsubsection{\mpifunc{MPI_FILE_SEEK_SHARED}}
% \subsubsection{\mpifunc{MPI_FILE_SET_ATOMICITY}}
% \subsubsection{\mpifunc{MPI_FILE_SET_ERRHANDLER}}
% \subsubsection{\mpifunc{MPI_FILE_SET_INFO}}
% \subsubsection{\mpifunc{MPI_FILE_SET_SIZE}}
% \subsubsection{\mpifunc{MPI_FILE_SET_VIEW}}
% \subsubsection{\mpifunc{MPI_FILE_SYNC}}
% \subsubsection{\mpifunc{MPI_FILE_WRITE}}
% \subsubsection{\mpifunc{MPI_FILE_WRITE_ALL_BEGIN}}
% \subsubsection{\mpifunc{MPI_FILE_WRITE_AT}}
% \subsubsection{\mpifunc{MPI_FILE_WRITE_AT_ALL_BEGIN}}
% \subsubsection{\mpifunc{MPI_FILE_WRITE_ORDERED}}
% \subsubsection{\mpifunc{MPI_FILE_WRITE_ORDERED_BEGIN}}
% \subsubsection{\mpifunc{MPI_FILE_WRITE_SHARED}}

% Related constants
% \mpiconst{MPI_MODE_APPEND}
% \mpiconst{MPI_MODE_CREATE}
% \mpiconst{MPI_MODE_DELETE_ON_CLOSE}
% \mpiconst{MPI_MODE_NOCHECK}
% \mpiconst{MPI_MODE_RDONLY}
% \mpiconst{MPI_MODE_SEQUENTIAL}
% \mpiconst{MPI_MODE_UNIQUE_OPEN}
% \mpiconst{MPI_SEEK_CUR}
% \mpiconst{MPI_SEEK_END}
% \mpiconst{MPI_SEEK_SET}
% \mpiconst{MPI_DISPLACEMENT_CURRENT}

\section{Portability}

This section discusses how the MPICH implementation is written to
provide portability to a wide variety of systems.  MPICH will continue
to rely on \code{configure}.  However, the use of \code{configure}
will rely on more carefully defined macros, along with more
information stored in external files, allowing for simpler adaptation
to site-specific environments, such as special compilers and runtime
environments. 

Question: What about shared libraries?  Using libtool is awkward for
development, but not using it is awkward for portability (libtool
knows a \emph{lot} about making shared libraries).  However, we
\emph{must} have support for shared libraries.  The plan is to develop
a simple perl program to extract the information stored in the
\code{libtool} source to take advantage of that source of information
without requiring the \code{libtool} development environment.

\subsection{Configure}
\label{sec:configure}
We will use autoconf version 2.13 or later.  The top-level configure will only
test for items used in the implementation of the MPI routines.  For
other parts of the package, such as the ADI implementation, the
top-level configure will invoke a configure or other setup script for
each package.

Question: configure understands how to invoke configure for other
packages.  If we use a level of indirection between the ADI configure
(e.g., a setup script), the top-level configure won't know about
this.  Do we care?  If we don't care, how do we ensure that
re-executing \file{config.status} executes the correct routines?

Standard \code{configure} and \code{Makefile} variables
\begin{description}
\item[\texttt{COPTIONS}]Use this to pass options for the C compiler that are
  used by the package.  For example, the standard configure option
  \cfgoption{--enable-strict} sets this to strict compilation (e.g.,
  \code{-Wall -Wstrict-prototypes} etc.) when using the \code{gcc} compiler.

\item[\texttt{CFLAGS}]Do not set this value.  Allow the user to use this to
  change, for example, the optimization level or the debugging level.  For
  example, the user should be able to do
  \begin{verbatim}
  setenv CFLAGS -g
  make clean
  make
  \end{verbatim}
  to rebuild the package with \code{-g} added to all compile lines.
\end{description}

If \code{automake} is used, add
\begin{verbatim}
# Use AM_xFLAGS to modify compiler behavior
AM_CFLAGS=${COPTIONS}
\end{verbatim}
to the \file{Makefile.am}. 

\subsection{Supporting Cross-compilation}
\label{sec:cross-compile}
In some cases (e.g., IBM SP or ASCI Red), the compiler that must be used to
compile parallel programs produces executables that must be run under
the parallel environment, which may be difficult and time consuming.
This is a type of cross-compilation.  To support this, configure must
be carefully written both to support cross-compilation and to provide
for a way to specify the results that would have been determined by
running a program.

For each test that requires running a program, a variable of the form
\code{CROSS_xxx} must be defined and documented.  For example, for
variable sizes, \code{CROSS_SIZEOF_INT} will give the size of an
integer in bytes.

We need to document all \code{CROSS_xxx} variables.  Here is a start at that
list:
\begin{description}
\item[\texttt{CROSS_BYTE_ORDERING}]
\item[\texttt{CROSS_STRUCT_ALIGNMENT}]Structure alignment.  One of 
    \begin{description}
    \item[\texttt{packed}]No padding
    \item[\texttt{largest}]Aligned on the largest item
    \item[\texttt{two}]Aligned to two bytes
    \item[\texttt{four}]Aligned to four bytes
    \item[\texttt{eight}]Aligned to eight bytes
    \item[\texttt{other}]Unable to determine
    \end{description}
    A program to determine these already exists and is part of the
    current MPICH.
\item[\texttt{CROSS_SIZEOF_INT}]\code{sizeof(int)}
\item[\texttt{CROSS_SIZEOF_VOID_P}]\code{sizeof(void*)}
\item[\texttt{CROSS_SIZEOF_CHAR}]\code{sizeof(char)}
\item[\texttt{CROSS_SIZEOF_LONG}]\code{sizeof(long)}
\item[\texttt{CROSS_SIZEOF_FLOAT}]\code{sizeof(float)}
\item[\texttt{CROSS_SIZEOF_DOUBLE}]\code{sizeof(double)}
\item[\texttt{CROSS_SIZEOF_LONG_DOUBLE}]\code{sizeof(long double)}
\item[\texttt{CROSS_F77_SIZEOF_INTEGER}]The size of an \code{INTEGER} in
  Fortran (as if there was a \code{sizeof} operator in Fortran)
\item[\texttt{CROSS_F77_SIZEOF_REAL}]Ditto for \code{REAL}
\item[\texttt{CROSS_F77_SIZEOF_DOUBLE_PRECISION}]Ditto for
\code{DOUBLE PRECISION}.
\item[\texttt{CROSS_F90_INTEGER_KIND}]The Fortran 90 kind for an integer that
  corresponds to a C \code{int}.
\item[\texttt{CROSS_OFFSET_KIND}]The Fortran 90 kind for an integer that
  corresponds to a \code{MPI_Offset}.
\item[\texttt{CROSS_ADDRESS_KIND}]The Fortran 90 kind for an integer that
  corresponds to a \code{MPI_Aint}.
\end{description}

Other items, such as the allowed types for the f90 modules, also needs
to be specifyable from the environment.

See also mpi-maint request 5556 for the values needed by the Intel
Tflops system.  It should be possible to specify these with a special
site file.  Also note the need to export variables; we may want to
create a step that does something like \code{set | grep 'CROSS_' | 
sed ...}.  

\subsubsection{Complex Configuration Data}
Much of the complexity in the current configure system comes from
handling special cases, particularly for compiler and linker options.
I propose replacing this code with a separate (yet simple) data list
that can be edited separately from the configure script, and which
contains information on various compilers and linkers.
This file can be considered a very simple database, where each record
contains the following information (some of the information is used
only by compile steps; other by link or shared library steps.  There
is enough overlap that the combined list is given).
\begin{description}
\item[kind]C, C++, Fortran 77, or Fortran 90.  Perhaps Java as well.
\item[action]compile, link, create static library, or create shared library
\item[name]E.g., cc, xlc, pgcc, gcc
\item[ostype]OS that has this compiler.  For some compilers, this is *
(e.g., gcc, pgcc, cc)
\item[optimize]Options for optimizing.
\item[debug]Options for debugging.
\item[ansi]Options to force ANSI (or a superset)
\item[posix]Options to force Posix
\item[threads]Options to allow threads
\item[sharedobject]Options to create a shared object
\item[version]Options to generate a version and name string.  See signature
\item[signature]A regular expression that should match the value
generated by version.  This may be slightly extended to allow a
particular line of the output to be matched.
\item[size32]Options for 32bit pointers
\item[size64]Options for 64bit pointers
\item[size=xx]Options for other sizes (128 bit pointers, anyone?) or
for special versions (e.g., IRIX n32)
\item[searchdir]Options to specify search directories for shared
libraries
\item[sharedlib]Options to create a shared library
\item[cross]Specify values for \code{CROSS_xxx} for cross-compiler case
\item[verbose]Options to generate verbose output, particularly showing
the command-line options passed to other tools such as ld when a
compiler command is used to link a program.  This is useful in
determining the libraries needed for multilanguage programs.
\item[verboseoptionsep]Option separator for the output from verbose.
Often either space or comma.
\item[strict]Options for strict (lint-like) compilation
\end{description}
Question: are these enough?  We should check what libtool uses.
We may also want a option that says ``check for a clean compile before
accepting these values''.  Compiler version numbers may also be
important; for that, there needs to be a command specified to extract and
compare the version number.

This file should have both specific rules and generic rules.  For
example, a generic description of \code{cc} would specify \code{-g}
for debugging, \code{-O} for optimization, and little else.  

In some cases, we may want to try several options.  For example, for
optimization, we may want to try \code{-Ofast} or \code{-O2} before \code{-O}.
Question:
what should the syntax for this be?

I recommend that the syntax for this file be key=value pairs, using a
trailing backslash to continue lines:
\begin{verbatim}
# comment describing compiler
kind=c action=compile name=xlc system=AIX \
  optimize="-O3 -qarch=native" \
  etc.
kind=c action=compile name=cc system=* \
  optimize=-O \
  debug=-g \
  etc.
\end{verbatim}
This is most easily processed using perl or (possibly) python, though
another alternative is to bootstrap by using the usual configure
macros to find a C compiler and then compile a simple program to read
this file.  In the cross compilation case, either a different compiler
may be used, the user can prebuild the program, or all of the
necessary data can be supplied through environment variables.

\subsection{Coding Rules}
We will attempt to enforce the coding rules by building tools that
check for conformance to these rules.

Tools already available:
\begin{description}
\item[checkforglobs]Check for global symbol names
\item[tool-to-be-created]Check for use of banned routines (e.g., \code{printf},
\code{alloca})
\item[gcc -Wall etc.]Missing prototypes, return values,
etc. (Question: add an \code{--enable-strict} to the standard
configure macros?)
\item[tool-to-be-created]Check for OS name or system type in a preprocessor
  statement (look for \emph{all} \code{\#ifdef} or \code{\#if defined()} and
  check against permitted values).
\end{description}

We also need tools for the following:
\begin{itemize}
\item Look for functions and header files that are not universal and
ensure that they are properly guarded.
\item Look for functions that are not universal (e.g., \code{rindex}).
Question: Do we have a list of banned routines?  Permitted ones?
\end{itemize}

\subsubsection{Printing and Other Messages}
Rule: do not use \code{printf} or \code{fprintf} except (possibly) for
messages intended only the for the developers of MPICH2.

Even for developer messages, using \code{printf} is not a good idea.
It is better to call a routine that can handle recording the results.

% See PETSc approach for printing.  

Question: Should we define \code{PRINTF} etc. as we have in MPICH, or
ban those values entirely?  Should we define a general output routine
that can be implemented to output to a file, stdout, or a graphical display?

\subsection{NT Friendly}

Declare all user-visible routines \code{EXPORT_MPI_API}\index{EXPORT_MPI_API}.
This is a 
macro that is defined as empty for UNIX and as the appropriate
Microsoft-specific extension for Windows.

Question: we can use a file to list these functions instead of
\code{EXPORT_MPI_API}.  Should we do that instead?

Avoiding \code{printf} is important for Windows applications.

Question: What else do we need to consider here?

\subsection{Fortran Support}
Fortran support has two parts: Fortran 77 and Fortran 90/95.  

There are a few functions unique to Fortran.  They include
\mpifunc{MPI_SIZEOF} and \mpifunc{MPI_TYPE_CREATE_F90_COMPLEX}, 
\mpifunc{MPI_TYPE_CREATE_F90_REAL}, and 
\mpifunc{MPI_TYPE_CREATE_F90_INTEGER}.  Note that the \code{TYPE_CREATE}
routines create types that must return the corresponding combiner names when
\mpifunc{MPI_TYPE_GET_ENVELOPE} is called and that these must be
``predefined'' types; that is, they cannot be freed.

\subsubsection{\mpifunc{MPI_SIZEOF}}
This must be implemented in an MPI module.  The implementation looks something
like this:
\begin{verbatim}
       MODULE MPI2__REAL_s
       PRIVATE
       PUBLIC :: MPI_SIZEOF
       INTERFACE MPI_SIZEOF
           MODULE PROCEDURE MPI_SIZEOF_T
       END INTERFACE
       CONTAINS
       SUBROUTINE MPI_SIZEOF_T( X, SIZE, IERROR )
       REAL X
       INTEGER SIZE, IERROR
       IERROR = 0
       SIZE   = 4
       END SUBROUTINE MPI_SIZEOF_T       
       END MODULE
\end{verbatim}
Each Fortran type (including each type passed as an array, and for
each number of dimensions of the array) requires a similar
definition.  The actual size (four in the example above) must be determined by
configure or provided by the user.
These can be created automatically in a way similar to the current
Fortran 90 interface.

\subsubsection{\mpifunc{MPI_TYPE_CREATE_F90_INTEGER}}
This searches through a global list (pointed at by
\mpidconst{MPID_F90_Predefined_types_head}) to see if the requested type has
already been allocated.  If so, it returns that type.  
If not, it must allocate a new predefined type, initialize all of the fields,
including the envelope type of \mpiconst{MPI_COMBINER_F90_INTEGER} and the
\code{digits} field, and returns that new type.
In a multithreaded case, this list must be locked while the update is taking
place. 

Questions:
Should we just have a small array instead of a list?  If the user asks for too
many distinct types, we can return an \mpiconst{MPI_ERR_OTHER} class
indicating the problem.

Question:
Should there be a separate routine to initialize this array and to free the
datatypes that are created during \mpifunc{MPI_Finalize}?  (I think so,
particularly when trying to streamline codes to requiring only single-language
support.)  What are the names of the routines?  Is there a common file that
contains \mpidconst{MPID_F90_Predefined_types_head} as well as the routines to
allocate and free these predefined types?  See also the finalize
callbacks in \mpifunc{MPI_Finalize}.

\subsubsection{\mpifunc{MPI_TYPE_CREATE_F90_REAL}}
See \mpifunc{MPI_TYPE_CREATE_F90_INTEGER}.

\subsubsection{\mpifunc{MPI_TYPE_CREATE_F90_COMPLEX}}
See \mpifunc{MPI_TYPE_CREATE_F90_INTEGER}.

\subsubsection{Fortran Wrappers}
One added complexity of the Fortran wrappers is handling the possibility that
the types \code{MPI_Fint} and \code{int} have different sizes.  When the
Fortran codes simply call the C codes, this results in copying array arguments
to a temporary array, calling the C code, and freeing the array.  Allocation
and deallocation of small arrays can be avoided by using local arrays (to be
thread-safe).  However, I'd prefer to avoid this whenever possible.  Thus, we
need a CPP value that indicates whether \code{MPI_Fint} and \code{int} are the
same size, such as \mpifunc{MPICH_FINT_EQ_INT}.

In addition, the MPICH Fortran wrapper code is intended for use with MPE and
other MPI implementations, and makes no assumptions about the structure of
MPI opaque objects, necessitating a transfer of values for each array-value
argument.  I'd like to avoid this as well.  Can we ignore the other MPI's, or
use special routines as part of the MPE support?  Alternately, should we
define \code{MPICH_REQUEST_C_IS_F77} etc.?

\subsection{C++ Support}
\label{sec:c++}
I'd like to consider adding more native C++ support.  While the Notre
Dame C++ code is valuable and helpful, there are some difficulties:
\begin{enumerate}
\item Supports only MPI-1.

\item Their configure is based too much on particular systems rather than on
capabilities. 

\item Because it is a separate package, the build process is awkward.

\item Testing is separate from the MPICH testing, causing inadequate
testing of the MPICH/C++ combination.

\item Layering makes some things difficult; it can be awkward because
some data structures must be duplicated since MPI itself leaves them
opaque.  For example, the C++ \code{Comm} could contain a pointer to the
\mpidconst{MPID_Comm}, rather than the opaque object \code{MPI_Comm}.

\item Does not pass our (sometimes stricter) coding standards.

\item Their's must deal with a wide variety of partial C++
implementations.  Perhaps by 2002, there will be fewer bugs in C++ compilers.
\end{enumerate}

\section{Standard Features}
\label{sec:standard-features}
(this section contains the standard features, such as command line
handling, environment variables, configure options for error checking,
etc.)

\subsection{Command line and environment}
The implementation must provide the \emph{service} of providing
command-line arguments to each process.  If the startup environment
does not do so, the implementation must (see the discussion of
\mpidfunc{MPID_Init}).  

\subsection{Standard I/O}
I/O handling, particularly for stdin, remains troublesome.  We need to
be more precise about what is available and what isn't.  We need to
pay closer attention to the value of the attribute \mpiconst{MPI_IO},
and we may want to consider adding new keyvals with more control, such
as \mpidconst{MPICH_IO_STDIN}, \mpidconst{MPICH_IO_STDOUT}, and
\mpidconst{MPICH_IO_STDERR}. 

\subsection{Documentation and Man Pages}
The MPI standard requires that the MPI implementation document certain
features and capabilities.  For example, any signal used by the MPI
implementation must be documented, as must any limitation on the use of
\mpifunc{MPI_Comm_join}.  

Since many of these depend on the capabilities of the device, each device
should provide files \file{signal_limits.txt} and \file{join_limits.txt} that
document any limitations in the use of signals or \mpifunc{MPI_Comm_join}.  If
no file is provide, no limits exist.

\section{Testing}
\label{sec:testing}

The tests for MPICH2 need to be more organized that for MPICH. They
should follow these principles:
\begin{enumerate}
\item Require no user intervention (e.g., to inspect the results)
\item Should be implementation independent (i.e., useful for
\emph{any} MPI implementation) unless they are testing \emph{only}
features specific to MPICH2.  An example is a test of error messages
and error class/code values; any test that expects a particular
message must not be combined with a general test of standard
conformance.
\item Require no extra files (the \file{.std} files in the MPICH
version).
\item Testing should be applied to a range of communicators and
datatypes, not just \code{MPI_COMM_WORLD}.  
\item Testing should be controlled by a file listing the tests rather
than a script.  That is, the script \code{runtests} should be generic,
working with a file in each testing directory that lists the test
programs and any special options (e.g., number of processes, command
line arguments, environment variables).
\item Testing that enables memory tracing should be simple and
organized so that it can be run regularly.
\item Should have short duration, so that the testing program can
signal failure because a program has not completed within the given
time.
\item All tests should have positive output.  Not all MPI
implementation are reliable at indicating a non-zero return code or at
flushing output.  A requirement for an output of \code{Test passed} is
necessary. 
\item [BRT] Errors should not be reported as a result of a feature
having been disabled (i.e., tests for the long long type should not
report a failure when the system is configured with
\code{--disable-long-long}),
\end{enumerate}

Results should be maintained in a database which should include
\begin{enumerate}
\item Configuration options
\item System description (including compiler version and machines file)
\item CVS tag (or nearest value, e.g., weekly tag + date that source
cut was made).
\item Results summary (success or failure; if failure, reason).
\item We should consider an XML format for the output.
\end{enumerate}

Testing should also contain performances tests (see the current Chiba
tests for an example: \file{/home/gropp/projects/chiba/perf/getperf}.
\begin{enumerate}
\item Configuration options
\item System description (including compiler version and machines file)
\item CVS tag (or nearest value, e.g., weekly tag + date that source
cut was made).
\item Results for the following tests:
    \begin{enumerate}
    \item \code{mpptest -logscale} To get the general trend in performance
    \item \code{mpptest -auto} To get details for the short message
performance. 
    \item bisection bandwidth test for a large number of processes.
    We may want to use beff instead of mpptest.
    \item \code{mpptest -logscale -halo -npartner 8} To get more
    realistic communication performance
    \item \code{mpptest -logscale -gop -dsum} 
    \item Similar test for alltoallv. 
    \item Tests for I/O (Rajeev and Rob to identify)
    \item Tests for put/get/accumulate as those become available.
    Start with \code{mpptest -logscale -put} and \code{mpptest
-logscale -halo -npartner 8}
    \end{enumerate}
\item This should also be in XML format.
    Question: Should we add an XML output format to \code{mpptest}?
Is there any clear XML definition yet for 2-d data?

\end{enumerate}
We should maintain the same data for vendor and other MPI
implementations, and we should ask for a standard set of tests.  

\subsection{Communication Tests}
\label{sec:testing-comm}

The MPI communication routines are quite general and allow many
combinations of arguments.  A comprehensive test needs to check many
of these combinations.  

\begin{enumerate}
\item Datatypes. It is particularly important to check cases where the
sender and receiver use datatypes with the same type signature but
different type maps (e.g., contiguous data at the sender and indexed
at the receiver).  
     \begin{enumerate}
     \item All predefined datatypes
     \item Some mixed types
     \item Different patterns (contig, vector (block count of one,
           two, 13), indexed (monotone
           increasing and nonmonotone)) 
     \item \code{MPI_PACKED} sent but received either with
           \code{MPI_PACKED} or with the matching type signature
     \end{enumerate}
     To implement this, with each datatype, we need routines to
     \begin{enumerate}
     \item Allocate the send buffer.  Leave room for sentinals at both
     ends of the buffer.
     \item Initialize the send buffer.  Make sure that most bits are
           set.  E.g., 64-bit int values should include values with
           bits set in the high 32 bits.
     \item Allocate receive buffer.
     \item Check that the correct data has been received (and that no
           other data has been set).  This takes the data buffer, the
           count, the status value, and the structure containing the
           description of this datatype.
     \item Release buffers.
     \end{enumerate}
     This list suggests that a routine be used to return an array of
     structures that contain both the datatypes to use (providing
     separate send and receive datatypes), the functions to allocate,
     initialize, and check the data buffers.  This function can query
     the test initialization routine to determine the number of
     datatypes to provide; for example, choosing all types, only the
     most popular types (e.g., \code{MPI_INT} and \code{MPI_DOUBLE}),
     or a specific type (e.g., an environment variable containing a
     string that names the datatype).  The allocation routines must
     take an \code{nelm} that specifies the number of basic
     types.  If there is a required divisor of \code{nelm} (e.g., the
     datatype is a vector with a block count of 13), that is
     specified, allowing the testing code to compute valid \code{nelm}s.

     Question:  For collective scatter and gather routines
     (particularly the ``v'' versions), do we also need routines to
     allocate, initialize, and check the communication buffers?
\item Communicators
    \begin{enumerate}
    \item \code{MPI_COMM_WORLD}
    \item Dup of \code{MPI_COMM_WORLD}
    \item rerank from $r$ to $n-r-1$ of \code{MPI_COMM_WORLD} (i.e.,
    reverse the order of the ranks
    \item Split into communicators containing only the odd and only
    the even ranks from \code{MPI_COMM_WORLD}
    \item \code{MPI_COMM_SELF} (note that this can't be used for some
    blocking communication)
    \item Intercommunicator (built from the groups containing odd,
or even ranks in MPI 
    \end{enumerate}
\item Message Sizes from zero to at least 128 KB
\end{enumerate}

Test harness

\subsection{Debugger Interface}
\label{sec:debug-interface}
We need a test that the DLL that provides access to the internal
symbols is correct, as well as a test that the MPICH library correctly
identifies the location of the DLL.

To test the DLL, we need a program \file{mpichdlltest.c} that can
load the dll and call the routines, ensuring that it runs correctly.
This program should use the ADI's include files to provide the data
structures that the DLL accesses.

Note that the debugger interface must be compiled in 32-bit mode on
platforms with mixed 32 and 64-bit modes.  We don't currently do this
(which is a bug) but we need to.  To do this, we need a configure step
that may need to know that a system has both options (in the case that
\code{sizeof(void *)} is 64.  See ``complex configuration
options'' above.  We should add a macro \code{PAC_PROG_CC32} that
determines the 32-bit compiler (if any); the DLL's makefile should use
\code{CC32} instead of \code{CC} \emph{or} it should have its own
\code{configure}. 

Still needed: Information needed for debugger startup and message
queues.

\subsection{Performance and Tracing Data}
\label{sec:tracing}

One critical piece of information that is hard currently to acquire is
the amount of idle time spent in completing a communication operation.
This is information that a tracing library would like to have.  Should
there be an \code{MPID_xxx} call that could be made available to a tracing
package?  For example, it could have semantics roughly like
\code{MPI_Wtime}, except that it would contain cumulative idle time.  In a
multithreaded environment, it could give a per-thread cumulative idle
time (or could it)?  Should we define \code{MPID_Idle_time}, and modify MPE
to look for that name in the MPI library?

To tune the layered routines, including the collective routines, we
should include from the beginning standardized tracing for:
\begin{enumerate}
\item All (major?) MPID calls
\item Idle time.  Note that to ensure that this is really close to the
actual idle time, it may be necessary to separate some actions into a
``check for ready'' and ``perform operation''.  For example, in the
TCP case, you'd want to use \code{select} to determine the idle time
rather than ever use a blocking I/O operation.  
\item Context switches (if using threads)
\item Resource usage
\item Flow control
\end{enumerate}
In addition to these, a tracing layer can benefit from access to the
context id and to a message sequence number (necessary for matching
messages in the multi-threaded case).  

%\let\SaveBibliography=\thebibliography
%\def\thebibliography#1{\SaveBibliography{#1}\addcontentsline{toc}{section}{References}}

% \section{Appendix}
% This section contain \emph{all} of the MPI functions and terms.  These
% will be moved into the body of this document; an index will then
% replace this appendix.  Until then, this section serves as a place to
% cache these items.

%\subsection{Constants}
%(I need to get the union of MPI-1 and MPI-2 constants)

% \subsection{Typedefs}
% \code{MPI_Aint}\\
% \code{MPI_Comm_copy_attr_function}\\
% \code{MPI_Comm_errhandler_fn}\\
% \code{MPI_Datarep_conversion_function}\\
% \code{MPI_Datarep_extent_function}\\
% \code{MPI_File_errhandler_fn}\\
% \code{MPI_Grequest_cancel_function}\\
% \code{MPI_Type_copy_attr_function}\\
% \code{MPI_Win_copy_attr_function}\\
% \code{MPI_Win_errhandler_fn}\\

% %%
% %% Temporary
% \clearpage
% Accessing this document.  Use one of:
% \begin{enumerate}
% \item \code{cvs -d /home/MPI/cvsMaster checkout mpich2-coding}
% \item \code{cvs -d /home/MPI/cvsMaster checkout mpich2all; cd doc/mpich2}
% \item \code{cd mpich2 ; cvs -q update -d}, only if you have previously checked
%   \code{mpich2} out with \code{mpich2all} instead of \code{mpich2}.
% \end{enumerate}
% %% End of Temporary
\section{ToDo List}
\label{sec:todo}
This section contains a ToDo list for this document; that is, the outstanding
issues and questions.  The process for resolving each of these is to have each
item choosen by one person who is responsible for writing the text (the
section author) and one
other person who is the ``immediate reviewer.''  The section author is
responsible for 
organizing and leading any discussion necessary to complete the text.
Anyone may read and comment on the document at any time, but should check with
the section author first to make sure that the document is up-to-date.
Once a section is written, it should be read by everyone and we should
``vote'' on the section.  Once a section is ``passed,'' the section author
should update the ADI-3 document to match the section.  This may involve
changing, adding, and/or deleting routines from the ADI-3 document.  Once that
step is completed, the routines in the section can be written.  The section
author is \emph{not} responsible for implementing the routines in the
section (though they can be; the point is that authoring a section is separate
from implementing a section).  

Many of these sections can be implemented independently, once the
infrastructure list is settled. 
\begin{enumerate}
\item Infrastructure

  These are necessary before any coding can commence.
  \begin{enumerate}
    \item Directory structure
    \item MPI routine source code template
    \item Partial definitions for key structures, such as communicators and
      datatypes, and macros for thread-safe operations
    \item Coding standards (documentation, style, associated tests)
    \item Integral profiling and data collection.  Definition of macros for
      collecting timing data and for generating slog records.
  \end{enumerate}
  In addition, the error reporting routines and guidelines to error handling
  are needed before much coding is done.

\item MPI Major Sections
  Each of these sections should consider
  \begin{enumerate}
    \item Thread safety,
    \item Error handling,
    \item Core ADI for 3rd parties (non-multimethod),
    \item Core method ADI for 3rd parties (as part of our multimethod device),
      and 
    \item Performance in MPI communication (whether point-to-point,
      collective, or RMA).
    \end{enumerate}
    In addition, scalability to 10,000 processes is required and scalability
    beyond that to 1,000,000 processes should be considered.  However, if
    scalability to a million processes complicates the design or the code, we
    should design for fewer processes and document the reasons in the
    Rationale (Section~\ref{sec:rationale}).

    The highest-priority items are: Point-to-point, collective, RMA, and
    dynamic processes, along with the communication agent.  Of course, these
    will require some specification of datatypes, communicators, groups, etc.,
    but they will also drive the details of those objects (e.g., datatypes
    must be defined to support the operations needed for communication).

  \begin{enumerate}
  \item Attributes. 
  \item Info.
  \item Datatypes.
  \item Groups.
  \item Point-to-point.  
    \begin{enumerate}
    \item Make the scenario (Section~\ref{sec:pt-2-pt-scenarios})
      consistent with the description of the individual routines and with
      the most current discussions of the active queue implementation,
      including the support of collective communication.
    \item Define the communicator data structure, including the handling
      of processes that are not part of the original
      \mpiconst{MPI_COMM_WORLD}.
    \item Address the issues of the multiple completion routines (e.g.,
      \mpifunc{MPI_Waitsome}). 
    \end{enumerate}
  \item Communication agent. 
    This ensures the progress of MPI communication including passive RMA
    access.  As such, it it closely connected to the point-to-point and RMA
    sections.  
  \item Collective.  
    \begin{enumerate}
    \item \mpifunc{MPI_Bcast}, \mpifunc{MPI_Scatter}, and \mpifunc{MPI_Reduce}
      with ``stream'' operations.
    \item Plan for developing algorithms for the other collective routines.
    \item Design to allow implementors to replace any collective routine with
      a device-specific version
    \end{enumerate}
  \item Communicators.
    \begin{enumerate}
    \item Basic routines for communicator construction.  Coordinate intercomm
      creation with the dynamic process routines (\mpifunc{MPI_Comm_spawn},
      \mpifunc{MPI_Comm_connect}, \mpifunc{MPI_Comm_attach}, and
      \mpifunc{MPI_Join}).  
    \end{enumerate}
  \item Topology.
  \item RMA.  Everything (including scenarios illustrating BSP-style defered
    updates).  
  \begin{enumerate}
    \item Scenarios
  \end{enumerate}
  \item Starting and Ending MPI (e.g., init, finalize, and abort).
  \item Dynamic processes.
  \item Name service.
  \item User-defined requests (needed for I/O).
  \item Error handlers. (These are the MPI error handlers, not the error
    reporting routines.)
  \item Handle Transfers (e.g., \code{MPI_Request_c2f})
  \item Timers.
  \item I/O.  For the most part, we will take ROMIO without any changes for
    now.  However, there are a few things to handle:
    \begin{enumerate}
    \item Replace ROMIO's \code{MPIO_Request} and \code{MPIO_Wait} etc. with
      regular \code{MPI_Request}s (possibly using the generalized requests).
    \item Update error reporting with new routines
    \item Check on datatype handling
    \end{enumerate}
  \item Runtime Environment.  (Processor name and MPI version.)
  \item Profiling.
  \item MPI command environment (\code{mpiexec}, \code{mpicc}, etc.)
  \end{enumerate}
\item Source code, portability, and framework
  These are miscellaneous (though important) items that need to be completed
  before much coding is done.

  \begin{enumerate}
  \item Error reporting routines, particularly the handling of
    instance-specific messages and internationalization.
  \item mpich2 bug list.
  \item Runtime parameter access (e.g., socket buffer sizes from an
    environment variable or \file{.mpichrc} file).
  \item Configure and automake, particularly a style-sheet on modifying the
    autoconf and automake input files
  \item Cross compilation and compilation environment (using different
    compilers from the ones MPICH is built with)
  \item Fortran
  \item C++
  \item Testing.  Needs new test harness; intelligent sampling of the possible
    combinations; archiving of results (including performance tests).  The
    tests must work with any MPI, not just MPICH (just like the current test
    suite).  Separate tests for MPICH-specific features should be provided in
    a separate suite of tests.
  \end{enumerate}
\end{enumerate}

% \section{Development Plan}
% \label{sec:development}
% This section proposes a development and implementation plan.  The
% emphasis here is on independent subprojects, allowing development to
% proceed without waiting for all of the pieces to be in place.

% \begin{enumerate}
% \item ADI Implementation
%     \begin{enumerate}
%     \item Design Development
%         \begin{enumerate}
%         \item Design and implement a subset of general datatype pack/unpack
%         \item Design and implement a subset (e.g., Allreduce, Bcast, and
%           Alltoall) of collective operations using the Stream and Segment MPID
%           functions.
%         \item Test improved collectives for performance and functionality
%         \item Similar design/implement/test for RMA and Dynamic; details yet
%           to be determined.
%         \end{enumerate}
%     \item Multi-method ADI for shared memory, VIA, and TCP/UDP
%         \begin{enumerate}
%         \item Design and implement general datatype point-to-point
%         \item Design and implement general datatype collective
%         (layered on ADI segment and stream interfaces)
%         \item Design and implement RMA
%         \item Design and implement Dynamic and connect with BNR
%         \end{enumerate}
%     \item ADI on \code{MPID_CORE}
%         \begin{enumerate}
%         \item Implement all ADI-3 modules on top of the core.  Refine
%         the design of the core during this process.   
%         \end{enumerate}
%     \item Implement common ADI services.  These are modules that are
%     needed by other ADI modules that do not (directly) involve
%     interprocess communication but will be common for most ADI
%     implementations, including both the core and the multi-method
%     implementation. 
%         \begin{enumerate}
%         \item Error handling
%         \item Attributes
%         \item Info
%         \item Runtime parameters
%         \item Topology
%         \item Datatype
%         \item Group
%         \item Timer
%         \item Utility
%         \end{enumerate}
%     \end{enumerate}
% \item MPICH Implementation on ADI
%     \begin{enumerate}
%     \item For each MPI routine, write the ``top'': structured comment,
%     routine header and error checking.
%     \item Implement the action of each MPI routine in terms of the
%     full ADI-3 design.
%     \end{enumerate}
% \item Test Suite
%     \begin{enumerate}
%     \item Design standard test harness: test communicators, datatypes,
%     operation mixtures, run script, and result checking.
%     \end{enumerate}
% \item Deployment
%     \begin{enumerate}
%     \item Implement new configure.  Design database for both configure
%     macros and for data (such as compiler options and names) that
%     configure cannot determine.
%     \item Shared library support.
%     \end{enumerate}
% \end{enumerate}

%
% Eventually include table or a project as Postscript from project.
%
\appendix
\section{Rationale}
\label{sec:rationale}

This appendix contains some of the discussion about the particular choices
made in the design of the MPICH2 implementation, and include both design
alternatives and discussion of constraints that may not be obvious to a casual
reader of the MPI standard.  This appendix is organized by major section.

\subsection{Attributes} 
\subsection{Info}
\subsection{Datatypes}
\subsection{Groups}
\subsection{Point-to-point}  
\subsection{Communication agent}
\subsection{Collective}  
\subsection{Communicators}
\subsection{Topology}
\subsection{RMA}  
\subsection{Starting and Ending MPI}
\subsection{Dynamic processes}
\subsection{Name service}
\subsection{User-defined requests}
\subsection{Error handlers}
\subsection{Handle Transfers}
\subsection{Timers}
\subsection{I/O}  
\subsection{Runtime Environment}
\subsection{Profiling}
\subsection{MPI command environment}

\let\SaveBibliography=\thebibliography
\def\thebibliography#1{\SaveBibliography{#1}\addcontentsline{toc}{section}{References}}
\bibliography{/home/MPI/allbib,/home/gropp/Update/new/gropp,mpich2}
\bibliographystyle{plain}

% Index
%\openin\testfile{adi3man.ind}
%\ifeof\testfile\else
\let\SaveIndex=\theindex
\def\theindex{\SaveIndex\addcontentsline{toc}{section}{Index}}
\input mpich2.ind
%\fi
%\closein\testfile

\end{document}
