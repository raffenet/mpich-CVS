%
% Design document for MPICH2
%
% This document should be used with the ADI3 document
\documentclass{article}
\usepackage{/home/gropp/data/share/refman}
\usepackage{/home/gropp/sowing-proj/sowing/docs/doctext/tpage}
\usepackage{epsf}
\textheight=9in
\textwidth=6.1in
\oddsidemargin=.2in
\topmargin=-.50in
\newread\testfile

%
% Modify the way titles are handled for no breaks between pages
%\def\mantitle#1#2#3{\pagerule\nobreak
%\ifmancontents\addcontentsline{toc}{subsection}{#1}\fi}

\makeindex

\begin{document}

\markright{MPICH Design Document}

\def\nopound{\catcode`\#=13}
{\nopound\gdef#{{\tt \char`\#}}}
\catcode`\_=13
\def_{{\tt \char`\_}}
\catcode`\_=11
\def\code#1{{\tt #1}}

\def\mpifunc#1{\code{#1}\index{#1}}
% We may want to separate the MPI and MPID index entries
\def\mpidfunc#1{\code{#1}\index{#1}}
% ditto for constants
\def\mpiconst#1{\code{#1}\index{#1}}
% Environment variables
\def\eindex#1{\index{#1}}

%\tpageoneskip
\ANLTMTitle{MPICH2 Design Document\\
Draft of \today}{\em 
William Gropp\\
Ewing Lusk\\
Mathematics and Computer Science Division\\
Argonne National Laboratory}{00}{\today}

\clearpage

\pagenumbering{roman}
\tableofcontents
\clearpage

\pagenumbering{arabic}
\pagestyle{headings}

\section{Introduction}
This document discusses how the MPICH2 implementation is written using
the ADI-3 \cite{adi3man} for the supporting functions.  This document
contains guidelines for the MPICH2 implementation.  One important
purpose of this document is to provide common guidelines for writing
the MPICH code.

See also the Coding Standards docment \cite{coding-standards}.

To date, this document contains primarily comments on the rules for
writing the code.  Few comments on the use of ADI-3 routines have been
added yet.  No part of this document is final.

A major challenge is developing an interface that requires fewer (or
at least simpler) routines to implement.  This is particularly
difficult since the MPI standard is defined to encourage efficient
implementations.  While it is possible to meet the functional
definitions of MPI with fewer routines, achieving performance requires
something relatively close to what MPI defines.

One possibility is to consider a few classes of systems.  Pure
distributed memory is one important case.  Another is shared memory,
or at least some common shared memory.  Of course, multi-method
devices make this more difficult.  Some more discussion of this is in
the RMA section.

% Outline:
\section{General}
This section contains general recommendations are requirements for
writing the code.

\subsection{Error reporting}

MPI routines should check all possible error conditions early, before
calling any ADI routines.  The ADI routines perform little error
checking.  To allow error handling to be enabled or disabled both at compile
time 
and at runtime, the tests should be placed within the following block:
\begin{verbatim}
#ifdef HAVE_ERROR_CHECKING
BEGIN_ERROR_CHECKS
...
END_ERROR_CHECKS
#endif
\end{verbatim}
The macros \code{BEGIN_ERROR_CHECKS} and \code{END_ERROR_CHECKS} can
expand, depending on configuration settings, into either null (i.e.,
no runtime control) or 
\begin{verbatim}
#define BEGIN_ERROR_CHECK if (MPIR_Perform_Error_Tests) {
#define END_ERROR_CHECKS }
\end{verbatim}

There must be a configure option, \code{--disable-error-checking},
that prevents \code{HAVE_ERROR_CHECKING} from being defined.

\subsection{Error handling and Fault Tolerance}
In order to support fault tolerance, errors should be handled as
gracefully as possible.  If it is possible to remain in a consistent
state, the process should not abort (unless, of course, the error
handler requires it, as the default \mpifunc{MPI_ERRORS_FATAL} does).  
If it is not possible to recover from an error, then the process
should call \mpidfunc{MPID_Abort} but specify \mpifunc{MPI_COMM_SELF} as the
communicator.  

\subsection{Memory Allocation}
Avoid memory allocation wherever pratical.  Use \mpidfunc{MPID_Malloc}
etc. instead of \code{malloc}, as described in the ADI-3 manual
\cite{adi3man}. 

At normal exit from the MPI library in \mpifunc{MPI_Finalize}, the
routine \mpidfunc{MPID_Trdump} must be called if the environment variable
\code{MPICH_TRDUMP}\eindex{MPICH_TRDUMP} is set.  

\subsection{MPI Opaque Objects}

Most objects in MPI (with the exception of \mpifunc{MPI_Status}) are
opaque.  In the MPICH2 implementation, opaque objects are represented
by integers.  This simplifies the implementation of the functions for
transfering handles between C/C++ and Fortran.  

Question:  One interesting idea is to encode information in the opaque
handle.  For example, the \code{sizeof} a basic datatype, and whether
a type is basic, could be part of the handle for an
\code{MPI_Datatype}. In that case, an implementation can avoid looking
up the datatype (e.g., by using the integer of the opaque type as an
array index) and instead perform a few simple operations on the handle
(e.g., mask and test and mask and shift).  Similarly, for
communicators that are dups of \code{MPI_COMM_WORLD}, the handle could
contain the \code{context_id}, again avoiding the need to look up the
communicator, since in addition the mapping from rank to local pid for 
communicators similar to \code{MPI_COMM_WORLD} is the identity
mapping.  Thus a single bit test on the opaque handle could eliminate
a number of tests and memory references for an important common case.

The questions are: do we want to make this possible?  If so, do we
want to implement it from the beginning?

\subsection{MPI and PMPI Routines}
Each routine should implement the \code{PMPI} version of the routine.
Where possible, a weak symbol pragma may be used to define the
\code{MPI} version of the routine.  If weak symbol support is not
available, the \file{Makefile}s will support recompiling each file
with the definition \code{MPICH_MPI_FROM_PMPI} made.  This can value
can also be used to protect code that is used by the \code{PMPI}
version of the routine.  For example, 
\begin{verbatim}
#include "mpich_impl.h"

#ifdef MPICH_MPI_FROM_PMPI
#define PMPI_Attr_get MPI_Attr_get
#else
int MPIR_Attr_get_util( int keyval, MPID_Comm *comm, void *attr )
{
...
}
#endif
int PMPI_Attr_get( ... etc ... )
\end{verbatim}

If it is necessary to create the \code{MPI} versions separately, the
object files should be renamed, allowing them to be placed into the
same library.  To handle the event that the library cannot hold over
500 files (250 for PMPI MPI 1 and 2, plus a version for the MPI
routines), the name of the library containing the profiling versions
should be separate.

Question: If we do have two libraries, then we must link with both,
even when the profiling routines are not otherwise used because any
internal functions may be defined in the PMPI versions.  We may need
to do this anyway, because any function that calls an MPI routine will
actually be calling the PMPI version.  Is this what we want to do?  Is
it the best thing to do?  Note that we do this now, but through the
confusing approach of using the MPI names, but redefining them as PMPI
with a file containing a redefinition of every single MPI routine.

Question: The pragma code is very ugly and hard to read.  It is also hard to
update.
It would be relatively
easy to create the correct form on the fly, using configure (or
another program, run by configure).  But that
would require at least an include file for each MPI routine.
An alternate approach is to make the inclusion entirely machine
generated.  In that case, updating it requires only re-running the
update editor.  In this case, the pragmas for implementing the
profiling interface should be placed between two clear markers, such
as 
\begin{verbatim}
/* -- begin profiling interface -- */
/* DO NOT EDIT.  Use the program xxx to update */
...
/* -- end profiling interface -- */
\end{verbatim}

\subsection{Coding Practices}
This section reviews some coding practices for the MPICH code.

\begin{description}
\item[function prototypes.]
All routines should be prototyped.  The prototypes should be in the
smallest scope possible.  For example, if the routine is used only
within the files in a subdirectory, the prototype should be in an
include file within that directory.
Question: do we really want this?

\item[static functions.]
Functions used entirely within a single file should be declared
\code{static}.  Functions that are not static must follow the naming
convention of starting with \code{MPI_} or \code{PMPI_} (for routines
implementing the MPI Standard) or \code{MPIR_} for internal routines.  
Global symbols, such as keyvals, should use \code{MPICH_} as the
prefix.  

\item[Parameter declarations.]
Parameters (with the exception of the MPI routines defined by the
standard) should follow the guidelines in \code{coding}.  In
particular, \code{const} and \code{restrict} should be used.
Parameters that are semantically arrays should be declared as arrays
(using \code{[]}).  

\item[Indentation style.]
Question: should we specify an indentation style?  If so, we need to
specify it at the top of each file (as an emacs command) so that emacs
will use the correct style for users with other styles.

\item[File header.]
There should be a standard file header, containing the copyright and
standardized includes (e.g., something like "mpiimpl.h").  This header
template should be part of the development environment.  Question:
what should this header be?

\end{description}

\subsection{Other Subsystems}
In MPICH, there is code that is not directly part of the MPI
implementation, such as the MPE and test suite code.  For MPICH2,
these should be cleanly separated.  To simplify the construction of a
full MPICH distribution, there should be a Makefile (and configure
options) that know how to build MPICH2 with MPE, perftest, the test
suite, and other options.  These should \emph{not} be part of the base
MPICH2 project (as far as CVS is concerned).  This will encourage
better separation of the projects.

Question: What about Fortran and C++?  These are required for full MPI support
(unlike MPE or the performance tests), but have many unique requirements
during configurations.  I believe that these should continue to have a
separate configure, though it is not necessary to make them standalone (e.g.,
work with other implementations of MPI).

\section{Include Files}
The include file \file{mpi.h} should not require any \code{-Dxxx}
definitions by the compiler.  This will require generating the
\file{mpi.h} from another file (and probably not using \code{configure} and an
\file{mpi.h.in}).  It should not include any other
files, with the (possible) exception of \file{mpi++.h} for C++ support.

The \file{mpif.h} (of \file{mpif.h.in}) should be created from
\file{mpi.h} so that the various integer values (e.g., error classes,
datatypes, etc.) are guaranteed consistent.

\section{MPI Operations}
This section describes the implementation of the MPI operations.  The
descriptions may include discussion of some implementation issues.
These are split up according to function, and roughly (but not
exactly) match the MPI standard.  Each of these has a corresponding
directory in the MPI source tree.  Note that this means that the
directory structure does not exactly match the chapter structure of
the MPI Standard.

\subsection{Attributes and Info}

\subsubsection{\mpifunc{MPI_ATTR_DELETE}}
\subsubsection{\mpifunc{MPI_ATTR_GET}}
\subsubsection{\mpifunc{MPI_ATTR_PUT}}
\subsubsection{\mpifunc{MPI_KEYVAL_CREATE}}
\subsubsection{\mpifunc{MPI_KEYVAL_FREE}}

\subsubsection{\mpifunc{MPI_COMM_CREATE_KEYVAL}}
\subsubsection{\mpifunc{MPI_COMM_FREE_KEYVAL}}
\subsubsection{\mpifunc{MPI_COMM_NULL_COPY_FN}}
%\subsubsection{\mpifunc{MPI_COMM_DUP_FN}}

\subsubsection{\mpifunc{MPI_TYPE_DELETE_ATTR}}
\subsubsection{\mpifunc{MPI_TYPE_FREE_KEYVAL}}
\subsubsection{\mpifunc{MPI_TYPE_NULL_COPY_FN}}
%\subsubsection{\mpifunc{MPI_TYPE_DUP_FN}}

\subsubsection{\mpifunc{MPI_WIN_CREATE_KEYVAL}}
\subsubsection{\mpifunc{MPI_WIN_FREE_KEYVAL}}
\subsubsection{\mpifunc{MPI_WIN_SET_ATTR}}
%\subsubsection{\mpifunc{MPI_WIN_NULL_COPY_FN}}

\subsubsection{\mpifunc{MPI_INFO_CREATE}}
\subsubsection{\mpifunc{MPI_INFO_DELETE}}
\subsubsection{\mpifunc{MPI_INFO_DUP}}
\subsubsection{\mpifunc{MPI_INFO_FREE}}
\subsubsection{\mpifunc{MPI_INFO_GET}}
\subsubsection{\mpifunc{MPI_INFO_GET_NKEYS}}
\subsubsection{\mpifunc{MPI_INFO_GET_NTHKEY}}
Question:  Do we want to have this return an error if the info list is 
modified by another thread?  Is there any way to actually do this?

\subsubsection{\mpifunc{MPI_INFO_GET_VALUELEN}}
\subsubsection{\mpifunc{MPI_INFO_SET}}

\mpiconst{MPI_ERR_INFO_KEY}
\mpiconst{MPI_ERR_KEYVAL}
\mpiconst{MPI_MAX_INFO_KEY}

\subsection{Datatypes}
The ADI defines a datatype structure that is believed to be a good
choice for implementing operatoins that involve datatypes, such as
\mpifunc{MPI_Pack} and \mpifunc{MPI_Unpack}.

\subsubsection{\mpifunc{MPI_ADDRESS}}

\subsubsection{\mpifunc{MPI_GET_ADDRESS}}

\subsubsection{\mpifunc{MPI_TYPE_CONTIGUOUS}}

\subsubsection{\mpifunc{MPI_TYPE_COMMIT}}

\subsubsection{\mpifunc{MPI_TYPE_HINDEXED}}

\subsubsection{\mpifunc{MPI_TYPE_HVECTOR}}

\subsubsection{\mpifunc{MPI_TYPE_INDEXED}}

\subsubsection{\mpifunc{MPI_TYPE_STRUCT}}

\subsubsection{\mpifunc{MPI_TYPE_VECTOR}}

\subsubsection{\mpifunc{MPI_TYPE_CREATE_DARRAY}}

\subsubsection{\mpifunc{MPI_TYPE_CREATE_F90_COMPLEX}}

\subsubsection{\mpifunc{MPI_TYPE_CREATE_HINDEXED}}

\subsubsection{\mpifunc{MPI_TYPE_CREATE_INDEXED_BLOCK}}

\subsubsection{\mpifunc{MPI_TYPE_CREATE_KEYVAL}}

\subsubsection{\mpifunc{MPI_TYPE_CREATE_RESIZED}}

\subsubsection{\mpifunc{MPI_TYPE_CREATE_STRUCT}}

\subsubsection{\mpifunc{MPI_TYPE_CREATE_SUBARRAY}}

\subsubsection{\mpifunc{MPI_TYPE_FREE}}
This first calls \mpidfunc{MPID_Datatype_incr} with an increment of $-1$.  If
the returned value is zero, it should free the contents of \code{dataloop} and
then call \mpidfunc{MPID_Datatype_free}.

\subsubsection{\mpifunc{MPI_TYPE_EXTENT}}
Simply uses the \code{extent} field in the structure.

\subsubsection{\mpifunc{MPI_TYPE_LB}}
Simply uses the \code{lb} field in the structure.

\subsubsection{\mpifunc{MPI_TYPE_SIZE}}
Simply uses the \code{size} field in the structure.

\subsubsection{\mpifunc{MPI_TYPE_UB}}
Simply uses the \code{ub} field in the structure.

\subsubsection{\mpifunc{MPI_TYPE_GET_TRUE_EXTENT}}
Simply uses the \code{true_extent} field in the structure.

\subsubsection{\mpifunc{MPI_TYPE_GET_CONTENTS}}
Uses the \code{dataloop} field to access the data uses to create the datatype.

\subsubsection{\mpifunc{MPI_TYPE_GET_ENVELOPE}}
Uses \code{dataloop} field to identify how the datatype was constructed.
The \code{combiner} type must be one of 
\mpiconst{MPI_COMBINER_NAMED},
\mpiconst{MPI_COMBINER_DUP},
\mpiconst{MPI_COMBINER_CONTIGUOUS}, 
\mpiconst{MPI_COMBINER_VECTOR},
\mpiconst{MPI_COMBINER_HVECTOR_INTEGER},
\mpiconst{MPI_COMBINER_HVECTOR},
\mpiconst{MPI_COMBINER_INDEXED},
\mpiconst{MPI_COMBINER_HINDEXED_INTEGER},
\mpiconst{MPI_COMBINER_HINDEXED},
\mpiconst{MPI_COMBINER_INDEXED_BLOCK},
\mpiconst{MPI_COMBINER_STRUCT_INTEGER},
\mpiconst{MPI_COMBINER_STRUCT},
\mpiconst{MPI_COMBINER_SUBARRAY},
\mpiconst{MPI_COMBINER_DARRAY},
\mpiconst{MPI_COMBINER_F90_REAL},
\mpiconst{MPI_COMBINER_F90_COMPLEX},
\mpiconst{MPI_COMBINER_F90_INTEGER},
\mpiconst{MPI_COMBINER_RESIZED}.

\subsubsection{\mpifunc{MPI_TYPE_GET_EXTENT}}
Simply uses ???

\subsubsection{\mpifunc{MPI_TYPE_GET_NAME}}
Uses the \code{name} field.

\subsubsection{\mpifunc{MPI_TYPE_SET_NAME}}
Sets the \code{name} field.

\subsubsection{\mpifunc{MPI_PACK}}
\subsubsection{\mpifunc{MPI_PACK_SIZE}}

\subsubsection{\mpifunc{MPI_UNPACK}}

\subsubsection{Heterogeneity}
\label{sec:hetero}
Optimizing for the common case of machines or clusters with a common
data representation is important.  

In MPICH, macros were used to include code that handled heterogeneous
systems.  For MPICH2, I'd prefer to use clearer blocks of code rather
than special macros.  For example,
\begin{verbatim}
#define MPICH_IS_HETERO
...
#else
...
#endif
\end{verbatim}

Questions about the implementation of datatypes:

\begin{enumerate}
\item Should we require alignment of data when packing/unpacking?  The
   problem is in the heterogeneous case, where we'd need to communicate
   the alignment rules, along with byte ordering and data lengths.

\item For nested datatypes, should we allow loop interchange (as NEC did
   in their ``flattening on the fly'' paper)?  We can implement this
   within the current representation by creating new dataloop
   structures for the re-ordered loops.  

\item We need to provide for the important special cases of aligned moves
   of sizes 1, 2, 4, 8, and perhaps 16.

\item We could even compile code to pack and unpack the given datatype
   and dynamically load the code.  PETSc has code for this for some
   user-interface convenience functions.  In general, we could
   consider allowing the pack and unpack functions to be specified as
   part of the datatype, with defaults based on the dataloop
   structures.  A datatype attribute could be used to decide when to
   create a datatype-specific routine.

\item We need to include instrumentation on the pack/unpack functions
   themselves so that we can gather information about the performance
   of the pack/unpack.  Should this be stored by datatype instance?
   Datatype kind (e.g., vector, indexed, struct)?  pack/unpack?

\item Do we need separate pack and unpack descriptions (e.g., if we
   optimize for the transfers by reordering loops, will we want
   different versions for each direction)?

\item In dataloop, kind should include information on basic alignment
   and/or length (to allow fast loops using wide moves based on long
   or double instead of char).

\item For types that do not contain struct, we can preload the entire
   processing stack, since the elements never change (just the
   position on the stack).  This is close to creating a simple nested
   loop structure for an interpreter.  We may want the datatype to
   have a field indicating that it has this feature; alternately, we
   might encode this by specifying a different pack/unpack routine,
   one that preloads the stack and eliminates any code to fill the
   stack during processing.  Another approach that would apply to the
   more general case would be to cause datatypes that have simple
   nested structure to load the entire stack and switch the stack
   interpreter into a mode that knew that the stack had been loaded.

\item Struct alignment (pad) should have optional rules.  That is, we
   need to support all alignment options that a compiler might pick
   (we currently test for this in the MPICH configure).  For systems
   where different padding rules can be specified (e.g., xlc has 4
   different choices), we should allow an environment variable to
   select a different padding rule.  We might implement this by using
   a separate routine for each type of padding, and call a routine to
   compute the padding towards the end of creating a struct datatype.

\item For pack and unpack code, we need to handle the tests for sizes of
    the output buffers efficiently, hoisting the tests out of the
    loops where possible.  

\item For the homogeneous case, some struct types (those that contain
    only basic datatypes) can be changed into indexed types (as if
    they were all \code{MPI_BYTE}).

\item Structs with no gaps (except at the ends, possibly because of 
    structure padding, an \code{MPI_UB}, or an explicit resize,
    should be replaced with a strided type.  In the heterogeneous
    case, this can only be done when the struct contains a single
    basic type.

\item In the heterogeneous case, we may want two different
    representations: one for homogenous communication and one for
    heterogeneous communication.  Thus the datatype structure needs
    several dataloop entries, at least in the heterogeneous case.

\end{enumerate}

\subsection{Groups}

Groups are simple.  The key point here is to make sure that the use of
groups to map from ranks in a communicator to particular destination
process is fast.
To provide a fast implementation of \mpifunc{MPI_COMM_GROUP}, groups must
have reference counts.  

Question: Do we want a special case for the groups of self and comm
world?  This would eliminate a lookup in the group table for the a very
common case.  Should there be a \code{MPID_GROUP_WORLD}?  

Question: What data structure should be used to represent groups?
MPICH-1 used a simple array that mapped rank in a group to rank in the
group of \code{MPI_COMM_WORLD}.  In MPI-2, we can't use
\code{MPI_COMM_WORLD}.  ADI-3 defines ``local process ids'', which
refers simply to the processes known to the current process.

Question: How do we implement \mpifunc{MPI_Group_translate_ranks}?

\subsubsection{\mpifunc{MPI_GROUP_COMPARE}}
\subsubsection{\mpifunc{MPI_GROUP_DIFFERENCE}}
\subsubsection{\mpifunc{MPI_GROUP_EXCL}}
\subsubsection{\mpifunc{MPI_GROUP_FREE}}
\subsubsection{\mpifunc{MPI_GROUP_INCL}}
\subsubsection{\mpifunc{MPI_GROUP_INTERSECTION}}
\subsubsection{\mpifunc{MPI_GROUP_RANGE_EXCL}}
\subsubsection{\mpifunc{MPI_GROUP_RANGE_INCL}}
\subsubsection{\mpifunc{MPI_GROUP_RANK}}
\subsubsection{\mpifunc{MPI_GROUP_SIZE}}
\subsubsection{\mpifunc{MPI_GROUP_TRANSLATE_RANKS}}
\subsubsection{\mpifunc{MPI_GROUP_UNION}}

\subsection{Communicators}

Communicators have two main features: a context id and a group.  In
addition, communicators that are created with \mpifunc{MPI_Comm_dup} must
copy attributes (where requested) from the old communicator.

Question: do we want a ``hidden'' communicator for implementing the
collective routines?  Just a hidden context?  One concern when there
are two communicators: which do you lock (e.g., with
\mpidfunc{MPID_Comm_thread_lock})?  How do you avoid deadly embraces?  How do
you 
ensure that the expected communicator is acted on?

If a hidden communicator is used, it should have its error handler permanently
set to \mpifunc{MPI_ERRORS_RETURN} so that an error action that is appropriate to
the collective routine, not the routine called, may be used.

\subsubsection{\mpifunc{MPI_COMM_COMPARE}}
\subsubsection{\mpifunc{MPI_COMM_CREATE}}
\subsubsection{\mpifunc{MPI_COMM_DUP}}
\subsubsection{\mpifunc{MPI_COMM_FREE}}
\subsubsection{\mpifunc{MPI_COMM_GROUP}}
Return a duplicate (shallow copy) of the local group.

\subsubsection{\mpifunc{MPI_COMM_RANK}}
Return the \code{rank} field.

\subsubsection{\mpifunc{MPI_COMM_REMOTE_GROUP}}
Return a duplicate (shallow copy) of the remote group.

\subsubsection{\mpifunc{MPI_COMM_REMOTE_SIZE}}
See \code{MPI_COMM_SIZE}/

\subsubsection{\mpifunc{MPI_COMM_SIZE}}
Return the \code{size} field.  This is really the size of the local
group, which for an intercommunicator may be different from the remote
size.  Note that for point-to-point communication, error checking for
destination or source ranks must look at the remote size.  Should
there be a \code{local_size} field in the communicator?

\subsubsection{\mpifunc{MPI_COMM_SPLIT}}
\subsubsection{\mpifunc{MPI_COMM_TEST_INTER}}
\subsubsection{\mpifunc{MPI_INTERCOMM_CREATE}}
\subsubsection{\mpifunc{MPI_INTERCOMM_MERGE}}

\subsubsection{\mpifunc{MPI_COMM_C2F}}
\subsubsection{\mpifunc{MPI_COMM_F2C}}
\subsubsection{\mpifunc{MPI_COMM_CLONE}}
This is a special C++ function.

\subsubsection{\mpifunc{MPI_COMM_CALL_ERRHANDLER}}
\subsubsection{\mpifunc{MPI_COMM_CREATE_ERRHANDLER}}
\subsubsection{\mpifunc{MPI_COMM_GET_ERRHANDLER}}
\subsubsection{\mpifunc{MPI_COMM_SET_ERRHANDLER}}

\subsubsection{\mpifunc{MPI_COMM_GET_NAME}}
Return a copy of the \code{name} field.

\subsubsection{\mpifunc{MPI_COMM_SET_NAME}}
Set the \code{name} field.

\subsection{Point to Point Communication}

The ADI provides a relatively close match to the point-to-point
communication routines.  

Questions that remain:  Handling of persistent requests.  The ADI
contains memory registration.  Is anything else needed for persistent
requests? 

\subsubsection{\mpifunc{MPI_PROBE}}
\subsubsection{\mpifunc{MPI_BSEND}}
\subsubsection{\mpifunc{MPI_BSEND_INIT}}
\subsubsection{\mpifunc{MPI_BUFFER_ATTACH}}
\subsubsection{\mpifunc{MPI_BUFFER_DETACH}}
\subsubsection{\mpifunc{MPI_CANCEL}}
\subsubsection{\mpifunc{MPI_GET_COUNT}}
\subsubsection{\mpifunc{MPI_GET_ELEMENTS}}
\subsubsection{\mpifunc{MPI_IBSEND}}
\subsubsection{\mpifunc{MPI_IPROBE}}
\subsubsection{\mpifunc{MPI_IRECV}}
\subsubsection{\mpifunc{MPI_IRSEND}}
\subsubsection{\mpifunc{MPI_ISEND}}
\subsubsection{\mpifunc{MPI_ISSEND}}
\subsubsection{\mpifunc{MPI_RECV}}
\subsubsection{\mpifunc{MPI_RECV_INIT}}
\subsubsection{\mpifunc{MPI_REQUEST_GET_STATUS}}
\subsubsection{\mpifunc{MPI_REQUEST_FREE}}
\subsubsection{\mpifunc{MPI_RSEND}}
\subsubsection{\mpifunc{MPI_RSEND_INIT}}
\subsubsection{\mpifunc{MPI_SEND}}
\subsubsection{\mpifunc{MPI_SENDRECV}}
\subsubsection{\mpifunc{MPI_SENDRECV_REPLACE}}
\subsubsection{\mpifunc{MPI_SEND_INIT}}
\subsubsection{\mpifunc{MPI_SSEND}}
\subsubsection{\mpifunc{MPI_SSEND_INIT}}
\subsubsection{\mpifunc{MPI_START}}
\subsubsection{\mpifunc{MPI_STARTALL}}
\subsubsection{\mpifunc{MPI_STATUS_SET_CANCELLED}}
\subsubsection{\mpifunc{MPI_STATUS_SET_ELEMENTS}}
\subsubsection{\mpifunc{MPI_TEST}}
\subsubsection{\mpifunc{MPI_TESTALL}}
\subsubsection{\mpifunc{MPI_TESTANY}}
\subsubsection{\mpifunc{MPI_TESTSOME}}
\subsubsection{\mpifunc{MPI_TEST_CANCELLED}}
\subsubsection{\mpifunc{MPI_WAIT}}
\subsubsection{\mpifunc{MPI_WAITALL}}
\subsubsection{\mpifunc{MPI_WAITANY}}
\subsubsection{\mpifunc{MPI_WAITSOME}}

\subsection{Collective Communication and Computation}
One of the major changes in MPICH2000 is in the implementation of the
collective routines.  The MPICH2000 implementation will exploit
pipelining and store and forward algorithms; these are supported by
the \code{MPID_Stream_xxx} routines.  

Since each system may have some feature that provides for even faster
implementation of the collective routines, it will be possible to
substitute a system-specific implementation for any of the collective
routines.  The purpose of the implementations provided with MPICH is
to provide a good performance and a level of performance that will be adequate
for many users.  

The $\alpha$-tree approach described in
\cite{bern:mpi-collective:hpcn99} should be considered; this is a
simple variation on the binomial tree approach used in the MPICH
implementations of many of the collective routines.  We will consider
combining this with the pipelining and scatter/gather approaches
championed by van de Geijn (\cite{vandegeijn} isn't quite the right
reference but it will do for now).


\subsubsection{Redution functions}
The reduction functions must use the \code{restrict} qualifier.

Question:
The MPICH code uses one gigantic file, \file{global_ops.c}, to implement
all of the reduction functions.  There are two problems with this.
First, some compilers become unhappy with it and do not optimize it
very well.  Second, all applications must load all of the routines
even though only a few (typically one) reduction function is used.  We
could break this into separate routines for each operation.  Those
could further be broken down by basic datatype, since the datatype is
known by the routine that calls the specific reduction function.  
For example, we could have \code{MPIR_SUM_Double},
\code{MPIR_SUM_Int}, etc.  This would also allow us to use Fortran
code for some or all of these routines, since Fortran compilers
typically produce better code for this kind of operation (though the
new definitions, using 'restrict', may be much better).

The down side of this is that, particularly in unstripped code, each
file (particularly if it includes any significant header files)
includes a significant amount of information.  A latency, if you will,
for each file.  That is, if putting all of the routines into a single
file takes $n$ bytes, putting them into $k$ files takes $n + (k-1)m$,
where $m$ is the size of the header.  

If we want to use the Fortran compiler for some or all of these, we'll
need a Fortran compiler and a backup when there is no Fortran compiler.

\subsubsection{Code Structure for the Implementation of the Collective
  functions} 

The MPICH code uses one gigantic file, \file{interops.c}, to provide a
generic implementation of each collective operation.  Each
communicator has a structure of pointers to functions.  Unless
otherwise set, each communicator points to the predefined structure
'MPIR_intra_collops' which is initialized to point to \emph{all} of
these functions.  

For MPICH2, I'd prefer that the default case have no structure, and
each of the MPI functions (in its file) contains the generic
implementation of the collective operation, based (most likely) on the
stream operations.  This will simplify the process of tuning each
operation; it will also reduce the size of executables since few if
any programs use all of the collective operations.
See the discussion of the implementation of PMPI.

Question: Now that MPI-2 defines intercommunicator collective
routines, do we want these in the same file as the intracommunicator
routines, or in a alternate file.  E.g., should \file{bcast.c} contain
the intracommunicator implementation of \mpifunc{MPI_Bcast} and
\file{ibcast.c} contain the intercommunicator implementation.

\subsubsection{Intracommunicator Collective Operations}
The following section (will) briefly describe the algorithms used to implement
the intracommunicator collective operations.

Several functions support \mpiconst{MPI_IN_PLACE} as an argument.

\subsubsection{\mpifunc{MPI_ALLGATHER}}
For short data, use recursive doubling algorithm.  For long data, consider the
bucket brigade algorithm.

\subsubsection{\mpifunc{MPI_ALLREDUCE}}
Consider recursive doubling algorithm, with care taken to ensure that all
results are the same, particularly in the presence of extended registers
(e.g., 80 bit intermediate quantities on Intel).

\subsubsection{\mpifunc{MPI_ALLTOALL}}
For short data and for all data on completely connected networks, use a
hypercube algorithm: each process exchanges with its partner in that dimension
the data needed by the partner and that partner's subsequent partners (in the
remaining dimensions).

For long data on less capable networks, use a bucket brigade algorithm.

\subsubsection{\mpifunc{MPI_ALLTOALLV}}
This is similar to \code{MPI_ALLTOALL}, but the decision on size of data is
more complicated. 

\subsubsection{\mpifunc{MPI_ALLTOALLW}}

\subsubsection{\mpifunc{MPI_BARRIER}}
Barrier is a special case of allreduce with no operation.

\subsubsection{\mpifunc{MPI_BCAST}}
Broadcast will be implemented by a scatter followed by an allgather.  These
will use an ordering of nodes from \mpidfunc{MPID_Topology_xxx}, rather than
the rank ordering of the communicator.

\subsubsection{\mpifunc{MPI_EXSCAN}}
See \code{MPI_SCAN}.

\subsubsection{\mpifunc{MPI_GATHER}}
This will use a MST for short gathers to reduce the impact of latency.  

\subsubsection{\mpifunc{MPI_GATHERV}}

\subsubsection{\mpifunc{MPI_REDUCE}}
Use a spanning tree.  Pipeline for long vectors.

\subsubsection{\mpifunc{MPI_REDUCE_SCATTER}}

\subsubsection{\mpifunc{MPI_SCAN}}
Use a spanning tree.  Pipeline for long vectors

\subsubsection{\mpifunc{MPI_SCATTER}}
For short data, use an MST. 

\subsubsection{\mpifunc{MPI_SCATTERV}}
For short data, use an MST. 

\subsubsection{Intercommunicator Collective Operations}
\subsubsection{\mpifunc{MPI_ALLGATHER}}
\subsubsection{\mpifunc{MPI_ALLGATHERV}}
\subsubsection{\mpifunc{MPI_ALLTOALL}}
\subsubsection{\mpifunc{MPI_ALLTOALLV}}
\subsubsection{\mpifunc{MPI_ALLTOALLW}}
\subsubsection{\mpifunc{MPI_BARRIER}}
\subsubsection{\mpifunc{MPI_BCAST}}
\subsubsection{\mpifunc{MPI_ALLREDUCE}}
\subsubsection{\mpifunc{MPI_REDUCE}}
\subsubsection{\mpifunc{MPI_EXSCAN}}
\subsubsection{\mpifunc{MPI_GATHER}}
\subsubsection{\mpifunc{MPI_GATHERV}}
\subsubsection{\mpifunc{MPI_SCAN}}
\subsubsection{\mpifunc{MPI_SCATTER}}
\subsubsection{\mpifunc{MPI_SCATTERV}}

\subsubsection{\mpifunc{MPI_OP_CREATE}}
\subsubsection{\mpifunc{MPI_OP_FREE}}

\subsection{Topology}

How do we implement \mpifunc{MPI_Cart_create} and \mpifunc{MPI_Dims_create}
with the MPID routines?   
Do we need an \mpidfunc{MPID_Topology_cart} and
\mpidfunc{MPID_Topology_cart_dims}?  Constructing a mesh from the 
hierarchical description that we've included can only be done
approximately.

The MPICH implementation uses private attributes to hold this information 
within a communicator.  Do we still want to do that?  Do we want to make the
topology routines a separate module so that, at the MPI level, it is easy to
substitute for these routines?  How do we define the routine called by
\mpifunc{MPI_Init} to initialize these routines (e.g., acquire attribute keys
for the topology information)?

One advantage to using attributes (or equivalently a pointer to a
structure) is it allows any information to be saved in with the
communicator, not just some predefined fields.  

If attributes are used for holding topology information, we need to
\emph{add} a routine to initialize the corresponding keyval; this will
allow the topology routines to be replaced as a module.

Question: how do we define routines needed to support the MPI (not
MPID) calls?  Should we have \code{MPIi} routines?  For topology, we
could have \code{MPIi_Topo_init}.

\subsubsection{\mpifunc{MPI_CARTDIM_GET}}
\subsubsection{\mpifunc{MPI_CART_CREATE}}
\subsubsection{\mpifunc{MPI_CART_GET}}
\subsubsection{\mpifunc{MPI_CART_MAP}}
\subsubsection{\mpifunc{MPI_CART_RANK}}
\subsubsection{\mpifunc{MPI_CART_SHIFT}}
\subsubsection{\mpifunc{MPI_CART_SUB}}
\subsubsection{\mpifunc{MPI_DIMS_CREATE}}
\subsubsection{\mpifunc{MPI_GRAPHDIMS_GET}}
\subsubsection{\mpifunc{MPI_GRAPH_CREATE}}
\subsubsection{\mpifunc{MPI_GRAPH_GET}}
\subsubsection{\mpifunc{MPI_GRAPH_MAP}}
\subsubsection{\mpifunc{MPI_GRAPH_NEIGHBORS}}
\subsubsection{\mpifunc{MPI_GRAPH_NEIGHBORS_COUNT}}
\subsubsection{\mpifunc{MPI_TOPO_TEST}}

\subsection{RMA}

My original plan was to implement this using the \code{Segment},
\code{Rhcv}, \code{Put_contig} and \code{Get_contig} routines.  We
will need code to support datatype caching at the destination process.
We may want to provide a way to define datatypes in globally shared
memory for systems like large SMPs that provide global access to at
least some memory.  Currently, there is no ADI interface for that.
I have since added additional put/get for the case where the origin
and target datatypes are the same.  

Question:  Should there be a model of remotely-defined datatypes that
would allow processes to avoid caching the description?  How would
this work in the multi-method case where some processes might have
shared memory and others might not?

For systems with ordered delivery, we may want a simpler completion
model, one that has completion per destination process (or per process
per window) rather than per RMA operation.  This is a further reason
to require that completion flags be created, and that this creation
contain both destination process and window.  Where operations are
ordered, this flag can simply count the number of started but not
completed operations, or it could contain a sequence number of some
sort for the most recent operation.  

Question:  For this to work with the waitflags and testflags, we
really need a flag set for the RMA window, which each RMA operation
takes (instead of a separate flag address).  How should the API for
both the flag set creation, reference, and completion work?  

The current ADI-3 interface defines put and get operations for both
contiguous data (at both origin and target) and for the case where the
same datatype is used at both origin and target.  Who is responisble
for the other cases?  The MPICH code or the ADI code?

The completion flags for the \code{MPID_Put_contig} etc. operations
have not be throughly though out.  For example, there is no explicit
support for the group-based window completion (\code{MPI_Win_post}
etc.), nor is there simple support for systems like the Cray T3E that
have (roughly) hardware support for \mpifunc{MPI_Win_fence}.

Question: The MPI RMA design is actually pretty lean and general, and
without further constraints or properties, it is hard to create a
simpler interface.  However, we might be able to simplify by
considering three important cases:
\begin{description}
\item[Shared Memory]This is not fully shared, but shared memory
segments or shared \code{mmap} regions. There may need to be special
calls to enforce memory ordering and coherency.
\item[Distributed Memory with DMA]This is for systems that support
some one-sided data delivery, such as VIA or LAPI.
\item[Distributed Memory with no DMA]This is for simple
network-connected processes, such as Unix processes connected by TCP.
\end{description}
Question: are these sufficient?  Should we put these classes into the
method-based interface instead?

For example, where shared memory is available, the synchronization and
lock operations can act directly on the shared memory area that is
allocated as part of the window object.  For example, the
start/post/complete/wait can use counters and flags in shared memory.
Locks can be acquired directly and quickly in shared memory, and (for
the passive target operations), the RMA operations can then be done
directly in shared memory.

In contrast, in the distributed memory case, particular with high
latency interconnects, deferred synchronization can be used.  For
example, a \code{MPI_Win_lock} in that case could return immediately.
At the first RMA operation, particularly if the amount of data is
small, the request for a lock can be piggy-backed on the RMA request.
In fact, following the BSP style, all of the RMA operations could be
held until the \code{MPI_Win_unlock}.

Clearly, the choice of immediate or deferred locks depends on the kind
of communication between processes.


\subsubsection{\mpifunc{MPI_ACCUMULATE}}
\subsubsection{\mpifunc{MPI_PUT}}
\subsubsection{\mpifunc{MPI_GET}}

\subsubsection{\mpifunc{MPI_WIN_FENCE}}

\subsubsection{\mpifunc{MPI_ALLOC_MEM}}
\subsubsection{\mpifunc{MPI_FREE_MEM}}

\subsubsection{\mpifunc{MPI_WIN_C2F}}
Question:  are there any special values similar to the ones considered
for datatypes and communicators?  For example, one bit could indicate
whether all windows are in shared memory.

\subsubsection{\mpifunc{MPI_WIN_F2C}}

\subsubsection{\mpifunc{MPI_WIN_CREATE}}
\subsubsection{\mpifunc{MPI_WIN_FREE}}

\subsubsection{\mpifunc{MPI_WIN_CREATE_ERRHANDLER}}
\subsubsection{\mpifunc{MPI_WIN_CALL_ERRHANDLER}}
\subsubsection{\mpifunc{MPI_WIN_GET_ERRHANDLER}}
\subsubsection{\mpifunc{MPI_WIN_SET_ERRHANDLER}}

\subsubsection{\mpifunc{MPI_WIN_GET_GROUP}}

\subsubsection{\mpifunc{MPI_WIN_GET_NAME }}
\subsubsection{\mpifunc{MPI_WIN_SET_NAME }}

\subsubsection{\mpifunc{MPI_WIN_LOCK} and \mpifunc{MPI_WIN_UNLOCK}}
There are two types of lock and unlock implementations.  In the most
obvious, based on the name, \code{MPI_WIN_LOCK} waits until the
request process acknowledges the lock.. This is appropriate when the
window is in memory that is shared among the processes in the window
object, such as a fully shared-memory implementation or a distributed
shared memory implementation.

For systems without direct access to the memory, an alternate but
equally valid approach is to make the lock a local operation, and wait
to issue it until the first RMA operation.  This is particularly
appropriate when the RMA operation (e.g., the put or accumulate)
involves a small amount of data and the interprocess communications
have high latency.  In fact, in the high-latency case, we may prefer
to hold all operations until the \code{MPI_WIN_UNLOCK} and then issue
them in a single communication.  I believe this is similar to what BSP
does, but for fence operations (I need the same discussion under fence).

Question.  For the nonblocking lock case, should we have an info key
for \mpifunc{MPI_WIN_CREATE} that asks for the blocking lock?  

\subsubsection{\mpifunc{MPI_WIN_POST}}
\subsubsection{\mpifunc{MPI_WIN_START}}
\subsubsection{\mpifunc{MPI_WIN_COMPLETE}}
\subsubsection{\mpifunc{MPI_WIN_WAIT}}

\subsection{Starting and Ending MPI}

\subsubsection{\mpifunc{MPI_ABORT}}
\subsubsection{\mpifunc{MPI_INIT_THREAD}}
\subsubsection{\mpifunc{MPI_QUERY_THREAD}}
\subsubsection{\mpifunc{MPI_IS_THREAD_MAIN}}
\subsubsection{\mpifunc{MPI_FINALIZED}}
\subsubsection{\mpifunc{MPI_INIT}}

As part of the error checking code, each routine should check the
state of the \code{is_initialized} flag.  Should there be an 
\begin{verbatim}
    enum { MPICH_PRE_INIT=0, MPICH_IS_INITIALIZED=1,
           MPICH_IS_FINALIZED=2 } MPIR_Initialized;
\end{verbatim}


\subsubsection{\mpifunc{MPI_INITIALIZED}}
\subsubsection{\mpifunc{MPI_FINALIZE}}
The MPI-2 standard requires that \code{MPI_Finalize} first delete the
attributes associated with \mpiconst{MPI_COMM_SELF}, even before
\mpifunc{MPI_FINALIZED} would return true.  This allows any number of
modules to attach ``end-of-job'' actions to \code{MPI_Finalize}.


\subsection{Dynamic Processes}

This section has not been written.  A big question is how or if the BNR
interface to a parallel process manager is used in the implementation
of the MPICH routines.  

\subsubsection{\mpifunc{MPI_CLOSE_PORT}}
\subsubsection{\mpifunc{MPI_COMM_CONNECT}}
\subsubsection{\mpifunc{MPI_COMM_DISCONNECT}}
\subsubsection{\mpifunc{MPI_COMM_GET_PARENT}}
\subsubsection{\mpifunc{MPI_COMM_JOIN}}
\subsubsection{\mpifunc{MPI_COMM_SPAWN}}
\subsubsection{\mpifunc{MPI_COMM_SPAWN_MULTIPLE}}

\subsubsection{\mpifunc{MPI_LOOKUP_NAME}}
Question:  Do we want to define an API or a wire protocol for the name
service.  

\subsubsection{\mpifunc{MPI_PUBLISH_NAME}}
\subsubsection{\mpifunc{MPI_UNPUBLISH_NAME}}

\subsubsection{\mpifunc{MPI_OPEN_PORT}}

\subsection{User-Defined Requests}

Question:  What ADI support is required for these?  Note that the
request is under the control of the device, so many of the fields
aren't defined yet.

Note that if \mpidfunc{MPID_Waitsome} is implemented in the ADI, then the
ADI must understand these requests (or at least be able to ignore
them).
Note that a user-defined request is started with callbacks (functions
to call for query, cancel, and free); these need to be associated with
the request.  

\subsubsection{\mpifunc{MPI_GREQUEST_START}}
\subsubsection{\mpifunc{MPI_GREQUEST_COMPLETE}}

(Need to describe the user-request handling)

\subsection{Error Handlers}


\subsubsection{\mpifunc{MPI_ERRHANDLER_CREATE}}

\subsubsection{\mpifunc{MPI_ERRHANDLER_FREE}}

\subsubsection{\mpifunc{MPI_ERRHANDLER_GET}}

\subsubsection{\mpifunc{MPI_ERRHANDLER_SET}}

\subsubsection{\mpifunc{MPI_ERROR_CLASS}}

\subsubsection{\mpifunc{MPI_ERROR_STRING}}

\subsubsection{\mpifunc{MPI_ADD_ERROR_CLASS}}

\subsubsection{\mpifunc{MPI_ADD_ERROR_CODE}}

\subsubsection{\mpifunc{MPI_ADD_ERROR_STRING}}

\subsection{Handle Transfers}

\subsubsection{\mpifunc{MPI_ERRHANDLER_F2C}}
\subsubsection{\mpifunc{MPI_ERRHANDLER_C2F}}
\subsubsection{\mpifunc{MPI_GROUP_F2C}}
\subsubsection{\mpifunc{MPI_GROUP_C2F}}
\subsubsection{\mpifunc{MPI_INFO_F2C}}
\subsubsection{\mpifunc{MPI_INFO_C2F}}
\subsubsection{\mpifunc{MPI_OP_F2C}}
\subsubsection{\mpifunc{MPI_OP_C2F}}
\subsubsection{\mpifunc{MPI_REQUEST_F2C}}
\subsubsection{\mpifunc{MPI_REQUEST_C2F}}
\subsubsection{\mpifunc{MPI_STATUS_F2C}}
\subsubsection{\mpifunc{MPI_STATUS_C2F}}
\subsubsection{\mpifunc{MPI_TYPE_C2F}}
\subsubsection{\mpifunc{MPI_TYPE_F2C}}

\subsection{I/O}
MPI I/O will be provided by ROMIO.  We expect to update ROMIO to exploit both
MPI-2 functions and to make use of MPID functions (such as the Stream and
Segment modules) when ROMIO is part of MPICH.

% \code{MPI_FILE_C2F}\\
% \code{MPI_FILE_CALL_ERRHANDLER }\\
% \code{MPI_FILE_CLOSE}\\
% \code{MPI_FILE_CREATE_ERRHANDLER}\\
% \code{MPI_FILE_DELETE}\\
% \code{MPI_FILE_F2C}\\
% \code{MPI_FILE_GET_AMODE}\\
% \code{MPI_FILE_GET_ATOMICITY}\\
% \code{MPI_FILE_GET_BYTE_OFFSET}\\
% \code{MPI_FILE_GET_ERRHANDLER}\\
% \code{MPI_FILE_GET_GROUP}\\
% \code{MPI_FILE_GET_INFO}\\
% \code{MPI_FILE_GET_POSITION}\\
% \code{MPI_FILE_GET_POSITION_SHARED}\\
% \code{MPI_FILE_GET_SIZE}\\
% \code{MPI_FILE_GET_TYPE_EXTENT}\\
% \code{MPI_FILE_GET_VIEW}\\
% \code{MPI_FILE_IREAD}\\
% \code{MPI_FILE_IREAD_AT}\\
% \code{MPI_FILE_IREAD_SHARED}\\
% \code{MPI_FILE_IWRITE}\\
% \code{MPI_FILE_IWRITE_AT}\\
% \code{MPI_FILE_IWRITE_SHARED}\\
% \code{MPI_FILE_OPEN}\\
% \code{MPI_FILE_PREALLOCATE}\\
% \code{MPI_FILE_READ}\\
% \code{MPI_FILE_READ_ALL_BEGIN}\\
% \code{MPI_FILE_READ_AT}\\
% \code{MPI_FILE_READ_AT_ALL_BEGIN}\\
% \code{MPI_FILE_READ_ORDERED}\\
% \code{MPI_FILE_READ_ORDERED_BEGIN}\\
% \code{MPI_FILE_READ_SHARED}\\
% \code{MPI_FILE_SEEK}\\
% \code{MPI_FILE_SEEK_SHARED}\\
% \code{MPI_FILE_SET_ATOMICITY}\\
% \code{MPI_FILE_SET_ERRHANDLER}\\
% \code{MPI_FILE_SET_INFO}\\
% \code{MPI_FILE_SET_SIZE}\\
% \code{MPI_FILE_SET_VIEW}\\
% \code{MPI_FILE_SYNC}\\
% \code{MPI_FILE_WRITE}\\
% \code{MPI_FILE_WRITE_ALL_BEGIN}\\
% \code{MPI_FILE_WRITE_AT}\\
% \code{MPI_FILE_WRITE_AT_ALL_BEGIN}\\
% \code{MPI_FILE_WRITE_ORDERED}\\
% \code{MPI_FILE_WRITE_ORDERED_BEGIN}\\
% \code{MPI_FILE_WRITE_SHARED}\\

% Related constants
% \code{MPI_MODE_APPEND}\\
% \code{MPI_MODE_CREATE}\\
% \code{MPI_MODE_DELETE_ON_CLOSE}\\
% \code{MPI_MODE_NOCHECK}\\
% \code{MPI_MODE_RDONLY}\\
% \code{MPI_MODE_SEQUENTIAL}\\
% \code{MPI_MODE_UNIQUE_OPEN}\\
% \code{MPI_SEEK_CUR}\\
% \code{MPI_SEEK_END}\\
% \code{MPI_SEEK_SET}\\
% \code{MPI_DISPLACEMENT_CURRENT}\\

\section{Portability}

This section discusses how the MPICH implementation is written to
provide portability to a wide variety of systems.  MPICH will continue
to rely on \code{configure}.  However, the use of \code{configure}
will rely on more carefully defined macros, along with more
information stored in external files, allowing for simpler adaptation
to site-specific environments, such as special compilers and runtime
environments. 

Question: What about shared libraries?  Using libtool is awkward for
development, but not using it is awkward for portability (libtool
knows a \emph{lot} about making shared libraries).  However, we
\emph{must} have support for shared libraries.  The plan is to develop
a simple perl program to extract the information stored in the
\code{libtool} source to take advantage of that source of information
without requiring the \code{libtool} development environment.

\subsection{Configure}
We will use autoconf version 2.13 or later.  The top-level configure will only
test for items used in the implementation of the MPI routines.  For
other parts of the package, such as the ADI implementation, the
top-level configure will invoke a configure or other setup script for
each package.

Question: configure understands how to invoke configure for other
packages.  If we use a level of indirection between the ADI configure
(e.g., a setup script), the top-level configure won't know about
this.  Do we care?  If we don't care, how do we ensure that
re-executing \file{config.status} executes the correct routines?

\subsection{Supporting Cross-compilation}
In some cases (e.g., IBM SP or ASCI Red), the compiler that must be used to
compile parallel programs produces executables that must be run under
the parallel environment, which may be difficult and time consuming.
This is a type of cross-compilation.  To support this, configure must
be carefully written both to support cross-compilation and to provide
for a way to specify the results that would have been determined by
running a program.

For each test that requires running a program, a variable of the form
\code{CROSS_xxx} must be defined and documented.  For example, for
variable sizes, \code{CROSS_SIZEOF_INT} will give the size of an
integer in bytes.

We need to document all \code{CROSS_xxx} variables.  Here is a start at that
list:
\begin{description}
\item[\texttt{CROSS_BYTE_ORDERING}]
\item[\texttt{CROSS_STRUCT_ALIGNMENT}]Structure alignment.  One of 
    \begin{description}
    \item[\texttt{packed}]No padding
    \item[\texttt{largest}]Aligned on the largest item
    \item[\texttt{two}]Aligned to two bytes
    \item[\texttt{four}]Aligned to four bytes
    \item[\texttt{eight}]Aligned to eight bytes
    \item[\texttt{other}]Unable to determine
    \end{description}
\item[\texttt{CROSS_SIZEOF_INT}]\code{sizeof(int)}
\item[\texttt{CROSS_SIZEOF_VOID_P}]\code{sizeof(void*)}
\item[\texttt{CROSS_SIZEOF_CHAR}]\code{sizeof{char}}
\item[\texttt{CROSS_SIZEOF_LONG}]\code{sizeof{long}}
\item[\texttt{CROSS_SIZEOF_FLOAT}]\code{sizeof{float}}
\item[\texttt{CROSS_SIZEOF_DOUBLE}]\code{sizeof{double}}
\item[\texttt{CROSS_SIZEOF_LONG_DOUBLE}]\code{long double}
\item[\texttt{CROSS_F77_SIZEOF_INTEGER}]The size of an \code{INTEGER} in
  Fortran (as if there was a \code{sizeof} operator in Fortran)
\item[\texttt{CROSS_F77_SIZEOF_REAL}]
\item[\texttt{CROSS_F77_SIZEOF_DOUBLE_PRECISION}]
\item[\texttt{CROSS_F90_INTEGER_KIND}]The Fortran 90 kind for an integer that
  corresponds to a C \code{int}.
\item[\texttt{CROSS_OFFSET_KIND}]The Fortran 90 kind for an integer that
  corresponds to a \code{MPI_Offset}.
\item[\texttt{CROSS_ADDRESS_KIND}]The Fortran 90 kind for an integer that
  corresponds to a \code{MPI_Aint}.
\end{description}

Other items, such as the allowed types for the f90 modules, also needs
to be speicfieable from the environment.

See also mpi-maint request 5556 for the values needed by the Intel
Tflops system.  It should be possible to specify these with a special
site file.  Also note the need to export variables; we may want to
create a step that does something like '\code{set | grep 'CROSS_' | 
sed ...}'.  

\subsubsection{Complex Configuration Data}
Much of the complexity in the current configure system comes from
handling special cases, particularly for compiler and linker options.
I propose replacing this code with a separate (yet simple) data list
that can be edited separately from the configure script, and which
contains information on various compilers and linkers.
This file can be considered a very simple database, where each record
contains the following information (some of the information is used
only by compile steps; other by link or shared library steps.  There
is enough overlap that the combined list is given).
\begin{description}
\item[kind]C, C++, Fortran 77, or Fortran 90.  Perhaps Java as well.
\item[action]compile, link, create static library, or create shared library
\item[name]E.g., cc, xlc, pgcc, gcc
\item[ostype]OS that has this compiler.  For some compilers, this is *
(e.g., gcc, pgcc, cc)
\item[optimize]Options for optimizing.
\item[debug]Options for debugging.
\item[ansi]Options to force ANSI (or a superset)
\item[posix]Options to force Posix
\item[threads]Options to allow threads
\item[sharedobject]Options to create a shared object
\item[version]Options to generate a version and name string.  See signature
\item[signature]A regular expression that should match the value
generated by version.  This may be slightly extended to allow a
particular line of the output to be matched.
\item[size32]Options for 32bit pointers
\item[size64]Options for 64bit pointers
\item[size=xx]Options for other sizes (128 bit pointers, anyone?) or
for special versions (e.g., IRIX n32)
\item[searchdir]Options to specify search directories for shared
libraries
\item[sharedlib]Options to create a shared library
\item[cross]Specify values for \code{CROSS_xxx} for cross-compiler case
\item[verbose]Options to generate verbose output, particularly showing
the command-line options passed to other tools such as ld when a
compiler command is used to link a program.  This is useful in
determining the libraries needed for multilanguage programs.
\item[verboseoptionsep]Option separator for the output from verbose.
Often either space or comma.
\item[strict]Options for strict (lint-like) compilation
\end{description}
Question: are these enough?  We should check what libtool uses.
We may also want a option that says ``check for a clean compile before
accepting these values''.  Compiler version numbers may also be
important; for that, there needs to be a command specified to extract and
compare the version number.

This file should have both specific rules and generic rules.  For
example, a generic description of \code{cc} would specify \code{-g}
for debugging, \code{-O} for optimization, and little else.  

In some cases, we may want to try several options.  For example, for
optimization, we may want to try \code{-Ofast} or \code{-O2} before \code{-O}.
Question:
what should the syntax for this be?

I recommend that the syntax for this file be key=value pairs, using a
trailing backslash to continue lines:
\begin{verbatim}
# comment describing compiler
kind=c action=compile name=xlc system=AIX \
  optimize="-O3 -qarch=native" \
  etc.
kind=c action=compile name=cc system=* \
  optimize=-O \
  debug=-g \
  etc.
\end{verbatim}
This is most easily processed using perl or (possibly) python, though
another alternative is to bootstrap by using the usual configure
macros to find a C compiler and then compile a simple program to read
this file.  In the cross compilation case, either a different compiler
may be used, the user can prebuild the program, or all of the
necessary data can be supplied through environment variables.

\subsection{Coding Rules}
We will attempt to enforce the coding rules by building tools that
check for conformance to these rules.

Tools already available:
\begin{description}
\item[checkforglobs]Check for global symbol names
\item[tool-to-be-created]Check for use of banned routines (e.g., \code{printf},
\code{alloca})
\item[gcc -Wall etc.]Missing prototypes, return values, etc.
\item[tool-to-be-created]Check for OS name or system type in a preprocessor
  statement 
\end{description}

We also need tools for the following:
\begin{itemize}
\item Look for functions and header files that are not universal and
ensure that they are properly guarded.
\item Look for functions that are not universal (e.g., \code{rindex}).
\end{itemize}

\subsubsection{Printing and Other Messages}
Rule: do not use \code{printf} or \code{fprintf} except (possibly) for
messages intended only the for the developers of MPICH2.

Even for developer messages, using \code{printf} is not a good idea.
It is better to call a routine that can handle recording the results.

% See PETSc approach for printing.  

Question: Should we define \code{PRINTF} etc. as we have in MPICH, or
ban those values entirely?  Should we define a general output routine
that can be implemented to output to a file, stdout, or a graphical display?

\subsection{NT Friendly}

Declare all user-visible routines \code{EXPORT_MPI_API}\index{EXPORT_MPI_API}.
This is a 
macro that is defined as empty for UNIX and as the appropriate
Microsoft-specific extension for Windows.

Avoiding \code{printf} is important for Windows applications.

\subsection{Fortran Support}

\subsection{C++ Support}

I'd like to consider adding more native C++ support.  While the Notre
Dame C++ code is valuable and helpful, there are some difficulties:
\begin{enumerate}
\item Supports only MPI-1.
\item Their configure is based too much on particular systems rather than on
capabilities. 
\item Because it is a separate package, the build process is awkward.
\item Testing is separate from the MPICH testing, causing inadequate
testing of the MPICH/C++ combination.
\item Layering makes some things difficult. 
\item Does not pass our (sometimes stricter) coding standards. 
\end{enumerate}

\section{Standard Features}
(this section contains the standard features, such as command line
handling, environment variables, configure options for error checking,
etc.)

The implementation must provide the \emph{service} of providing
command-line arguments to each process.  If the startup environment
does not do so, the implementation must (see the discussion of
\mpidfunc{MPID_Thread_init}).  

\section{Testing}

The tests for MPICH2 need to be more organized that for MPICH. They
should follow these principles:
\begin{enumerate}
\item Require no user intervention (e.g., to inspect the results)
\item Should be implementation independent (i.e., useful for
\emph{any} MPI implementation) unless they are testing \emph{only}
features specific to MPICH2.  An example is a test of error messages
and error class/code values; any test that expects a particular
message must not be combined with a general test of standard
conformance.
\item Require no extra files (the \file{.std} files in the MPICH
version).
\item Testing should be applied to a range of communicators and
datatypes, not just \code{MPI_COMM_WORLD}.  
\item Testing should be controlled by a file listing the tests rather
than a script.  That is, the script \code{runtests} should be generic,
working with a file in each testing directory that lists the test
programs and any special options (e.g., number of processes, command
line arguments, environment variables).
\item Testing that enables memory tracing should be simple and
organized so that it can be run regularly.
\item Should have short duration, so that the testing program can
signal failure because a program has not completed within the given
time.
\end{enumerate}

Results should be maintained in a database which should include
\begin{enumerate}
\item Configuration options
\item System description (including compiler version and machines file)
\item CVS tag (or nearest value, e.g., weekly tag + date that source
cut was made).
\item Results summary (success or failure; if failure, reason).
\item We should consider an XML format for the output.
\end{enumerate}

Testing should also contain performances tests.
\begin{enumerate}
\item Configuration options
\item System description (including compiler version and machines file)
\item CVS tag (or nearest value, e.g., weekly tag + date that source
cut was made).
\item Results for the following tests:
    \begin{enumerate}
    \item \code{mpptest -logscale} To get the general trend in performance
    \item \code{mpptest -autodx} To get details for the short message
performance. 
    \item bisection bandwidth test for a large number of processes.
    We may want to use beff instead of mpptest.
    \item \code{mpptest -logscale -halo -npartner 8} To get more
    realistic communication performance
    \item \code{mpptest -logscale -gop dsum} This isn't quite right.
    We need both the scaling and the performance as a function of
    message size.
    \item Similar test for alltoallv. 
    \item Tests for I/O (Rajeev and Rob to identify)
    \item Tests for put/get/accumulate as those become available.
    Start with \code{mpptest -logscale -put} and \code{mpptest
-logscale -halo -npartner 8}
    \end{enumerate}
\item This should also be in XML format.
\end{enumerate}
We should maintain the same data for vendor and other MPI
implementations, and we should ask for a standard set of tests.  

\subsection{Debugger Interface}
We need a test that the DLL that provides access to the internal
symbols is correct, as well as a test that the MPICH library correctly
identifies the location of the DLL.

To test the DLL, we need a program \file{mpichdlltest.c} that can
load the dll and call the routines, ensuring that it runs correctly.
This program should use the ADI's include files to provide the data
structures that the DLL accesses.

Note that the debugger interface must be compiled in 32-bit mode on
platforms with mixed 32 and 64-bit modes.  We don't currently do this
(which is a bug) but we need to.  To do this, we need a configure step
that may need to know that a system has both options (in the case that
\code{sizeof(void *)} is 64.  See ``complex configuration
options'' above.

Still needed: Information needed for debugger startup and message
queues.

\subsection{Performance and Tracing Data}

One critical piece of information that is hard currently to acquire is
the amount of idle time spent in completing a communication operation.
This is information that a tracing library would like to have.  Should
there be an \code{MPID_xxx} call that could be made available to a tracing
package?  For example, it could have semantics roughly like
\code{MPI_Wtime}, except that it would contain cumulative idle time.  In a
multithreaded environment, it could give a per-thread cumulative idle
time (or could it)?  Should we define \code{MPID_Idle_time}, and modify MPE
to look for that name in the MPI library?

To tune the layered routines, including the collective routines, we
should include from the beginning standardized tracing for:
\begin{enumerate}
\item All (major?) MPID calls
\item Idle time
\item Context switches (if using threads)
\item Resource usage
\item Flow control
\end{enumerate}

%\let\SaveBibliography=\thebibliography
%\def\thebibliography#1{\SaveBibliography{#1}\addcontentsline{toc}{section}{References}}

\section{Appendix}
This section contain \emph{all} of the MPI functions and terms.  These
will be moved into the body of this document; an index will then
replace this appendix.  Until then, this section serves as a place to
cache these items.

\subsection{Constants}
\code{MPI::ERRORS_THROW_EXCEPTIONS}\\
\code{MPI_ADDRESS_KIND}\\
\code{MPI_ANY_TAG}\\
\code{MPI_APPNUM}\\
\code{MPI_ARGVS_NULL}\\
\code{MPI_ARGV_NULL}\\
\code{MPI_BOTTOM}\\
\code{MPI_CONVERSION_FN_NULL}\\
\code{MPI_DISTRIBUTE_BLOCK}\\
\code{MPI_ERRCODES_IGNORE}\\
\code{MPI_ERR_ACCESS}\\
\code{MPI_ERR_AMODE}\\
\code{MPI_ERR_BASE}\\
\code{MPI_ERR_CONVERSION}\\
\code{MPI_ERR_DUP_DATAREP}\\
\code{MPI_ERR_FILE_IN_USE}\\
\code{MPI_ERR_NAME}\\
\code{MPI_ERR_NO_MEM}\\
\code{MPI_ERR_NO_SUCH_FILE}\\
\code{MPI_ERR_PORT}\\
\code{MPI_ERR_SERVICE}\\
\code{MPI_ERR_SPAWN}\\
\code{MPI_FILE_NULL}\\
\code{MPI_F_STATUSES_IGNORE}\\
\code{MPI_INFO_NULL}\\
\code{MPI_INTEGER_KIND}\\
\code{MPI_LASTUSEDCODE}\\
\code{MPI_LOCK_EXCLUSIVE}\\
\code{MPI_MAX_DATAREP_STRING}\\
\code{MPI_MAX_OBJECT_NAME}\\
\code{MPI_MAX_PORT_NAME}\\
\code{MPI_OFFSET_KIND}\\
\code{MPI_ORDER_C}\\
\code{MPI_ORDER_FORTRAN}\\
\code{MPI_REPLACE}\\
\code{MPI_ROOT}\\
\code{MPI_STATUSES_IGNORE}\\
\code{MPI_STATUS_IGNORE}\\
\code{MPI_SUBVERSION}\\
\code{MPI_THREAD_FUNNELED}\\
\code{MPI_TYPECLASS_COMPLEX}\\
\code{MPI_UNIVERSE_SIZE}\\
\code{MPI_VERSION}\\
\code{MPI_WCHAR}\\
\code{MPI_WIN_BASE}\\
\code{MPI_WIN_NULL}\\
\code{MPI_WIN_SIZE}\\

\subsection{Info keys}
\code{access_style}\\
\code{appnum}\\
\code{arch}\\
\code{cb_block_size}\\
\code{external32}\\
\code{false}\\
\code{file}\\
\code{file_perm}\\
\code{host}\\
\code{internal}\\
\code{io_node_list}\\
\code{ip_address}\\
\code{native}\\
\code{nb_proc}\\
\code{no_locks}\\
\code{num_io_nodes}\\
\code{path}\\
\code{random}\\
\code{soft}\\
\code{striping_factor}\\
\code{wdir}\\
\code{write_mostly}\\

\subsection{MPI-1 Functions}
\code{MPI_GET_PROCESSOR_NAME}\\
\code{MPI_PCONTROL}\\
\code{MPI_WTICK}\\
\code{MPI_WTIME}\\

\subsection{MPI-2 Functions}
This also includes MPI-1 functions with significant new functionality,
such as the collective communication routines that are extended, in
MPI-2, from intracommunicators to intercommunicators.

\code{MPI_GET_VERSION}\\
\\
\code{MPI_PACK_EXTERNAL}\\
\code{MPI_PACK_EXTERNAL_SIZE}\\
\code{MPI_UNPACK_EXTERNAL}\\
\\
\code{MPI_REGISTER_DATAREP}\\
\\
\code{MPI_SIZEOF}\\
\\
\code{MPI_TYPE_DUP}\\
\code{MPI_TYPE_MATCH_SIZE}\\
\\


\subsection{Typedefs}
\code{MPI_Aint}\\
\code{MPI_Comm_copy_attr_function}\\
\code{MPI_Comm_errhandler_fn}\\
\code{MPI_Datarep_conversion_function}\\
\code{MPI_Datarep_extent_function}\\
\code{MPI_File_errhandler_fn}\\
\code{MPI_Grequest_cancel_function}\\
\code{MPI_Type_copy_attr_function}\\
\code{MPI_Win_copy_attr_function}\\
\code{MPI_Win_errhandler_fn}\\

\section{Development Plan}
\label{sec:development}
This section proposes a development and implementation plan.

\begin{enumerate}
\item ADI Implementation
    \begin{enumerate}
    \item Multi-method ADI for shared memory, VIA, and TCP/UDP
	\begin{enumerate}
	\item Design and implement general datatype point-to-point
	\item Design and implement general datatype collective
        (layered on ADI segment and stream interfaces)
  	\item Design and implement RMA
	\item Design and implement Dynamic and connect with BNR
	\end{enumerate}
    \item ADI on \code{MPID_CORE}
        \begin{enumerate}
	\item Implement all ADI-3 modules on top of the core.  Refine
        the design of the core during this process.   
        \end{enumerate}
    \item Implement common ADI services.  These are modules that are
    needed by other ADI modules that do not (directly) involve
    interprocess communication but will be common for most ADI
    implementations, including both the core and the multi-method
    implementation. 
	\begin{enumerate}
	\item Error handling
	\item Attributes
	\item Topology
	\item Datatype
	\item Group
	\item Timer
	\item Utility
	\end{enumerate}
    \end{enumerate}
\item MPICH Implementation on ADI
    \begin{enumerate}
    \item For each MPI routine, write the ``top'': structured comment,
    routine header and error checking.
    \item Implement the action of each MPI routine in terms of the
    full ADI-3 design.
    \end{enumerate}
\item Test Suite
    \begin{enumerate}
    \item Design standard test harness: test communicators, datatypes,
    operation mixtures, run script, and result checking.
    \end{enumerate}
\item Deployment
    \begin{enumerate}
    \item Implement new configure.  Design database for both configure
    macros and for data (such as compiler options and names) that
    configure cannot determine.
    \item Shared library support.
    \end{enumerate}
\end{enumerate}

%
% Eventually include table or a project as Postscript from project.
%

\let\SaveBibliography=\thebibliography
\def\thebibliography#1{\SaveBibliography{#1}\addcontentsline{toc}{section}{References}}
\bibliography{/home/MPI/allbib,/home/gropp/Update/new/gropp,mpich2}
\bibliographystyle{plain}

% Index
%\openin\testfile{adi3man.ind}
%\ifeof\testfile\else
\let\SaveIndex=\theindex
\def\theindex{\SaveIndex\addcontentsline{toc}{section}{Index}}
\input mpich2.ind
%\fi
%\closein\testfile

\end{document}
