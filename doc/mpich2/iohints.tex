\subsection{I/O Hints}

\subsubsection{Controlling Collective I/O}

MPI provides a mechanism based on MPI \emph{hints} to allow the user to
provide information that an improve performance.  Note that all hints do
\emph{not} change the semantics of the operations but can influence the
algorithms used by the MPI implementation.  For collective I/O, MPI defines
two \code{MPI_Info} key values:
\begin{description}
\item[\texttt{cb_buffer_size}] specifies the size of the buffer used for
  aggregation, and might allow a user to do some tuning.
\item[\texttt{cb_nodes}]specifies the number of processes that will actually
  do I/O, although it doesn't let you name them.
\end{description}
These provide information that can be used by an implementation that uses
collective buffering and aggregation algorithms to perform collective I/O in
MPI.  Note that ROMIO implements both of these.

% This solves your first problem, which is to constrain the number of processes
% doing I/O and to control the amount of memory dedicated to it, but doesn't
% allow you to specify exactly which processes do it.

However, for some systems configurations, more control is needed to specify
which hardware resources (processors or nodes in an SMP) are preferred for
collective I/O.  MPICH will provide the additional \code{MPI_Info} key name
\begin{description}
\item[\texttt{cb_config_list}]specifies a comma-separated list of strings,
  each string specifying a particular node and an optional limit on the number
  of processes to be used for collective buffering on this node.  Here, node
  refers to a particular processing element with a unit of I/O connectivity;
  typically, this is a single SMP unit.
\end{description}
This refers to the same processes that \code{cb_nodes} refers to, but
specifies the available nodes more precisely.

The format of the value of \code{cb_config_list} is given by the following BNF:
\begin{verbatim}
cb_config_list => hostspec [ ',' cb_config_list ]
hostspec       => hostname [ ':' maxprocesses ]
hostname       => <alphanumeric string>
               |  '*'
maxprocesses   => <digits>
               |  '*'
\end{verbatim}
The value \code{hostname} identifies a processor.  This name must match the
name returned by \code{MPI_Get_processor_name}\footnote{The MPI standard
  requires that the output from this routine identify a particular piece of
  hardware; some MPI implementations may not conform to this requirement.
  MPICH does conform to the MPI standard.} for the specified hardware.
The value \code{*} matches all processors.  The value of \code{maxprocesses}
may be any nonnegative integer (zero is allowed).
% The footnote is for SGI

Question: There is a conflict between our use of node and the processor name
returned by \code{MPI_Get_processor_name}.  An implementation that identified
the processor within the SMP would conform to the MPI specification but would
not meet our requirements above.  We may want to change the discussion above
to indicate that it will use \code{MPI_Get_processor_name}, but that a second
routine, \code{MPIR_Get_node_name} may be used instead.  This routine would
identify a ``node'' according to the above definition.

The value \code{maxprocesses} specifies the maximum number of processes that
may be used for collective buffering on the specified host.
If no value is specified, the value one is assumed.  If \code{*} is specified,
then as many processes as there are processors may used (in other words, all
MPI processes with the same hostname).

All three of these info keys can be specified when a file is opened with
\code{MPI_File_open} or after the file is opened with \code{MPI_File_set_info}
or \code{MPI_File_set_view}.  In all three cases, the values of these info
keys must be the same for all processes in these calls (i.e., the calls are
collective and each process must receive the same hint value for these
collective buffering hints).

The set of hints used with a file is available through the routine
\code{MPI_File_get_info}, as documented in the MPI-2 standard.  As an
additional feature in the MPICH implementation, wildcards will be expanded to
indicate the precise configuration used with the file, with the hostnames in
the rank order used for the collective buffering algorithm.

% We propose a new hint, to be called cb_config_list, which is a
% comma-separated 
% list of strings, each string of which is a hostname:number-of-processes-to-
% perform-real-I/O-from-this-node, where "*" means all possible hosts.

Here are some examples of what this might look like:

\begin{description}
\item[\texttt{*:1}]One process per hostname, i.e., one process per node)
\item[\texttt{box12:30,*:0}] Thirty processes on one machine, namely box12,
  and none anywhere else.
\item[\texttt{n01,n11,n21,n31,n41}]Processes on these specific nodes only.
\end{description}

When the values specified by \code{cb_config_list} conflict with other hints
(e.g., the number of collective buffering nodes specified by \code{cb_nodes}),
the implementation is encouraged to take the minimum of the two values.  In
other words, if \code{cb_config_list} specifies ten processors on which I/O
should be performed but \code{cb_nodes} specifies a smaller number, an
implementation is encouraged to use only \code{cb_nodes}.  If
\code{cb_config_list} specifies fewer processes than \code{cb_nodes}, only the
number in \code{cb_config_list} should be used.

The processors specified by \code{cb_config_list} may include processors that
are not part of the MPI job.  In this case, the MPI implementation may ignore
those processors.  This allows a single \code{cb_config_list} specification to
identify all prefered I/O nodes, without requiring the user to first restrict
that list to those nodes available to the MPI job.

The implementation is also encouraged to assign processes in the order that
they are listed in \code{cb_config_list}.  

% Question: do we want to take the minimum of the two values?  E.g., if
% \code{cb_nodes=3,cb_config_list=*:1}, use only three nodes?

To provide more control over the behavior of MPICH, the default values of
these hints may be set at runtime using the standard MPICH runtime parameter
method.  The environment variables and command-line parameters for
\code{mpiexec} are:
\begin{center}
\begin{tabular}{l|l}
Environment variable&Command Line Parameter\\\hline
\code{MPICH_IO_CB_BUFFER_SIZE}&\code{--mpich-io-cb-buffer-size=n}\\
\code{MPICH_IO_CB_NODES}&\code{--mpich-io-cb-nodes=n}\\
\code{MPICH_IO_CB_CONFIG_LIST}&\code{--mpich-io-cb-config-list=list}
\end{tabular}
\end{center}
These provide default values that may be overridden by an explicit use of an
\code{MPI_Info} parameter in a program.

The support for environment variables and command-line arguments will be
available in MPICH2.
