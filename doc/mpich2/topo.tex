\ifx\indocument\undefined
\documentclass{article}
\usepackage[dvipdfm]{hyperref}
\usepackage{mpidoc}
\let\subsubsection=\section
\begin{document}
\fi

\subsubsection{Proposed Interface}
The device may implement \code{MPID_Cart_map} and/or \code{MPID_Graph_map} .
These are very similar to the MPI routines of the same names, and can be used
to communication information on the topology of the underlying interconnect
and process layout to the MPI routines.  If the device does implement these
routines, it must define the corresponding C preprocessor value to indicate
that the routine is available.  If the device does not provide the routine,
then the MPICH implementation will provide a simple default.  Note that 
the \code{MPID_Cart_map} and \code{MPID_Graph_map} routines are sufficient for
implementing the MPI topology routines, as described in the MPI-1 standard.

The specifics are

Use
\begin{verbatim}
    #define MPID_HAVE_CART_MAP 
\end{verbatim}
if the device provides \code{MPID_Cart_map}.  The binding for this routine is
\begin{verbatim}
int MPID_Cart_map( MPID_Comm *comm_ptr, int dims, const int dims[], 
                   const int periods[], int *newrank )
\end{verbatim}
Use 
\begin{verbatim}
#define MPID_HAVE_GRAPH_MAP
\end{verbatim}
if the device provides \code{MPID_Graph_map}.  The binding for this routine is
\begin{verbatim}
int MPID_Graph_map( MPID_Comm *comm_ptr, int nnodes, const int index[], 
                    const int edges[], int *newrank )
\end{verbatim}
Use
\begin{verbatim}
#define MPID_HAVE_DIMS_CREATE
\end{verbatim}
if the device provides \code{MPID_Dims_create}.  The binding for this routine
is 
\begin{verbatim}
int MPID_Dims_create( int nnodes, int ndims, int *dims )
\end{verbatim}

These routines should return valid MPI error codes (not classes!) if an error
is detected.  They \emph{may} assume that the input communicator and output
pointer are valid (checked in the calling routine). 

These routines should perform any initialization that they require on the
first call.  If they allocate resources (e.g., malloc memory), they must
register a finalize handler to clean up on exit. 

It is the long-term goal of the MPICH group to provide sample implementations
of these for several important classes of machine interconnects.  However,
until that time, these routines provide a way for a device implementor to
communicate topology information to the MPI routines. 

\subsubsection{Proposed Interface 2}
An alternative to the above interface would be an interface that allowed
the device to specify one of the following topology types:
\begin{description}
\item[cart]Cartesian; the device provides the number of dimensions, the size 
of each dimension, and whether the dimension is periodic (i.e., a
torus or mesh).  Systems with SMPs connected on a mesh can use a first
dimension with size 2 and periodic.
\item[heirarchical]The device provides, for each process, a set of
  levels and the color and key of the process within that level.
  These arguments have meanings similar to \mpifunc{MPI_Comm_split}.
  Processes that belong to the same group at a particular level (e.g.,
  an SMP or a cluster) have the same value of \code{color}; each
  process has a distinct value of \code{key}.  The number of levels is
  provided by the device, allowing the description of an arbitrary
  hierarchy of processes.  In addition, at any level, the processes
  with the same \code{color} may have additional structure; e.g., they
  may have cartesian topology.
\item[switched]The processes are connected by a switched network that
  either provides or approximates a complete connection network.  This
  is appropriate for systems with full bisection bandwidth independent
  of the number of processes and with good handling of contention.
  Note that most large systems will only approximate this, but it may
  still be an appropriate choice because the details of the
  interconnect are too complex to be exploited.
\item[bus]The processes are connected by a shared resource, such as a
  bus, non-switched Ethernet (e.g., using hubs), or even with switched
  networks that do not have adequate bandwidth to handle all processes
  at one time.  One additional parameter may be the number of
  processes that may communicate simultaneously without significant
  contention.
\end{description}

\ifx\indocument\undefined
\gdef\LAST{\end{document}}
\afterassignment\LAST
\fi
% Dummy to trigger the \LAST
\count9=0
