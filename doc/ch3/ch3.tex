% 
%   This is a latex file that generates a reference manual for 
%   ADI-3 
%
\documentclass{article}
\usepackage{tpage}
\usepackage{refman}
\usepackage{url} 
\usepackage{epsf}
\usepackage{psfig}
%hyperref - do not remove this comment

\textheight=9in
\textwidth=6.1in
\oddsidemargin=.2in
\topmargin=-.50in
\newread\testfile

%
% For now, let findex be the same as index.  This will allow us to
% more easily separate function and nonfunction index entries later.
\let\findex=\index
%
% Modify the way titles are handled for no breaks between pages
\def\mantitle#1#2#3{\pagerule\nobreak
\ifmancontents\addcontentsline{toc}{subsection}{#1}\fi
\index{#1}}

\makeindex

\begin{document}

\markright{CH3 Reference Manual}

\def\nopound{\catcode`\#=13}
{\nopound\gdef#{{\tt \char`\#}}}
\catcode`\_=13
\def_{{\tt \char`\_}}
\catcode`\_=11
\def\code#1{{\tt #1}}
%\let\url=\code
\def\makeussubscript{\catcode`\_=8}
\def\makeustext{\catcode`\_=11}
%\tpageoneskip
\ANLTMTitle{MPICH Abstract Device Interface\\
Version 3.3\\
Reference Manual\\\ \\Draft of \today}{\em 
William Gropp\\
Ewing Lusk\\
Brian R. Toonen\\
Your Name Here\\
Mathematics and Computer Science Division\\
Argonne National Laboratory}{00}{\today}

\clearpage

\pagenumbering{roman}
\tableofcontents
\clearpage

\pagenumbering{arabic}
\pagestyle{headings}

\section{Introduction}
\section{Functions}

% Index
%\openin\testfile{ch3.ind}
%\ifeof\testfile\else
\let\SaveIndex=\theindex
\long\def\theindex#1{\SaveIndex{#1}\addcontentsline{toc}{section}{Index}}
\input ch3.ind
%\fi
%\closein\testfile

\end{document}
